Є~рюкзак місткістю $V$~мм$^3$.
Є~$N$~сипучих речовин;
першої з них є у наявності $v_1$~мм$^3$, причому вона має питому вартість $p_1$~{коп}/{мм$^3$};
другої є $v_2$~мм$^3$, з питомою вартістю $p_2$~{коп}/{мм$^3$};
і так далі, до {$N$-ої},
якої є $v_N$~мм$^3$, з питомою вартістю $p_N$~{коп}/{мм$^3$}.
Яку найбільшу вартість цих речовин можна набрати, не~перевищивши місткість рюкзака?
Припускаємо, ніби сипучі речовини можна розділити одну від одної, зовсім не~втративши на цьому корисний об'єм рюкзака.


\InputFile
Перший рядок містить об'єм рюкзака~$V$ ($10\leqslant V\leqslant 10^9$).
Др{\it у}гий рядок містить кількість речовин~$N$ ($1\leqslant N\leqslant 10^5$).
Кожен з подальших $N$ рядків містить спочатку кількість відповідної сипучої речовини $v_i$ ($1\leqslant v_i\leqslant 10^7$), потім (через пробіл) її питому вартість $p_i$ ($1\leqslant v_i\leqslant 10^7$).
$V$ та $v_i$ вимірюються у мм$^3$,
$p_i$ --- у {коп}/{мм$^3$},
$N$ у штуках (безрозмірна).


\OutputFile
Ваша програма повинна вивести єдине число --- максимальну можливу вартість (у копійках) вибраних речовин.

\Example

\begin{example}
\exmp{200
3
10 40000
50 2000
2000 5}{500700}%
\end{example}
