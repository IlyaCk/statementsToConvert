За~правилами змагань ICPC, задача може бути лише або зарахованою, або~ні (часткових балів не~буває).
Тому природньо, що при визначенні переможців найважливішим фактором є кількість розв'язаних задач.
У~випадку, коли різні команди розв'язали однакову кількість задач, вони ранжуються за так званим штрафом.
Штраф нараховується так: 
\begin{itemize}
\item
якщо задача не~зарахована, вона не~приносить штрафа (незалежно від того, чи~пробувала команда її здавати);
\item
якщо зарахована, то штраф дорівнює кількості хвилин (повних чи не~повних) від моменту початку туру до моменту, коли зарахована ця задача, плюс по 20~хвилин за кожну невдалу спробу здачі;
\item
однак, якщо спроба відбувається вже після того, як задача зарахована, вона --- байдуже, вдала чи ні --- не~впливає на штраф. 
\end{itemize}
 
Команда Dream Team має дві надзвичайні властивості:
\begin{enumerate}
\item
вона вміє не~лише миттєво прочитати всі умови всіх задач, а ще й визначити, скільки хвилин займе процес розв'язування кожної з них;
\item
вона ніколи не~допускає помилок: якщо почне розв'язувати деяку задачу, то успішно здасть правильний розв'язок рівно через стільки хвилин, як визначила з самого початку.
\end{enumerate}

Разом з тим, команда Dream Team не~вміє у розпаралелювання обов'язків між членами команди, тому завжди і всюди може переходити до розв'язування наступної задачі лише після того, як (успішно) здала попередню.

Команда Dream Team в курсі, що на ICPC дозволяється здавати задачі в будь-якому порядку, і хоче вибрати такий порядок здачі, щоб {\it за рівно 300~хвилин (5~годин)} змагання отримати якнайкращий результат.

\InputFile
Перший рядок містить кількість задач $N$ ($1\leqslant N\leqslant 100$).
Др{\it у}гий рядок містить $N$ цілих чисел від~1 до~$10^{9}$ --- визначені командою Dream Team тривалості розв'язування {1-ї}, {2-ї},~\dots, {$N$-ї} задач.

\OutputFile
Знайдіть найкращий можливий результат (кількість задач та штраф), який може забезпечити команда Dream Team, якщо вибере найкращий можливий порядок розв'язування задач.

\Examples
\begin{example}
\exmp{5
100 500 20 180 200}{3 440}%
\end{example}

\Note
Раніше вже стверджували, що в поточному формулюванні задачі є певний слизький момент.
Однак, він цілком вирішується, якщо враховувати, що приклад вхідних даних та результатів теж є частиною умови.