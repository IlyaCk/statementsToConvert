На площині задані координати $n$ ($4\leqslant n\leqslant 12$) різних вершин.

Знайти довжину найкоротшого замкнутого маршруту, що починається і закінчується в 1-й вершині і відвідує всі інші вершини по одному разу. Дозволяється (якщо так виявляється вигідно) «проїжджати через вершину, не зупиняючись» (див. приклад~1).

Довжина маршруту обчислюється як сума довжин складових його ребер, довжини окремих ребер обчислюються згідно звичайної евклідової метрики, як $\sqrt{(x_A-x_B)^2+(y_A-y_B)^2}$.

\InputFile
Перший рядок містить кількість вершин $n$ ($4\leqslant n\leqslant 12$). Кожен з наступних $n$ рядків містить по два розділених пропуском числа з плаваючою точкою --- $x$- та $y$-координати відповідної вершини.

\OutputFile
Результати повинні містити єдине число (з~плаваючою крапкою) --- знайдену мінімальну довжину замкнутого маршруту.

\Examples
\begin{example}
\exmp{4
0 0
2 0.2
7 0.7
5 0.5}{1.40698258695692E+0001}
\exmp{5
1 0
4 4
3 2
4 0
1 1}{1.24721359549995E+0001}
\end{example}

\Note
Задача з такими обмеженнями, по ідеї, повинна вирішуватися хоч методом гілок і меж, хоч динамічним програмуванням по підмножинам, хоч самими лише відтинаннями пошуку з поверненням (backtracking), що не~намагається оцінювати можливий діапазон довжини ще~не~збудованої частини шляху. 