 Завдання в цілому аналогічне завданням «Структура (колекція) set» та «Структура (колекція) set від власного типу — 1» (і наполегливо рекомендується зробити спочатку їх, саме в такому порядку), але елементами повинні бути не числа, а дещо складніші об'єкти: прямокутники, кожен з яких описується довжинами двох своїх сторін {\it a}, {\it b} (два цілі числа) та кольором ({\tt string}) {\it col}. 

 Cлід постійно підтримувати чотири множини, виходячи з таких чотирьох можливих трактувань задачі:
\begin{enumerate}
\item
    повертати прямокутники заборонено (довжина повинна бути довжиною, ширина повинна бути шириною), колір враховується; інакше кажучи, елемент-прямокутник належить множині тоді й тільки тоді, коли множина містить елемент-прямокутник, в якого такі самі довжина і ширина (в такому самому порядку), причому такого самого кольору (в смислі звичайної case-sensitive рівності рядків);
\item
    повертати прямокутники заборонено (довжина повинна бути довжиною, ширина повинна бути шириною), колір ігнорується; інакше кажучи, елемент-прямокутник належить множині тоді й тільки тоді, коли множина містить елемент-прямокутник, в якого такі самі довжина і ширина (в такому самому порядку), але колір не перевіряється, може бути хоч однаковим, хоч різним;
\item
    повертати прямокутники дозволено (можна хоч залишити довжину довжиною та ширину шириною, хоч повернути, зробивши довжину шириною та ширину довжиною), колір враховується; інакше кажучи, елемент-прямокутник належить множині тоді й тільки тоді, коли множина містить елемент-прямокутник, в якого такі самі розміри, але порядок цих розмірів може бути хоч таким самим, хоч переставленим; при цьому колір повинен бути таким самим (у смислі звичайної case-sensitive рівності рядків);
\item
    повертати прямокутники дозволено (можна хоч залишити довжину довжиною та ширину шириною, хоч повернути, зробивши довжину шириною та ширину довжиною), колір ігнорується; інакше кажучи, елемент-прямокутник належить множині тоді й тільки тоді, коли множина містить елемент-прямокутник, в якого такі самі розміри, але порядок цих розмірів може бути хоч таким самим, хоч переставленим; при цьому колір не перевіряється, може бути хоч однаковим, хоч різним.
\end{enumerate}

(див. також пояснення до прикладів). 

Кожен із запитів {\tt ADD} та/або {\tt PRESENT} має по три параметри {\it a b col}, саме в такому порядку. Вони задаються в одному рядку через пробіли.

Напишіть програму, яка виконуватиме послідовність запитів виду {\tt ADD} {\it a b col}, {\tt PRESENT} {\it a b col} та {\tt COUNT} (без параметрів). Програму обов'язково слід писати за допомогою бібліотечного типу (колекції) set (її реалізації в~конкретних мовах програмування можуть називатися {\tt HashSet}, {\tt TreeSet}, {\tt SortedSet},~\dots).

Виконання кожного запиту виду {\tt ADD} {\it a b col} повинно додавати прямокутник до всіх чотирьох множин (якщо такий само прямокутник вже є, додавання ще однієї копії не змінює множину; слід розуміти, що можливі ситуації, коли з точки зору деяких із трактувань 1, 2, 3, 4 такий прямокутник вже є й додавання ще однієї копії не змінить множину, а з точки зору інших трактувань такого прямокутника нема і його слід обов'язково додати). На екран при цьому нічого не виводиться. 

При виконанні кожного запиту виду {\tt PRESENT} {\it a b col} повинно видаватися рівно чотири повідомлення «{\tt YES}» чи «{\tt NO}» (великими літерами, в одному рядку через пробіли, без лапок), на позначення того, чи наявний такий прямокутник у кожній з цих множин, згідно описаних вище трактувань 1, 2, 3, 4, саме в такому порядку. Значення множин при цьому не змінюються. 

При виконанні кожного запиту виду {\tt COUNT} повинно видаватися (цифрами, в одному рядку через пробіли) рівно чотири кількості елементів, згідно описаних вище трактувань 1, 2, 3, 4, саме в такому порядку. Значення множин при цьому не змінюються.


\InputFile

У першому рядку задано кількість запитів $N$ ($5\leqslant N\leqslant 2\cdot10^5$), далі йдуть $N$ рядків, кожен з яких містить по одному запиту згідно з описаним форматом. Всі параметри {\it a} та {\it b} є цілими числами від 1 до 1234567890. Всі параметри {\it col} є рядками, що містять від 1 до 15 латинських (англійських) літер, без будь-яких інших символів. 



\OutputFile

Виводьте на стандартний вихід (екран) в окремих рядках результати запитів {\tt PRESENT} (рівно чотири «{\tt YES}» чи «{\tt NO}» великими літерами, в одному рядку через пробіли, без лапок) та COUNT (рівно чотири числа, в одному рядку через пробіли); на запити ADD нічого виводити не слід. 

\Examples

\begin{example}
\exmp{12
ADD 5 2 red
COUNT
PRESENT 5 2 red
PRESENT 5 2 ReD
PRESENT 2 5 red
ADD 2 5 green
COUNT
ADD 2 5 red
COUNT
PRESENT 7 17 yellow
ADD 3 3 brown
COUNT}{1 1 1 1
YES YES YES YES
NO YES NO YES
NO NO YES YES
2 2 2 1
3 2 2 1
NO NO NO NO
4 3 3 2}
\end{example}

\Note

Перша відповідь {\tt 1 1 1 1} правильна, бо для будь-якого трактування правильно, що в той' момент множина містить рівно один елемент-прямокутник {\tt 5 2 red}.

Друга відповідь {\tt YES YES YES YES} правильна, бо коли питають про наявність прямокутника, з такими самими розмірами {\tt 5 2} в такому ж порядку, причому такого самого кольору {\tt red}, то за будь-яким з трактувань він вже наявний.

Третя відповідь {\tt NO YES NO YES} правильна, бо коли розміри {\tt 5 2} такі самі в такому ж порядку, але колір інший (case-sensitive передбачає, що {\tt ReD} не~дорівнює {\tt red}), то лише за трактуваннями 2 та 4, які ігнорують колір, можна вважати, що такий прямокутник вже~є.

Четверта відповідь {\tt NO NO YES YES} правильна, бо коли розміри {\tt 2 5} є тими самим числами, але в іншому порядку, то лише за трактуваннями, які дозволяють повертати прямокутник, тобто трактуванням 3 (враховуючи однаковість кольору {\tt red}) та 4, можна вважати, що такий прямокутник вже~є.

П'ята відповідь 2 2 2 1 правильна, бо за трактуванням 4 розміри 2 5 рівноцінні розмірам 5 2, а колір не перевіряється, тому додавання ще однієї копії не змінює множину у трактуванні 4; за будь-яким іншим трактуванням або одна, або обидві з причин «в уже наявного прямокутника інший колір, тому цей новий» та/або «в уже наявного прямокутника розміри в іншому порядку, тому цей новий» роблять необхідним додавання нового елемента-прямокуника.

Шоста відповідь 3 2 2 1 правильна, бо за трактуванням 1 прямокутник 2 5 red новий і його треба додати, тоді як за всіма іншими трактуваннями такий прямокутник вже був (як 2 5 green або 5 2 red).

Сьома відповідь NO NO NO NO правильна, бо нічого досить схожого на прямокутник 7 17 yellow раніше не згадувалося, навіть якщо дозволяти переставляння сторін місцями та/або ігнорування кольору.

Восьма відповідь 4 3 3 2 правильна, бо прямокутник 3 3 brown (який є також квадратом, але це неважливо) є новим за всіма трактуваннями і його треба додати.

Абсолютно повне виконання завдання передбачає, що воно виконане і через бібліотечну колекцію HashSet (з написанням власних GetHashCode та Equals), і через бібліотечну колекцію SortedSet (з написанням власного компаратора). Само собою, можна виконати частину, на відповідну частину балів.

Рекомендується описати чотири різні власні класи, які відповідають прямокутникам згідно кожного з чотирьох трактувань. Чи організовувати їх так, щоб якісь з них були нащадками інших, чи робити, щоб вони всі були нащадками одного абстрактного класу, чи ще якось; чи користуватися поліморфізмом — все це повністю на вибір студента. ООП взагалі не є предметом розгляду курсу «Алгоритми та структури даних». Якщо вдасться зробити завдання, використавши лише лямбда-функції без написання чотирьох різних власних класів — теж можна. Але навряд чи вийде зробити це завдання, взагалі не використавши ні власні класи, ні лямбда-функції.) 

Перша відповідь {\tt 1} правильна, бо на той момент множина містить рівно один елемент-прямокутник {\tt 5 2 red}.

Друга відповідь {\tt YES} правильна, бо коли питають про наявність прямокутника, з такими самими розмірами {\tt 5 2} в такому ж порядку, причому такого самого кольору {\tt red}, то він вже наявний.

Третя відповідь {\tt NO} правильна, бо хоч розміри {\tt 5 2} такі самі в такому ж порядку, але колір інший (case-sensitive передбачає, що {\tt ReD} не дорівнює {\tt red}).

Четверта відповідь {\tt YES} правильна, бо розміри {\tt 2 5} і є тими самим числами (хоч і в іншому порядку), що й {\tt 5 2}, і колір правильний.

П'ята відповідь {\tt 2} правильна, бо перед тим був запит {\tt ADD} з елементом-прямокутником, що відрізняється від вже наявного кольором, тому тоді його додали, від чого кількість елементів множини змінилася.

Шоста відповідь {\tt 2} правильна, бо перед тим був запит {\tt ADD} з елементом-прямокутником {\tt 2 5 red}, що (незважаючи на переставляння місцями чисел) вважається рівним одному з уже наявних у множині прямокутників {\tt 5 2 red}, тому множина не змінилася й продовжує містити рівно 2 елементи-прямокутники.

Сьома відповідь {\tt 2} правильна, бо перед тим був запит {\tt ADD} з елементом-прямокутником, за всіма ознаками рівним вже наявному (найпершому) елементу-прямокутнику, тому множина не змінилася й продовжує містити рівно 2 елементи-прямокутники.

Абсолютно повне виконання завдання передбачає, що воно виконане і через бібліотечну колекцію {\tt HashSet} (з написанням власних {\tt GetHashCode} та {\tt Equals}), і через бібліотечну колекцію {\tt SortedSet} (з написанням власного компаратора). В разі виконання іншою мовою програмування, з'ясуйте відповідні деталі самостійно, але це повинен бути бібліотечний спосіб через хеш-таблиці та бібліотечний спосіб через дерева. Само собою, можна виконати частину, на відповідну частину балів. 
