 Напишіть програму, яка реалізовуватиме пошук у ширину в орієнтованому незваженому не мульти графі без петель. Вершини графа є цілими невід’ємними числами в діапазоні від 0 до $N–1$ (де~$N$ --- кількість вершин у графі). Граф обов’язково подавати списками суміжності. (Наприклад, як {\tt List<int>[]}, але чіткої вимоги щодо конкретного типу не~ставиться. А~вимога просто використати списки суміжності ставиться; власне, інакше взагалі неясно як отримати правильний результат із прийнятними витратами часу й пам'яті.)

\InputFile
В першому рядку задано число $NUM$ --- кількість різних пошуків у ширину, які треба виконати (на різних графах). Далі йдуть $NUM$ блоків, кожен з яких має таку структуру.

Перший рядок блоку містить два числа $N$ та $M$, розділені пробілом --- кількість вершин та кількість ребер графа. Далі йдуть $M$ рядків, кожен з яких містить по два числа (розділені пробілом, від 0 до $N–1$ кожне) --- початок та кінець відповідної дуги. Далі, в останньому рядку блоку, записане єдине число від 0 до $N–1$ --- вершина, починаючи з якої треба запустити пошук.

Точні обмеження на $N$ та $M$ свідомо не~повідомляються; приблизні --- $N$ не~перевищує кількадесят тисяч, $M$ не~перевищує кількасот тисяч.


\OutputFile
Виведіть $NUM$ рядків, у кожному з яких по $N_i$ чисел, розділених пробілами --- відстані від указаної стартової вершини орграфа до його 0-ої, 1-ої, 2-ої і~т.~д. вершин. Якщо деяка вершина недосяжна з указаної початкової, замість відстані виводьте число 987654321.

У попередньому абзаці сказано про $N_i$ (а~не~$N$), щоб підкреслити: кількості вершин різних графів можуть бути різними, тому різні рядки результатів можуть бути різної довжини.



\Examples

\begin{example}
\exmp{2
3 2
0 1
1 2
1
4 4
0 1
0 3
1 2
2 3
0}{987654321 0 1 
0 1 2 1}
\end{example}


\Note

\begin{enumerate}
\item 
Переконайтеся, що Ваша програма правильно враховує, що за один запуск слід обробити кілька різних графів (зокрема, працює, коли «менший» граф йде після «більшого»).

\item
У наведеному прикладі треба зробити 2 пошуки вшир:
один --- на орграфі з 3 вершин та 2 ребер
$0\to 1$ і
$1\to 2$,
починаючи з вершини~1;
інший --- на орграфі з 4 вершин та 4 ребер
$0\to 1$,
$0\to 3$,
$1\to 2$ і
$2\to 3$,
починаючи з вершини~0.



\end{enumerate}

