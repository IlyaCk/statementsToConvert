{\it Відрізняється від попередньої версії {\bf виключно} тим, що результати потрібно вивести більш детально. Якщо граф незв'язний, то, як і в попередній задачі, відповідь має бути просто рядком з єдиним словосполученням ``{\tt NON-CONNECTED}''. А для зв'язного графу відповідь повинна описувати саме дерево.} 

Напишіть програму, яка знаходитиме ОДМВ (остовне дерево мінімальної ваги) неорієнтованого зваженого графу з додатними довжинами ребер. Програма повинна працювати швидко для великих розріджених графів (максимальна кількість вершин --- десятки тисяч, ребер --- сотні тисяч).

\InputFile
Перший рядок містить два числа {\it N} та {\it M}, розділені пробілом --- кількість вершин та кількість ребер графа. Далі йдуть {\it M} рядків, кожен з яких містить по три цілі числа, розділені пробілами. Перші два з них різні, у межах від 0 до {\it N}–1 кожне, і позначають кінці відповідного ребр{\it а}, третє — у межах від 1 до $10^9$ і позначає довжину цього ребр{\it а}. Гарантовано, що всі р{\it е}бра мають різні довжини.

\OutputFile
Якщо граф незв'язний, виведіть єдиний рядок з єдиним словосполученням ``{\tt NON-CONNECTED}'' (великими латинськими буквами, без лапок).

Якщо граф зв'язний, то перший рядок має містити єдине ціле число --- сумарну довжину всіх ребер ОДМВ; далі мають бути {\it N} рядків, кожен з яких має містити список суміжності відповідної вершини. Формат списку суміжності: пишеться, яка вершина, пробіл, двокрапка, далі усі вершини, куди йдуть ребра з даної, у вигляді: пробіл, номер вершини, куди йде ребро, відкривна дужка ``('', довжина ребра, закривна дужка ``)''. Рядки мають бути виведені згідно порядку номерів вершин (тих що зліва від `` : ''). Переліки всередині кожного рядка теж повинні бути впорядковані за номерами вершин. Між кожними сусідніми різними ребрами одного переліку повинні бути кома і пробіл. Закінчувати рядок комою чи комою з пробілом (ставити їх після опису останнього в рядку ребра) заборонено.

\Examples

\begin{example}
\exmp{5 7
1 2 5
1 3 2
2 3 4
2 4 3
3 4 6
0 3 20
0 4 10}{19
0 : 4(10)
1 : 3(2)
2 : 3(4), 4(3)
3 : 1(2), 2(4)
4 : 0(10), 2(3)}
\exmp{100 1
17 42 111}{NON-CONNECTED}    
\end{example}

\Note

Див. примітки до попередньої задачі «Алгоритм Краскала» та до задачі «Алгоритм Дейкстри за M log N».