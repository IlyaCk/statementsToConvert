Напишіть програму, яка виконуватиме послідовність запитів виду {\tt ADD} {\it num}, {\tt PRESENT} {\it num} та {\tt COUNT} (без параметра). Програму обов'язково слід писати за допомогою бібліотечного типу (колекції) set (її реалізації в~конкретних мовах програмування можуть називатися {\tt HashSet}, {\tt TreeSet}, {\tt SortedSet},~\dots).

Виконання кожного запиту виду {\tt ADD} {\it num} має додавати елемент {\it num} у~множину (якщо такий елемент вже~є, додавання ще~однієї копії не~змінює множину), на~екран при цьому нічого не~виводиться.

При виконанні кожного запиту виду {\tt PRESENT} {\it num} має видаватися повідомлення {\tt YES} або {\tt NO} (великими літерами, в~окремому рядку), відповідно до того, чи~є такий елемент у множині; значення множини при цьому не~змінюється.

При виконанні кожного запиту виду {\tt COUNT} має видаватися на екран в~окремому рядку кількість різних елементів у~множині; значення множини при цьому не~змінюється.


\InputFile

У першому рядку задано кількість запитів $N$ ($1\leqslant N\leqslant 10^5$), далі йдуть $N$ рядків, кожен з яких містить по одному запиту згідно з описаним форматом.

Значення чисел є цілими і не~перевищують за модулем 100~000~000  (інакше кажучи, належать проміжку $[-10^8; +10^8]$).


\OutputFile

Виводьте окремими рядками результати запитів {\tt PRESENT} та {\tt COUNT}; на запити {\tt ADD} нічого не~виводьте.


\Examples

\begin{example}
\exmp{7
ADD 5
ADD 7
COUNT
PRESENT 3
PRESENT 5
ADD 3
COUNT
}{2
NO
YES
3
}
\end{example}