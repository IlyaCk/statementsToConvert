Є $n$ дерев, розміщених уздовж дороги в точках з координатами $x_1$, $x_2$, \dots, $x_n$. Кожне дерево має свою висоту $h_i$. Дерево можна зрубати і повалити або ліворуч, або праворуч. Тоді воно буде займати або відрізок $[x_i - h_i, x_i]$, або $[x_i; x_i + h_i]$ відповідно. Поки дерево не~зрубане, воно займає точку з координою~$x_i$. Дерево можна повалити, якщо на відрізку, який воно має займати після звалювання, немає жодної зайнятої точки.

Яку найбільшу кількість дерев можна повалити?

\InputFile
У першому рядку задане ціле число $n$ ($1\leqslant n\leqslant 10^5$) --- кількість дерев.
У наступних $n$ рядках задані пари цілих чисел $x_i, h_i$ ($1\leqslant n\leqslant 10^9$) — координата і
висота $i$-го дерева.
Пари задані у порядку зростання $x_i$. Жодні два дерева не знаходяться в точці з однаковою
координатою.

\OutputFile
Потрібно вивести одне число --- максимальну кількість дерев, які можна зрубати й повалити згідно зазначених правил.

\Examples
\begin{example}
\exmp{4
10 4
15 1
19 3
20 1}{4}
\exmp{5
1 2
2 1
5 10
10 9
19 1}{3}
\end{example}