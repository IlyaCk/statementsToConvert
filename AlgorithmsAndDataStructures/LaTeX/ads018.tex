Напишіть програму, яка знаходить всі розв'язки задачі про $n$ ферзів. Прочитати постановку цієї задачі можна, наприклад, \href{https://uk.wikipedia.org/wiki/%D0%97%D0%B0%D0%B4%D0%B0%D1%87%D0%B0_%D0%BF%D1%80%D0%BE_%D0%B2%D1%96%D1%81%D1%96%D0%BC_%D1%84%D0%B5%D1%80%D0%B7%D1%96%D0%B2}{у вікіпедії}.

\InputFile
В єдиному рядку задане єдине число $n$ ($1\leqslant n\leqslant 12$) --- розмір дошки.

\OutputFile
Виведіть спочатку всі знайдені розміщення, наприкінці їх кількість (символ-у-символ дотримуючись описаного далі формату).

Кожне окреме розміщення, де $n$ ферзів не~б'ють один одного, слід виводити окремим рядком такого вигляду: спочатку позначку вертикалі ``{\tt a}'' (саму букву, без лапок), потім число, що означає, в якій горизонталі розміщено ферзя у цій вертикалі~``{\tt a}'', спочатку позначку ``{\tt b}'' та номер горизонталі ферзя у вертикалі ``{\tt b}'', і так далі; все це в один рядок без пробілів чи будь-яких інших роздільників. Якщо $10\leqslant n\leqslant 12$, деякі з номерів горизонталей виявляться одноцифрові, деякі двоцифрові; так і виводити; зокрема, для $n=10$ першим з рядків відповіді повинно бути {\tt a1b3c6d8e10f5g9h2i4j7}. Всі~ці~рядки, що описують можливі розміщення ферзів, обов'язково повинні бути впорядковані с{\it а}ме так, як виходить при класичному порядку перебору, а с{\it а}ме: в першу чергу за номером горизонталі у вертикалі ``{\tt a}'', при однаковості цих номерів --- за номером горизонталі у вертикалі ``{\tt b}'', і так далі. (При $n\leqslant 9$ це збігається також зі стандартним словниковим порядком рядків, але при $n\geqslant 10$ це не~так; який рядок повинен бути першим при $n=10$, вже наведено.)

Наприкінці повинен бути ще один рядок, вигляду ``{\tt (totally ...)}'' (без лапок), де замість трикрапки слід записати кількість знайдених розміщень. Рядок повинен містити лише один пробіл (між словом totally та знайденою кількістю) і завершуватися символом переведення рядка.

\Examples
\begin{example}
\exmp{1}{a1
(totally 1)}
\exmp{2}{(totally 0)}
\exmp{4}{a2b4c1d3
a3b1c4d2
(totally 2)}
\exmp{6}{a2b4c6d1e3f5
a3b6c2d5e1f4
a4b1c5d2e6f3
a5b3c1d6e4f2
(totally 4)}
\end{example}

\Note
У цьому завданнi треба строго дотримуватися формату виведення, бо зараховуватися будуть тільки точні збіги байт-у-байт. 