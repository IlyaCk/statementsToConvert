Є прямокутна таблиця розміром $N$ рядків на $M$ стовпчиків. У кожній клітинці записане невід'ємне ціле число. 
По ній потрібно пройти згори донизу, починаючи з будь-якої клітинки верхнього рядка, 
далі переходячи щоразу в одну з <<нижньо-сусідніх>> і закінчити маршрут у якій-небудь клітинці нижнього рядка. 
<<Нижньо-сусідня>> означає, що з клітинки $(i,j)$ можна
перейти в ${(i+1, j-1)}$,
або в ${(i+1,j)}$,
або в ${(i+1,j+1)}$, 
але не~виходячи за межі таблиці (у крайньому лівому стовпчику перший з наведених варіантів стає неможливим, а у крайньому правому --- останній).

Напишіть програму, яка знаходитиме максимально можливу \emph{щасливу} суму значень пройдених клітинок 
серед усіх допустимих шляхів.
Як широко відомо у вузьких колах, щасливими є ті й тільки ті ч{\it и}сла, десятковий запис яких містить лише цифри 4 та/або~7 
(можна обидві, можна лише якусь одну; але ніяких інших цифр використовувати не~можна).
Зверніть увагу, що щасливою повинна бути с{\it а}ме сума, а обмежень щодо окремих доданків нема.

\InputFile
У першому рядку записані $N$ та $M$ --- кількість рядків і кількість стовпчиків 
($1\leqslant N,\,M\leqslant 12$); далі у кожному з наступних $N$ рядків 
записано рівно по $M$ розділених пробілами невід'ємних цілих чисел, 
кожне не~більш ніж з 12 десяткових цифр --- значення клітинок.

\OutputFile
Вивести або єдине ціле число (знайдену максимальну серед щасливих сум за маршрутами зазначеного вигляду), 
або рядок ``\texttt{impossible}'' (без лапок, маленькими латинськими буквами). 
Рядок ``\texttt{impossible}'' слід виводити тільки у разі, коли жоден з допуcтимих маршрутів не~має щасливої суми.

\begin{example}
\exmp{3 4
3 0 10 10
5 0 7 4
4 10 5 4}{7}
\end{example}


\Note
Взагалі-то максимально можливою сумою є $27=10+7+10$, але число 27 не~є щасливим.
Тому відповіддю буде максимальна серед щасливих сума $7=3+0+4$, 
яка досягається уздовж маршруту \texttt{a[1][1]}$\to$\texttt{a[2][2]}$\to$\texttt{a[3][1]}.

Наскільки відомо автору задачі, автором <<широко відомого у вузьких колах>> такого трактування <<щасливого числа>>
є Василь Білецький, випускник Львівського національного університету імені Івана Франка,
котрий був капітаном першої з українських команд, що вибороли золоту медаль на фіналі першості світу ACM~ICPC,
і тривалий час входив у десятку найсильніших спортивних програмістів світу за рейтингом TopCoder.
