Напишіть програму, яка знаходитиме перелік прямокутних паралелепіпедів, що мають об'єм~$V$ та площу поверхні~$S$. Враховувати лише паралелепіпеди, в яких всі три розміри виражаються натуральними числами.

\InputFile
Єдиний рядок містить розділені одинарним пробілом два натуральні числа $V$~$S$, обидва у межах від 1 до $10^6$ (мільйон).

\OutputFile
Кожен рядок повинен містити розділені пробілами цілочисельні розміри $a$~$b$~$c$ чергового паралелепіпеда. Ці~трійки обов'язково виводити у порядку зростання першого розміру~$a$, при рівних~$a$ --- у порядку зростання др{\it у}гого розміру~$b$ (а~різних відповідей, в~яких рівні і~$a$, і~$b$, не~буває). 

У~випадку, якщо жодного паралелепіпеда з потрібними $V$, $S$ не~існує, виводьте єдиний рядок \texttt{0~0~0}.

\Note
Якщо прямокутний паралелепіпед має розміри $a$~$b$~$c$, то його об'єм дорівнює ${a\,{\cdot}\,b\,{\cdot}\,c}$, а площа поверхні ${2\,{\cdot}\bigl(a\,{\cdot}\,b}+{b\,{\cdot}\,c}+{c\,{\cdot}\,a}\bigr)$, 
бо є 
дві грані (наприклад, передня й задня) площі ${a\,{\cdot}\,b}$, 
дві (наприклад, ліва та права) площі ${b\,{\cdot}\,c}$ 
та
дві (наприклад, верхня й нижня) площі ${c\,{\cdot}\,a}$.

Гарантовано, що у кожному з тестів, що використовуються для оцінювання цієї задачі, кількість трійок-відповідей строго менша~100.

\Examples
\begin{example}
\exmp{7 7}{0 0 0}%
\exmp{6 22}{1 2 3
1 3 2
2 1 3
2 3 1
3 1 2
3 2 1}%
\exmp{1 6}{1 1 1}%
\exmp{1600 1120}{4 20 20
5 8 40
5 40 8
8 5 40
8 40 5
20 4 20
20 20 4
40 5 8
40 8 5}%
\end{example}