{\it
Зверніть увагу, що тут Вас просять відновити детальну відповідь (сукупність банкнот), ще й 
{\bf в умовах відносно невеликого обсягу дозволеної пам'яті}.
}

У деякій державі в обігу перебувають банкноти певних номіналів. 
Національний банк хоче, щоб банкомат видавав будь-яку запитану суму 
за допомогою мінімального числа банкнот, вважаючи, 
що запас банкнот кожного номіналу необмежений. 
Допоможіть Національному банку вирішити цю задачу.

\InputFile
Перший рядок містить натуральне число~$N$, 
що не~перевищує~100 --- кількість номіналів банкнот у~обігу. 
Другий рядок вхідних даних містить $N$ різних натуральних чисел 
$x_1$, $x_2$,\dots, $x_N$, 
що не~перевищують $10^6$ --- номінали банкнот. 
Третій рядок містить натуральне число~$S$, 
що не~перевищує~$10^6$ --- суму, яку необхідно видати.

\OutputFile
Програма повинна знайти подання числа $S$ у вигляді суми 
доданків з~множини~$x_i$, що містить мінімальне число доданків, 
і вивести це подання у вигляді послідовності чисел, розділених пробілами.

Якщо таких подань існує декілька, то програма повинна вивести 
будь-яке (одне) з~них.
Якщо такого подання не~існує, то програма повинна вивести рядок 
``{\tt No solution}'' (без лапок, перша літера велика, решта маленькі).


\Examples
\begin{example}
\exmp{7
1 2 5 10 20 50 100
72}{50 20 2}%
\exmp{2
20 50
60}{20 20 20}%
\end{example}
