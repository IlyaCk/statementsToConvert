Шеф завжди приділяє всім відвідувачам рівні проміжки часу (наприклад, кожному по п'ять хвилин); щоб попасти на прийом, слід напередодні записатися у секретарки.

При реєстрації відвідувач вказує єдиний інтервал часу, що задається парою $[A_i; B_i]$ (початковий та кінцевий моменти, коли він згоден \emph{заходити} на прийом). $A_i$ і $B_i$ --- невід’ємні цілі числа, що означають кількість інтервалів прийому, що пройшли з початку робочого дня Шефа (отже, момент початку робочого часу Шефа має номер~0). Допоможіть секретарці обробляти зібрані записи і складати графік прийому.

\InputFile
Перший рядок містить кількість відвідувачів ($2\leqslant N\leqslant 10^4$), далі йдуть ще $N$ рядків, у кожному з яких по два числа $А_i$ і $B_i$, $0\leqslant A_i\leqslant B_i\leqslant 2N$.

\OutputFile
Програма повинна вивести на екран 1 (якщо встановити графік
прийому можливо), або 0 (якщо неможливо). 

\Examples
\begin{example}
\exmp{3
1 2
0 1
2 2}{1}
\exmp{3
1 2
1 2
1 2}{0}
\exmp{3
1 2
1 2
2 4}{1}
\end{example}

\Note
У наступній задачі приклад описує для кожного з цих самих прикладів вхідних даних також один з можливих порядків, у якому слід заходити відвідувачам.