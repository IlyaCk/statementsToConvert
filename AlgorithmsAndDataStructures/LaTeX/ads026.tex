Щоб зобразити за допомогою паркету Супер-Креативний Візерунок, треба:
\begin{itemize}
\item
$N_1$ дощечок розмірами $1{\times}1$,
\item
$N_2$ дощечок розмірами $2{\times}1$,
\item
$N_3$ дощечок розмірами $3{\times}1$,
\item
$N_4$ дощечок розмірами $4{\times}1$,
\item
$N_5$ дощечок розмірами $5{\times}1$.
\end{itemize}
Купити можна лише дощечки розмірами $5{\times}1$. Дощечки можна різати, але не~можна склеювати. Наприклад, коли потрібні п’ять дощечок $2{\times}1$, їх не~можна зробити з двох дощечок $5{\times}1$, але можна з трьох. Для цього дві з них розріжемо на три частини $2{\times}1$, $2{\times}1$ та $1{\times}1$ кожну, а третю --- на дві частини $2{\times}1$ та $3{\times}1$. Отримаємо потрібні п’ять дощечок $2{\times}1$, а дві дощечки $1{\times}1$ та одна $3{\times}1$ підуть у відходи.


Напишіть програму, яка, прочитавши кількості дощечок $N_1$, $N_2$, $N_3$, $N_4$ та $N_5$, знайде, яку мінімальну кількість дощечок $5{\times}1$ необхідно купити.

\InputFile	Вхідні дані слід прочитати зі стандартного входу (клавіатури). Це будуть п’ять чисел $N_1$, $N_2$, $N_3$, $N_4$ та $N_5$ (саме в такому порядку), розділені пропусками (пробілами).

\OutputFile	Єдине число (скільки дощечок треба купити) виведіть на стандартний вихід (екран).

\Examples
\begin{example}%
\exmp{0 5 0 0 0}{3}%
\exmp{1 1 1 1 1}{3}%
\end{example}
