{\it Ця задача відрізняється від наступної \underline{лише} тим, що тут потрібно знаходити й виводити лише мінімальну сумарну висоту, а в наступній ще й розподіл блоків по рядкам.}

Абзац містить блоки різної висоти (наприклад, звичайні слова й математичні системи). Цей абзац навряд чи поміщається весь в один рядок, тому його, найімовірніше, потрібно розбити на рядки. Висота кожного рядка визначається за найвищим з блоків цього рядка. Висота абзацу дорівнює сумі висот всіх рядків (ніби друкуємо не на окремих сторінках, а на довгому свитку). Довжина кожного рядка визначається як сумарна ширина блоків, включених до цього рядка (враховувати пробіли між блоками не~потрібно). Можливість розбиття блоку для перенесення з рядка на рядок не~розглядається. Змінювати порядок блоків не~можна (і це природньо: ну куди це годиться, щоб у друкарні взяли й попереставляли слова в тексті, який вони мали лише надрукувати, а~не~відредагувати).

Потрібно знайти таке розбиття абзацу на рядки, щоб висота абзацу була мінімальною. Ширина і висота кожного блоку $(w_i, h_i)$ та максимально допустима довжина рядка $TW$ (скорочення від TextWidth) задаються у вхідних даних.

\InputFile
У першому рядку записано два числа: $TW$ (максимально допустима довжина рядка) і $N$ (кількість блоків в абзаці, де $5\leqslant N\leqslant 5000$). 
У наступних $N$ рядках записано по два числа $w_i$ та $h_i$ — ширина і висота чергового блоку.

Всі розміри (окремих блоків та $TW$) — натуральні числа, не~більші~$10^6$. Гарантовано, що для кожного окремо взятого блоку $w_i\leqslant TW$.

\OutputFile
Виведіть єдине число — мінімальну висоту абзацу.

\Examples
\begin{example}    
\exmp{7 6
3 1
2 1
2 3
1 1
3 3
3 1}{5}
\end{example}

\Note
Як отримати відповідь 5 при цих вхідних даних, див. у наступній задачі. 