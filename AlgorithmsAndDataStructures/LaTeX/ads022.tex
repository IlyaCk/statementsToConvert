Як відомо, {\it простим} називають таке натуральне число, яке має рівно два дільники --- одиницю й самого себе. Перші десять простих чисел --- 2, 3, 5, 7, 11, 13, 17, 19, 23, 29. 

Напишіть програму, яка знайде усі підряд, у порядку зростання, прості ч{\it и}сла у~проміжку від $A$ до~$B$ (обидві межі включно).

\InputFile
У єдиному рядку через пробіл задані два натуральні числ{\it а} $A$ та~$B$, які є межами проміжку.
Обмеження:
\begin{itemize}
\item
$1\leqslant A$;
\item
$B\leqslant 10^{13}$;
\item
$A\leqslant B\leqslant A+100$.
\end{itemize}

\OutputFile
Виведіть усі прості ч{\it и}сла проміжку, кожне у окремому рядку. Якщо буде введений проміжок, що не~містить жодного простого числ{\it а}, слід нічого не виводити (навіть символа завершення рядка).


\Examples
\begin{example}
\exmp{2 5}{2
3
5}%
\exmp{4 4}{}%
\end{example}

\Note