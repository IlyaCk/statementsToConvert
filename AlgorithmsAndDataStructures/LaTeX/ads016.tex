 Нагадаємо, що розміщення з $n$ по $k$ --- виборки по $k$ елементів з $n$ можливих, причому порядок елементів у вибірці важливий («1 2» відрізняється від «2 1»).

Наприклад, повний перелік всіх можливих розміщень з 3 по 2: (1,2), (1,3), (2,1), (2,3), (3,1), (3,2).

Напишіть програму, яка за заданими $n$ і $k$ генеруватиме всі можливі розміщення з чисел 1, 2, ..., $n$ по $k$. Розміщення повинні виводитися в лексикографічному (словниковому, але за числами) порядку.

\InputFile
В єдиному рядку через пропуск задано два числа --- спочатку $n$, потім $k$.

Виконуються умови: $k\leqslant n\leqslant 20$; $1\leqslant k\leqslant 7$; $1\leqslant A_n^k < 10^4$ (де $A_n^k$ --- кількість розміщень, які слід згенерувати).

\OutputFile
Виведіть знайдені розміщення, кожне в окремому рядку, в лексикографічному (за числами) порядку. Усередині рядка числа повинні відділятися один від одного одиничними пробілами.

\Examples
\begin{example}
\exmp{3 2}{1 2
1 3
2 1
2 3
3 1
3 2}
\end{example}

\Note

Під «лексикографічним (за числами) порядком» мається на увазі: спочатку треба виводити всі перестановки, де на першому (крайньому зліва) місці 1, потім усі, де на першому місці 2, тощо, насамкінець усі, де на першому місці $n$. У~свою чергу, всі перестановки, де перші числа однакові між собою, мають бути відсортовані за другими числами; всі перестановки, де однакові як перші так і другі числа, мають бути відсортовані за третіми; тощо.

Примітка «за числами» означає, що якби спочатку сформували всі перестановки як рядки, і відсортували рядки, то порядок виявився б таким самим при $n\leqslant 9$, але при $n\geqslant 10$ з'являється відмінність. Наприклад, при $n=10$ у цій задачі потрібно виводити перестановки, що починаються з 10, наприкінці, а якби поформували рядки й відсортували їх як рядки, то перестановки, що починаються з 10, потрапили б між перестановками, що починаються з 1, і перестановками, що починаються з 2 (що в цій задачі не~буде зараховуватися).