{\it Нема жорсткої вимоги робити цю задачу саме з використанням бінарного дерева пошуку, що підтримує операцію «знайти за номером», але це --- один зі способів. Гарантовано, що вхідні дані згенеровані суто випадково, тому небалансоване дерево не~має істотних переваг над балансованим (процес балансування можна й не~писати). Можна використати й деяку іншу структуру даних (взагалі не~дерево, а~зовсім іншу). Однак, просто масив, начебто, повинен виявлятися не~досить ефективним.}

Напишіть програму, яка реалізовуватиме деяку колекцію, яка підтримує дії «вставити» та «знайти» (за~значенням). Програма повинна виконувати  послідовність запитів вигляду ``{\tt ADD} {\it v}'', ``{\tt SEARCH} {\it v}'', ``{\tt INDEX} {\it i}'', ``{\tt VAL2INDEX} {\it v}'' та ``{\tt COUNT}'', де {\it v}, {\it i} --- цілі числа.

Для кожного запиту ``{\tt ADD} {\it v}'' слід виконати такі дії: якщо вказаного числа {\it v} ще нема в колекції, то вставити (додати) його, якщо вже є --- залишити як було (не~вставляти додаткову копію). При виконанні запитів {\tt ADD} ніякого результату не~виводиться, лише змінюється внутрішній стан колекції.

Для кожного запиту ``{\tt SEARCH} {\it v}'' слід вивести на екран слово ``{\tt YES}'' (якщо значення {\it v} знайдене) або слово ``{\tt NO}'' (якщо не~знайдене); при виконанні запитів {\tt SEARCH} колекція не~змінюється.

Для кожного запиту ``{\tt INDEX} {\it i}'', слід вивести на екран значення, яке мало би вказаний порядковий номер {\it i}, якби колекція була відсортованим масивом з нумерацією з нуля. Інакше кажучи, таке значення з колекції, що рівно~{\it i} інших значень колекції строго менші за знайдене. У~випадку, якщо такого елемента не~існує (значення {\it i} занадто велике або занадто мале), слід вивести слово ``{\tt NO}''. В~будь-якому разі, при виконанні запитів {\tt INDEX} колекція не~змінюється.

Для кожного запиту ``{\tt VAL2INDEX} {\it n}'', якщо значення {\it n} є у колекції, то слід вивести, під яким індексом воно було б у колекції, якби колекція була відсортованим масивом з нумерацією з нуля. Інакше кажучи, якщо значення {\it n} є у колекції, то слід вивести, скільки інших елементів колекції строго менші вказаного~{\it n}. У~випадку, якщо колекція не~містить такого значення, слід вивести слово ``{\tt NO}''. В~будь-якому разі, при виконанні запитів {\tt VAL2INDEX} колекція не~змінюється.

Для кожного запиту ``{\tt COUNT}'', слід вивести поточну кількість елементів у колекції; колекція при цьому не~змінюється.

\InputFile
В кожному рядку вхідних даних записаний один із запитів ``{\tt ADD} {\it v}'', або ``{\tt SEARCH} {\it v}'', або ``{\tt INDEX} {\it i}'', або ``{\tt VAL2INDEX} {\it v}'', або ``{\tt COUNT}'' (без лапок; слова записані великими латинськими буквами; для всіх запитів, що передають число (всіх крім {\tt COUNT}), число відділене від слова одинарним пробілом, і для запитів ``{\tt COUNT}'' перебуває в межах від 0 до $10^6$, а для решти запитів у межах від $-10^9$ до $+10^9$).
Загальна кількість запитів не~перевищує $5\cdot10^5$ (пів мільйона). 
{\it Гарантовано, що вхідні дані згенеровані суто випадково, тому небалансоване дерево не~має істотних переваг над балансованим (процес балансування можна й не~писати).}

\OutputFile
Для кожного запиту слід виводити відповідь на нього, в окремому рядку; якщо відповідь є словом, то великими латинськими буквами; в будь-якому разі, без лапок.

\Examples

\begin{example}
\exmp{20
ADD 50
ADD -6
COUNT
ADD -2
ADD 57
INDEX 1
ADD -1
VAL2INDEX 50
ADD -2
ADD 29
ADD 19
INDEX 6
ADD 27
ADD 22
PRESENT -3
ADD -7
VAL2INDEX -7
VAL2INDEX 73
VAL2INDEX 22
PRESENT -7}{2
-2
3
57
NO
0
NO
5
YES}
\end{example}

\Note
Детальніше пояснимо приклад:

{\tt 2} --- у відповідь на COUNT, коли колекція містить –6, 50

{\tt -2} --- у відповідь на INDEX 1, коли колекція містить –6, –2, 50, 57.

{\tt 3} --- у відповідь на VAL2INDEX 50, коли колекція містить –6, –2, –1, 50, 57.

{\tt 57} --- у відповідь на INDEX 6, коли колекція містить –6, –2, –1, 19, 29, 50, 57 (причому, повторне включення –2 не~змінило вміст колекції).

{\tt NO} --- у відповідь на PRESENT –3, коли колекція містить –6, –2, –1, 19, 22, 27, 29, 50, 57.

{\tt 0} --- у відповідь на VAL2INDEX -7, коли колекція містить –7, –6, –2, –1, 19, 22, 27, 29, 50, 57. 

{\tt 0} --- у відповідь на VAL2INDEX 73, коли колекція містить –7, –6, –2, –1, 19, 22, 27, 29, 50, 57. 

{\tt 5} --- у відповідь на VAL2INDEX 22, коли колекція містить –7, –6, –2, –1, 19, 22, 27, 29, 50, 57. 

{\tt YES} --- у відповідь на PRESENT -7, коли колекція містить –7, –6, –2, –1, 19, 22, 27, 29, 50, 57.
