У банка є клієнти. Кожен клієнт має рівно один рахунок.

Напишіть програму, яка виконуватиме послідовність запитів таких двох видів:

\begin{enumerate}
\item
починається з числа 1, потім через пробіл іде ім’я клієнта (слово з латинських букв), далі через пробіл іде сума грошей, яка додається до рахунку поточного клієнта (ціле число, не~перевищує за модулем $10\,000$).
\item
починається з числа 2, через пробіл іде ім’я клієнта. На кожен такий запит програма повинна відповісти? яка сума в даний момент є на рахунку заданого клієнта. Якщо таке ім’я клієнта поки що ні разу не згадувалося в запитах виду~1, виводьте замість числа слово \texttt{ERROR}.
\end{enumerate}

На початку роботи програми у всіх клієнтів на рахунку~0. Потім суми можуть ставати як додатними, так і від’ємними.

Зверніть увагу, що у ситуації, коли клієнт зняв грошей сумарно рівно стільки ж, скільки поклав, сума на рахунку стає рівною~0; але, раз його ім’я вже зустрічалося, нульове значення \textbf{не} є підставою виводити \texttt{ERROR}.

\InputFile
Перший рядок вхідних даних --- кількість запитів $N$ (обмеження на максимальне значення $N$ --- приблизно до $10^5=100\,000$). 

Далі йдуть $N$ рядків, у кожному з яких задано один запит одного з двох вищеописаних видів.

\OutputFile
Формат виведення результатів. На кожний запит 2-го виду потрібно вивести поточне значення на рахунку заданого клієнта (або слово \texttt{ERROR}).


\Examples

\begin{example}
\exmp{7
1 asdf 3
1 zxcv 5
2 asdf
1 asdf -2
2 asdf
2 lalala
2 zxcv}{3
1
ERROR
5}
\end{example}

\Note
Для абсолютно повного виконання задачі, її рекомендується зробити двома різними способами. 

Один --- з використанням бібліотечної словникової структури даних. Відповідна готова колекція чи контейнер може називатися (залежно від мови програмування) \texttt{Dictionary}, \texttt{dict}, \texttt{map}, \texttt{HashMap}, \texttt{SortedMap}, тощо.

Інший --- із використанням власноруч написаних hash-таблиць (не~використати біблотечні, а~написати спрощений їх аналог вручну, користуючись лише масивами та/або списками).