{\it У цій задачі трохи інше, чим зазвичай, трактування жадібності. Тут є етап «перепробуємо таку, таку, \dots{} відповіді, та виберемо з них максимальну»; однак, ідея, завдяки якій цей максимум не~доводиться шукати серед усіх $\Theta(N^2)$ варіантів (пари кожної стінки з кожною іншою), а вдається обійтися значно меншою кількістю ($\Theta(N)$), все ж може бути розцінена як жадібна.}

Герой відомого мультсеріала Коливан вирішив побудувати собі басейн. Оскільки він дуже скупий, він намагається використати будівлю, що вже існує. 

Будівля являє собою абсолютно рівний коридор одиничної ширини, в якому є $N$ перегородок. Якщо розмістити вісь $Ox$ вздовж коридору, усі перегородки будуть знаходитись точно в її цілочисельних координатах з~кроком~1, причому по ширині перегородки займають увесь коридор, а вис{\it о}ти можуть відрізнятися. Басейн, що створюється, повинен мати максимально великий можливий об’єм за умови, що з усіх існуючих перегородок потрібно залишити {\it тільки} дві та збільшувати їх висоту заборонено.

\InputFile
Програма зчитує в першому рядку ціле число $N$ ($2\leqslant N\leqslant 10^5$) --- кількість перегородок. Потім програма зчитує в другому рядку $N$ цілих чисел $a_i$ ($1\leqslant a_i\leqslant 10^9$) --- вис{\it о}ти перегородок. 

\OutputFile
Програма виводить єдине число --- максимально можливий об’єм створюваного басейна з урахуванням вказаних обмежень. 

\Examples
\begin{example}
\exmp{4
1 2 1 3}{4}
\end{example}
