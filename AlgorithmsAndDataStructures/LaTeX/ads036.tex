{\it Задача відрізняється від попередньої \emph{лише} тим, що потрібно вивести детальнішу відповідь (якщо прийняти всіх відвідувачів згідно з їх побажаннями можна, то потрібен також порядок («розклад»)).}

Шеф завжди приділяє всім відвідувачам рівні проміжки часу (наприклад, кожному по п'ять хвилин); щоб попасти на прийом, слід напередодні записатися у секретарки.

При реєстрації відвідувач вказує єдиний інтервал часу, що задається парою $[A_i; B_i]$ (початковий та кінцевий моменти, коли він згоден \emph{заходити} на прийом). $A_i$ і $B_i$ --- невід’ємні цілі числа, що означають кількість інтервалів прийому, що пройшли з початку робочого дня Шефа (отже, момент початку робочого часу Шефа має номер~0). Допоможіть секретарці обробляти зібрані записи і складати графік прийому.

\InputFile
Перший рядок містить кількість відвідувачів ($2\leqslant N\leqslant 10^4$), далі йдуть ще $N$ рядків, у кожному з яких по два числа $А_i$ і $B_i$, $0\leqslant A_i\leqslant B_i\leqslant 2N$.

\OutputFile
Програма повинна вивести на екран 1 (якщо встановити графік прийому можливо), або 0 (якщо неможливо). Якщо відповідь позитивна (1), то далі в тому ж рядку через пропуски програма має вивести послідовність чисел-номерів відвідувачів у порядку, як вони потрапляють на прийом. Якщо потрібно, щоб в якийсь момент ніхто не заходив на прийом, слід виводити $–1$.


\Examples
\begin{example}
\exmp{3
1 2
0 1
2 2}{1 2 1 3}
\exmp{3
1 2
1 2
1 2}{0}
\exmp{3
1 2
1 2
2 4}{1 -1 1 2 3}
\end{example}

\Note
Зверніть увагу, що $B_i$ є останнім моментом, коли відвідувач усе ще \emph{згоден} заходити на прийом (і в першому, і в третьому прикладах лише завдяки цьому і вдається побудувати порядок прийому).

Зверніть увагу, що справді бувають ситуації, коли розклад можна побудувати лише таким, коли в деякі моменти часу ніхто не~заходить на прийом.
