{\it Ця задача відрізняється від попередньої задачі «Платформи з обмеженням на кількість суперприйомів» \underline{лише} обмеженнями на кількість платформ та на дозволений обсяг пам'яті.}

  У багатьох старих іграх з двовимірною графікою можна зіткнутися з такою ситуацією. Який-небудь герой стрибає по платформах (або острівцям), 
  які висять у повітрі. Він повинен перебратися від одного краю екрану до іншого. При цьому при стрибку з  платформи на сусідню, 
  герой витрачає $|y_2-y_1|$ одиниць енергії, де $y_1$ і $y_2$ ---
  вис{\it о}ти, на яких розташовані ці платформи. Крім того, у героя є суперприйом, який дозволяє перескочити через платформу, 
  але на це витрачається $3\cdot|y_3-y_1|$ одиниць енергії. (Суперприйом \emph{можна} застосовувати багатократно.) 
  {\bf
  Кількість використань суперприйому обмежена й повинна перебувати в межах від $k_{\min}$ до $k_{\max}$ разів (обидві межі включно).   
  }
  Звісно, енергію слід витрачати максимально економно.
       


	Нехай вам відомі координати всіх платформ у порядку від лівого краю до правого. 
	Чи зможете ви знайти, яка мінімальна кількість енергії необхідна герою, щоб дістатися з першої платформи до останньої?

\InputFile
  У першому рядку записано кількість платформ $n$ ($1\leqslant n\leqslant 10000$). Другий рядок містить $n$ натуральних чисел, 
  що не~перевищують 30000 --- вис{\it о}ти, на яких розташовані платформи.
  Третій рядок містить два цілі невід'ємні числ{\it а} $k_{\min}$ та $k_{\max}$ ($0\leqslant k_{\min}\leqslant k_{\max}\leqslant \frac{n-1}{2}$). 

\OutputFile
  Виведіть єдине число --- мінімальну кількість енергії, яку має витратити гравець.

\Examples
\begin{example}
\exmp{3
1 5 10
0 1}{9}
\exmp{3
1 5 10
1 1}{27}
\exmp{3
1 5 2
0 1}{3}
\exmp{3
1 5 2
0 0}{7}
\exmp{5
1 2 3 30 31
0 1}{30}
\exmp{5
1 2 3 30 31
1 2}{34}
\end{example}

\Note

{\bf Тест 1} Вигідно стрибати, не користуючись суперприйомом (використавши його 0 разів).

{\bf Тест 2} Герой зобов’язаний використати суперприйом рівно один раз, і не має іншого вибору, крім як стрибати з першої платформи на останню.

{\bf Тест 3} Вигідно використати один суперприйом, щоб стрибнути з першої платформи на останню.

{\bf Тест 4} Суперприйомів фактично нема (кількість=0), тож нема іншого вибору, крім як стрибати послідовно через усі платформи одна за одною.

{\bf Тест 5} Вигідно стрибати, не користуючись суперприйомом (використавши його 0 разів)

{\bf Тест 6} Герой зобов’язаний використати суперприйом або один раз, або двічі; вибираючи, чи краще використати його двічі (з 1 на 3, потім з 3 на 31), чи один раз з 1 на 3, чи один раз з 2 на 30, чи один раз з 3 на 31, бачимо, що найвигідніше використати один раз з 1 на 3.

