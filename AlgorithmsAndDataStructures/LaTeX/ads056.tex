Напишіть програму, яка знаходитиме відстані в неорієнтованому зваженому графі з невід'ємними довжинами ребер, від зазначеної вершини до всіх інших. Програма повинна працювати швидко для великих розріджених графів.

\InputFile
У першому рядку вхідних даних задано число {\it NUM} --- кількість різних запусків алгоритму Дейкстри (на~різних графах). Далі слідують {\it NUM} блоків, кожен з яких має таку структуру.

Перший рядок блоку містить два числа {\it N} і {\it M}, розділені пропуском --- кількість вершин і кількість ребер графа. Далі слідують {\it M} рядків, кожен з яких містить по три цілих числа, розділені пробілами. Перші два з них в межах від 0 до {\it N}–1 кожне і позначають кінці відповідного ребр{\it а}, третє --- в~межах від 0 до 20000 і позначає довжину цього ребр{\it а}. Далі, в останньому рядку блоку, записане єдине число від 0 до {\it N}–1 --- вершина, відстані від якої треба шукати.

Кількість різних графів в одному тесті {\it NUM} не~перевищує~5. Кількість вершин не~перевищує 60000, ребер --- 200000.

\OutputFile
Виведіть {\it NUM} рядків, у кожному з яких по $N_i$ чисел, розділених пропусками --- відстані від зазначеної початкової вершини зваженого неорієнтованого графа до його 0-ї, 1-ї, 2-ї і~т.~д. вершин (допускається зайвий пробіл після останнього числа). Якщо деяка вершина недосяжна від зазначеної початкової, замість відстані виводите число 2009000999 (гарантовано, що всі реальні відстані менші).

\Examples

\begin{example}
\exmp{1
5 7
1 2 5
1 3 2
2 3 4
2 4 3
3 4 6
0 3 20
0 4 10
1}{18 0 5 2 8} 	
\end{example}


\Note

Переконайтеся, що програма правильно враховує, що за один запуск слід обробити кілька різних графів.

Щоб забезпечити асимптотичну оцінку часу виконання $O(M\log N)$, слід:

\begin{enumerate}
\item
Подавати граф списками суміжності і робити перебір сусідів аналогічно завданню «Пошук в ширину — 1».
(Однак, списками суміжності слід подати {\it зважений} граф, тому елементами списків повинні бути вже не числ{\it а}, а, наприклад, структури з двох полів (вершина, куди йде ребро, і довжина ребра).)

\item
При виборі, яка саме вершина статусу 1 має найнижчу оцінку відстані, користуватися деякою структурою даних, що вміє робити це ефективно. Зокрема (але~не~тільки), це може бути будь-який з таких способів:
\begin{enumerate}
\item
деяка реалізація priority queue, наприклад, піраміда;
\item
SortedSet.
\end{enumerate}
В будь-якому з цих випадків, зберігати в піраміді чи в SortedSet'і чи в ще якомусь аналогу треба не~самі лише оцінки відстаней, а такі структури чи класи, де кожен примірник містить і оцінку відстані, і номер вершини, а компаратор заданий так, щоб зручно й ефективно знаходити мінімальну оцінку відстані.
\end{enumerate}

Додаткову проблему створює те, що при виконанні алгоритму Дейкстри можлива зміна оцінки відстані до вершини статусу~1 (новий маршрут виявляється коротшим, чим знайдений раніше), після чого піраміда (чи SortedSet, чи ще якийсь аналог) має містити новий примірник структури/класу з тією ж вершиною, але меншою оцінкою відстані. Якщо лише вставляти новий примірник додатково до старого, це призводитиме до багатократного виймання з піраміди (чи SortedSet'а, чи ще якогось аналога) однієї з тієї ж вершини. Якщо таких вершин чимало, і з них виходить чимало ребер --- це може призводити до істотного сповільнення роботи всього алгоритму. Наприклад, неважко побудувати граф, де оцінка відстані деякої вершини зменшується $\approx0,4\cdot N$ разів, і с{\it а}ме з цієї вершини виходить $\approx0,5\cdot N$ ребер. Погана реалізація алгоритму Дейкстри може $\approx(0,4\cdot N)-1$ раз переглядати оцінки всих цих $\approx0,5\cdot N$ вершин (без жодної на те потреби, бо саме в цьому випадку пізніше побудовані оцінки будуть гіршими (більшими) за раніше побудовані). Тобто, погана реалізація алгоритму Дейкстри запросто може виконати $O(N^2)$ відверто зайвої роботи, що у випадку розріджених графів прямо протирічить меті «реалізувати алгоритм Дейкстри складністю $O(M\log N)$». І в тестах цієї задачі такі (та ще деякі схожі) випадки є.

Щоб уникати таких непродуктивних ситуацій, варто забезпечити одну з двох властивостей:

\begin{itemize}
\item
коли для деякої вершини зменшується оцінка відстані, відповідний примірник структури/класу слід вийняти з тієї структури даних, що використовується для пошуку мінімальної оцінки --- так, щоб там взагалі ніколи не~було кількох різних примірників, відповідних одній і тій самій вершині;
\item
коли деякий примірник структури/класу виймають з тієї структури даних, що використовується для пошуку мінімальної оцінки, треба додатково перевірити, чи така сама вершина ще~не~була вийнята раніше (іншими словами, чи вершина все~ще статусу~1, а~не~статусу~2).
\end{itemize}

(Яку з цих властивостей краще підтримувати? Якщо використовувати піраміду, то асимптотично ефективно забезпечити першу властивість важко (піраміда не~підтримує ефективних вилучень не~кореневих елементів), а другу неважко, тому варто забезпечувати другу. Якщо використовувати SortedSet, то забезпечувати першу стає легше (підтримання другої при цьому не~ускладнюється), й тому більш-менш байдуже.
Чи можна забезпечувати відразу обидві ці властивості? Можна, але не~варто, бо будь-яка одна вже позбавить від вищезгаданої відверто зайвої роботи.)