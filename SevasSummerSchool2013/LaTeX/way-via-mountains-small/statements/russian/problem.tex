\begin{problem}{Путь через горы (малые ограничения)}{stdin}{stdout}{2 секунды}{64 мегабайта}

Поверхность Земли в горной местности можно представить в виде ломаной линии.
Вершины ломаной расположены в точках $(x_1,y_1)$, $(x_2,y_2)$,~\ldots, 
$(x_N,y_N)$, при этом $x_i<x_{i+1}$.
Обычный горный маг находится в точке $(x_1,y_1)$ и очень хочет попасть 
в точку $(x_N,y_N)$. При этом он может перемещаться только пешком. 
Он может ходить по поверхности Земли (т.~е. вдоль ломаной). А~может
сотворить в воздухе мост и пройти по нему. Мост может соединять две вершины
ломаной: мост не~может начинаться и заканчиваться не~в~вершине ломаной, 
и мост не~может проходить под землей (в~т.~ч. не~может быть туннелем в~горе),
но~мост может каким-то своим участком проходить по поверхности земли. 
Длина моста не~может быть больше~$R$. Суммарно маг может построить 
не~более $K$ мостов. После прохождения моста, он (мост) растворяется 
в воздухе. Какое наименьшее расстояние придётся пройти магу, 
чтобы оказаться в точке $(x_N,y_N)$?


\InputFile
Программа должна прочитать сначала натуральное число $N$ 
($2\leqslant N\leqslant 42$);
затем натуральное число $K$ ($1\leqslant K\leqslant 23$) ---
максимальное количество мостов;
далее целое число $R$ ($0\leqslant R\leqslant 10000$) --- максимальную 
возможную длину моста. Далее координаты $(x_1,y_1)$, $(x_2,y_2)$,~\dots, 
$(x_N,y_N)$. 
Все координаты --- целые числа, не~превышающие по~модулю $10000$, 
для всех~$i$ от~$1$ до~$N$--1 выполняется $x_i<x_{i+1}$.


\OutputFile
Программа должна вывести одно число --- минимальную длину пути, 
которую придётся пройти магу (как по земле, так и по мостам). 
Ответ выведите с точностью 6 цифр после десятичной точки.


\Examples

\begin{example}
\exmp{5 2 5
0 0
2 2
3 -1
4 1
5 0
}{6.4787086646190746
}%
\end{example}

\end{problem}
