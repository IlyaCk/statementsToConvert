\begin{problem}{Раскладывание Пасьянса}{stdin}{stdout}{1 секунда}{64 мегабайта}

<<N-T пасьянс>>~--- карточная игра для одного игрока. В~игре используется $4N$ 
($3\leqslant N\leqslant 15)$ карт, причем
каждой карте соответствует уникальная пара её значения (целое число 
в диапазоне~$1.\,.N$) и масти ($\spadesuit$, $\clubsuit$, $\heartsuit$
или $\diamondsuit$). В~начальном положении все карты разложены в~$T$ 
($4\leqslant T\leqslant 12)$ стопок; при этом каждая из первых
$(4N)\%T$ стопок содержит по~$(4N/T)+1$ карт, остальные~--- 
по~$4N/T$ карт (здесь ``/'' и ``\%''~--- целочисленное деление и 
остаток от деления соответственно). Если сумма значений верхних карт
двух стопок равна~$N+1$, то эти две карты можно переместить в отбой 
(независимо от их мастей). Это единственный способ перемещать карты.

Напишите программу, которая будет определять, какое максимальное
количество карт можно переместить в отбой, и как это сделать. 


\InputFile
Первая строка содержит два целых числа~$N$ и~$T$, далее идут $T$ строк 
с описаниями карт соответствующей стопки. Каждая карта описывается 
её значением (целое число) и мастью (символ с ASCII-кодом 03($\heartsuit$), 
04($\diamondsuit$), 05($\spadesuit$), или 06($\clubsuit$)) без пробела 
между ними. Описания разных карт одной стопки разделены ровно одним пробелом,
направление описания слева направо соответствует
порядку карт снизу вверх.


\OutputFile
Ваша программа должна вывести целое число $S$ --- максимально возможное количество карт, которые можно переместить в отбой.
Затем $S/2$ пар чисел, по паре в строке: номера стопок, с которых надо снимать карты на очередном ходу. 

\Examples

\begin{example}
\exmp{3 5
2$\spadesuit$ 2$\clubsuit$ 2$\heartsuit$
2$\diamondsuit$ 3$\diamondsuit$ 1$\heartsuit$
3$\clubsuit$ 1$\spadesuit$
1$\clubsuit$ 3$\heartsuit$
1$\diamondsuit$ 3$\spadesuit$
}{10
2 4
2 3
1 2
4 5
3 5
}%
\end{example}

\Note
Если существует несколько способов раскладывания (с одинаковым максимальным суммарным количеством снятых карт), выведите любой.

\end{problem}
