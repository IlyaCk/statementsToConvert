\begin{problem}{Unfolding the Solitaire}{stdin}{stdout}{1 second}{64 megabytes}

<<N-T solitaire>> is a card game for one player. There are $4N$ 
($3\leq N\leq 15)$ cards in the game and each card corresponds to a unique pair of it's value (an~integer in the range~$1.\,.N$) and suit ($\spadesuit$, $\clubsuit$, $\heartsuit$ or~$\diamondsuit$). In the initial position all cards are laid out in $T$ ($4\leq T\leq 12)$ piles; moreover, each of $(4N)\%T$ first piles has $(4N/T)+1$ cards, others have $4N/T$ cards each (here ``/'' and ``\%''~--- integer division and remainder of division, respectively). If the sum of the values of upper cards of two piles is $N+1$, then these two cards can be moved to discard pile (irrespective of their suits). This is the only way to move the cards.

Write a program that will determine the maximum number of cards that one can move to the discard pile and how to do this.

\InputFile
The first line contains two integers $N$ and~$T$, then there are $T$~lines with descriptions of the corresponding piles. Each card is described as its value (an integer) and a suit (the char with ASCII-code 03($\heartsuit$), 04($\diamondsuit$), 05($\spadesuit$), or~06($\clubsuit$)) without space between. Descriptions of the cards inside the same pile are single-space separated. Description's direction from left to right corresponds to the order of cards from bottom to up.

\OutputFile
Your program should print in the first line a single integer --- the maximum number of cards that can be moved to the discard pile.
Then $S/2$ pairs of integers, one pair per line: the numbers of piles, from which cards should be removed.

\Examples

\begin{example}
\exmp{3 5
2$\spadesuit$ 2$\clubsuit$ 2$\heartsuit$
2$\diamondsuit$ 3$\diamondsuit$ 1$\heartsuit$
3$\clubsuit$ 1$\spadesuit$
1$\clubsuit$ 3$\heartsuit$
1$\diamondsuit$ 3$\spadesuit$
}{10
2 4
2 3
1 2
4 5
3 5
}%
\end{example}

\Note
If there are multiple ways to unfold (with the same maximum quantity of cards in discard pile), output any one.

\end{problem}
