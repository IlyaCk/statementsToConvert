\begin{problem}{Mathematical platforms}{stdin}{stdout}{1 second}{256 megabytes}

Can you remember at least single your acquaintance, who is less than twenty years old and didn’t play computer games in childhood? If yes, you might not know this kind of entertainment. But it wouldn’t prevent you to solve this problem.

In older games, which have 2D graphics, one can run into the next situation. The hero jumps along the platforms or islands that hang in the air. He must move himself from one side of the screen to the other. The hero can jump from any platform number~$i$ to any platform number~$k$, spending $(i-k)^2\cdot(y_i-y_k)^2$ energy, where $y_i$ and $y_k$ are the heights where these platforms hang. Obviously energy should be spent frugally.

You are given the heights of the platforms in order from the left side to the right. Can you find the minimum amount of energy to get from the 1-st (start) platform to the $n$-th (last)?

\InputFile
The first line contains the number of platforms $n$ ($1\leq n\leq 4000$). The second line gives $n$ integers --- the heights of the platforms, which absolute values are not greater than 200000.

\OutputFile
Print the singe integer which is the minimum amount of energy to get from the 1-st platform to the $n$-th (assuming, that player is forbidden to use cheat codes).

\Examples

\begin{example}
\exmp{4
1 2 3 30
}{731
}%
\end{example}

\end{problem}
