\begin{problem}{Математические платформы}{stdin}{stdout}{1 секунда}{256 мегабайт}

Вы можете вспомнить хоть одного своего знакомого до двадцатилетнего возраста, который в детстве не играл в компьютерные игры? Если да, то может быть вы и сами не знакомы с этим развлечением? Впрочем, трудностей при решении этой задачи это создать не должно.

Во многих старых играх с двумерной графикой можно столкнуться с подобной ситуацией. Какой-нибудь герой прыгает по платформам (или островкам), которые висят в воздухе. Он должен перебраться от одного края экрана до другого. Игрок может прыгнуть с любой платформы~$i$ на любую платформу~$k$, затратив при этом $(i-k)^2\cdot(y_i-y_k)^2$ единиц энергии, где $y_i$ и $y_k$ --- высоты, на которых расположены эти платформы. Конечно же, энергию следует расходовать максимально экономно.

Предположим, что вам известны координаты всех платформ в порядке от левого края до правого. Сможете ли вы найти, какое минимальное количество энергии потребуется герою, чтобы добраться с первой платформы до последней? 

\InputFile
В первой строке записано количество платформ $n$ ($1\leqslant n\leqslant 4000$). Вторая строка содержит $n$ целых чисел, не превосходящих по модулю 200000 --- высоты, на которых располагаются платформы.

\OutputFile
Выведите единственное число --- минимальное количество энергии, которую должен потратить игрок на преодоление платформ (конечно же в предположении, что cheat-коды использовать нельзя).

\Examples

\begin{example}
\exmp{4
1 2 3 30
}{731
}%
\end{example}

\end{problem}
