На прямокутному полi $N{\times}M$ клiтинок у лiвому верхньому кутку стоїть фішка. Ця~кутова клітинка гарантовано вільна (не~замінована); абсолютно кожна інша клітинка п{\it о}ля може бути хоч вільною, хоч замінованою. 
Гра полягає в тому, що два гравцi поперемiнно рухають згадану фішку на якусь кiлькiсть клiтинок праворуч або донизу (кожен гравець сам вирiшує, в якому з цих двох напрямків i на скільки клітинок рухати; не~можна ні~лишати фішку на мiсцi, ні~ставати нею в заміновану клітинку, ні~перестрибувати нею через заміновану клітинку).
Програє той, хто не~може нiкуди походити (і~знизу, і~праворуч або край п{\it о}ля, або міна). Відповідно, його суперник виграє.

Напишіть програму, яка визначатиме, хто виграє при правильній грі обох гравців. 
Іншими словами, хто може забезпечити собі виграш, хоч би як не грав інший.
Якщо виграє перший, то програма повинна знайти також сукупність усіх його виграшних перших ходів.

\InputFile
Перший рядок містить два цілі числ{\it а} $N$ та~$M$, розділені одним пропуском (пробілом)~--- спочатку кількість рядків, потім кількість стовпчиків. Обидва ці значення у межах від~1 до~2023.

Далі йдуть $N$ рядків, що задають мінне поле. Кожен з них містить рівно по~$M$ символів \texttt{.} (позначає вільну клітинку) та/або \texttt{*} (позначає заміновану клітинку). Ці~символи йдуть без роздільників, і кожен з цих $N$ рядків містить лише ці символи та переведення рядка наприкінці.

\OutputFile
Перший рядок має містити єдине ціле число, або \texttt{1} (якщо перший гравець може забезпечити собі виграш), або \texttt{2} (якщо др\it{у}гий). Якщо відповідь з першого рядка~\texttt{2}, то на цьому виведення слід припинити. А~якщо відповідь з першого рядка~\texttt{1}, то далі треба вивести також перелік всіх можливих перших ходів першого гравця, після яких др\it{у}гий (при правильній грі першого) вже ні\'{я}к не~зможе виграти. Цей перелік виводити в такому форматі: спочатку, якщо існує виграшний хід униз, то вивести його як велику латинську літеру~\texttt{D}, одинарний пропуск (пробіл) та довжину ходу (на~скільки клітинок іти вниз); потім, якщо існує виграшний хід направо, то аналогічно, але на~початку велику латинську~\texttt{R}. (Заодно {\it доведіть}, що в цій задачі не~може бути багатьох виграшних ходів униз чи багатьох виграшних ходів направо.)

\Examples
\begin{example}
\exmp{2 4
....
.**.
}
{1
D 1
R 2
}
\exmp{1 1
.
}
{2
}
\end{example}

\Note
Спочатку детально проаналізуємо перший приклад.

Перший гравець може піти \texttt{D~1} (на~одну клітинку вниз), після чого др{\it у}гому не~буде куди йти. А~ще~перший гравець може піти першим ходом \texttt{R~2}; тоді др{\it у}гому не~лишиться ніякого вибору, крім як піти \texttt{R~1} (на~ще~одну клітинку праворуч), після чого перший іде \texttt{D~1} (вниз), і знову др{\it у}гому не~буде куди йти. 

Перший хід \texttt{R~3} не~виграшний, бо після нього др{\it у}гий гравець іде \texttt{D~1} і виграє. 
Перший хід \texttt{R~1} теж не~виграшний, бо тут може бути як ситуація <<др{\it у}гий піде \texttt{R~2}, перший \texttt{D~1} і виграє>>, так і ситуація <<др{\it у}гий піде \texttt{R~1}, перший не~матиме іншого вибору, як іти \texttt{R~1}, др{\it у}гий піде \texttt{D~1} і виграє>>. Тобто, в разі першого ходу першого гравця \texttt{R~1} др{\it у}гий гравець може перехопити ініціативу, якщо зуміє. А~питали перелік таких ходів, щоб др{\it у}гий (при правильній грі першого) вже ні{\it я}к не~міг виграти.

У др{\it у}гому прикладі, першому гравцю відразу ж нема куди йти, і він негайно програ{\it є} (відповідно, вигра{\it є} др{\it у}гий гравець).