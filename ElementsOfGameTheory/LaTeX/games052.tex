% \begin{problemAllDefault}{Палички з ходами, залежними від попереднього; інтерактив}
% \label{prob:201819-2-E-sticks-with-spec-moves-interactive}

Гра та сама, що у попередній 
% задачі~\ref{problem:sticks-with-dependence-on-preevious-move}.
задачі «Палички з ходами, залежними від попереднього; хто переможе?».

Напишіть програму, яка інтерактивно гратиме за першого гравця. 

На початку, один раз, Ваша програма повинна прочитати одне ціле число в окремому рядку\nolinebreak[3] --- початкову кількість паличок $N$ ($2{\<}N{\<}12345$).

Потім слід повторювати такий цикл:
\begin{enumerate}
\item\looseness=-1
Вивести єдине число в окремому рядку\nolinebreak[3] --- свій хід, тобто кількість паличок, які вона зараз забирає з купки.
На першому ході це повинно бути 1 або~2, на подальших\nolinebreak[2] --- ціле число від~1 до подвоєної кількості, щойно забраної про\-гра\-мою-суперницею, причому не~більше за поточну кількість паличок у купці. 
\item
Якщо після цього купка стає порожньою, 
% вивести рядок із фразою ``\texttt{I~won!}''
вивести окремим рядком фразу ``\texttt{I~won!}''
(без лапок, символ-у-символ згідно зразку) і завершити. % роботу.
\item
Прочитати хід програми-суперниці, тобто кількість паличок, які вона зараз забирає з~купки (єдине % ціле 
число, в окремому рядку).
Якщо Ваша програма правильно визначила, що гра % ще~
не~закінчилася і цей хід % взагалі 
відбудеться, то гарантовано, що він допустимий (%введене 
число є цілим від~1 до подвоєної кількості, щойно забраної Вашою програмою, і не~перевищує поточну кількість паличок у купці).
\item
Якщо після цього купка стає порожньою, 
вивести окремим рядком 
фразу ``\texttt{You~won...}''
(без\nolinebreak[3] лапок, символ-у-символ згідно зразку) і завершити. % роботу.
\end{enumerate}
% Ці дії повинні повторюватися, 
Це слід повторювати, 
доки\nolinebreak[3] якась із програм-гравців не~в\'{и}грає.
% доки\nolinebreak[3] не\nolinebreak[2] будуть забрані всі палички (тобто, \mbox{доки} якась із програм-гравців не~в\'{и}грає).
Програма-суперниця не~виводить фраз ``\texttt{I~won!}''\nolinebreak[2] / ``\texttt{You~won...}'' 
чи\nolinebreak[2] якихось їх аналогів.

\Scoring
Тести оцінюються кожен окремо (без~блоків).
У\nolinebreak[3] \mbox{1-му}\nolinebreak[2] тесті\nolinebreak[2] ${N\,{=}\,3}$, 
у\nolinebreak[3] \mbox{2-му}\nolinebreak[2] ${N\,{=}\,5}$,
і ці тести не~приносять балів.
Решта тестів приносять однакові бали.
У\nolinebreak[3] 20\%\nolinebreak[3] тестів ${2\,{\<}\,N\,{\<}\,25}$ (${N\,{\neq}\,3}$, ${N\,{\neq}\,5}$), програма-суперниця ідеальна (не~робить помилок).
Ще\nolinebreak[3] у\nolinebreak[2] 20\%, ${100\,{<}\,N\,{\<}\,1234}$, суперниця ідеальна.
Ще\nolinebreak[3] у\nolinebreak[2] 20\%, ${1234\,{<}\,N\,{\<}\,12345}$, суперниця ідеальна.
Ще\nolinebreak[3] у\nolinebreak[2] 20\%, ${100\,{<}\,N\,{\<}\,1234}$, суперниці інші.
Ще\nolinebreak[3] у\nolinebreak[2] 20\%, ${1234\,{<}\,N\,{\<}\,12345}$, суперниці інші. 
%%
Ці\nolinebreak[3] інші програми-суперниці (їх\nolinebreak[3] кілька різних) роблять ходи, де гарантовано дотримані вимоги \textsl{<<забирати лише від~1 палички до подвоєної щойно забраної кількості>>} та \textsl{<<забирати не~більше паличок, чим~є у~купці>>}, але\nolinebreak[2] дотримуються кожна власних уявлень, як\nolinebreak[2] треба грати, частенько вибираючи не~найкра\-щий з допустимих ходів.

Буде оцінюватися і~вміння Вашої програми виграти там, де це % точно 
можливо,
і~вміння Вашої програми 
% достойно, згідно правил, 
гідно, дотримуючись правил гри,
програти, де виграш неможливий, 
і~вміння Вашої програми скористатися (теж\nolinebreak[2] згідно правил) помилками чи іншими неадекватностями програми-суперниці, якщо такі будуть.
За\nolinebreak[3] будь-яке порушення правил гри з\nolinebreak[3] боку Вашої програми, відповідний тест буде оцінено на~0~балів.

\Examples

\begin{exampleWidthsAndFileNames}{7em}{7em}{Вхід, суперник}{Ваша програма}
\exmp{3
~
2}{~
1
~
You won...}%
\end{exampleWidthsAndFileNames}

У купці спочатку 3 палички. Ваша програма забирає одну, лишається дві; програма-суперниця забирає обидві й виграє.

\noindent\hrulefill

\begin{exampleWidthsAndFileNames}{7em}{7em}{Вхід, суперник}{Ваша програма}
\exmp{3
~
1}{~
2
~
You won...}%
\end{exampleWidthsAndFileNames}

Спробуємо забрати не~одну, а~дві з трьох паличок, тобто лишити одну; програма-суперниця забирає її і теж виграє.

\noindent\hrulefill

\begin{exampleWidthsAndFileNames}{7em}{7em}{Вхід, суперник}{Ваша програма}
\exmp{5
~
1
~
2}{~
1
~
1
~
You won...}%
\end{exampleWidthsAndFileNames}

У купці спочатку п'ять паличок.
Ваша програма забирає одну, лишається чотири;
програма-суперниця забирає одну, Вашій програмі дістається три палички, й вона ніяк не може виграти з вищеописаних причин.

\noindent\hrulefill

\begin{exampleWidthsAndFileNames}{7em}{7em}{Вхід, суперник}{Ваша програма}
\exmp{5
~
3}{~
2
~
You won...}%
\end{exampleWidthsAndFileNames}

Спробуємо забрати дві з п'яти паличок (лишається три); програма-суперниця забирає всі три (має право, бо Ваша програма щойно взяла дві) й теж виграє.

\Notes
(1)~Всі наведені послідовності ходів є прикладами правильної гри.
Ваша програма не~зобов'язана при різних запусках для однієї початкової кількості паличок робити різні ходи. 
Але вона має таке право.
Якщо Ваша програма при різних запусках грає по-різному, 
% й це призводить до різних результатів, 
система автоматичної перевірки не~шукатиме ні\nolinebreak[2] найкращий, ні~найгірший з результатів, а~просто оцінюватиме перший.
(2)~Вводити/виводити порожні рядки не~треба; додаткові вертикальні відступи у пр\'{и}кладах зроблені умовно, щоб краще було видно, хто коли ходить.

% \end{problemAllDefault}