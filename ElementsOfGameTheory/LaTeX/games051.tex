% \begin{problemAllDefault}{Палички з ходами, залежними від попереднього; хто переможе?}%
% \label{problem:sticks-with-dependence-on-preevious-move}

Є одна купка, яка спочатку містить $N$ паличок.
Двоє грають у таку гру.
Спочатку перший 
може забрати з купки або~1, або~2 палички.
На~кожному подальшому ході кожен з гравців може забрати 
будь-яку кількість паличок від~1 до подвоєної кількості, щойно забраної суперником (обидві межі включно).
Інших
варіантів ходу нема. 
Ходять гравці по черзі, пропускати хід не~можна.
Виграє той, хто забирає останню паличку (можливо, разом з деякими іншими).

Напишіть програму, яка визначатиме, хто виграє при правильній грі обох гравців. 
Іншими словами, хто з гравців може забезпечити собі виграш, хоч би як не грав інший.



\InputFile
Єдине ціле число~$N$\nolinebreak[3] --- початкова кількість паличок у купці
($2\dib{{\<}}N\dib{{\<}}1234567$).

\OutputFile
Єдине ціле число, або \texttt{1} (якщо перший гравець може забезпечити собі виграш, як~би не~грав др\'{у}гий), або \texttt{2} (якщо др\'{у}гий, як~би не~грав перший).

\Examples
\begin{exampleSimple}{5em}{5em}
\exmp{2}{1}
\exmp{3}{2}
\exmp{4}{1}
\exmp{5}{2}
\exmp{1024}{2}\end{exampleSimple}


\Note Ці приклади частково пояснені також у прикладах до наступної задачі.

{

\looseness=-1
\Scoring\phantomsection\label{text:201819-2-D-scoring-begin}
\mbox{1-й}\nolinebreak[3] блок (тести\nolinebreak[3] \mbox{1--5}) містить тести з умови, перевіряється завжди, але безпосередньо не~оцінюється.
%
\mbox{2-й}\nolinebreak[3] блок (тести\nolinebreak[3] \mbox{6--10}) містить усі значення з\nolinebreak[3] діапазону $6\dib{{\<}}N\dib{{\<}}10$ і оцінюється у\nolinebreak[2] 30\%\nolinebreak[3] балів; перевіряється й оцінюється завжди.
%
\mbox{3-й}\nolinebreak[3] блок (тести\nolinebreak[3] \mbox{11--20}) має обмеження $11\dib{{\<}}N\dib{{\<}}100$ й оцінюється у\nolinebreak[2] 20\%\nolinebreak[3] балів; перевіряється й оцінюється завжди.
%
\mbox{4-й}\nolinebreak[3] блок (тести\nolinebreak[3] \mbox{21--30}) має обмеження $101\dib{{\<}}N\dib{{\<}}1234$ й оцінюється у\nolinebreak[2] 20\%\nolinebreak[3] балів; перевіряється й оцінюється завжди.
%
\mbox{5-й}\nolinebreak[3] блок (тести\nolinebreak[3] \mbox{31--40}) має обмеження $12345\dib{{\<}}N\dib{{\<}}43210$ і оцінюється у\nolinebreak[2] 15\%\nolinebreak[3] балів; перевіряється й оцінюється, лише якщо успішно пройдено всі попередні блоки.
%
\mbox{6-й}\nolinebreak[3] блок (тести\nolinebreak[3] \mbox{41--50}) має обмеження $123456\dib{{\<}}N\dib{{\<}}1234567$ і оцінюється у\nolinebreak[2] 15\%\nolinebreak[3] балів; перевіряється й оцінюється, лише якщо успішно пройдено всі попередні блоки.
%
Для кожного окремо взятого блоку, бали нараховуються, лише якщо успішно пройдено \emph{всі} тести блоку.\phantomsection\label{text:201819-2-D-scoring-end}

}

% \end{problemAllDefault}