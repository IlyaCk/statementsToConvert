$N$ карток викладені в ряд зліва направо. На~кожній картці написане ціле число. Два гравці по черзі забирають по одній картці, причому забирати можна або крайню ліву, або крайню праву. Закінчується гра, коли забрано всі картки (поки картки~є, гравець зобов'язаний робити один із можливих ходів). Мета гри~--- отримати якомога більшу суму (чисел, записаних на забраних картках).

Яку максимальну суму гарантовано зможе набрати перший гравець?


\InputFile

У першому рядку вказано кількість карток $N$ ($1\leqslant N\leqslant 2013$). У~другому рядку через пробіли задані $N$ цілих чисел (що~не~перевершують за~модулем~$10^3$; інакше кажучи, належать проміжку $[-1000; +1000]$) --- значення, записані на картках.


\OutputFile

Виведіть єдине ціле число~--- максимальну суму, яку гарантовано зможе набрати перший гравець.


\Examples

\begin{example}
\exmp{4
1 2 9 3}{10}%
\end{example}


\Note

Якщо на першому ході забрати~1, то суперник у відповідь буде змушений забрати або 3, або~2; у~будь-якому з цих випадків, перший гравець зможе забрати собі~9, і, таким чином, гарантовано отримати суму~10 (після чого другий гравець забирає останню картку, і гра закінчується).

Якби перший гравець на першому ходу забирав~3, то другий міг~би у відповідь забрати~9, і~в~результаті перший отримав би лише $3+2 = 5$. При великій дурості, другий гравець міг би відповісти на хід "3" і ходом "1"; у цьому випадку перший гравець міг би отримати суму $3+9=12$. Але перший гравець не~може гарантувати, що другий зробить такий дурний хід, тому й відповідь не~12, а~10.