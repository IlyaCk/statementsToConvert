Є $N$ купок, кожна з яких містить деяку кількість паличок.
Двоє грають у таку гру.
Кожен з гравців на кожному своєму ході може забрати з будь-якої однієї купки будь-яку кількість паличок, від~1 до зразу всіх паличок цієї купки. Палички можна лише забирати (ні~додавати, ні~перекладувати з~купки в~купку не~можна).
Ніяких інших варіантів ходу нема. 
Коли купка стає порожньою (кількість паличок=0), гра просто продовжується для решти купок.
Ходять гравці по черзі, пропускати хід не~можна.
Виграє той, хто забирає останню паличку (можливо, разом із ще деякими) з останньої купки.
(Інакше кажучи, виграє той, після чийого ходу не~лишається жодної палички в жодній купці.)

Напишіть програму, яка визначатиме, хто виграє при правильній грі обох гравців. 
Іншими словами, хто може забезпечити собі виграш, хоч~би~як не~грав інший.
Якщо виграє перший, то програма повинна знайти також сукупність усіх його виграшних перших ходів.

\InputFile
Перший рядок містить єдине ціле число~$N$ ($1\leqslant N\leqslant 123$) --- кількість купок.
Др{\it у}гий рядок містить рівно $N$ чисел $k_1$, $k_2$,~\dots, $k_N$, розділених одинарними пропусками (пробілами) --- початкові кількості паличок у кожній з купок. Всі ч{\it и}сла $k_1$, $k_2$,~\dots, $k_N$ є цілими, у межах від~1 до~123456, серед них можуть бути як однакові, так і різні.

\OutputFile
Перший рядок має містити єдине ціле число, або \texttt{1} (якщо перший гравець може забезпечити собі виграш), або \texttt{2} (якщо др{\it у}гий).
Якщо відповідь з першого рядка~\texttt{2}, то на цьому виведення слід припинити. А~якщо відповідь з першого рядка~\texttt{1}, то далі треба вивести також перелік всіх можливих перших ходів першого гравця, після яких др{\it у}гий (при правильній грі першого) вже ні{\it я}к не~зможе виграти. Цей перелік виводити в такому форматі: кожен такий хід в~окремому рядку; кожен хід записується як пара чисел через пропуск: спочатку з~якої купки слід забрати палички, потім скільки штук паличок треба забрати; якщо є багато різних виграшних перших ходів, вони повинні бути відсортовані за зростанням номера купки, а якщо є багато різних виграшних перших ходів, де палички беруться з однієї й тієї ж купки, то за зростанням кількості паличок, що беруться.

Кількість перших виграшних ходів виводити не~треба.
Вважати, що купки занумеровані з~одиниці: 1, 2,~\dots,~$N$.

\Examples

\begin{example}
\exmp{2
3 4}{1
2 1}
\exmp{2
5 5}{2}
\exmp{4
1 2 5 7}{1
1 1
3 1
4 1}\end{example}

\Note
В умові {\it є} одне дрібне неможливе й непотрібне уточнення. Приблизно у стилі {\it «якщо $2+2=5$, то виведіть слово "хрпщ"»} --- його виводити все'дно не~доведеться, бо все'дно $2+2\neq 5$. Знайти це дрібне неможливе й непотрібне уточнення~--- одне із завдань, які дуже бажано зробити у процесі розв'язування цієї задачі.