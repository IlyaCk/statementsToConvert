{\it Ця задача відрізняється від задачі «Нім — 1» в~точності тим, що за один хід можна забирати не~більше~3-х паличок.}

Є $N$ купок, кожна з яких містить деяку кількість паличок.
Двоє грають у таку гру.
Кожен з гравців на кожному своєму ході може забрати з будь-якої однієї купки або~1, або~2, або~3 палички (але, звісно, не~більше, чим їх є в цій купці). Палички можна лише забирати (ні~додавати, ні~перекладувати з~купки в~купку не~можна).
Ніяких інших варіантів ходу нема. 
Коли купка стає порожньою (кількість паличок=0), гра просто продовжується для решти купок.
Ходять гравці по черзі, пропускати хід не~можна.
Виграє той, хто забирає останню паличку (можливо, разом із ще деякими) з останньої купки.
(Інакше кажучи, виграє той, після чийого ходу не~лишається жодної палички в жодній купці.)

Напишіть програму, яка визначатиме, хто виграє при правильній грі обох гравців. 
Іншими словами, хто може забезпечити собі виграш, хоч~би~як не~грав інший.

\InputFile
Перший рядок містить єдине ціле число~$N$ ($1\leqslant N\leqslant 123$) --- кількість купок.
Др{\it у}гий рядок містить рівно $N$ чисел $k_1$, $k_2$,~\dots, $k_N$, розділених одинарними пропусками (пробілами) --- початкові кількості паличок у кожній з купок. Всі ч{\it и}сла $k_1$, $k_2$,~\dots, $k_N$ є цілими, у межах від~1 до~123456, серед них можуть бути як однакові, так і різні.

\OutputFile
Перший рядок має містити єдине ціле число, або \texttt{1} (якщо перший гравець може забезпечити собі виграш), або \texttt{2} (якщо др{\it у}гий).

\Examples

\begin{example}
\exmp{2
3 4}{1}
\exmp{2
5 5}{2}
\exmp{3
1 2 3}{2}
\exmp{2
4 8}{2}%
\end{example}
