$N$ карток викладені в ряд зліва направо. На~кожній картці написане ціле число. Два гравці по черзі забирають по одній картці, причому забирати можна лише з правого боку, або~1~картку, або~2~картки, або~3~картки (але, звісно, не~більше карток, чим їх лишилося). Закінчується гра, коли забрано всі картки (поки картки~є, гравець зобов'язаний робити один із можливих ходів). Мета гри~--- отримати якомога більшу суму (чисел, записаних на забраних картках).

Яку максимальну суму гарантовано зможе набрати перший гравець?


\InputFile

У першому рядку вказано кількість карток $N$ ($1\leqslant N\leqslant 1234567$). У~другому рядку через пробіли задані $N$ цілих чисел (що~не~перевершують за~модулем~$10^3$; інакше кажучи, належать проміжку $[-1000; +1000]$) --- значення, записані на картках.


\OutputFile

Виведіть єдине ціле число~--- максимальну суму, яку гарантовано зможе набрати перший гравець.


\Examples

\begin{example}
\exmp{7
11 11 11 11 11 11 11}{44}%
\exmp{8
3 2 999 4 6 7 -5 1}{1000}%
\exmp{9
23 -17 2 -13 5 3 11 -7 19}{30}
\end{example}

\Note

У першому прикладі, всі числа додатні й абсолютно однакові, тому гравцям вигідно забирати якнайбільше штук карток: 1-й забирає три штуки, потім 2-й забирає три штуки, потім 1-й забирає останню і цим завершує гру з сумою $(11+11+11)+(11)=44$.

У др{\it у}гому прикладі, значення 999 сильно перевищує всю решту, тому 1-му гравцю вигідно дбати в першу чергу про те, щоб це найбільше число дісталося саме йому.
І~перший хід «взяти дві картки 1 та –5» це забезпечує: як би на нього не відповів 2-й (чи забере лише~7, чи дві картки 7 і~6, чи~три картки 7, 6 і~4), все'дно на наступному ході 1-й гравець зможе взяти собі зокрема й картку~999. А~інші можливі перші ходи 1-го гравця не~гарантують йому взяття картки~999: якщо забрати на першому ході три картки, 2-й гравець зможе забрати 999 собі наступним же ходом; якщо забрати на першому ході одну картку, 2-й гравець у відповідь може взяти собі лише одну картку, тобто (згідно з уже наведеними міркуваннями) забезпечити, що картка~999 дістанеться йому. Звісно, для остат{\it о}чної відповіді потрібно не~лише забрати картку~999, а й отримати точне значення суми; це $(1+(-5))+(999+2+3)=1000$, що досягається при такому сценарії гри: 1-й забирає 1 та~(-5); 2-й забирає 7, 6 та 4 (це найкраще, що йому лишається після того, як його позбавили надії взяти~999); 1-й забирає 999, 2 і 3.

Зверніть увагу: щойно розглянутий приклад містить дуже контрінтуїтивний, на перший абсолютно погляд неприродний перший хід: треба взяти від'ємне значення –5 (хоча його можна й не~брати), і не~взяти після цього додатне значення~7 (хоч його й можна взяти разом з 1 і –5). Однак, усе це нівелюється тим, що с{\it а}ме 1-й гравець може гарантувати, що потім візьме~999.

Очевидно, що при «середніх» значеннях (і~різних, і~нема яскраво вираженого найбільшого) вищезгадані підходи не~працюють, а~працює лише специфічне застосування динамічного програмування. Можливість від'ємних значень на картках додатково зменшує ймовірність правильності простих-але-не-завжди-правильних евристик, не~заважаючи при цьому динамічному програмуванню.

У третьому прикладі, 30 досягається так:
1-й гравець забирає одну картку 19;
2-й гравець забирає три картки –7, 11, 3;
1-й гравець забирає одну картку 5;
2-й гравець забирає дві картки –13, 2;
1-й гравець забирає дві картки –17, 23, на чому гра закінчується;
$(19)+(5)+(-17+23)=30$.