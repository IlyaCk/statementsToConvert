На прямокутному полi $N{\times}M$ клiтинок у лiвому верхньому кутку стоїть фішка. Ця~кутова клітинка гарантовано вільна (не~замінована); абсолютно кожна інша клітинка п{\it о}ля може бути хоч вільною, хоч замінованою. 
Гра полягає в тому, що два гравцi поперемiнно рухають згадану фішку на якусь кiлькiсть клiтинок або праворуч, або донизу (кожен гравець сам вирiшує, в якому з цих двох напрямків i на скільки клітинок рухати; не~можна ні~лишати фішку на мiсцi, ні~ставати нею в заміновану клітинку, ні~перестрибувати нею через заміновану клітинку).
Програє той, хто не~може нiкуди походити (і~знизу, і~праворуч або край п{\it о}ля, або міна). Відповідно, його суперник виграє.

Напишіть програму, яка визначатиме, хто виграє при правильній грі обох гравців. 
Іншими словами, хто може забезпечити собі виграш, хоч би як не грав інший.

\InputFile
Перший рядок містить два цілі числ{\it а} $N$ та~$M$, розділені одним пропуском (пробілом)~--- спочатку кількість рядків, потім кількість стовпчиків. Обидва ці значення у межах від~1 до~12.

Далі йдуть $N$ рядків, що задають мінне поле. Кожен з них містить рівно по~$M$ символів \texttt{.} (позначає вільну клітинку) та/або \texttt{*} (позначає заміновану клітинку). Ці~символи йдуть без роздільників, і кожен з цих $N$ рядків містить лише ці символи та переведення рядка наприкінці.

\OutputFile
Єдине ціле число, або \texttt{1} (якщо перший гравець може забезпечити собі виграш), або \texttt{2} (якщо др\it{у}гий).

\Examples
\begin{example}
\exmp{2 4
....
.**.
}
{1
}
\exmp{1 1
.
}
{2
}
\end{example}

\Note
У першому прикладі, перший гравець може, наприклад, піти на одну клітинку вниз, після чого др{\it у}гому не~буде куди йти, і він прогр{\it а}є. 
Якби перший гравець сильно помил{\it и}вся і пішов на першому кроці на три клітинки праворуч, то др{\it у}гий пішов~би на одну вниз і виграв. Але це значило~б, що перший грає неправильно, а~питають про ситуацію правильної гри обох гравців.

У др{\it у}гому прикладі, першому гравцю відразу ж нема куди йти, і він негайно програ{\it є} (відповідно, вигра{\it є} др{\it у}гий гравець). Звісно, з точки зору загальнолюдських уявлень про справедливість це якась дуже нечесна гра\dots  
Але вже яка є, описані в умові задачі обмеження формально дотримані.