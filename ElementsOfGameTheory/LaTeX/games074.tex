% \begin{problemAllDefault}{Плюси й мінуси}
% \label{problem:red-is-good}

Є~$a$~карток, на яких написано~\mbox{``+''}, 
та\nolinebreak[3]
$b$~карток, на яких написано~\mbox{``–''};
ці\nolinebreak[3] написи є лише з\nolinebreak[3] одного боку, а\nolinebreak[3] з\nolinebreak[3] іншого боку ці картки (всі\nolinebreak[2] ${a\,{+}\,b}$\nolinebreak[2] штук) однакові.
Всі\nolinebreak[3] ці\nolinebreak[3] картк\'{и} якось випадково перемішані, й\nolinebreak[2] усі лежать догори тією стороною, з\nolinebreak[3] якої вони однакові.

Єдиний гравець може бр\'{а}ти ці картк\'{и} по\nolinebreak[3] одній, перевертати й дивитися, чи\nolinebreak[3] взяв~\mbox{``+''},\nolinebreak[2] чи~\mbox{``–''}.
За~кожну картку, на~якій написано~\mbox{``+''}, виграш гравця збільшується\nolinebreak[3] на~1,
а\nolinebreak[3] за\nolinebreak[3] кожну, на~якій написано~\mbox{``–''}, зменшується\nolinebreak[3] на~1, і\nolinebreak[3] може ставати від'ємним.
Зокрема, якщо забрати всі картк\'{и}, то виграш 
% становитиме\nolinebreak[2] 
буде\nolinebreak[2] 
$a\,{-}\,b$.

Але гравець не~зобов'язаний забирати всі картк\'{и}; він має право в будь-який момент (після взяття будь-якої к\'{а}ртки, вже побачивши її позначку, або % навіть 
на с\'{а}мому початку, ще\nolinebreak[3] нічого не~взявши) заявити \textsl{«закінчую»} і припинити гру. Тоді його виграш дорівнює сумі, яку він уже набрав на той момент (включно з останньою взятою карткою, якщо така була).

Яке максимальне матсподівання виграшу може забезпечити собі гравець при правильній грі?


\InputFile
У єдиному рядку через пропуск (пробіл) вказано два числа $a$, $b$ (обидва цілі, з\nolinebreak[3] проміжку від~0 до~100)\nolinebreak[3] — кількості карток з позначками \mbox{``+''}\nolinebreak[3]
та\nolinebreak[3] \mbox{``–''} відповідно.

\OutputFile
Виведіть єдине дійсне число~--- матсподівання виграшу для найкращої можливої стратегії єдиного гравця, якщо картк\'{и} перемішані випадково, а\nolinebreak[3] гравець знає ч\'{и}с\-ла\nolinebreak[3] $a$,~$b$. Формат виведення може бути будь-яким зі стандартних (зокрема, байдуже, чи\nolinebreak[3] вивести\nolinebreak[3] \texttt{0.5}, чи\nolinebreak[3] \texttt{0.500000000}, чи\nolinebreak[3] \mbox{\texttt{5e-1}}), важливо лише забезпечити точність, вказану в~«Оцінюванні».

\Examples

\noindent
\begin{exampleSimple}{16em}{5em}
\exmp{2 0}{2}%
\exmp{0 2}{0}%
\exmp{7 50}{0}%
\exmp{1 1}{0.5}%
\exmp{7 9}{0.299038462}%
\end{exampleSimple}

\Notes
У~першому прикладі, всі картк\'{и} мають позначки~\mbox{``+''}, їх вигідно забрати всі (обидві).
У~др\'{у}гому прикладі, всі картк\'{и} мають позначки~\mbox{``–''}, їх вигідно взагалі не~бр\'{а}ти.
У~третьому прикладі, мінусів настільки більше, чим плюсів, що теж вигідно взагалі не~бр\'{а}ти.
У~четвертому прикладі, вигідно взяти першу картку, далі так:
\begin{itemize}
\item
% якщо на ній~``$+$'', то на цьому спинитися, й виграш буде~1;
якщо на ній~``$+$'', то спинитися; виграш буде~1;
\item
якщо на ній~``$–$'', то взяти й наступну (останню), на ній точно буде~``$+$''; виграш буде ${({-}1)\,{+}\,1}\dib{{=}}0$.
\end{itemize}
Ймовірності кожного з цих випадків $\frac{1}{2}$, тому матсподівання % виходить
$
{\frac{1}{2}\cdot1}\dib{{+}}
{\frac{1}{2}\cdot0}\dibbb{{=}}
{\frac{1}{2}}\dib{{+}}
{0}\dibbb{{=}}
{\frac{1}{2}}
$.
Також зверніть увагу, що міркування \textsl{«раз плюсів і мінусів однаково, то й сума буде~0»}, хоч і може здатися природнім, насправді нічого не~дає в цій задачі. \mbox{С\'{а}ме}\nolinebreak[3] тому, що в~деяких ситуаціях видно, що вигідніше зупинитися й не~забирати решту карток.
П'ятий приклад надто громіздкий, щоб пояснити число-відповідь детально; але зверніть увагу, що вміння вчасно зупинятися може дати додатне матсподівання виграшу, навіть коли мінусів (трохи) більше, чим плюсів.

\Scoring
Потестове (кожен тест перевіряють і оцінюють неза\-леж\-но від решти).
Тест зараховують, коли абсолютна або відносна похибка (хоча~б одна з~двох) не~перевищує~$10^{-9}$.

% \end{problemAllDefault}
