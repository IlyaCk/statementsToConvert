{\it Це інтерактивна задача. Прочитайте умовну повністю, щоб зрозуміти, як працювати з такими задачами.}

$N$ карток викладені в ряд зліва направо. На~кожній картці написане ціле число. Два гравці по черзі забирають по одній картці, причому забирати можна лише з правого боку, або~1~картку, або~2~картки, або~3~картки (але, звісно, не~більше карток, чим їх лишилося). Закінчується гра, коли забрано всі картки (поки картки~є, гравець зобов'язаний робити один із можливих ходів). Мета гри~--- отримати якомога більшу суму (чисел, записаних на забраних картках).

Напишіть програму, яка інтерактивно гратиме за першого гравця. 

Ця задача є інтерактивною: 
Ваша програма не~отримає всіх вхідних даних на~початку,
а~отримуватиме по~мірі виконання доуточнення, 
що залежатимуть від попередніх дій Вашої програми. 
Тим~не~менш, {\it її перевірка буде
\underline{автоматичною}}. 
Тому, слід чітко дотримуватися формату спілкування з~програмою, яка грає роль суперника.


\Interaction

На початку, один раз, Ваша програма повинна прочитати початкову позицію гри, яка задається у двох рядках.
Перший рядок містить єдине число~$N$ ($1\leqslant N\leqslant 1000$) --- кількість карток.
У~другому рядку через пробіли задані $N$ цілих чисел (що~не~перевершують за~модулем~$10^3$; інакше кажучи, належать проміжку $[-1000; +1000]$) --- значення, записані на картках.

Потім Ваша програма повинна повторювати такий цикл:
\begin{enumerate}
\item
Вивести окремий рядок, що містить єдине ціле число від~1 до~3 (але не~більше, чим поточна кількість карток)~--- свій хід, який позначає, скільки карток з правого боку забирає Ваша програма.

\item
Якщо забрана картка була останньою, 
вивести окремим рядком фразу формату "\texttt{FINISH} {\it <myScore> <yourScore>}"
(без лапок; 
слово \texttt{FINISH} треба так і вивести, символ-у-символ згідно зразку;
замість {\it <myScore>} повинна бути сума чисел на картках, забраних Вашою програмою;
замість {\it <yourScore>} повинна бути сума чисел на картках, забраних програмою-суперницею) 
й завершити роботу.

\item
Інакше, прочитати хід програми-суперниці, в~такому~ж форматі, як описано в позаминулому пункті.

\item
Якщо забрана картка була останньою, 
вивести окремим рядком фразу в абсолютно такому самому форматі, як описано в позаминулому пункті, й завершити роботу.
\end{enumerate}

Все вищезгадане повинно повторюватися, доки~не~будуть забрані всі картки (тобто, доки гра не~завершиться).

Програма-суперниця не~виводить фраз формату "\texttt{FINISH} {\it <myScore> <yourScore>}" чи якогось аналогічних.

Наполегливо рекомендується, щоб Ваша програма після кожного свого виведення 
робила дію \verb"flush(output)" (Pascal), 
вона~ж \verb"cout.flush()" (C++), 
вона~ж \verb"fflush(stdout)" (C), 
вона~ж \verb"sys.stdout.flush()" (Python),
вона~ж \verb"System.out.flush()" (Java).
Це~істотно зменшує ризик, 
що~проміжна відповідь <<загубиться>> десь по~дорозі, 
не~дійшовши до програми-суперниці.




\Example

\begin{example}
\exmp{8
3 2 999 4 6 7 -5 1

3
}{

2

3
FINISH 1000 17}%
\exmp{8
3 2 999 4 6 7 -5 1

1

1}{

2

3

1
FINISH 1008 9}%
\end{example}


\Note

Перший приклад є грою ідеальних суперників, які обидва завжди вибирають найкращі ходи:
спочатку 1-й гравець забирає \texttt{2} картки (1 та~(-5)); 
потім 2-й забирає \texttt{3} картки (7, 6 та 4); 
потім 1-й забирає \texttt{3} картки (999, 2 та 3).
Протягом цієї гри 1-й гравець набрав $(1+(-5))+(999+2+3)=1000$, а 2-й гравець набрав $(7+6+4)=17$.

У др{\it у}гому прикладі послідовність карток така с{\it а}ма, але програма-суперниця грає неправильно. Внаслідок цього, Вашій програмі вдається набрати більше: $(1+(-5))+(6+4+999)+(3)=1008$. Але Ваша програма зарані не~знає того, чи~оптимально грає програма-суперниця, й повинна бути готова до будь-яких дозволених правилами гри варіантів розвитку подій. Також зверніть увагу, що в цьому прикладі Ваша програма повинна вивести наприкінці \texttt{FINISH 1008 9}, згідно фактичної гри (а~не~попередню теоретичну оцінку цих сум, яка така само \texttt{1000 17}. як і в попередньому прикладі).


\Scoring

Оцінювання потестове (кожен тест запускається й оцінюється незалежно від результатів проходження інших тестів; оцінка за розв'язок є сумою оцінок за проходження окремих тестів).

Вашій програмі треба буде грати як з ідеальними програмами-суперницями, так і іншими. Вашій програмі при цьому невідомо, чи конкретно в цьому запуску йдеться про ідеального суперника, чи~ні. Незалежно від того, якою є програма-суперниця, Ваша програма зобов'язана набрати суму, більшу або рівну сумі, обчисленій за правилами попередньої (не~інтерактивної) задачі (якщо Ваша програма набере строго менше, оцінка за відповідний тест буде~0; якщо Ваша програма набере більше, ніяких підвищень балів за це не~буде).

Решта вимог до Вашої програми: довести гру до кінця; вчасно помітити, що гра закінчилася, після чого негайно вивести фінальне повідомлення й завершити роботу; правильно порахувати, хто яку суму набрав (фактично).