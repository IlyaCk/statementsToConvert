Кількакрокова послідовна гра на дереві рішень відбувається так:
спочатку 1-й гравець повинен прийняти рішення, вибравши один з можливих варіантів свого ходу; 
деякі з цих варіантів ходу можуть відразу мати числові значення, скільки від того виграв 1-й гравець і скільки 2-й, 
решта передбачають, що тепер повинен прийняти рішення (вибрати варіант свого ходу) 2-й гравець, 
вже знаючи, яке рішення перед тим прийняв (який варіант свого ходу вибрав) 1-й гравець.
Серед варіантів ходу 2-го гравця теж деякі можуть мати готові числові значення, скільки від того виграв 1-й гравець і скільки 2-й,
а~решта передбачати, що тепер повинен прийняти рішення (вибрати варіант свого ходу) знову 1-й гравець.
І~так далі.

(Цю~схему можна узагальнити на більшу кількість гравців, але в цій задачі гравців рівно двоє, й рішення приймають спочатку 1-й, потім 2-й, потім 1-й, потім 2-й, \dots) 

Всі гілки такого д{\it е}рева мусять колись закінчитися «листками» з готовими числовими значеннями, скільки виграв 1-й гравець і скільки 2-й, але шляхи від кореня (початкового вибору 1-го гравця) до різних «листків» є різними, й кількості проміжних вершин-рішень у цих різних шляхах можуть бути різними (однаковими теж можуть).

% % % % % % % % \begin{footnotesize}

% % % % % % % Наприклад, на рисунку зображено модель того, як може відбуватися взаємодія між власником підприємства («Вл.») та особою, що бажає найнятися працівником («Пр.») зразу після співбесіди між ними.
% % % % % % % (Звісно, це лише модель, яка враховує лише спрощену частину реальності. Чому розглянуті с{\it а}ме такі варіанти рішень і звідки взяті конкретні ч{\it и}сла у вузлах-«листках»~--- окреме складн{\it е} питання, часто складіше за основний алгоритм розв'язання задачі, коли все це вже задано. З~одного боку, «вони виражають індивідуальні особливості конкретного власника й конкретного працівника» та «слід просто подивитися на реальність і описати, хто що може зробити й кому що наскільки вигідно»; з~іншого~--- це якраз зовсім не~«просто».)

% % % % % % % % \end{footnotesize}

% % % % % % % % \includegraphics{https://static.eolymp.com/content/i6/i6s9di4t5p2tn402r42aiakon0.png}
% % % % % % % %%% \input dec-tree-1

% % % % % % % % \begin{footnotesize}

% % % % % % % Власник може відразу відмовити у взятті на роботу, тоді обидві сторони отримують помірно-від'ємні «виграші» (–3) та (–2) відповідно (час витратили, власник не~отримав працівника, потенційний працівник не~отримав роб{\it о}ти), на чому взаємодія між ними закінчується. Два альтернативні варіанти~--- власник може взяти на роботу, призначивши або низьку зарплату~(ЗП), або добру~ЗП. У~кожному з цих випадків результат залежить від подальшого рішення працівника (який вже знає рівень призначеної~ЗП). 

% % % % % % % Якщо ЗП~добра і~працівник працює добре~--- обидві сторони отримують непогані виграші (+25 та +20 відповідно). 
% % % % % % % Якщо ж, при добрій~ЗП, працівник працює ліниво~--- власнику слід прийняти рішення, чи~змиритися й лишити як~є (це найкращий варіант для працівника~(+50), але поганий для власника), чи~звільнити працівника.

% % % % % % % Якщо ще на першому кроці власник призначив низьку~ЗП, то працівник теж має аналогічні варіанти «працювати добре» і «працювати ліниво» (але в них і~числові вираження виграшів інші, і~вибір працювати ліниво не~потребує додаткового рішення власника, він з цим змирився ще коли призначав низьку~ЗП), і крім того ще варіанти «сам{\it о}му відмовитися від такої роботи» та «почати працювати добре, але постійно вимагати підвищити~ЗП» (після чого власник мусить або підвищити~ЗП, або звільнити працівника).

% % % % % % % % \end{footnotesize}

% % % % % % % Такі задачі слід розв'язувати, починаючи від найдальших від кореня «листків». 

% % % % % % % Наприклад, якщо вже сталося, що власник призначив добру~ЗП, а працівник працює ліниво, то власнику слід вибрати, змиритися чи звільнити працівника; відповідні вузли є «листками», і серед чисел (–12) та (–10) більшим є (–10), тому власник вибирає варіант «звільнити», а отже~--- пара оцінок «(–10) собі, (+1) працівникові» переноситься на проміжний вузол.
% % % % % % % Зверніть увагу: те, що у парах $(-10,+1)$ та $(-12,+50)$ дуже різний виграш працівника, ніяк не~впливає, власник враховує лише свій виграш.

% % % % % % % % \includegraphics{https://static.eolymp.com/content/6i/6itgsm9lsd1dn8srlab55t1i80.png}
% % % % % % % %%% \input dec-tree-1-a

% % % % % % % Далі все відбувається аналогічно:
% % % % % % % \begin{enumerate}
% % % % % % % %%% \refstepcounter{enumi}
% % % % % % % \item
% % % % % % % у ситуації «ЗП~низька, працівник працює добре й вимагає підвищити~ЗП» власник при виборі між $(+10,+25)$ і $(-6,-3)$ вибирає
% % % % % % % $(+10,+25)$ (бо +10=max(+10,–6));
% % % % % % % \item
% % % % % % % у ситуації «ЗП~добра», працівник при виборі між $(-10,+1)$ і $(+25,+20)$ вибирає $(+25,+20)$ (бо +20=max(+1,+20));
% % % % % % % \item
% % % % % % % у ситуації «ЗП~низька», працівник при виборі між $(-4,0)$, $(+1,+4)$, $(+40,-10)$ і $(+10,+25)$ вибирає $(+10,+25)$ (бо +25=max(0, +4, –10, +25));
% % % % % % % \item
% % % % % % % у корені (початковій ситуації), власник при виборі між $(+25,+20)$, $(+10,+25)$ і $(-3,-4)$ вибирає $(+25,+20)$ (бо +25=max(+25, +10, –3)).

% % % % % % % \end{enumerate}

% % % % % % % (У цьому переліку, {\it деякі} (але не всі) пункти можна переставити місцями.)

% % % % % % % \includegraphics{https://static.eolymp.com/content/i4/i4sp2paij976v9a052eontfvoc.png}
% % % % % % % %%% \input dec-tree-1-b

\InputFile
Єдиний рядок, який містить дерево, закодоване таким способом:
\begin{enumerate}
\item
кожен окремий вузол-«листок» подається у вигляді $(p_1\,\,\,p_2)$, тобто: відкривна кругла дужка, величина виграшу 1-го гравця, пробіл, величина виграшу 2-го гравця, закривна кругла дужка;
\item
кожен вузол, що не~є «листком» (отже, в ньому приймається рішення й різні варіанти ведуть до різних подальших вузлів) подається у вигляді: відкривна квадратна (якщо ходить 1-й гравець) чи кутова (якщо 2-й) дужка, пробіл, код вузла, куди веде перший варіант рішення, пробіл, код вузла, куди веде др{\it у}гий варіант рішення, пробіл, \dots, код вузла, куди веде останній варіант рішення, пробіл, закривна квадратна чи кутова дужка (так само, \texttt{]} для 1-го і \texttt{>} для 2-го).
При цьому вкладені коди вузлів можуть відповідати чи то попередньому пункту (якщо вони вже не~мають розгалужень), чи то поточному (якщо мають).
\end{enumerate}

Наприклад, \texttt{[ (2 3) (4 5) (6 1) ]} подає дерево, де єдиний проміжний вузол має розгалуження на три варіанти, які всі є «листками», виграші яких становлять 
2 для 1-го і 3 для 2-го,
4 для 1-го і 5 для 2-го,
6 для 1-го і 1 для 2-го.
% Також, на рисунку нижче зображено, яким рядком кодується дерево, неоднократно зображене вище (але тепер без словесних позначок на ребрах), причому частково вказано, яким піддеревам відповідають деякі підрядки.

% \includegraphics{https://static.eolymp.com/content/28/28u19lr9jl7et6b19n2g4b2ec4.png}
%%% \input dec-tree-1-code

Гарантовано, що:
\begin{enumerate}
\item
дерево містить щонайменше один вузол;
\item
рядок, що кодує дерево, має не~більше $10^5$ символів (усіх, включно з пробілами та дужками);
\item
для проміжних вузлів, кількість варіантів вибору перебуває в межах від~2 до~6;
\item
всі виграші обох гравців (рахуючи по всім вузлам-«листкам») є різними числами, кожне з яких поміщається у 32-бітовий знаковий \end{enumerate}

\OutputFile
Програма виводить два числа в одному рядку, розділені пропуском: виграші, які мають отримати 1-й та 2-й гравці відповідно, якщо застосувати описаний алгоритм зворотньої індукції.

\Example

\begin{scriptsize}
\begin{examplewide}
% \begin{example}
\exmp{[ < [ (-12 50) (-10 1) ] (25 20) > < (-4 -2) (1 4) (40 -10) [ (10 25) (6 -7) ] > (-3 -2) ]}{25 20}
% \end{example}
\end{examplewide}
\end{scriptsize}


% % % % % % \Note

% % % % % % Практичну цінність\&застосовність цього алгоритму сильно знижує поєднання таких факторів:
% % % % % % \begin{enumerate}
% % % % % % \item
% % % % % % Для застосування цього алгоритму кожен гравець повинен знати все дерево, включно з величинами виграшів у всіх «листках», причому як свого виграшу, так і суперника.
% % % % % % Наприклад, коли працівник робить вибір у ситуації «власник призначив низьку~ЗП», то вибір варіанту «почати працювати добре, але постійно вимагати підвищити~ЗП» обґрунтовується тим, що при його виборі на наступному ході власнику буде вигідніше все-таки підвищити~ЗП, чим звільнити працівника, який почав працювати добре. Але якщо працівник думає, що це так, а~насправді власник настільки не~любить змінювати своє рішення щодо~ЗП, що справжні велич{\it и}ни виграшу в цьому «листку» не~$(+10,+25)$, а~$(-10,+25)$, тож власнику краще звільнити такого «бунтівливого» працівника~--- той с{\it а}мий вибір працівника призвів би вже до його звільнення власником, що працівнику менш вигідно, чим, наприклад, варіант «лінива праця». Навіть якщо працівник чудово знає сукупність варіантів д{\it е}рева ходів і правильно розраховує велич{\it и}ни власних виграшів у кожному вузлі~--- все~це не~рятує від невдалого рішення, спричиненого неправильною інформацією про сам{\it у} лише чужу величину виграшу. Ще~одна схожа ситуація в цьому ж дереві: якщо власник думає, що всі велич{\it и}ни виграшів такі, як на~рисунках, а~насправді виграш працівника у ситуації «ЗП~добра, працівник працює добре» становить не~(+20), а~0, то власнику насправді вигідно призначати низьку~ЗП, а~не~високу (при високій~ЗП такий працівник все'дно вибере «працювати ліниво», що власнику невигідно).
% % % % % % \item
% % % % % % Застосування цього алгоритму дає чіткий однозначний результат {\it лише} якщо ніколи не~трапиться ситуація, що варіанти, з~яких треба обирати, мають однакові оцінки. Нехай, наприклад, варіанти відповіді власника на ситуацію, коли він уже призначив низьку~ЗП, а працівник вже почав працювати добре, але вимагати підвищення~ЗП, мають виграші $(-6,-7)$ («звільнити») та $(-6,+25)$ («підвищити~ЗП»). Тоді працівник взагалі ні{\it я}к не~може оцінити, чи його вибір працювати добре, але вимагати підвищення~ЗП призведе до підвищення~ЗП (отже, виграшу~(+25)), чи до звільнення (отже, виграшу~(–7)). Через~це, працівник ні{\it я}к не~може визначити, чи вибір «працювати добре, але вимагати підвищення~ЗП» для нього кращий за вибір «працювати ліниво», чи гірший.
% % % % % % Що~у~свою чергу призводить до того, що взагалі неможливо визначити оцінку для вузла «власник вже взяв на роботу, призначивши низьку~ЗП, тепер хід працівника».

% % % % % % Тому, при використанні цього алгоритму зазвичай вимагають, щоб всі значення виграшу були різними (точніше, всі значення виграшу кожного окремо взятого гравця були різними, бо цей алгоритм ніколи не~порівнює виграші різних гравців). 

% % % % % % \end{enumerate}

% % % % % % Ще, цьому алгоритму при виборі максимального виграшу серед можливих варіантів по суті байдуже, чи максимальний «трохи більше» за інші, чи «набагато більше».
% % % % % % Тому, при використанні цього алгоритму іноді вимагають, щоб для д{\it е}рева з $n$ «листками» значення виграшу в листках були числами від~1 до~$n$. (Наприклад, якщо маємо відсортовані значення виграшів працівника –10, –3, –2, 0, 1, 4, 20, 25, 50, то, на думку прихильників цієї вимоги, слід замінити –10 на~1, –3 на~2, –2 на~3, 0 на~4, \dots, 50 на~9.)

% % % % % % Також слід розуміти, що (на відміну від більшості задач комплекту) знайдені величини виграшів {\it не~є} величинами, які кожен окремо гравець може забезпечити собі, хоч би як не грав інший. Алгоритм прямо розраховує на те, що суперник діятиме, намагаючись збільшити свій виграш, а~не~«як~завгодно» і не~«аби~створити проблеми своєму супернику~(нам)». З~одного боку, це чудово, бо створює шанси вибрати якийсь із {\it взаємо}вигідних варіантів (якщо такі існують). Але з~іншого боку це означає, що навіть коли нам відоме все дерево рішень, значення усіх оцінок (включаючи чужі) гарантовано правильні, і ми вміємо все обчислити за цим алгоритмом --- все'дно знайдена цим алгоритмом оцінка для нашого гравця не~гарантована: якщо суперник помилиться чи зіграє безграмотно, він може в результаті погіршити не~лише свій виграш, а~й~наш.
