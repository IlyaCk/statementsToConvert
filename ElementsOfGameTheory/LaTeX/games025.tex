Є одна купка, яка спочатку містить $N$ паличок.
Двоє грають у таку гру.
Кожен з гравців на кожному своєму ході може забрати з купки деяку кількість паличок (звісно, не~більше, чим їх є в купці), {\bf причому кількості паличок, які можна забирати, для різних суперників різні: 1-й гравець може забирати або~4, або~8, або~16, або~32 палички, тоді як 2-й --- або~2, або~7, або~14}.
Ніяких інших варіантів ходу нема. 
Ходять гравці по черзі, пропускати хід не~можна.
Програє той, хто не~може походити. Зверніть увагу: оскільки в цій задачі гравці не~можуть забирати 1~паличку, можливі також і ситуації, коли палички ще є, а походити вже не~можна. Точніше кажучи, 1-й вже не~може ходити не~лише коли йому не~лишили паличок, але також і коли лишили 1, 2 або 3 палички; 2-й вже не~може ходити не~лише коли йому не~лишили паличок, але також і коли лишили 1 паличку.

\InputFile
Єдине ціле число~$N$ ($1\leqslant N\leqslant 12345$) --- початкова кількість паличок у купці.

\OutputFile
Перший рядок має містити єдине ціле число, або \texttt{1} (якщо перший гравець може забезпечити собі виграш), або \texttt{2} (якщо др{\it у}гий).
Якщо відповідь з першого рядка~\texttt{2}, то на цьому виведення слід припинити. А~якщо відповідь з першого рядка~\texttt{1}, то далі треба вивести також перелік всіх можливих перших ходів першого гравця, після яких др{\it у}гий (при правильній грі першого) вже ні{\it я}к не~зможе виграти. Цей перелік виводити в такому форматі: вибрати лише потрібні ч{\it и}сла з переліку допустимих ходів 4, 8, 16, 32, і записати в другому рядку всі вибрані («виграшні») в порядку зростання через пробіли. Кількість «виграшних ходів» виводити не~треба і не~можна.

\Examples
\begin{example}
\exmp{7}{2}
\exmp{10}{1
4}
\end{example}

\Note
У першому тесті, 1-й гравець фактично може забрати лише 4 палички; після цього, у 2-го теж нема вибору (може лише забрати 2 палички), але 2-й походити ще зміг, а 1-й, якому лишається 1 паличка, більше не~має ходів і програ{\it є}.

У другому тесті, у 1-го гравця на першому ході фактично є вибір «забирати~4 чи забирати~8», причому якщо він забере~8, то 2-й забере 2~палички і тим в{\it и}грає, бо 1-му гравцю нема паличок і тому нема ходів. (Отже, 1-му гравцю не~варто забирати~8 на своєму першому ході.) А от якщо 1-й гравець на першому ході здогадається забрати 4~палички (залишиться~6), то у 2-го нема вибору (може лише забрати 2 палички, лишається~4), після чого 1-й гравець може забрати 4~палички і тим виграти, бо 2-му гравцю нема паличок і тому нема ходів.