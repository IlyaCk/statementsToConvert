{\it Це інтерактивна задача. Прочитайте умову повністю, щоб зрозуміти, як працювати з такими задачами.}

На прямокутному полi $N{\times}M$ клiтинок у лiвому верхньому кутку стоїть фішка. Ця~кутова клітинка гарантовано вільна (не~замінована); абсолютно кожна інша клітинка п{\it о}ля може бути хоч вільною, хоч замінованою. 
Гра полягає в тому, що два гравцi поперемiнно рухають згадану фішку на якусь кiлькiсть клiтинок праворуч або донизу (кожен гравець сам вирiшує, в якому з цих двох напрямків i на скільки клітинок рухати; не~можна ні~лишати фішку на мiсцi, ні~ставати нею в заміновану клітинку, ні~перестрибувати нею через заміновану клітинку).
Програє той, хто не~може нiкуди походити (і~знизу, і~праворуч або край п{\it о}ля, або міна). Відповідно, його суперник виграє.

Напишіть програму, яка інтерактивно гратиме за першого гравця. 

Ця задача є інтерактивною: 
Ваша програма не~отримає всіх вхідних даних на~початку,
а~отримуватиме по~мірі виконання доуточнення, 
що залежатимуть від попередніх дій Вашої програми. 
Тим~не~менш, {\it її перевірка буде
\underline{автоматичною}}. 
Тому, слід чітко дотримуватися формату спілкування з~програмою, яка грає роль суперника.


\Interaction

На початку, один раз, Ваша програма повинна прочитати поле гри.
Перший рядок містить два цілі числ{\it а} $N$ та~$M$, розділені одним пропуском (пробілом) --- спочатку кількість рядків, потім стовпчиків. Обидва ці значення у межах від~1 до~123.

Далі йдуть $N$ рядків, що задають мінне поле. Кожен з них містить рівно по~$M$ символів \texttt{.} (позначає вільну клітинку) та/або \texttt{*} (позначає заміновану клітинку). Ці~символи йдуть без роздільників, і кожен з цих $N$ рядків містить лише ці символи та переведення рядка наприкінці.

Потім вона повинна повторювати такий цикл:
\begin{enumerate}
\item
Якщо йти нема куди, вивести фразу "\texttt{I~lost...}" (без лапок, символ-у-символ згідно зразку) й завершити роботу.
\item
Вивести в окремому рядку свій хід, тобто спочатку або велику латинську~\texttt{D} (якщо хід униз), або велику латинську~\texttt{R} (якщо направо), потім пробіл, потім єдине ціле число~--- на~скільки клітинок переміститися. Ця~довжина переміщення повинна бути цілою, строго додатною, не~виводити за межі п{\it о}ля, не~приводити у клітинку з~міною і не~перестрибувати ні через одну клітинку з~міною.
\item
Якщо після цього програмі-суперниці нема куди йти, вивести фразу "\texttt{I~won!}" (без лапок, символ-у-символ згідно зразку) й завершити роботу.
\item
Прочитати хід програми-суперниці, у~форматі в~точності як у позаминулому пункті. Гарантовано, що цей хід відповідає всім вимогам, сформульованим у позаминулому пункті. Само собою, ця гарантія дійсна лише за умови, що Ваша програма правильно визначила, що гра ще~не~закінчилася.
\end{enumerate}
Все вищезгадане повинно повторюватися, доки фішка не~потрапить у клітинку, з~якої за~правилами не~можна вийти (тобто, доки якась із програм-гравців не~прогр{\it а}є).
Програма-суперниця не~виводить фраз "\texttt{I~won!}" / "\texttt{You~won...}" 
чи якихось їх аналогів.

Наполегливо рекомендується, щоб Ваша програма після кожного свого виведення 
робила дію {\tt flush(output)} (Pascal), 
вона~ж {\tt cout.flush()} (C++), 
вона~ж {\tt fflush(stdout)} (C), 
вона~ж {\tt sys.stdout.flush()} (Python),
вона~ж {\tt System.out.flush()} (Java).
Це~істотно зменшує ризик, 
що~проміжна відповідь <<загубиться>> десь по~дорозі, 
не~дійшовши до програми-суперниці.


\Examples

\begin{example}
\exmp{2 4
....
.**.
~
R 1}{~
~
~
R 2
~
D 1
I won!}%
\exmp{2 4
....
.**.}{~
~
~
D 1
I won!}%
\end{example}


\Note

Нібито порожні рядки між різними ходами у прикладі зроблені суто для того, щоб краще було видно, хто коли ходить; вводити/виводити їх не~треба.

Обидві наведені послідовності ходів є прикладами правильної гри.
Ваша програма не~зобов'язана при різних запусках для одного поля робити різні ходи. 
Але вона має таке право.
При цьому
автоматична перевірка не~шукатиме кращий чи гірший результат, а~просто оцінюватиме перший.


\Scoring

У~приблизно половині тестів Ваша програма матиме справу з ідеальною програмою-суперницею, яка не~робить помилок.
У~іншій приблизно половині~--- з різними програмами-суперницями, які грати не~вміють~--- тобто, роблять лише ходи, які дотримуються формальних вимог гри, 
але можуть вибирати не~найкращий з допустимих ходів, дотримуючись кожна своїх власних уявлень про те, як варто грати в цю гру.
Буде оцінюватися 
як уміння Вашої програми виграти там, де це гарантовано можливо,
так і вміння Вашої програми достойно, згідно правил гри, програти, 
так і вміння Вашої програми скористатися (теж згідно правил) помилками чи іншими неадекватностями програми-суперниці, якщо такі будуть.

За будь-яке порушення правил гри з~боку Вашої програми, відповідний тест оцінюватиметься як не~пройдений.



\Examples

