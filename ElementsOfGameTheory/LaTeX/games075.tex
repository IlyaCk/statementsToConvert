% \begin{problemAllDefault}{Гра на максимум суми (1/2/3) з випадковими втратами карток}
% \label{problem:cards-maxsum-1-2-3-random}

$N$ карток викладені в ряд зліва направо. На~кожній картці написане ціле число. Два гравці по черзі забирають картк\'{и}, причому забирати можна лише з\nolinebreak[3] правого боку, або~1~картку, або~2~картки, або~3~картки (але, звісно, не~більше карток, чим~їх~є). Закінчується гра, коли забрано всі картк\'{и} (поки хоч\nolinebreak[3] одна картка є, гравець зобов'язаний робити один із можливих ходів). 
Крім того, після кожного ходу кожного з гравців з\nolinebreak[3] імовірністю~10\% стається (отже, з\nolinebreak[3] імовірністю~90\% не~стається) випадкове віднесення вітром картки, яка щойно стала крайньою правою.
(Це~\emph{найголовніша} відмінність цієї задачі від задачі\nolinebreak[3] % \ref{problem:cards-maxsum-1-2-3}.)
«Гра на максимум суми (1/2/3)».)
Мета гри~--- отримати 
якнайбільшу 
суму (чисел, записаних на забраних картках).

Які матсподівання сум, що їх наберуть гравці, якщо обидва гратимуть якнайкраще, 
намагаючись
максимізувати кожен своє матсподівання?


\InputFile
У першому рядку вказано кількість карток $N$ ($1\dib{{\<}}N\dib{{\<}}1234$). У~другому рядку через пробіли задані $N$ цілих чисел (що~не~перевищують за~модулем~$10^3$; інакше кажучи, з\nolinebreak[2] проміжку $[-1000; +1000]$) --- значення, записані на картках.


\OutputFile
Виведіть в одному рядку через пропуск два дійсні числ\'{а}~--- 
матсподівання сум, що наберуть гравці (спочатку\nolinebreak[3] \mbox{1-й}, потім\nolinebreak[3] \mbox{2-й}).
Формат виведення може бути будь-яким зі стандартних (зокрема, байдуже, чи\nolinebreak[3] вивести\nolinebreak[3] \texttt{41.91}, чи\nolinebreak[3] \texttt{41.910000000}, чи\nolinebreak[3] \mbox{\texttt{4.191e+1}}), важливо лише забезпечити точність, вказану в~«Оцінюванні».

\Examples

\noindent
\hspace*{-1.25em}\begin{exampleSimple}{15.125em}{7.875em}
\exmp{7
11 11 11 11 11 11 11}{41.91 33}%
\exmp{8
3 2 999 4 6 7 -5 1}{810.17 116.2}%
\exmp{9
23 -17 2 -13 5 3 11 -7 19}{26.7633 2.215}%
\end{exampleSimple}

\Notes
У першому тесті, 
числа додатні й 
однакові, тому гравцям вигідно забирати якнайбільше 
карток: 
спочатку \mbox{1-й} забирає три штуки;
потім вітер чи\nolinebreak[3] то\nolinebreak[3] відносить одну картку, чи\nolinebreak[3] то\nolinebreak[3] ні, й карток лишається чи\nolinebreak[3] то\nolinebreak[3] три штуки, чи\nolinebreak[3] то\nolinebreak[3] чотири;
в~будь-якому разі, \mbox{2-й} забирає три штуки;
залежно від вітру, повторний хід \mbox{1-го}\nolinebreak[2] гравця може бути, а\nolinebreak[3] може й не~бути: він буде з\nolinebreak[3] ймовірністю ${0{,}9\,{\cdot}\,0{,}9}\dib{{=}}0{,}81$,
якщо вітер не~віднесе картки ні\nolinebreak[3] після ходу \mbox{1-го}\nolinebreak[2] гравця, ні\nolinebreak[3] після ходу\nolinebreak[2] \mbox{2-го}; в~цьому разі, лишиться одна картка, яку вигідно забрати. Тобто, \mbox{1-й} з\nolinebreak[3] імовірністю~$81\%$ набере\nolinebreak[2] $33\dib{{+}}11\dibbb{{=}}44$, а\nolinebreak[3] з\nolinebreak[3] імовірністю\nolinebreak[2] ${100\%\,{-}81\%}\dib{{=}}19\%$ набере~$33$; матсподівання дорівнює ${0{,}81\cdot44}\dib{{+}}{0{,}19\cdot33}\dibbb{{=}}35{,}64\dib{{+}}6{,}27\dibbb{{=}}41{,}91$.
Для цих вхідних даних, вітер не~може завадити \mbox{2-му} гравцеві взяти три картки по\nolinebreak[3] 11\nolinebreak[3] кожна, й\nolinebreak[3] результат\nolinebreak[3] \mbox{2-го} завжди ${11{+}11{+}11}\dib{{=}}33$. 
% (Попри\nolinebreak[3] те, що 
(Від\nolinebreak[2] вітру залежить, \emph{які} це будуть три картки; але на кожній з\nolinebreak[3] них % все'дно
однакове~11, тому на\nolinebreak[3] результат не~впливає.)


У др\'{у}гому прикладі, як\nolinebreak[3] і\nolinebreak[2] в\nolinebreak[3] задачі\nolinebreak[3] 
% \ref{problem:cards-maxsum-1-2-3}, 
«Гра на максимум суми (1/2/3)»
значення\nolinebreak[3] 999 усе\nolinebreak[3] ще настільки перевищує всю решту, що все\nolinebreak[3] ще варто дбати в першу чергу про те, щоб узяти це найбільше число.
Через те, що тепер є випадкові впливи вітру, перший хід \textsl{«узяти дві картки 1 та~\mbox{(–5)}»} вже не~гарантує цього, але все\nolinebreak[3] ще дає найбільшу ймовірність ${0{,}9\,{\cdot}\,0{,}9}\dib{{=}}0{,}81$,
якщо вітер не~віднесе картки ні\nolinebreak[3] після ходу \mbox{1-го}\nolinebreak[2] гравця, ні\nolinebreak[3] після ходу\nolinebreak[2] \mbox{2-го}. 

Як\nolinebreak[2] від\nolinebreak[2] ще\nolinebreak[3] детальніших пояснень відповіді др\'{у}\-гого тесту, так і від будь-яких коментарів щодо третього тесту утримаюся.

\Scoring
Потестове (кожен тест перевіряють і оцінюють неза\-леж\-но від решти).
Тест зараховують, коли абсолютна або відносна похибка (хоча~б одна з~двох) не~перевищує~$10^{-9}$.

% \end{problemAllDefault}
