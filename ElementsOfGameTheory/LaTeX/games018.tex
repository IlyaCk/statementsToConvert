{\it Це інтерактивна задача. Прочитайте умовну повністю, щоб зрозуміти, як працювати з такими задачами.}

Є стрічка шириною в одну клітинку і довжиною в $N$ клітинок.
Двоє грають у таку гру.
Кожен з гравців на кожному своєму ході може закреслити кілька клітинок. 
Якщо закреслення відбувається скраю смужки, або безпосередньо поруч з уже закресленою клітинкою,
то закреслити можна або~1, або~2 клітинки підряд.
Якщо закреслення відбувається не~згідно попереднього речення, а~десь всер{\it е}дині досі суцільного фрагмента стрічки (так, щоб по обидва боки від закресленого були незакреслені фрагменти), то закреслити можна або~2, або~4 клітинки підряд.
Ніяких інших варіантів ходу нема. 
Ходять гравці по черзі, пропускати хід не~можна.
Виграє той, хто закреслює останню клітинку (можливо, разом із ще кількома).

Напишіть програму, яка інтерактивно гратиме за першого гравця. 

Ця задача є інтерактивною: 
Ваша програма не~отримає всіх вхідних даних на~початку,
а~отримуватиме по~мірі виконання доуточнення, 
що залежатимуть від попередніх дій Вашої програми. 
Тим~не~менш, {\it її перевірка буде
\underline{автоматичною}}. 
Тому, слід чітко дотримуватися формату спілкування з~програмою, яка грає роль суперника.


{\it Пару прикладів суто для пояснення правил гри.}

Повний перелік усіх можливих ходів для позиції <<суцільна стрічка з 4-х клітинок>>:

% \includegraphics{https://static.eolymp.com/content/hn/hng3u5gvt11qt3e08hns7i37ag.png}

З~кожного з країв можна закреслити 1 або 2 клітинки; 
посередині можна закреслити 2~клітинки, 
але не~4, бо <<всер{\it е}дині>> вимагає, щоб з кожного з~країв хоч~щось лишилося.
Всього є 5 варіантів ходу.


Повний перелік усіх можливих ходів для позиції <<10~клітинок, 3-тя та 4-та вже закреслені раніше>>:

% \includegraphics{https://static.eolymp.com/content/9k/9k59vqipup6ql3fs15o2ktc6g0.png}

(Суцільно-чорні --- закреслені раніше; заштриховані --- закреслені на поточному ході.)
У~лівому фрагменті можна закреслити будь-яку одну крайню, або двома різними способами (дві крайні зліва чи дві крайні справа) закреслити обидві.
Якщо креслити з~країв правого фрагмента, виходить чотири варіанти: закреслити 
одну зліва, або
дві зліва, або
одну справа, або
дві справа.
Якщо креслити всер{\it е}дині правого фрагмента, є один варіант закреслити зразу чотири,
і три варіанти закреслити дві.
Всього є 11 варіантів ходу.


\Interaction

На початку, один раз, Ваша програма повинна прочитати початкову позицію гри, яка задається єдиним числом~$N$ ($1\leqslant N\leqslant 2000$) --- кількість клітинок у єдиній неперервній смужці.

Потім Ваша програма повинна повторювати такий цикл:
\begin{enumerate}
\item
Вивести два числ{\it а}, розділені пропуском, у окремому рядку~--- свій хід, тобто починаючи з якої клітинки вона закреслює, і скільки клітинок, згідно з раніше описаними правилами.
Клітинки занумеровані з~одиниці (1, 2,~\dots,~$N$), й ці номери лишаються закріпленими за клітинками протягом усієї гри (хоч би як не викреслювали інші клітинки).

\item
Якщо при цьому відбулося викреслення останньої клітинки, 
вивести окремим рядком фразу "\texttt{I~won!}"
(без лапок, символ-у-символ згідно зразку) й завершити роботу.
\item
Інакше, прочитати хід програми-суперниці, в~такому~ж форматі (починаючи з якої клітинки вона закреслює, і скільки клітинок; ці числа подані в одному рядку, розділені одинарним пробілом).
Гарантовано, що хід допустимий: відповідає описаним вище правилам, і жодна з нині закреслюваних клітинок не~була закреслена раніше.
Само собою, ця гарантія дійсна лише за умови, що Ваша програма правильно визначила, що гра ще~не~закінчилася.
\item
Якщо при цьому відбулося викреслення останньої клітинки, 
вивести окремим рядком фразу "\texttt{You~won...}"
(без~лапок, символ-у-символ згідно зразку) й завершити роботу.
\end{enumerate}
Все вищезгадане повинно повторюватися, доки~не~будуть викреслені всі клітинки 
(тобто, доки якась із програм-гравців не~в{\it и}грає).
Програма-суперниця не~виводить фраз "\texttt{I~won!}" / "\texttt{You~won...}"
чи якихось їх аналогів.

Наполегливо рекомендується, щоб Ваша програма після кожного свого виведення 
робила дію \verb"flush(output)" (Pascal), 
вона~ж \verb"cout.flush()" (C++), 
вона~ж \verb"fflush(stdout)" (C), 
вона~ж \verb"sys.stdout.flush()" (Python),
вона~ж \verb"System.out.flush()" (Java).
Це~істотно зменшує ризик, 
що~проміжна відповідь <<загубиться>> десь по~дорозі, 
не~дійшовши до програми-суперниці.




\Examples

\begin{example}
\exmp{7

4 1

1 1
}{
2 2

6 2

5 1
I won!}
\end{example}


\Note

Нібито порожні рядки між різними ходами у прикладі зроблені суто для того, щоб краще було видно, хто коли ходить; вводити/виводити їх не~треба.

Загальний хід гри з прикладу можна прокоментувати так: 

\begin{tabular}{lc|c|c|c|c|c|c|c|}
коментар
&
хід
&
\multicolumn{7}{c}{стан стрічки}
\\ \hline
\\ \hline
до початку гри жодна з 7 клітинок не закреслена
&  
& . & . & . & . & . & . & . 
\\ \hline
\\ \hline
Ваша програма закреслює 2-у та 3-ю клітинки
&
2 2
& . & x & x & . & . & . & .  
\\ \hline
\\ \hline
Програма-суперниця закреслює 4-у клітинку
&
4 1
& . & x & x & o & . & . & .  
\\ \hline
\\ \hline
Ваша програма закреслює 6-у та 7-у клітинки
&
6 2
& . & x & x & o & . & x & x  
\\ \hline
\\ \hline
Програма-суперниця закреслює 1-у клітинки
&
1 1
& o & x & x & o & . & x & x  
\\ \hline
\\ \hline
Ваша програма закреслює 5-у клітинку й виграє
&
5 1
& o & x & x & o & x & x & x  

\end{tabular}

\Scoring
У~більшості тестів Ваша програма матиме справу з ідеальною програмою-суперницею, яка не~робить помилок.
Однак, невелика кількість тестів передбачатиме також і гру з різними програмами-суперницями, які грати не~вміють~---  роблять ходи, які дотримуються формальних правил гри, 
але можуть вибирати не~найкращий з допустимих ходів, дотримуючись кожна своїх власних уявлень про те, як краще грати в цю гру.
Буде оцінюватися 
як уміння Вашої програми виграти там, де це гарантовано можливо,
так і вміння Вашої програми достойно, згідно правил гри, програти, 
так і вміння Вашої програми скористатися (теж згідно правил) помилками чи іншими неадекватностями програми-суперниці, якщо такі будуть.
Уміння перехоплювати ініціативу в неідеальної суперниці оцінюватиметься лише на досить великих тестах.

За будь-яке порушення правил гри з~боку Вашої програми, відповідний тест оцінюватиметься як не~пройдений.

