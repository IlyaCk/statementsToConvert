{\it Це інтерактивна задача. Прочитайте умовну повністю, щоб зрозуміти, як працювати з такими задачами.}

Є одна купка, яка спочатку містить $N$ паличок.
Двоє грають у таку гру.
Кожен з гравців на кожному своєму ході може забрати з купки або~1, або~2, або~3 палички (але, звісно, не~більше, чим їх є в купці).
Ніяких інших варіантів ходу нема. 
Ходять гравці по черзі, пропускати хід не~можна.
Виграє той, хто забирає останню паличку (можливо, разом із ще однією або ще двома).

Напишіть програму, яка інтерактивно гратиме за першого гравця.

Ця задача є інтерактивною: 
Ваша програма не~отримає всіх вхідних даних на~початку,
а~отримуватиме по~мірі виконання доуточнення, 
що залежатимуть від попередніх дій Вашої програми. 
Тим~не~менш, \it{її перевірка буде
\underline{автоматичною}}. 
Тому, слід чітко дотримуватися формату спілкування з~програмою, яка грає роль суперника.


\Interaction 

На початку, один раз, Ваша програма повинна прочитати одне ціле число в окремому рядку~--- початкову кількість паличок $N$ ($1{\leqslant}N{\leqslant}12345$).
Потім вона повинна повторювати такий цикл:
\begin{enumerate}
\item
Вивести єдине число в окремому рядку~--- свій хід, тобто кількість паличок, які вона зараз забирає з купки.
Це~повинно бути ціле число від~1 до~3, причому не~більше за поточну кількість паличок у купці. 
\item
Якщо після цього купка стає порожньою, 
% вивести рядок із фразою ``\texttt{I~won!}''
вивести окремим рядком фразу ``\texttt{I~won!}''
(без лапок, символ-у-символ згідно зразку) й завершити роботу.
\item
Інакше, прочитати хід програми-суперниці, тобто кількість паличок, які вона зараз забирає з~купки (єдине ціле число, в окремому рядку).
Гарантовано, що хід допустимий (є цілим числом від~1 до~3 і не~перевищує поточного залишку паличок у купці).
Само собою, ця гарантія дійсна лише за умови, що Ваша програма правильно визначила, що гра ще~не~закінчилася.
\item
Якщо після цього купка стає порожньою, 
% вивести рядок із фразою ``\texttt{You~won...}''
вивести % окремим рядком 
фразу ``\texttt{You~won...}''
(без лапок, символ-у-символ згідно зразку) й завершити роботу.
\end{enumerate}
Все вищезгадане повинно повторюватися, доки~не~будуть забрані всі палички (тобто, доки якась із програм-гравців не~в\it{и}грає).
Програма-суперниця не~виводить фраз ``\texttt{I~won!}'' / ``\texttt{You~won...}'' 
чи якихось їх аналогів.

Наполегливо рекомендується, щоб Ваша програма після кожного свого виведення 
робила дію \verb"flush(output)" (Pascal), 
вона~ж \verb"cout.flush()" (C++), 
вона~ж \verb"fflush(stdout)" (C), 
вона~ж \verb"sys.stdout.flush()" (Python),
вона~ж \verb"System.out.flush()" (Java).
Це~істотно зменшує ризик, 
що~проміжна відповідь <<загубиться>> десь по~дорозі, 
не~дійшовши до програми-суперниці.


\Example

\begin{example}
\exmp{5

2

}{
1

2
I won!}
\end{example}

\Note
Нібито порожні рядки між різними ходами у прикладі зроблені суто для того, щоб краще було видно, хто коли ходить; вводити/виводити їх не~треба.

Загальний хід гри з прикладу можна прокоментувати так. У~купці спочатку 5 паличок. Ваша програма забирає одну, 
лишається 4; програма-суперниця забирає дві, лишається 2;
Ваша програма забирає обидві, повідомляє про свій виграш і завершує роботу.


\Scoring
У~приблизно половині тестів Ваша програма матиме справу з ідеальною програмою-суперницею, яка не~робить помилок.
У~іншій приблизно половині~--- з різними програмами-суперницями, які грати не~вміють~--- тобто, роблять лише ходи, які дотримуються формальних вимог <<забирати лише від~1 до~3 паличок>> та <<забирати не~більше паличок, ніж реально~є у~купці>>, 
але можуть вибирати не~найкращий з допустимих ходів, дотримуючись кожна своїх власних уявлень про те, як варто грати в цю гру.
Буде оцінюватися 
як уміння Вашої програми виграти там, де це гарантовано можливо,
так і вміння Вашої програми достойно, згідно правил гри, програти, 
так і вміння Вашої програми скористатися (теж згідно правил) помилками чи іншими неадекватностями програми-суперниці, якщо такі будуть.

Тобто, щоб отримати абсолютно повні бали, Ваша програма має врахувати все щойно перелічене. 
Конкретно в цій задачі, можливі часткові бали як у смислі «одні тести успішно пройдені, інші~ні», так і в смислі часткових балів за окремо взятий тест. Часткові бали за окремий тест можливі, коли програма грала з дотриманням усіх правил і вчасно завершила гру, але програла, коли можна було виграти, та/або вивела неправильне повідомлення про завершення гри чи не~вивела його взагалі, тощо.

За будь-яке порушення правил гри з~боку Вашої програми, відповідний тест оцінюватиметься як не~пройдений.


