Напишіть програму, яка застосує до вказаних матриць виграшів біматричної гри двох гравців послідовне виключення строго домінованих стратегій. Тобто, поки хоча б у одного з гравців є стратегії (хоча~б одна, можна більше), строго доміновані іншими його ж стратегіями, доміновані стратегії слід вилучати; цю дію треба виконати для обох гравців; може виявлятися (а може й не виявлятися), що, внаслідок вилучення стратегій одного з гравців, для іншого гравця деякі зі стратегій, які досі не~були строго домінованими, стають такими; якщо таке відбувається, їх теж слід вилучати. Див. також задачу «Домінування стратегій біматричної гри» та примітки наприкінці цієї умови.

\InputFile
У першому рядку через пропуск (пробіл) задано кількість стратегій першого гравця $N$ ($2{{\leqslant}}N{{\leqslant}}12$) та кількість стратегій другого гравця $M$ ($2{{\leqslant}}M{{\leqslant}}12$).
Далі йде один порожній рядок.
Наступні $N$~рядків містять платіжну матрицю виграшів першого гравця, тобто кожен з цих рядків містить рівно $M$ чисел, розділених одинарними пропусками (пробілами), причому {$j$-е} число {$i$-го} рядка являє собою виграш першого гравця у випадку, якщо перший гравець застосує стратегію~№$i$, а~другий стратегію~№$j$. Далі йде один порожній рядок, після якого в аналогічному форматі задано матрицю виграшів другого гравця: теж $N$~рядків по $M$ чисел, де {$j$-е} число {$i$-го} рядка являє собою виграш другого гравця у випадку, якщо перший гравець застосує стратегію~№$i$, а~другий стратегію~№$j$. Всі значення елементів матриць є цілими числами, що не~перевищують за модулем (абсолютною величиною)~999.

\OutputFile
Ваша програма повинна вивести <<решти>> матриць виграшів, що лишаються після викреслень, у такому вигляді: спочатку <<решту>> матриці виграшів прешого гравця, потім др\'{у}гого. Кожна з цих <<решт>> може бути утворена таким чином: взята відповідна матриця зі вхідних даних, зліва дописані номери стратегій першого гравця \texttt{a1}, \texttt{a2},~\dots, а згори номери стратегій другого гравця \texttt{b1}, \texttt{b2},~\dots, після чого відбуваються власне всі викреслення.
У~наведених прикладах результати матриці відформатовані так, щоб створювати матриці з акуратними стовпчиками, але при автоматичній перевірці це не~перевірятиметься (досить, щоб всередині кожного з рядків відповіді утворювалася та сама послідовність позначок стратегій та чисел).

\Examples

\begin{example}
\exmp{3 2

1 2
4 8
6 4

11 12
21 22
31 32}{
    b2
a2  8

    b2
a2  22}
\exmp{3 2

1 2
4 8
1 2

11 12
91 22
31 32}{
    b1
a2  4

    b1
a2  91}
\exmp{3 2

1 2
4 8
6 4

11 12
29 28
31 32}{
    b1 b2
a2  4  8
a3  6  4

    b1 b2
a2  29 28
a3  31 32}%
\end{example}

\Note
Перший тест передбачає, наприклад, такі перетворення:

% % \includegraphics{https://eolympusercontent.com/images/7nau3ifps53gn1q4oq5bipn9oc.png}

(Тут і далі, в~кожній комірці, перше число --- виграш першого гравця, друге --- виграш другого.
Слово «наприклад» вжите в тому смислі, що існують також інші порядки викреслювань, які дають такий самий результат.)

Другий тест передбачає, наприклад, такі перетворення:

% % \includegraphics[xscale=0.5]{https://eolympusercontent.com/images/jd5k4pi6rh64p9lodr8kphva3g.png}

Перше викреслення правильне, бо можна, щоб і~\texttt{a1}, і~\texttt{a3} одночасно були строго домінованими;
тут неважливо, що вони (для 1-го гравця) однакові, досить і того, що кожну з них строго домінує~\texttt{a2}.

У третьому тесті вдається вилучити лише стратегію~\texttt{a1}.
