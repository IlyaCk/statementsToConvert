% \begin{problemAllDefault}{Гра на максимум суми (L/R, два числа на картці)~— 2}
% \label{problem:cards-maxsum-L-R-two-values-2}

Загальні правила гри, нарахування виграшу та формат вхідних даних 
такі\nolinebreak[3] ж, як\nolinebreak[2] у\nolinebreak[3]
% задачі~\ref{problem:cards-maxsum-L-R-two-values-1}.
задачі «Гра на максимум суми (L/R, два числа на картці)~— 1».

Якими будуть результати гри при правильній грі обох гравців?
Яку максимальну суму гарантовано зможе набрати перший гравець?

\InputFile
У першому рядку вказано кількість карток $N$ ($1\dib{{\<}}N\dib{{\<}}30$). Далі йдуть $N$ рядків, кожен з яких містить записані через пропуск (пробіл) рівно два натуральних числа, записані на відповідній картці, спочатку верхнє цинанове, потім нижнє магентове. Гарантовано, що всі $2N$ чисел різні, всі є цілими степенями двійки і~перебувають у~межах від $2^0$ до $2^{60}$, обидві межі включно.


\OutputFile
Виведіть в першому рядку через пропуск два цілі числа~--- результати гри при правильній грі обох гравців, спочатку суму цианових чисел, яку набере \mbox{1-й} гравець, потім суму магентових чисел, яку набере \mbox{2-й} гравець.
Потім виведіть у др\'{у}гому рядку одне ціле число~--- максимальну суму, яку гарантовано зможе набрати \mbox{1-й} гравець, хоч би як не~грав \mbox{2-й}.

\Examples

\begin{exampleSimple}{5em}{6.5em}
\exmp{4
1024 8
256 512
4 2
16 65536}{1280 65538
1040}%
\exmp{4
1 2
8 64
1024 256
32 16}{1025 80
1025}%
\exmp{4
1 2
8 256
1024 64
32 16}{1056 258
1025}%
\end{exampleSimple}

% \end{problemAllDefault}
