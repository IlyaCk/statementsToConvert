% \begin{problemAllDefault}{Баше з додатковими ходами і циклами}
% \label{problem:Bachet-with-looping-moves}

Двоє грають у таку гру. Спочатку є $N$ ($1\dib{{\<}}N\dib{{\<}}10^5$) паличок. На кожному ході кожен гравець може забирати або одну, або дві, або три палички, але не~більше, чим їх є всього. Описані досі ходи (тотожні класичній 
грі Баше) 
% грі Баше з розд.\nolinebreak[3] \ref{sec:Bachet-game-theory}) 
будемо називати \emph{традиційними}. Крім них, існують \emph{спеціальні} ходи: задається кілька пар цілих чисел 
$(a_1, b_1)$,
$(a_2, b_2)$,\nolinebreak[3] \dots,
$(a_t, b_t)$,
які означають:
якщо кількість паличок дорівнює якомусь із $a_i$, то гравець може замінити $a_i$ паличок на~$b_i$. Якщо гравець може зробити спеціальний хід, він сам вирішує, чи робити його, чи якийсь із традиційних, але зробити якийсь один хід треба. Ситуація, коли відразу кілька пар мають однакове значення $a_i$, можлива; в\nolinebreak[3] такій ситуації гравець теж сам вирішує, яким із доступних ходів (спеціальних чи традиційних) скористатися. В\nolinebreak[3] будь-якому разі, після кожного ходу черга ходити переходить до іншого гравця, пропускати хід не можна. (Щоправда, користуватися ходом, де ${b_i\,{=}\,a_i}$, можна, і це в\nolinebreak[3] деякому смислі відповідає пропуску ходу; але ж це можливо, лише якщо поточна кількість паличок якраз рівна таким $b_i$\nolinebreak[3] та~$a_i$, для яких існує такий спеціальний хід.) Закінчується гра тоді, коли лишається 0~паличок (це може статися хоч після традиційного ходу, хоч, якщо існує пара, де ${b_i\,{=}\,0}$, після спеціального), і той, хто походив у\nolinebreak[3] цю\nolinebreak[3] позицію, виграв, а той, кому така позиція дісталася, програв. Однак, ця\nolinebreak[3] гра може й не~завершуватися,\linebreak[2] бо завдяки спеціальним ходам з ${b_i\,{>}\,a_i}$ кількість паличок може так ніколи\nolinebreak[2] й\nolinebreak[2] не\nolinebreak[3] стати~0. Такий результат кожен з гравців розцінює як гірший, чим виграш, але кращий, чим програш. 

Напишіть програму, яка визначатиме, хто виграє при ідеальній грі обох гравців. Щоб відповідь не~так\nolinebreak[3] легко було вгадати, задачу слід розв'язати, при одній і тій самій сукупності пар 
$(a_1, b_1)$,
$(a_2, b_2)$,\nolinebreak[3] \dots,
$(a_t, b_t)$,
для різних початкових кількостей паличок.

\InputFile
В першому рядку записане єдине ціле число~$t$
($0\dib{{\<}}t\dib{{\<}}12345$), 
яке задає кількість пар, що утворюють спеціальні ходи. Кожен з подальших $t$ рядків задає один спеціальний хід, у\nolinebreak[3] вигляді $a_i$~$b_i$ (два числ\'{а}, розділені пробілом). Наступний (\mbox{${(t{+}2})$-й}) рядок містить єдине ціле число~$k$
($1\dib{{\<}}k\dib{{\<}}12345$)\nolinebreak[3] задає кількість варіантів початкової кількості паличок, а ще наступний (\mbox{${(t{+}3})$-й}) рядок містить $k$ натуральних чисел $N_1$,\nolinebreak[3] $N_2$,~…,\nolinebreak[3] $N_k$ (кожне в межах $1\dib{{\<}}N_i\dib{{\<}}10^5$)\nolinebreak[3] – різні початкові кількості паличок, для яких слід розв'язати задачу. 

\OutputFile
Виведіть у один рядок без пропусків рівно $k$ великих латинських букв \texttt{W} та/або \texttt{L} та/або~\texttt{D}, де \texttt{W} позначає, що при відповідній початковій кількості паличок гарантувати собі виграш може \mbox{1-й} гравець, \texttt{L}\nolinebreak[3] – що\nolinebreak[3] \mbox{2-й}, а\nolinebreak[2] \texttt{D}\nolinebreak[3] – що  жоден з гравців не~може гарантувати собі виграшу, але \mbox{1-й} гравець може гарантувати, що або гра триватиме нескін\-ч\'{е}\-нно довго, або якщо \mbox{2-й} поступиться, то виграє \mbox{1-й}.


% \end{problemAllDefault}



% \end{multicols}

\Examples

\begin{exampleSimple}{15.75em}{7em}
\exmp{2
1 1
8 8
12
1 2 3 4 5 6 7 8 9 10 11 12}{WWWLWWWDDDDD}%
\end{exampleSimple}

\begin{examplewideSimple}
\hspace*{-3em}\exmp{12
3 5
5 0
8 9
11 4
14 4
12 0
13 0
18 9
19 9
13 18
18 13
1 4
22
1 2 3 4 5 6 7 8 9 10 11 12 13 14 15 16 17 18 19 20 21 22}{WWWLWWWDDDWWWWLWWWDDDDD}%
\end{examplewideSimple}

% \begin{multicols}{2}

\Notes
У першому прикладі додаткові ходи дозволяють не~забирати палички, коли їх у купці або~1, або~8.
Якщо в купці 1~паличка, її вигідніше забрати й виграти.
Якщо в купці 8~паличок, будь-який зі традиційних ходів веде до програшу, й вигідніше нічого не~забирати, що потім повторюється обома гравцями до нескін\-ч\'{е}\-нно\-сті.
Й~так виходить, що з усіх подальших позицій (9,~10,~\dots\nolinebreak[3] палчик) теж вигідніше якось (за один хід чи за кілька) прийти до позиції <<8~паличок>> й повторювати її до нескін\-ч\'{е}\-нно\-сті.

У другому прикладі все набагато складніше.