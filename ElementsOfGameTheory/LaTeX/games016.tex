Одного разу хлопець, що цікавиться олімпіадними задачами з програмування, пішов
на прогулянку до лісу. Ходячи поміж деревами лісу, він зустрів Злого Мішку, лісового
звіра, що не~любить, коли хтось заходить до його лісу. Мішка хотів принести хлопчика
в жертву, але дізнавшись, що хлопчик розуміється в програмуванні, вирішив зіграти з
ним у гру ``Лісові Шахи''.

Гра має наступні правила: на дошці розміром $3\times N$ у першому рядку дошки
знаходяться $N$ чорних пішаків Злого Мішки, у третьому рядку знаходяться $N$ білих
пішаків хлопчика, відповідно другий рядок пустий.

Злий Мішка та хлопчик ходять по черзі, починає хлопчик. На~кожному кроці гравець
обирає пішак, яким або робить хід, або б'є ворожого пішака. Відповідно до правил
пішаки ходять на одну клітину вперед, тобто якщо пішак білий, то він може піти з
третього рядка на другий, а потім з другого на перший. Якщо пішак чорний, то все
навпаки, він може піти з першого рядка на другий, а потім з другого на третій. Звісно,
та клітинка, на яку ходить пішак, має бути пустою. Пішак б'є фігури по діагоналі на
одну клітинку. В~грі ``Лісові Шахи'' бити фігури є обов'язковим, тобто якщо можна бити
фігуру, то її треба обов'язково побити на цьому кроці! Програє той, хто не~зможе
зробити хід. Ваша задача допомогти хлопчику вказати всі можливі перші ходи, що
100\% призведуть до його виграшу, якщо він буде дотримуватись оптимальної
стратеґії.

\InputFile
Програма читає з клавіатури одне число $N$ ($1\leqslant N\leqslant 1000$).

\OutputFile
Програма виводить на~екран спочатку число~$K$, що є кількістю перших ходів хлопчика,
що 100\% приведуть його до виграшу, якщо він буде дотримуватись оптимальної
стратеґії. Якщо таких немає, тоді треба вивести~0. Далі йдуть $K$ чисел у порядку
зростання, що показують, яким за~номером білим пішаком повинен піти хлопчик.

Білі пішаки нумеруються від 1 до~$N$ зліва направо. Всі числа виводяться через
пропуск.


\Examples
\begin{example}
\exmp{3}{1 2}
\exmp{2}{2 1 2}
\end{example}

