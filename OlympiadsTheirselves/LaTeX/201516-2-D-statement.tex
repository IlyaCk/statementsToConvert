% % % \begin{problem}{Кількість дільників на проміжку}{\stdinOrInputTxt}{\stdoutOrOutputTxt}{1~сек}{4 мегабайти}
\begin{problemAllDefault}{Кількість дільників на проміжку}

Напишіть програму, яка знайде суму кількостей дільників усіх чисел у проміжку від\nolinebreak[2] $A$ до~$B$ (обидві межі включно, $A\<B$).

\InputFile У єдиному рядку через пробіл задані два натуральні числа $A$ та~$B$, які являють собою межі проміжку.

\OutputFile Виведіть у одному рядку єдине число --- суму кількостей дільників усіх чисел проміжку.

\ifAfour
\vspace{-0.5\baselineskip}
\fi

\Example
\begin{exampleSimple}{5em}{3em}%
\exmp{119 122}{27}%
\end{exampleSimple}

\Note
Число\nolinebreak[3] 119 має 4\nolinebreak[3] дільники (1, 7, 17, 119); 
число\nolinebreak[3] 120 має 16\nolinebreak[3] дільників (1, 2, 3, 4, 5, 6, 8, 10, 12, 15, 20, 24, 30, 40, 60, 120); 
число\nolinebreak[3] 121 має 3\nolinebreak[3] дільника (1, 11, 121); 
число\nolinebreak[3] 122 має 4\nolinebreak[3] дільники (1, 2, 61, 122). 
Звідси відповідь \mbox{$4\,{+}\,16\,{+}\,3\,{+}\,4 = 27$}.

\Scoring
40\%~балів припадає на тести, в яких $1\,{\<}\,A\,{\<}\,B\,{\<}\,1000$.
Ще~20\%~балів припадає на тести, в~яких $1\,{\<}\,A\,{\<}\,B\,{\<}\,10^7$.
Ще~10\%~балів припадає на тести, в~яких $10^8\,{\<}\,B\,{\<}\,10^9$, але ${B\,{-}\,100}\,{\<}\,A\,{\<}\,B$.
Решта\nolinebreak[1] 30\%~балів припадає на тести, де виконуються обмеження $1\,{\<}\,A\,{\<}\,B\,{\<}\,10^{12}$ і не~виконуються обмеження попередніх блоків.
\ifAfour\else\par\fi
Писати треба одну програму, а~не~різні програми для різних випадків; єдина мета цього переліку різних блоків обмежень\nolinebreak[3] --- дати уявлення про\nolinebreak[3] те, скільки балів можна отримати, якщо розв’язати задачу правильно, але~не~ефективно.


\end{problemAllDefault}
% % % \end{problem}
