\begin{problem}{Відстань між мінімумом та максимумом--2}{\stdinOrInputTxt}{\stdoutOrOutputTxt}{2 сек}{64 мегабайти}

Напишіть програму, яка за заданим масивом цілих чисел знайде місце мінімального її елемента, максимального її елемента, та відстань (кількість проміжних елементів) між ними. У~випадку, якщо масив містить однакові мінімальні та\slash{}або однакові максимальні елементи на різних позиціях, слід вибрати такий з мінімальних та такий з максимальних елементів, щоб шукана відстань виявилася мінімальною (див.~також приклади~3~та~4).

\InputFile
Перший рядок містить єдине число~$N$ ($2\dib{{\<}}N\dib{{\<}}10^6$)\nolinebreak[3] --- кількість елементів масиву. Др\'{у}гий рядок містить (розділені пробілами) ч\'{и}сла--елементи масиву; значення цих чисел--елементів цілі, від~1 до~$10^6$.

\OutputFile
Виведіть єдине ціле число\nolinebreak[3] --- відстань (кількість проміжних елементів) між найближчими мінімальним і максимальним елементами.

\Examples


\vspace{-1.125\baselineskip}

\makeTableLongtrue
\noindent\begin{exampleThreeWithSpecNameColTwoAndLineStretch}{7.5em}{5em}{\ifAfour 11cm\else 10cm\fi}{Результати}{Примітки}{\ifAfour -2pt\else -5pt\fi}{-2pt}{0.625}
\exmp{8
3 9 7 5 2 8 6 4}{2}{\noindent\ifAfour\begin{minipage}[t]{11cm}\else\begin{minipage}[t]{10cm}\fi
\setstretch{0.875}
Між елементами 2 та 9 два проміжні елементи 5 та~7
\end{minipage}}%
\exmp{8
4 3 2 1 8 7 6 5}{0}{\noindent\ifAfour\begin{minipage}[t]{11cm}\else\begin{minipage}[t]{10cm}\fi
\setstretch{0.875}
Між елементами 1 та 8 нема проміжних елементів
\end{minipage}}%
\exmp{8
3 1 4 1 5 9 2 6}{1}{\noindent\ifAfour\begin{minipage}[t]{11cm}\else\begin{minipage}[t]{10cm}\fi
\setstretch{0.875}
Максимальний елемент (дев'ятка) єдиний; найближчою до єдиної дев'ятки є остання з мінімальних елементів (одиниць); між ними один проміжний елемент~5
\end{minipage}}%
\exmp{8
1 1 1 1 1 1 1 1}{0}{\noindent\ifAfour\begin{minipage}[t]{11cm}\else\begin{minipage}[t]{10cm}\fi
\setstretch{0.875}
Оскільки всі елементи однакові, кожен з них мінімальний і максимальний; між елементом і ним же самим нема проміжних 
\end{minipage}}%
\end{exampleThreeWithSpecNameColTwoAndLineStretch}

\vspace{-0.5\baselineskip}

% % % \noindent\hspace*{-1em}\begin{exampleSimpleThree}{7.5em}{5em}{10cm}{Примітки}
% % % \exmp{8
% % % 3 9 7 5 2 8 6 4}{2}{\noindent\begin{minipage}[t]{10cm}
% % % Між елементами 2 та 9 два проміжні елементи 5 та~7
% % % \end{minipage}}%
% % % \exmp{8
% % % 4 3 2 1 8 7 6 5}{0}{\noindent\begin{minipage}[t]{10cm}
% % % Між елементами 1 та 8 нема проміжних елементів
% % % \end{minipage}}%
% % % \exmp{8
% % % 3 1 4 1 5 9 2 6}{1}{\noindent\begin{minipage}[t]{10cm}
% % % Максимальний елемент (дев'ятка) єдиний; найближчою до єдиної дев'ятки є остання з мінімальних елементів (одиниць); між ними один проміжний елемент~5
% % % \end{minipage}}%
% % % \exmp{8
% % % 1 1 1 1 1 1 1 1}{0}{\noindent\begin{minipage}[t]{10cm}
% % % Оскільки всі елементи однакові, кожен з них мінімальний і максимальний; між елементом і ним же самим нема проміжних 
% % % \end{minipage}}%
% % % \end{exampleSimpleThree}

\Scoring
20\% балів припадає на тести, в~яких масив містить єдине мінімальне та єдине максимальне значення; 80\%\nolinebreak[3] --- на~тести, де і~мінімальне, і~максимальне повторюються.
% 
Розподіл балів за розмірами тестів такий:\hspace{1mm plus 1mm}\linebreak[2]
20\%\nolinebreak[3] балів припадає на тести, де ${2\,{\<}\,N\,{\<}\,100}$;\hspace{1mm plus 1mm}\linebreak[2]
30\%\nolinebreak[3]\nolinebreak[3] --- на тести, де ${4000\,{<}\,N\,{\<}\,10^4}$;\hspace{1mm plus 1mm}\linebreak[2]
20\%\nolinebreak[3]\nolinebreak[3] --- на тести, де ${5\,{\cdot}\,10^4{<}N\,{\<}\,10^5}$;\hspace{1mm plus 1mm}\linebreak[2]
30\%\nolinebreak[3]\nolinebreak[3] --- на тести, де ${7\,{\cdot}\,10^5\,{\<}\,N\,{\<}\,10^6}$.
% 
Писати різні програми для різних випадків не~треба;
розподіли вказані, щоб показати,
скільки приблизно балів можна отримати, розв'язавши задачу не~повністю.

% \begin{tabular}{c||c|c|c@{}}
% Кількість & \multicolumn{3}{c}{Максимальне значення елемента}\\
% елементів & $2{\<}max{\<}100$ & $101{\<}max{\<}30000$ & $30001{\<}max{\<}10^6$ \\\hline\hline
% $2{\<}N{\<}100$      & 10\% & 10\% & 10\% \\\hline
% $101{\<}N{\<}30000$  & 10\% & 10\% & 10\% \\\hline
% $30001{\<}N{\<}10^6$ & 10\% & 10\% & 20\% \\
% \end{tabular}

% Писати різні програми для різних випадків не~треба;
% мета таблиці\nolinebreak[2] --- показати, скільки приблизно балів можна отримати, розв'язавши задачу не~повністю.



\end{problem}