\Tutorial	
Тут ще істотніше, ніж у задачі~\ref{prob:201516-oioi-A-leap-years-easy}, що можна або реалізовувати все самостійно, або використати бібліотечні засоби.

\MyParagraph{Cамостійна реалізація.} Можна використати (згаданий у~задачі~\ref{prob:201516-oioi-A-leap-years-easy}) факт, що 15.10.1582 було п'ятницею, отже 01.01.1583\nolinebreak[3] --- субота (наприклад, за наданим % в умові цієї задачі
 календарем можна підрахувати, що 1~січня наступного року припадає на\nolinebreak[2] наступний після 15~жовтня день тижня).
%
Ще можна згадати (чи\nolinebreak[3] побачити у 
% наведеному 
нада\-ному
календарі), що не~високосний рік містить цілу кількість тижнів~+~1~день. Відповідно, високосний\nolinebreak[3] --- цілу кількість тижнів~+~2~дні. Отже:
\vspace{-0.5\baselineskip}
\begin{center}
\begin{tabular}{rcl}
01.01.1584 & неділя & (01.01.1583 субота+1) \\
01.01.1585 & вівторок & (01.01.1584 неділя+2, бо 1584~рік високосний) \\
01.01.1586 & середа & (01.01.1585 вівторок+1) \\
01.01.1587 & четвер & (01.01.1586 середа+1) \\
01.01.1588 & п'ятниця & (01.01.1587 четвер+1) \\
01.01.1589 & неділя & (01.01.1588 п'ятниця+2, бо 1588~рік високосний) \\
$\vdots$~~~ & $\vdots$ & ~~~$\vdots$
\end{tabular}
\end{center}
\vspace{-0.5\baselineskip}
% 01.01.1584\nolinebreak[3] --- неділя\nolinebreak[3] (субота+1);\linebreak[1]\hspace{0.5em plus 4em} 
% 01.01.1585\nolinebreak[3] --- вівторок\nolinebreak[3] (неділя+2, бо 1584~рік високосний);\linebreak[1]\hspace{0.5em plus 4em} 
% 01.01.1586\nolinebreak[3] --- середа\nolinebreak[3] (вівторок+1);\linebreak[1]\hspace{0.5em plus 4em} 
% 01.01.1587\nolinebreak[3] --- четвер\nolinebreak[3] (середа+1);\linebreak[1]\hspace{0.5em plus 4em} 
% 01.01.1588\nolinebreak[3] --- п'ятниця\nolinebreak[3] (четвер+1);\linebreak[1]\hspace{0.5em plus 4em} 
% і~т.~д.
% 01.01.1589\nolinebreak[3] --- неділя\nolinebreak[3] (п'ятниця+2, бо 1588~рік високосний);
% % % 01.01.1584\nolinebreak[3] --- Нд~(Сб+1); 
% % % 01.01.1585\nolinebreak[3] --- Вв~(Нд+2, 1584 високосний);
% % % 01.01.1586\nolinebreak[3] --- Ср~(Вв+1); 
% % % 01.01.1587\nolinebreak[3] --- Чт~(Ср+1); 
% % % 01.01.1588\nolinebreak[3] --- Пт~(Чт+1);
% % % 01.01.1589\nolinebreak[3] --- Нд~(Пт+2, 1588 високосний);
% % % і~т.~д.

Значить, можна перебрати усі роки від 1583 по $y{-}1$ (де\nolinebreak[3] $y$\nolinebreak[3] --- рік зі вхідних даних; кілька тисяч ітерацій\nolinebreak[3] --- для комп'ютера зовсім\nolinebreak[2] не\nolinebreak[3] багато) і щоразу додавати 1 чи~2, користуючись правильною перевіркою високосності з задачі~\ref{prob:201516-oioi-A-leap-years-easy}. Так\nolinebreak[2] узн\'{а}ємо, яким днем тижня починається  потрібний рік~$y$.

Тепер можна, наприклад, послідовно перебрати усі дні цього року, як у \IdeOne{m71JWc}. Можна і трохи оптимізувати, перебираючи лише місяці (а~не~дні), як у \IdeOne{gnuUXY}. Але раз раніше перебирали роки (можливо, кілька тисяч), виграш не~буде справді помітним. Ще,\nolinebreak[2] можна замість функції \verb|daysInMonth| тримати масив кількостей днів \texttt{(31, 28, 31, 30, 31, 30, 31, 31, 30, 31, 30, 31)} (у~випадку високосності року 28 міняється на~29). І~так далі\nolinebreak[3] --- у цієї задачі взагалі багато правильних розв'язків\dots

\MyParagraph{Розв'язки, що істотно спираються на бібліотечні засоби.} Вони істотно залежать від конкретних бібліотек та конкретної мови програмування, тож нема сенсу займати тут багато місця описами цих бібліотек. Наведемо лише приклади: \IdeOne{2jLmH1} (Delphi чи \mbox{Free}\discretionary{}{}{}\mbox{Pascal}), \IdeOne{X9xo5s} (Java), а~описи відповідних бібліотек пропонуємо знайти в Інтернеті самостійно.
Див. також зауваження наприкінці пояснень до задачі~\ref{prob:201516-oioi-A-leap-years-easy} цього змагання.