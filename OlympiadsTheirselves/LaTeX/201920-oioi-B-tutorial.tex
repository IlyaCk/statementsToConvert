\Tutorial
Само собою, потрібно з'ясувати, яка з трьох ситуацій має місце: (А)~потрібний проміжок повністю потрапляє на рівноприскорений рух ракети; (Б)~потрібний проміжок повністю потрапляє на рівномірний рух ракети; (В)~потрібний проміжок містить і\nolinebreak[3] кінець рівноприскореного руху, і~початок рівномірного.

Найпростішим є випадок~(Б); він має місце при ${\frac{ad^2}{2}\,{\<}\,b}$ (початок потрібного проміжку далі від старту, ніж точка, до якої триває прискорений рух), а~остаточна відповідь виражається простою формулою $\frac{c-b}{ad}$ (бо\nolinebreak[3] для рівномірного руху $\textnormal{час}\dib{{=}}\frac{\textnormal{відстань}}{\textnormal{швидкість}}$, а~за\nolinebreak[2] $d$~секунд рівноприскореного руху набувається швидкість~$ad$).

Випадок~(А) має місце при ${c\,{\<}\,\frac{ad^2}{2}}$; для нього слід перетворити згадану в умові формулу ${s\,{=}\,\frac{at^2}{2}}$ до ${t\,{=}\,\sqrt{\frac{2s}{a}}}$; підставивши двічі для початку й кінця фрагменту, маємо $\sqrt{\frac{2c}{a}}\dib{{-}}\sqrt{\frac{2b}{a}}$.

Лишається тільки випадок~(В); розпізнавати його у програмі, мабуть, краще не~шляхом виписування його умови, а~за~рахунок організації розгалужень з \texttt{else}-ами так, щоб легко виражати <<все, що не~(А) і~не~(Б)>>; для нього слід поєднати формулу пункту~(А) для проміжку від~$b$ до~$\frac{ad^2}{2}$ та формулу пункту~(Б) для проміжку від~$\frac{ad^2}{2}$ до~$c$.

% % % \ifnum\getpagerefnumber{sec:floating-point}=\getpagerefnumber{text:floating-point-end}\else

Тільки рівно такий розв'язок, найімовірніше, набере значну частину балів, та не~всі. Щось~із\nolinebreak[2] 210~з~250 (точна кількість може залежати від використаної мови програмування, а\nolinebreak[2] в\nolinebreak[2] деяких з мов\nolinebreak[3] --- від використаних типів). І~причина том\'{у}\nolinebreak[3] --- похибки (на\nolinebreak[3] що досить явно натякає й те, що при здачі такого розв'язку він отримує за деякі з тестів і~не~0, і~не~максимум). При обчисленні $\sqrt{\frac{2c}{a}}\dib{{-}}\sqrt{\frac{2b}{a}}$ доводиться віднімати ч\'{и}сла, які можуть бути досить близкими. Коли сам\'{і} $b$ та~$c$ досить великі, але при цьому ${c\,{-}\,b}$ становить всього кілька метрів, то різниця між $\sqrt{\frac{2c}{a}}$ та $\sqrt{\frac{2b}{a}}$ стає взагалі малопомітною. Це\nolinebreak[3] стандартна проблема (див.~також стор.~\pageref{sec:floating-point}\ifnum\getpagerefnumber{sec:floating-point}=\getpagerefnumber{text:floating-point-end}\else--\pageref{text:floating-point-end}\fi), і один з можливих шляхів її зменшення (повністю ліквідувати її неможливо, бо так уже влаштовані комп'ютери, що нема стандартних способів подавати ірраціональні ч\'{и}сла точно)\nolinebreak[3] --- спробувати аналітично перетворити формулу так, щоб смисл був той\nolinebreak[3] самий, але вплив похибок менший. І~тут це досить просто: $
\sqrt{\frac{2c}{a}}\dib{{-}}\sqrt{\frac{2b}{a}}
\dibbb{{=}}
\sqrt{\frac{2}{a}}\cdot{(\sqrt{c}\,{-}\,\sqrt{b})}
\dibbb{{=}}
\sqrt{\frac{2}{a}}\frac{c\,-\,b}{\sqrt{c}\,{+}\,\sqrt{b}}
$
(останнє перетворення спирається на тотожність $(x\,{-}\,y)\cdot(x\,{+}\,y)\dib{{=}}{x^2\,{-}\,y^2}$, де ${x\,{=}\,\sqrt{c}}$, ${y\,{=}\,\sqrt{b}}$). Само собою, для отримання повних балів слід зробити таке  перетворення і для ситуації~(Б), і~для ситуації~(В).