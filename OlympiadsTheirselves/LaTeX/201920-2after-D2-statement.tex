\begin{problemAllDefault}{Прямокутні максимуми}

\myflfigaw{\hspace*{-0.25em}\noindent\raisebox{0pt}[5\baselineskip][5\baselineskip]{\begin{tabular}{@{}c|c|c@{}}
% Мова
&
Прямий лінк на файл
&
\begin{footnotesize}\hspace{-0.5em}\begin{minipage}{0.085\textwidth}
\begin{center}
Його \mbox{потім} здавати?
\end{center}
\end{minipage}\end{footnotesize}
\\\hline
\multirow{3}{*}{\rotatebox[origin=c]{90}{\texttt{g++}}\hspace*{-0.25em}}
&
\begin{footnotesize}%
\href{https://ejudge.ckipo.edu.ua/201920-2a-D-header-g++.cpp}{ejudge.ckipo.edu.ua/201920-2a-D-header-g++.cpp}%
\end{footnotesize}
&
Ні
\\
% \texttt{g++}
&
\begin{footnotesize}%
\href{https://ejudge.ckipo.edu.ua/201920-2a-D-sample-g++.cpp}{ejudge.ckipo.edu.ua/201920-2a-D-sample-g++.cpp}%
\end{footnotesize}
&
Так
\\
% \texttt{g++}
&
\begin{footnotesize}%
\href{https://ejudge.ckipo.edu.ua/201920-2a-D-footer-g++.cpp}{ejudge.ckipo.edu.ua/201920-2a-D-footer-g++.cpp}%
\end{footnotesize}
&
Ні
\\\hline
\multirow{3}{*}{\rotatebox[origin=c]{90}{\texttt{fpc}}\hspace*{-0.25em}}
&
\begin{footnotesize}%
\href{https://ejudge.ckipo.edu.ua/201920-2a-D-header-fpc.pas}{ejudge.ckipo.edu.ua/201920-2a-D-header-fpc.pas}%
\end{footnotesize}
&
Ні
\\
% \texttt{fpc}
&
\begin{footnotesize}%
\href{https://ejudge.ckipo.edu.ua/201920-2a-D-sample-fpc.pas}{ejudge.ckipo.edu.ua/201920-2a-D-sample-fpc.pas}%
\end{footnotesize}
&
Так
\\
% \texttt{fpc}
&
\begin{footnotesize}%
\href{https://ejudge.ckipo.edu.ua/201920-2a-D-footer-fpc.pas}{ejudge.ckipo.edu.ua/201920-2a-D-footer-fpc.pas}%
\end{footnotesize}
&
Ні
\\\hline
\multirow{3}{*}{\rotatebox[origin=c]{90}{\texttt{javac}}\hspace*{-0.25em}}
&
\begin{footnotesize}%
\hspace*{-0.75em}\href{https://ejudge.ckipo.edu.ua/201920-2a-D-header-javac.java}{ejudge.ckipo.edu.ua/201920-2a-D-header-javac.java}\hspace*{-0.75em}%
\end{footnotesize}
&
Ні
\\
% \texttt{java}
&
\begin{footnotesize}%
\hspace*{-0.75em}\href{https://ejudge.ckipo.edu.ua/201920-2a-D-sample-javac.java}{ejudge.ckipo.edu.ua/201920-2a-D-sample-javac.java}\hspace*{-0.75em}%
\end{footnotesize}
&
Так
\\
% \texttt{java}
&
\begin{footnotesize}%
\hspace*{-0.5em}\href{https://ejudge.ckipo.edu.ua/201920-2a-D-footer-javac.java}{ejudge.ckipo.edu.ua/201920-2a-D-footer-javac.java}\hspace*{-0.5em}%
\end{footnotesize}
&
Ні
\end{tabular}}}

Алгоритмічна суть задачі повністю повторює задачу~D <<Прямокутні максимуми>> з ІІ~(районного/\nolinebreak[3]міського) етапу Всеукраїнської олімпіади з інформатики (програмування) по Черкаській області, що відбувся 14.12.2019 (див. змагання~70 сайту \EjudgeCkipoName).
\underline{\emph{Єдина}} відмінність 
від попередньої задачі\nolinebreak[3] --- 
потрібно знаходити максимальне число прямокутника, а~не~суму. 

Всі технічні умови такі~ж, як у попередній задачі цього комплекту, крім того, що: 
(1)~одна з підпрограм (функцій для C++ та Pascal, методів для Java), які Ви повинні реалізувати, називається \texttt{calcRectMax} замість \texttt{calcRectSum} та повинна вертати 32-бітовий тип (\texttt{int}/\nolinebreak[2]\texttt{integer}/\nolinebreak[2]\texttt{int} залежно від мови програмування) замість 64-бітового;
(2)~мовою Pascal, означення типу масиву має вигляд \texttt{type Arr2D = array[1..1234, 1..1234] of integer}.

% \begin{small}

\myflfigaw{\hspace*{-0.25em}\raisebox{0pt}[3.75\baselineskip][3.5\baselineskip]{\begin{small}\begin{tabular}{@{}c|c|p{0.425\textwidth}|c|c@{}}
блок\hspace*{-0.01\textwidth}
&
тести
&
\multicolumn{1}{c|}{обмеження}
&
передумови
&
бали
\\\hline\hline%\endhead
1 & 
1--5 & 
${2\,{\<}\,N,M,T,a_{ij}\,{\<}\,9}$ & 
нема & 
10\% 
\\\hline
2 & 
6--10 & 
${1\,{\<}\,N,M,T\,{\<}\,100}$; ${1\,{\<}\,a_{ij}\,{\<}\,10^4}$ &
блок~1 & 
10\% \\\hline
3 &
11--14 & 
${100\,{\<}\,N,M\,{\<}\,400}$; ${123\,{\<}\,T,a_{ij}\,{\<}\,43210}$ &
блоки \mbox{1--2} & 
15\% \\\hline
4 &
15--18 & 
${400\,{\<}\,N,M\,{\<}\,500}$; ${54321\,{\<}\,T\,{\<}\,222555}$; ${1\,{\<}\,a_{ij}\,{\<}\,10^9}$ &
блоки \mbox{1--3} & 
15\% \\\hline
5 &
19--22 & 
${555\,{\<}\,N,M\,{\<}\,1111}$; ${12345\,{\<}\,T\,{\<}\,54321}$; ${1\,{\<}\,a_{ij}\,{\<}\,10^9}$ &
блоки \mbox{1--3} & 
15\% \\\hline
6 &
23--26 & 
${500\,{\<}\,N,M\,{\<}\,700}$; ${5{\cdot}10^5\,{\<}\,T\,{\<}\,10^6}$; ${1\,{\<}\,a_{ij}\,{\<}\,10^9}$ &
блоки \mbox{1--4} & 
15\% \\\hline
7 &
27--32 & 
${12\,{\<}\,N,M\,{\<}\,1234}$; ${8{\cdot}10^5\,{\<}\,T\,{\<}\,10^6}$; ${1\,{\<}\,a_{ij}\,{\<}\,10^9}$ &
блоки \mbox{1--6} & 
20\%
\end{tabular}\end{small}}}

\Scoring
Аналогічно попередній задачі, але обмеження блоків та розподіл балів між ними 
трохи інші (див. праворуч).
%
Аналогічно попередній задачі, у~блоках 1~та~2 тести в~точності збігаються з відповідними тестами задачі~D <<Прямокутні максимуми>> з ІІ~(районного/\nolinebreak[3]міського) етапу Всеукраїнської олімпіади з інформатики (програмування) по Черкаській області, що відбувся 14.12.2019; зокрема, 1-й тест 1-го блоку відповідає тому тесту, котрий там наведений в умові. 
% В~подальших блоках однаковість тестів не~гарантується, і, як~правило, не~дотримана.
В~подальших блоках тести інші.

% \end{small}

% \vspace{-0.75\baselineskip}
 
\end{problemAllDefault}
