\Tutorial
Одна з основних складностей цієї задачі\nolinebreak[3] --- потреба і прочитати вхідні дані, як рядок (щоб переконатися, що довжина $\<$30~символів), і~виділити з нього число, щоб провести розрахунок похибки за наданою формулою (примітивною, там нема чого пояснювати) та написати правильні розгалуження. Тут\nolinebreak[3] важко щось сказати загально, бо все\nolinebreak[3] це\nolinebreak[3] --- більше про особливості % конкретних 
мов програмування, ніж про задачу. % в~цілому. 
Учасники, які зазвичай пишуть на python або\nolinebreak[3] C\#, тут опинилися в трохи виграшному становищі, бо їм постійно доводиться робити перетворення з рядкового подання у чисельне, й вони повинні~б це добре пам'ятати; але засоби такого перетворення % взагалі-то 
є в усіх дозволених на цій олімпіаді мовах програмування.
\TODO % add direct ideOne links
Пізніше сюди будуть дописані посилання на правильні р\'{о}зв'язки цієї задачі різними мовами програмування.

