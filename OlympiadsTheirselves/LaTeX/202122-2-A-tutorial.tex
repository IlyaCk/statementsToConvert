\Tutorial

Якщо обчислити кілька перших трійок 
(${1 + 2 - 3}\dibbb{{=}}0$;\linebreak[2]\hspace{0.125em plus 0.5em}
${4 + 5 - 6}\dibbb{{=}}3$;\linebreak[2]\hspace{0.125em plus 0.5em}
${7 + 8 - 9}\dibbb{{=}}6$;\linebreak[2]\hspace{0.125em plus 0.5em}
${10 + 11 - 12}\dibbb{{=}}9$;\nolinebreak[3]\hspace{0.125em plus 0.5em} \dots),
можна помітити, що результати таких трійок дорівнюють 
0,\nolinebreak[3] 
3,\nolinebreak[2]
6,\nolinebreak[2]
9,\nolinebreak[3]
\dots,
утворюючи тим самим арифметичну прогресію.

Таким чином, якщо порахувати кількість трійок через вираз 
\texttt{k:=n~div~3}\nolinebreak[2] (Pascal),
він\nolinebreak[3] же \texttt{k~=~n//3}\nolinebreak[2] (Python3),
\dots{} (в~разі потреби допишіть своєю мовою програмування самостійно),
то суму всіх трійок можна порахувати через формулу ${\frac{0+3(k-1)}{2}\cdot{}k}$.

Само собою, введене\nolinebreak[3] $n$ може бути й не~кратним~3, тобто можуть лишитися один чи два останні доданки, які не~входять ні\nolinebreak[3] в\nolinebreak[3] яку\nolinebreak[2] трійку; їх можна додати до результату окремими \texttt{if}-ами.


\MyParagraph{А без цих прогресій можна?}
Якщо Вас влаштовує 70\% балів, то можна просто порахувати в циклі суму всіх підряд натуральних чисел від~1 до~$n$, але доданки, кратні~3, не~додавати, а~віднімати (включивши \texttt{if} всер\'{е}дину\nolinebreak[2] \mbox{\texttt{for}-а}). 

Само собою, можна прийти до правильного результату і ще якось. Можна, наприклад, не~пам'ятаючи формулу суми арифметичної прогресії, вивести самостійно якийсь її аналог, враховуючи якісь особливості конкретно цієї задачі. Ніхто не~перевіряє, чи\nolinebreak[3] знаєте Ви, що це називається арифметичною прогресією, та чи\nolinebreak[3] пам'ятаєте стандартну формулу. Зумієте замінити чимось іншим, що теж вкладеться в обмеження часу\nolinebreak[3] --- ну то й добре.


\MyParagraph{А згорнути зовсім в один вираз, щоб обчислювати і без циклів, і без \mbox{\texttt{if}-ів}, можна?}
В~принципі, можна. Якби сума була не~знакозмінною, а~просто 
$1\dib{{+}}
2\dib{{+}}
3\dib{{+}}
4\dib{{+}}
5\dib{{+}}
{6\,{+}\,\ldots\,{+}\,n}$,
її можна було~б перетворити (наприклад, за все\nolinebreak[3] тією\nolinebreak[3] ж формулою суми арифметичної прогресії) до $\frac{n\cdot(n+1)}{2}$. Чим потрібна знакозмінна сума відрізняється від цієї? Тим, що доданки, кратні~3, віднімаються, а~не~додаються\dots{} але це\nolinebreak[3] с\'{а}ме можна сформулювати інакше: тим, що із $\frac{n\cdot(n+1)}{2}$ треба двічі відняти 
$3\dib{{+}}
6\dib{{+}}
{9\,{+}\,\ldots\,{+}\,3k}$ (де\nolinebreak[3] $k$, як і кілька абзаців тому, має смисл $n\bdiv 3$, а~віднімати двічі треба тому, що якщо просто відняти, то вийшла~б сума $1\dib{{+}}
2\dib{{+}}
4\dib{{+}}
5\dib{{+}}
7\dib{{+}}
8\dib{{+}}
{10,{+}\,\ldots}$,
тобто доданки, \mbox{кратні}~3, просто <<щезли>>~б, а~треба, щоб вони не~<<щезли>>, а\nolinebreak[3] стали від'ємними). Легко бачити, що 
$3\dib{{+}}
6\dib{{+}}
{9\,{+}\,\ldots\,{+}\,3k}
\dibbb{{=}}
3\cdot\bigl(
1\dib{{+}}
2\dib{{+}}
{3\,{+}\,\ldots\,{+}\,k}\bigr)
\dibbb{{=}}
3\cdot\frac{k\cdot(k+1)}{2}$,
а раз цю суму треба відняти двічі, маємо
$$
\textnormal{остат\'{о}чний результат}
=
\frac{n\cdot(n+1)}{2}
-
3\cdot k\cdot(k+1),
\quad
\textnormal{де }
k=n\bdiv 3.
$$

Також, для отримання повних балів важливо рахувати суму, щонайменше, в 64-бітовому цілочисельному типі, бо в менших (хоч 32-бітовому цілочисельному, хоч з рухомою комою (плаваючою точкою), як-то \verb"double") не~вистачає розрядів (відбувається переповнення (див.\nolinebreak[3] також стор.~\pageref{text:overflow-example}) чи 
похибка (див.\nolinebreak[3] також стор.~\pageref{text:floating-point-error-in-0.3}), що призводить до неправильної відповіді (наскільки сильно не~вистачає і наскільки неповні бали, може залежати від конкретної мови програмування та конкретних типів). Рахувати у ще\nolinebreak[3] більших типах (наприклад, \texttt{BigInteger} мови Java) теж можна.