\Tutorial

На перший погляд, слід просто порахувати суму всіх елементів 
$
a_1\dibbb{{+}}
a_2\dibbb{{+}}
\dots\dibbb{{+}}
a_N
$,
і поділити на~2 з заокругленням догори (наприклад, 
${\lceil\frac{1}{2}\rceil}\dib{{=}}{\lceil\frac{2}{2}\rceil\,{=}\,1}$,
${\lceil\frac{3}{2}\rceil}\dib{{=}}{\lceil\frac{4}{2}\rceil\,{=}\,2}$,
${\lceil\frac{5}{2}\rceil}\dib{{=}}{\lceil\frac{6}{2}\rceil\,{=}\,3}$,~\dots).
Це\nolinebreak[2] цілком <<обґрунтовується>> такими правдоподібними міркуваннями: почнемо стріляти з\nolinebreak[2] \mbox{1-ї}\nolinebreak[2] сторони; 
поки на\nolinebreak[2] ній лишається $\>\,$2 плит, стріляємо з обох стволів по цій стороні, доки\nolinebreak[3] не~лишиться або~0, або~1; якщо лишиться~2, то\nolinebreak[2] перейдемо до пострілів з\nolinebreak[2] обох стволів по\nolinebreak[2] наступній стороні, а~якщо лишиться~1, то\nolinebreak[2] зробимо один постріл, який збиває по одній плиті з поточної та наступної сторін, а\nolinebreak[3] потім продовжуємо залежно від кількості плит, що лишилися на цій наступній стороні, тощо. Й\nolinebreak[3] усе\nolinebreak[3] це начебто доводить, що цілком марнотратних пострілів (які\nolinebreak[2] не~збивають жодної плити) не~буде взагалі, а\nolinebreak[3] част\-ково-марно\-трат\-ним (який збиває лише одну плиту, а~не~дві) може виявитися хіба лише останній постріл, якщо так уже вийде, що на останній стіні лишилася тільки одна плита\dots

Однак, вищезгадане міркування не~враховує, що за\nolinebreak[3] умовою кількість плит на стороні може бути~й~0. Наприклад, при кількостях плит 0~1~0~1~0~1~0~1~0 доведеться витрачати окремий постріл на кожну плиту. Втім, навіть якщо нулі у вхідних даних~є, ті\nolinebreak[3] с\'{а}мі вищезгадані міркування все~ж частково правильні, просто застосовувати їх треба не~до\nolinebreak[3] всього масиву, а\nolinebreak[3] до\nolinebreak[2] кожного проміжку між нулями окремо. Наприклад, при кількостях плит 0~1~3~1~0~7~0~3~9~0 можна і варто порахувати 
окремо $\lceil\frac{1+3+1}{2}\rceil\dib{{=}}3$,
окремо $\lceil\frac{7}{2}\rceil\dib{{=}}4$,
окремо $\lceil\frac{3+9}{2}\rceil\dib{{=}}6$,
і потім додати ці результати (${3\,{+}\,4\,{+}\,6}\dibbb{{=}}13$).

Крім того, <<зацикленість>> многокутника (остання сторона сусідня з першою) може давати (а~може й не~давати) можливість зекономити один постріл за рахунок об'єднання фрагментів на\nolinebreak[3] початку й наприкінці. Наприклад, для кількостей плит 3~0~5~1~0~3 нема потреби робити 
$
\lceil\frac{3}{2}\rceil\dib{{+}}
\lceil\frac{5+1}{2}\rceil\dib{{+}}
\lceil\frac{3}{2}\rceil\dibbb{{=}}
{2\,{+}\,3\,{+}\,2}\dibbb{{=}}7
$
пострілів, досить і 
$
\lceil\frac{5+1}{2}\rceil\dib{{+}}
\lceil\frac{3+3}{2}\rceil\dibbb{{=}}
{3\,{+}\,3}\dibbb{{=}}6
$.

Все, що слід \emph{помітити}, вже перелічено. Лишається тільки реалізувати все~це, правильно розділивши весь масив на фрагменти між нулями, причому враховуючи, що нулів може й не~бути. 

Насамкінець, у~багатьох мовах програмування для <<заокруглення догори>> може бути корисною функція, що називається \verb"ceil", \verb"Ceiling" чи ще якось схоже. 
Можливий і спосіб, що не~потреб\'{у}є такої функції: 
якщо вищезгадана сума вже порахована у змінній~\texttt{s}, лишається обчислити
\verb"(s+1) div 2"\nolinebreak[2] (Pascal),
\verb"(s+1)//2"\nolinebreak[2] (Python~3),
\verb"(s+1)/2"\nolinebreak[2] (за умови цілочисельності~\texttt{s}, С-подібні мови та Python~2).
% % % В~такому разі, при п\'{а}рному значенні~\texttt{s}, 