\begin{problemAllDefault}{Максимальний добуток}

Дано чотири цілі числ\'{а}.
Виведіть два з~них, добуток яких максимальний.

\InputFile
Рівно чотири числ\'{а}, що задаються кожне в~окремому рядку. 
Серед цих чисел можуть (але\nolinebreak[2] не~зобов'язані) бути рівні.

{
\ifAfour\looseness=-1\hyphenpenalty=8000\fi

\OutputFile
Рівно два (кожне в~окремому рядку) з~\mbox{4-х} уведених чисел, 
які дають максимальний добуток. Це~повинні бути різні введені елементи. 
Інакше кажучи, повторити обидва рази одакове значення 
можна \emph{лише}\nolinebreak[3] якщо це~с\'{а}ме значення 
зустрічалося двічі (або~ще~більше разів) у~вхідних даних.

}

Якщо можливі різні правильні відповіді\nolinebreak[3] --- 
виводьте будь-яку одну з~них.

\Examples
\begin{exampleSimple}{5em}{5em}
\exmp{17
7
42
23}{42
23}%
\end{exampleSimple}
\begin{exampleSimple}{5em}{5em}
\exmp{42
17
42
23}{42
42}%
\end{exampleSimple}


\Scoring
25\% балів припадає на тести, де значення кожного з чисел у~межах від~1 до~100.
Ще~25\% балів припадає на тести, де значення кожного з чисел не~перевищує 100 за~модулем, але ч\'{и}сла можуть бути довільних знаків.
Ще~25\% балів припадає на тести, де значення кожного з чисел у~межах від~1 до~$10^9$.
Решта 25\% балів припадає на тести, де значення кожного з чисел не~перевищує $10^9$ за~модулем, але ч\'{и}сла можуть бути довільних знаків.

Здавати потрібно одну програму, а~не~чотири; різні обмеження вказані, щоб пояснити, 
скільки балів можна отримати, розв’язавши задачу не~повністю.

\end{problemAllDefault}

