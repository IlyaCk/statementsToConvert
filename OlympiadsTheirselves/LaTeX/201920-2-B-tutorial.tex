\Tutorial
Легко придумати ситуації, коли два к\'{о}ла мають 2 чи 4 спільні дотичні; ситуації, коли їх 0, 1 або 3, придумуються дещо важче; може здатися, ніби строго більше, ніж 4, взагалі не~буває. Але насправді такий випадок (один) все-таки~є:
\texttt{(X1}$\,$\texttt{=}$\,$\texttt{X2) and 
(Y1}$\,$\texttt{=}$\,$\texttt{Y2) and 
(R1}$\,$\texttt{=}$\,$\texttt{R2)}, тобто к\'{о}ла повністю однакові. Тоді абсолютно будь-яка дотична є спільною, і кількість виявляється нескінч\'{е}нною; це й треба позначати числом~7.

Тому для розв'язування 
% цієї задачі слід \emph{лише} вибрати, 
задачі досить вибрати, 
який із розглянутих випадків має місце, а для цього потрібні \emph{лише} радіуси %кіл 
$R1$, $R2$ та відстань між центрами %кіл 
$d\dib{{=}}\sqrt{(X1-X2)^2+(Y1-Y2)^2}$.
Для зручності подальших викладок забезпечимо $R1\>R2$ (якщо це не~так, досить обміняти їх місцями).

\begin{figure*}[h] 

\begin{small} 

\begin{tabular}{@{}p{0.14\textwidth}|p{0.15\textwidth}|p{0.14\textwidth}|p{0.14\textwidth}|p{0.15\textwidth}|p{0.14\textwidth}@{}}
\begin{large}$\infty$\end{large} спільних дотичних:
&
\begin{large}\textsf{0}\end{large} спільних дотичних:
&
\begin{large}\textsf{1}\end{large} спільна дотична:
&
\begin{large}\textsf{2}\end{large} спільні дотичні:
&
\begin{large}\textsf{3}\end{large} спільні дотичні:
&
\begin{large}\textsf{4}\end{large} спільні дотичні:
\\
\begin{mfpic}[4]{-5}{5}{-5}{5}
\circle{(0,0),5}
\lines{(7.027,0.792),(-2.746,8.585)}
\lines{(3.762,5.987),(-8.425,3.206)}
\lines{(-2.335,6.674),(-7.759,-4.588)}
\lines{(-6.674,2.335),(-1.251,-8.927)}
\lines{(-5.987,-3.762),(6.199,-6.544)}
\lines{(-0.792,-7.027),(8.981,0.767)}
\lines{(5,-5),(5,7.5)}
\end{mfpic}
&
\begin{mfpic}[6]{-5}{5}{-5}{5}
\circle{(0,0),5}
\circle{(-2,1),2}
\end{mfpic}
&
\begin{mfpic}[6]{-5}{5}{-5}{5}
\circle{(0,0),5}
\circle{(2.4,1.8),2}
\lines{(7,-1),(2.5,5)}
\end{mfpic}
&
\begin{mfpic}[5]{-5}{5}{-5}{5}
\circle{(0,0),5}
\circle{(3,4),2}
\lines{(5,-5),(5,5)}
\lines{(-5.24,3.68),(4.36,6.48)}
\end{mfpic}
&
\begin{mfpic}[4]{-5}{11}{-5}{5}
\circle{(0,0),5}
\circle{(5.6,4.2),2}
\lines{(7,-1),(1,7)}
\lines{(-7.194,3.64),(8.3,6.79)}
\lines{(1.48,-7.925),(8.842,6.067)}
\end{mfpic}
&
\begin{mfpic}[4]{-5}{11}{-5}{5}
\circle{(0,0),5}
\circle{(7,4),2}
\lines{(-8.099,3.924),(10.431,6.495)}
\lines{(-0.731,-8.97),(10.892,5.689)}
\lines{(-0.908,6.338),(7.708,1.262)}
\lines{(5,-4),(5,6)}
\end{mfpic}
\\\hline
к\'{о}ла повністю однакові
&
одне коло повністю все\-р\'{е}\-ди\-ні іншого
&
к\'{о}ла дотикаються внут\-рішньо
&
к\'{о}ла перетинаються двома точками
&
к\'{о}ла дотикаються зовнішньо
&
к\'{о}ла не~перетинаються
\\\hline
% % % $(X1=X2)$\mathop{and}(Y1=Y2)\mathop{and}(R1=R2)$
\mbox{$(X1=X2)$\texttt{and}}\linebreak[1]\mbox{$(Y1=Y2)$\texttt{and}}\linebreak[1]\mbox{$(R1=R2)$}
&
$d<R1-R2$
&
$d=R1-R2$
&
${R1-R2}\dib{{<}}d\dib{{<}}R1$
&
$d=R1$
&
$d>R1$
\\
\end{tabular}

\end{small}

\end{figure*}

З суто технічних (кодерських) питань слід зауважити: коли координати можуть сягати за модулем мільйон, то квадрат різниці координат вже не~поміщається у 32-бітовий тип, надійніше використати 64-бітовий чи <<дійсний>> (з~рухомою комою, він~же з плаваючою точкою). А~для когось може виявитися несподіванкою, що у FreePascal (якщо говорити конкретно про варіанти, які пропонує \EjudgeCkipoName, то лише \texttt{fpc}, не~\texttt{fpc-delphi} та не~\texttt{pasabc-linux}) тип \texttt{integer} взагалі досі 16-бітовий.
Див.\nolinebreak[3] також стор.~\pageref{text:overflow-example} та стор.~\pageref{text:notes-about-delphi-mode}.