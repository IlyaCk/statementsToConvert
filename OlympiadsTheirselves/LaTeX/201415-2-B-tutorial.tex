\Tutorial
Ця задача проста, щоб набрати \emph{частину} балів, але набрати повний бал чи значну частину балів не~так~просто. Розв'язок \IdeOne{w8Yu4L} набирає половину балів. 
І~тут є\nolinebreak[2] де помил\'{и}тися й отримати ще\nolinebreak[3] менше\nolinebreak[3] (важливо і\nolinebreak[3] правильно врахувати смисл~$m$, і\nolinebreak[3] взяти тип \texttt{int64}).

При ${n\,{\>}\,3{\cdot}10^6}$ (приблизно), сума $1^2\dib{{+}}2^2\dib{{+}}\dots\dib{{+}}n^2$ виходить навіть за межі 64-\nolinebreak[2]бітового типу. З~цим можна боротися, застосовуючи найпростіший прийом \emph{модульної арифметики}: робити додавання\nolinebreak[3] $i^2$ не~як\nolinebreak[3] \verb"s:=s+sqr(int64(i))", а\nolinebreak[2] як\nolinebreak[3] \verb"s:=(s+sqr(int64(i))) mod m". 
Реалізація з вчасними <<\texttt{\dots~mod~m}>>, набирає 150 балів (з~250). 
Тепер критичним стає те, що $\approx{}10^9$ ітерацій циклу (ще\nolinebreak[3] й з\nolinebreak[2] громіздкою операцією \texttt{mod}) не~поміщаються в обмеження часу (1~сек).

\MyParagraph{100\%-ий спосіб \textnumero$\,$1.} 
Якщо знати (або вивести під час туру) формулу $1^2\dib{{+}}2^2\dib{{+}}\dots\dib{{+}}n^2\dib{{=}}\frac{n(n+1)(2n+1)}{6}$, можна поєднати її з засобами модульної арифметики й отримати зовсім інший розв'язок \IdeOne{g6nSKa}. Він взагалі не~містить циклів (складність $\Theta(1)$), тож працює миттєво.

Щоб написати цей розв'язок, треба знати деякі математичні факти та властивості, і це може не~всім подобатися. Але, по-перше, задача має альтернативний розв'язок; по-друге, навіть якби цей розв'язок був єдиним, це не~суперечило~б традиціям Всеукраїнської олімпіади з~інформатики.

У~цьому розв'язку треба писати с\'{а}ме ``\texttt{\dots\nolinebreak[3] mod\nolinebreak[3] \mbox{(6*m)}}'' (а\nolinebreak[2] не\nolinebreak[3] ``\texttt{\dots\nolinebreak[3] mod~m}''), бо такі властивості модульної арифметики: хоча ${(a+b)\bmod p} \dib{{=}} {\bigl((a\bmod p)}\dib{{+}}{(b\bmod p)\bigr)\bmod p}$\phantomsection\label{text:trivial-proof-for-add-in-ring} --- правильна для всіх $a$,\nolinebreak[2] $b$,\nolinebreak[2] $p$ тотожність, і так само для множення, але для ділення не~так (наприклад, 
${\tfrac{12}{2}\bmod10}\dib{{=}}{6\bmod10}\dib{{=}}6\dib{{\neq}}1\dib{{=}}\tfrac{2}{2}\dib{{=}}\tfrac{12\bmod10}{2\bmod10}$). 
Детальніші відомості про модульну арифметику просимо знайти % в Інтернеті або літературі 
самостійно.

% % % Щодо того, як можна вивести формулу $1^2\dib{{+}}2^2\dib{{+}}\dots\dib{{+}}n^2\dib{{=}}\frac{n(n+1)(2n+1)}{6}$ самому під час туру --- див., наприклад, \verb"http://dxdy.ru/topic22151.html"\hspace{0.5em plus 1em} Звісно, доцільність витрачання часу туру на подібні виведення формул істотно залежить від умінь конкретного учасника та від того, чи\nolinebreak[3] має місце ситуація, коли \dib{{=}}\tfrac{2}{2}учасник знає, що така формула в~принципі існує, але не~пам'ятає її точно.
Щодо того, як вивести формулу $1^2\dib{{+}}2^2\dib{{+}}\dots\dib{{+}}n^2\dib{{=}}\frac{n(n+1)(2n+1)}{6}$ під час туру\nolinebreak[3] --- див., наприклад, \href{https://dxdy.ru/topic22151.html}{\texttt{\mbox{dxdy.ru/}\nolinebreak[2]\mbox{topic22151.html}}}. Звісно, доцільність витрачання часу туру на виведення формул істотно залежить від знань та умінь конкретного учасника.


\MyParagraph{100\%-ий спосіб \textnumero$\,$2.}
Навіть не~знаючи ні\nolinebreak[2] формули зі способу\nolinebreak[3] \textnumero$\,$1, ні\nolinebreak[2] модульної арифметики, можна побудувати інший повнобальний розв'язок, користуючись лише базовими, в~межах загальнообов'язкового мінімуму, знаннями математики, а~також спостережливістю та кмітливістю.

Якщо $n < 2\,{\cdot}\,10^6$, можна порахувати відповідь <<в~лоб>>, як на початку розбору.
Інакше (враховуючи, що ${m\<10^6}$, $n \> 2\,{\cdot}\,10^6$) вийде, що проміжок від~1 до~$n$ містить кілька (як\nolinebreak[3] мінімум, два, як\nolinebreak[3] максимум ---  сотні мільйонів) проміжків від\nolinebreak[2] 1\nolinebreak[2] до~$m$, від\nolinebreak[2] ${m\,{+}\,1}$\nolinebreak[2] до~$2m$, від\nolinebreak[2] ${2m\,{+}\,1}$\nolinebreak[2] до~$3m$, і~т.~д.

% % % \mytextandpicture{Розглянемо (праворуч) послідовність очевидних 
% % % тотожностей для проміжків від\nolinebreak[2] 1\nolinebreak[2] до~$m$ та від
% % % \nolinebreak[2] ${m{+}1}$\nolinebreak[2] до~$2m$. У~виразі ${m^2{+}2m{+}1}$
 % % % частина ${m^2{+}2m}$ кратна~$m$ і тому не~впливає на\nolinebreak[3] остат
% % % \'{о}чну відповідь задачі.}{\begin{tabular}{r@{}c@{}l|rcl}
% % % $1^2$ & $=$ & 1	& $(m{+}1)^2$ & $=$ & $m^2+2m+1$ \\
% % % $2^2$ & $=$ & 4	& $(m{+}2)^2$ & $=$ & $m^2+4m+4$ \\
% % % $3^2$ & $=$ & 9	& $(m{+}3)^2$ & $=$ & $m^2+6m+9$ \\
 % % % & $\vdots$ & & & $\vdots$\\
% % % $m^2$ &$=$&$m^2$& $(m{+}m)^2$ & $=$ & $m^2+2m^2+m^2$ 
% % % \end{tabular}}

% % % {\widowpenalty=0
% % % \clubpenalty=0

% % % % TODO: потребУє ретельного перегляду куди сАме вверсталося
% % % \myflfigaw{\begin{small}\begin{minipage}{20em}\vspace*{-0.3\baselineskip}\begin{tabular}{|r@{}c@{}l|rcl|}\hline
% % % $1^2$ & $=$ & 1	& $(m{+}1)^2$ & $=$ & $m^2+2m+1$ \\
% % % $2^2$ & $=$ & 4	& $(m{+}2)^2$ & $=$ & $m^2+4m+4$ \\
% % % $3^2$ & $=$ & 9	& $(m{+}3)^2$ & $=$ & $m^2+6m+9$ \\
 % % % & \begin{small}$\dots$\end{small} & & & \begin{small}$\dots$\end{small} & \\
% % % $m^2$ &$=$&$m^2$& $(m{+}m)^2$ & $=$ & $m^2+2m^2+m^2$\\\hline 
% % % \end{tabular}\vspace*{-1.3\baselineskip}\end{minipage}\end{small}}
% % % Розглянемо очевидні тотожності на проміжках від\nolinebreak[2] 1\nolinebreak[2] до~$m$ та від\nolinebreak[2] ${m{+}1}$\nolinebreak[2] до~$2m$. Помітимо, що сума ${m^2{+}2m}$ кратна~$m$ і тому не~впливає на\nolinebreak[3] остат\'{о}чну відповідь задачі.
% % % Аналогічно не~впливають ${m^2{+}4m}$,  ${m^2{+}6m}$, і~т.~д. Тобто, 
% % % ${(m{+}1)^2\bmod m}\dib{{=}}{1^2\bmod m}$,
% % % ${(m{+}2)^2\bmod m}\dib{{=}}{2^2\bmod m}$,
% % % ${(m{+}3)^2\bmod m}\dib{{=}}{3^2\bmod m}$,\nolinebreak[3] \dots, 
% % % а\nolinebreak[2] звідси\nolinebreak[3] --- сума усього др\'{у}гого проміжку $(m{+}1)^2\dib{{+}}(m{+}2)^2\dib{{+}}\dots\dib{{+}}(m{+}m)^2$ має той самий залишок від ділення на~$m$, що й сума усього першого $1^2\dib{{+}}2^2\dib{{+}}\dots\dib{{+}}m^2$.

{

\def\pyramideFirstPhrase{Розглянемо очевидні тотожності на проміжках від\nolinebreak[2] 1\nolinebreak[2] до~$m$ та від\nolinebreak[2] ${m\,{+}\,1}$\nolinebreak[2] до~$2m$. Помітимо, що сума ${m^2\,{+}\,2m}$ кратна~$m$ і тому не~впливає на\nolinebreak[3] остат\'{о}чну відповідь задачі.}
\def\pyramideSecondPhrase{Аналогічно не~впливають ${m^2\,{+}\,4m}$,  ${m^2\,{+}\,6m}$, і~т.~д. Тобто, 
${(m\,{+}\,1)^2\bmod m}\dibbb{{=}}{1^2\bmod m}$,
${(m\,{+}\,2)^2\bmod m}\dibbb{{=}}{2^2\bmod m}$,
${(m\,{+}\,3)^2\bmod m}\dibbb{{=}}{3^2\bmod m}$,\nolinebreak[3] \dots, 
а~звідси~--- сума усього др\'{у}гого 
проміжку ${(m\,{+}\,1)^2}\dibbb{{+}}{(m\,{+}\,2)^2}\dibbb{{+}}\dots\dibbb{{+}}{(m\,{+}\,m)^2}$ 
має той~\mbox{самий} залишок від ділення на~$m$, що й сума усього першого 
$1^2\dib{{+}}2^2\dib{{+}}\dots\dib{{+}}m^2$.}
\def\pyramideEquationsTable{\begin{tabular}{|r@{}c@{}l|rcl|}
$1^2$ & $=$ & 1	& $(m\,{+}\,1)^2$ & $=$ & $m^2\,{+}\,2m\,{+}\,1$ \\
$2^2$ & $=$ & 4	& $(m\,{+}\,2)^2$ & $=$ & $m^2\,{+}\,4m\,{+}\,4$ \\
$3^2$ & $=$ & 9	& $(m\,{+}\,3)^2$ & $=$ & $m^2\,{+}\,6m\,{+}\,9$ \\
 & $\vdots$ & & & $\vdots$ & \\
$m^2$ &$=$&$m^2$& $(m\,{+}\,m)^2$ & $=$ & $m^2\,{+}\,2m^2\,{+}\,m^2$
\end{tabular}}

\ifAfour
\myflfigaw{\pyramideEquationsTable}
\pyramideFirstPhrase{}
% no "\par"!!!
\pyramideSecondPhrase
\else
\mytextandpicture{\pyramideFirstPhrase}{\pyramideEquationsTable}
\par
\pyramideSecondPhrase
\fi

}


З~аналогічних причин, такий самий залишок % від ділення на~$m$ 
мають і сума усього третього проміжку ${(2m\,{+}\,1)^2}\dibbb{{+}}{(2m\,{+}\,2)^2}\dibbb{{+}}\dots\dibbb{{+}}{(2m\,{+}\,m)^2}$, і сума будь-якого подальшого. Тому досить цей однаковий для всіх проміжків залишок домножити на ${n\bdiv m}$, а\nolinebreak[3] потім, якщо % $n\bmod m\neq0$, 
$n$\nolinebreak[2] не\nolinebreak[3] кратне\nolinebreak[3] $m$, окремо порахувати й додати шматочок від ${(n\bdiv m){\cdot}\,m}\dibbb{{+}}1$ до~$n$ (або від\nolinebreak[2] 1\nolinebreak[2] до~$n\bmod m$). 

% % % }

Отже, сумарна кількість усіх ітерацій не\nolinebreak[3] перевищить $m\dib{{+}}{(n\bmod m)}\dibbb{{<}}2m\dibbb{{\<}}{2\,{\cdot}\,10^6}$. Це\nolinebreak[3] довше, ніж $\Theta(1)$ зі <<способу\nolinebreak[3] \textnumero$\,$1>>, але теж вкладається у\nolinebreak[2] 1~сек. Щоправда, якби обмеження було не\nolinebreak[3] ${m\,{\<}\,10^6}$, а\nolinebreak[2] ${m\,{\<}\,10^9}$, це стало~б не~так (а~<<способу\nolinebreak[3] \textnumero$\,$1>> байдуже, тому він в деякому смислі кращий).

