\begin{problem}{Відтинання квадратів}{\stdinOrInputTxt}{\stdoutOrOutputTxt}{0,5 сек}{64 мегабайти}

Від заданого прямокутника розміром $A\times{}B$ (натуральні ч\'{и}сла) щоразу відрізається квадрат найбільшого розміру. Знайти число таких квадратів.

\InputFile
Два рядки, кожен з яких містить єдине натуральне число: 
у~першому рядку\nolinebreak[3] $A$, тобто розмір по\nolinebreak[3] висоті; 
у~др\'{у}гому рядку\nolinebreak[3] $B$, тобто розмір по\nolinebreak[3] ширині.
Обмеження на значення $A$, $B$ див. далі.

\OutputFile
Єдине число\nolinebreak[3] --- кількість квадратів.

\setlength{\mytemplen}{\parindent}

{

\def\exmpI{\protect\exmp{12\\
8}{3}}
\def\exmpII{\protect\exmp{2018\\
2018}{1}}
\def\exmpIII{\protect\exmp{17\\
42}{12}}

\def\squaresCutsNotePhrase{У~першому тесті маємо ситуацію з рисунку: 
спочатку відрізають квадрат ${8{\times}8}$ і лишається ${4{\times}8}$,
потім відрізають квадрат ${4{\times}4}$ і лишається ${4{\times}4}$, що і є останнім третім квадратом.
\ifAfour\else\par\fi
У~др\'{у}гому тесті відразу маємо квадрат, відрізати взагалі не доводиться, але один квадрат (початковий) все-таки~є.}
\def\squaresCutsPicture{\begin{mfpic}[4]{0}{8}{0}{12}
\xyswap
\lines{(1,0),(1,8)}
\lines{(2,0),(2,8)}
\lines{(3,0),(3,8)}
\lines{(4,0),(4,8)}
\lines{(5,0),(5,8)}
\lines{(6,0),(6,8)}
\lines{(7,0),(7,8)}
\lines{(9,0),(9,8)}
\lines{(10,0),(10,8)}
\lines{(11,0),(11,8)}
%
\lines{(0,1),(12,1)}
\lines{(0,2),(12,2)}
\lines{(0,3),(12,3)}
\lines{(0,4),(12,4)}
\lines{(0,5),(12,5)}
\lines{(0,6),(12,6)}
\lines{(0,7),(12,7)}
%
\pen{2pt}
\lines{(0,0),(0,8)}
\lines{(8,0),(8,8)}
\lines{(12,0),(12,8)}
\lines{(8,4),(12,4)}
\lines{(0,0),(12,0)}
\lines{(0,8),(12,8)}
\end{mfpic}}

\ifAfour
\Examples
\begin{exampleSimpleExtraNarrow}{2.5em}{3em}
\exmpI%
\end{exampleSimpleExtraNarrow}
\begin{exampleSimpleExtraNarrow}{2.5em}{3em}
\exmpII%
\end{exampleSimpleExtraNarrow}
\begin{exampleSimpleExtraNarrow}{2.5em}{3em}
\exmpIII%
\end{exampleSimpleExtraNarrow}
\par
\mytextandpicture{\Notes\squaresCutsNotePhrase}{\squaresCutsPicture}\par\vspace{-0.25\baselineskip}
% % % \myflfigaw{\squaresCutsPicture}
% % % \Notes
% % % \squaresCutsNotePhrase
\else
\noindent
\begin{minipage}{11em}
\hspace*{\mytemplen}\Examples\par
\begin{exampleSimple}{5em}{5em}
\exmpI%
\exmpII%
\exmpIII%
\end{exampleSimple}
\end{minipage}
\begin{minipage}{21em}
\squaresCutsNotePhrase
\end{minipage}
\squaresCutsPicture
\fi

% % % 17 42
% % % 17 25
% % % 17 8
% % % 9 8
% % % 1 8
% % % 1 7
% % % 1 6
% % % 1 5
% % % 1 4
% % % 1 3
% % % 1 2
% % % 1 1

}

\Scoring
В цій задачі тести перевіряються й оцінюються незалежно.
Половина балів припадає на тести, в яких обидва розміри від~1 до~100.
Інша половина\nolinebreak[3] --- на тести, в яких більший-або-рівний із розмірів від\nolinebreak[3] 1000 до\nolinebreak[3] $10^9$ (мільярда), а\nolinebreak[3] менший-або-рівний від 1 до більшого-або-рівного.

Здавати треба одну програму, а не дві; Вас просто інформують, у скількох тестах які обмеження на розміри.

\ifStatementOnly
\vfill\par
Всі задачі цього змагання, включно з цією задачею, заборонено повторно здавати в ejudge після того, як вже здано повнобальний розв'язок.
\fi

\end{problem}
