\Tutorial	
\MyParagraph{Навіщо там від'ємні ч\'{и}сла?}
Добуток двох від'ємних чисел додатній, так що відповідь може (але не~зобов'язана)
являти собою два від'ємних числ\'{а}. наприклад, для четвірки чисел 2, 3, --50, --60 найбільшим можливим добутком є ${({-}50)\times({-}60)}\dib{{=}}3000$.

\MyParagraph{Навіщо вказані діапазони значень?} При значеннях до~100, добуток поміщається у~майже будь-який числовий тип (крім однобайтових, як-то \verb"byte"), так що можна й не~приділяти особливої уваги типу. При значеннях до~$10^9$, добуток гарантовано поміщається лише у 64-\nolinebreak[3]бітових типах (\texttt{Int64} у~Pascal, \texttt{long\nolinebreak[3] long} у~C/C++, тощо). Див.\nolinebreak[3] також стор.~\pageref{text:overflow-example}. Втім, після переходу навесні 2019~р. з \mbox{32-}\nolinebreak[3]біто\-вої на \mbox{64-}\nolinebreak[3]біто\-ву архітектуру сервера значна частина розв'язків, які раніше набирали неповний бал с\'{а}ме з цієї причини, почали набирати повний бал. Що,~втім, ні\'{я}к не~заважає тому, щоб аналогічна проблема виникала на якійсь іншій системі автоматичної перевірки\dots{}

\MyParagraph{<<Лобовий>> повний розв'язок.}
Можна просто перебрати всі шість варіантів пар і вибрати з~них максимальний за звичайними правилами пошуку максимума. Якщо скласти ч\'{и}сла у~масив, перебір буде дещо приємнішим та <<більш алгоритмічним>>, див.\nolinebreak[3] \IdeOne{yQlAq0}. Але, раз кількість чисел рівно~4, то можна обійтися й без масиву і навіть без циклів, див.\nolinebreak[3] \IdeOne{VYnuIW}.

\MyParagraph{Альтернативний розв'язок.} При виборі пари з~конкретно \mbox{4-х} чисел, цей спосіб не~кращий, а~просто інший. Можна довести (пропонуємо читачам зробити це доведення самостійно), що максимальний добуток завжди дають або два максимальні числ\'{а}, або два мінімальні (якщо вони від'ємні й великі за модулем). Іншими словами: якщо відсортувати масив, то відповідь утворять або останній та передостанній елементи, 
або перший та др\'{у}гий. 
% або \mbox{1-й} та \mbox{2-й}. 
Якщо мова програмування має готове (бібліотечне) сортування, це досить зручно (див.\nolinebreak[3] \IdeOne{exmAiY}), а за~умови ефективного сортування ще~й швидше за попередій спосіб (ця~відмінність стає помітною при кількостях чисел від кількох десятків тисяч). Можна пробувати й додатково прискорити цей підхід, бо навіть ефективне сортування відбувається повільніше, ніж вибір самих лише максимального, наступного максимального, мінімального та наступного мінімального елементів. Але\nolinebreak[2] це\nolinebreak[2] громіздко, в~ньому легко помил\'{и}тися, тож писати таке було~б варто \emph{лише якби} чисел було дуже багато (сотні тисяч чи ще~більше).