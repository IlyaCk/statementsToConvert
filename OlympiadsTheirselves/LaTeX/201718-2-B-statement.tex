{

\begin{problemAllDefault}{Точки та круг}

На площині задано круг з центром у початку координат та набір точок.
Напишіть програму, що знаходитиме, як(а/і) з точок леж(и/а)ть ззовні круга, але найближче до нього; якщо є різні такі точки, програма має знайти їх усі.

\InputFile
Перший рядок вхідних даних містить єдине ціле число $N$ (${1\,{\<}\,N\,{\<}\,12345}$)\nolinebreak[3] --- кількість точок. Другий рядок містить єдине ціле число $R$ (${1\,{\<}\,R{\<}\,12345}$)\nolinebreak[3] --- радіус круга. Кожен з подальших $N$ рядків містить рівно два цілі числа\nolinebreak[3] --- $x$- та $y$-координати відповідної точки. Всі координати є цілими числами, які не~перевищують за абсолютною величиною (модулем)~12345.

\OutputFile
Якщо шуканих точок не~існує (жодна з уведених точок не~ззовні круга), програма має  вивести єдиний рядок з єдиним числом~0.

Інакше, програма має вивести перелік номерів усіх тих точок, які лежать ззовні круга і мають однакову мінімальну відстань від нього. Кількість точок у переліку виводити не~треба, сам перелік виводити у порядку зростання номерів.



\Examples

\ifAfour
\vspace*{-0.75\baselineskip}
\fi
\noindent
\ifAfour
\hspace*{-1.25em}
\fi
\begin{exampleSimpleThree}{5em}{5em}{\ifAfour 7em\else24em\fi}{Примітки}
\exmp{2
50
30 40
17 23}{0}{Одна з точок на межі круга, інша всередині нього, жодної точки ззовні}%
\ifAfour%
\end{exampleSimpleThree}%
\hspace{-2em}
\begin{exampleSimpleThree}{5em}{5em}{\ifAfour 10em\else24em\fi}{Примітки}
\fi
\exmp{5
10
30 40
40 30
-30 -40
777 555
0 -50}{1 2 3 5}{Зразу чотири з п'яти точок мають однакову мінімальну відстань (50 до центру круга, або \mbox{50--10=40} до найближчої його частини)}%
\end{exampleSimpleThree}

\end{problemAllDefault}

}