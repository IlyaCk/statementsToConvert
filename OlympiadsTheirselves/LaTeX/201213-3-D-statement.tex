\begin{problem}{Кава --- coffee}{coffee.dat}{coffee.sol}{1 сек}{64 Мб}

\looseness=-1
Програміст Василь дуже полюбляє пити каву. Про нього ще говорять, що він перетворює каву в код. Василь знає, якщо він вип'є чашку кави перед виконанням певного завдання, то він витратить на нього на 20\% менше часу, ніж без кави. Але на заварювання кави теж необхідно витратити певний час.

Вам необхідно визначити, за яку мінімальну кількість робочих днів Василь зможе справитися з усіма своїми завданнями, якщо у нього є запас кави на $K$ чашок. Василь вже наперед визначив необхідну кількість часу для кожного завдання. Завдання необхідно виконувати послідовно. Якщо залишок робочого часу не~дозволяє виконати наступне завдання, то Василь почне його виконувати наступного дня. 

Зверніть увагу, що магічна дія чашки кави впливає лише на одне завдання, і що не~можна випивати перед виконанням завдання більше однієї чашки кави.

\Task
Напишіть програму \texttt{coffee}, яка б знаходила мінімальну кількість днів, які необхідно потратити на виконання всіх завдань.

\InputFile
Перший рядок вхідного файлу \texttt{coffee.dat} містить три цілих числа $N$, $K$,\nolinebreak[2] $L$\nolinebreak[3] ---  кількість завдань, кількість чашок кави та тривалість заварювання однієї чашки кави ($1\dib{{\<}}N\dib{{\<}}1000$,\hspace{0.25em plus 0.25em} $0\dib{{\<}}K\dib{{\<}}1000$,\hspace{0.25em plus 0.25em} $1\dib{{\<}}L\dib{{\<}}100$). Наступний рядок містить $N$ цілих чисел, розділених пробілами\nolinebreak[3] --- необхідний обсяг часу для виконання кожного завдання (час\nolinebreak[2] задано в хвилинах, кожне число не~менше\nolinebreak[2] за\nolinebreak[3] 1 і не~більше\nolinebreak[2] за\nolinebreak[3] 480).

\OutputFile
Ваша програма має створити текстовий файл \texttt{coffee.sol} і вивести туди єдине ціле число\nolinebreak[3] --- мінімальну кількість днів, які необхідно потратити на виконання всіх завдань.

\Examples
\ifAfour\else
\par\noindent
\fi
\begin{exampleWidthsAndDefaultFileNames}{8em}{3em}
\exmp{5 1 10
10 10 10 100 360}{1}%
\end{exampleWidthsAndDefaultFileNames}
\begin{exampleWidthsAndDefaultFileNames}{9.5em}{3em}
\exmp{5 2 5
200 281 240 240 480}{3}%
\end{exampleWidthsAndDefaultFileNames}

\end{problem}

