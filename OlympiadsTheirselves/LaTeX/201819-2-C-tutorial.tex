\Tutorial % \MyParagraph{Розв'язок на 75\%.}
\MyParagraph{Чому очевидний підхід набирає так мало балів?}
Справді обчислювати сам факторіал і шукати його останню ненульо\-ву цифру\nolinebreak[3] --- формально правильно, але набирає дуже мало балів. Факторіал дуже швидко зростає: $8!\dib{{=}}40\,320$ вже не\nolinebreak[3] поміщається у\nolinebreak[3] \mbox{16-бі}\-то\-вий знаковий цілий тип, $13!\dib{{=}}6\,227\,020\,800$\nolinebreak[3] --- у\nolinebreak[3] \mbox{32-бі}\-то\-вий,
$21!\dib{{=}}51\,090\,942\,171\,709\,440\,000$\nolinebreak[3] --- у\nolinebreak[3] \mbox{64-бі}\-то\-вий. Так що для мов програмування, в яких найбільшим цілочисельним типом є \mbox{64-бі}\-то\-вий, цей підхід правильний лише для ${N\,{\<}\,20}$.

Межу ${N\,{\<}\,20}$ можна збільшити до ${N\,{\<}\,23}$, якщо 
% при кожному домноженні, де з'являється 
щоразу, коли додається
нуль наприкінці, тут же відкидати його%. Якось так
: $1
\dib{{\xrightarrow{\*2}}}2
\dib{{\xrightarrow{\*3}}}6
\dib{{\xrightarrow{\*4}}}24
\dib{{\xrightarrow[/10]{\*5}}}12
\dib{{\xrightarrow{\*6}}}72
\dib{{\xrightarrow{\*7}}}504
\dib{{\xrightarrow{\*8}}}4032
\dib{{\xrightarrow{\*9}}}36288
\dib{{\xrightarrow[/10]{\*10}}}36288
\dib{{\xrightarrow{\*11}}}\dots$

Деякі з дозволених на олімпіаді мов програмування мають вбудовану довгу арифметику, тобто вміють працювати зі значно більшими цілими числами: Python\nolinebreak[3] --- просто; Java\nolinebreak[3] --- через бібліотечний тип \texttt{BigInteger}. (C\# теж має тип \texttt{BigInteger}, але там усе сумніше, бо засоби, якими \texttt{BigInteger} підключається на локальному Windows-комп'ютері, не~працюють на Linux-сервері ejudge, і авторам збірника невідомо ні як їх підключити там, ні, навіть, чи~можливо це взагалі.) Використання довгої арифметики (байдуже, готової чи написаної власноруч) дозволяє дещо розширити діапазон~$N$, але недостатньо: з~нею повинні проходити не~лише всі тести з $N\,{\<}\,23$, а й усі блоки тестів з $N\,{\<}\,555$. А~подальші блоки повинні не~проходити: наприклад, 50000! має понад 200~тис. десяткових цифр, так що обчислення не~мають шансів поміститися в обмеження за часом.

\MyParagraph{То що ж робити?}
Раз питають \emph{лише останню} ненульову цифру, \emph{нема потреби} обчислювати та зберігати \emph{всі} цифри факторіалів. 
% З~іншого боку, це також і 
Але це й 
\emph{не}~класична <<модульна арифметика>> (в~якій досить зберігати % одну 
лише останню цифру), бо питають \emph{останню ненульову}, кількість нулів наприкінці $N!$ зі зростанням~$N$ час від часу збільшується, 
% так що 
й потрібний розряд час від часу зсувається: 
при $1\dib{{\<}}N\dib{{\<}}4$ це розряд одиниць (остання цифра),
при $5\dib{{\<}}N\dib{{\<}}9$\nolinebreak[3] --- розряд десятків (передостання цифра), тощо.

% % % Чи~правда, що при домноженні на кожне число, кратне~5, наприкінці факторіалу з'являється ще\nolinebreak[3] один нуль? Так, якщо мати на\nolinebreak[2] увазі <<хоча~б ще~один>> (буде доведено трохи пізніше), але ні, якщо мати на\nolinebreak[2] увазі <<в~точності ще~один>>. Наприклад, при домноженні 4$!$\nolinebreak[2] на~5 з'являється ще~один ноль, а при домноженні 999$!$\nolinebreak[2] на~1000 з'являється зразу ще три нулі.

Спробуємо з'ясувати, коли скільки нулів додається наприкінці факторіалу. 
% Тут важливо, 
Важливо, 
що розглядаємо с\'{а}ме десяткову систему числення 
% (де\nolinebreak[3] 10\nolinebreak[2] розкладається на прості множники як\nolinebreak[3] $2^1\cdot5^1$) 
($10\dibbb{{=}}2^1\cdot5^1$) 
і с\'{а}ме $N! \dibbb{{=}} 1\dib{{\times}}2\dib{{\times}}\dots\dib{{\times}}N$. Яке\nolinebreak[3] б\nolinebreak[1] не\nolinebreak[3] було число вигляду\nolinebreak[2] ${5^{a}\times{}b}$, число ${2^{a}\times{}b}$ (при\nolinebreak[2] тих\nolinebreak[3] самих % цілих 
додатних\nolinebreak[2] $a$,~$b$) існує і менше\nolinebreak[2] ${5^{a}\times{}b}$. Так що, при послідовному домноженні $(1\dib{{\times}}2\dib{{\times}}\dots\dib{{\times}}(N{-}1))\dib{{\times}}N$, на кожну п'ятірку, скільки\nolinebreak[3] б\nolinebreak[2] їх\nolinebreak[2] не\nolinebreak[3] було в розкладенні~$N$, <<вже чек\'{а}є>> відповідна кількість двійок, наявних у розкладенні\nolinebreak[3] $(N{-}1)!$ і ще\nolinebreak[3] не\nolinebreak[3] використаних іншими п'ятірками. Так\nolinebreak[3] що кожне число, яке можна подати як ${5^{a}\times{}b}$ (де\nolinebreak[3] $b$\nolinebreak[2] не~кратне~5) додає наприкінці факторіала рівно $a$\nolinebreak[3] нулів.

\MyParagraph{Спосіб \textnumero$\,$1 отримати повні бали (крім Python).}
Оскільки $5^9\dib{{=}}1\,953\,125\dib{{<}}2\,018\,019\dib{{<}}9\,765\,625\dib{{=}}5^{10}$, % то % при вказаних %%% обмеженнях на~$N$ 
за одне % чергове 
домноження з'являтиметься щонайбільше дев'ять нулів. 
І~ніщо не~заважає поєднати такі % дві 
ідеї: (1)~відкидати нулі наприкінці, як\nolinebreak[3] тільки вони з'являються; 
%(див.\nolinebreak[3] др\'{у}гий абзац цього розбору); 
(2)~зберігати (у~\mbox{64-бі}\-то\-вій змінній) останні \mbox{10--12}\nolinebreak[2] десяткових цифр. 
% Якось, наприклад, так: 
Якось так: 
$1
\dibbb{{\xrightarrow{\*2}}}2
\dibbb{{\xrightarrow{\*3}}}6
\dibbb{{\xrightarrow{\*4}}}24
\dibbb{{\xrightarrow[/10]{\*5}}}12
\dibbb{{\xrightarrow{\*6}}}72
\dibbb{{\xrightarrow{\*7}}}504
\dibbb{{\xrightarrow{\*8}}}4032
\dibbb{{\xrightarrow{\*9}}}36288
\dibbb{{\xrightarrow[/10]{\*10}}}36288
\dibbb{{\xrightarrow{\*11}}}399168	
\dibbb{{\xrightarrow{\*12}}}4790016	
\dibbb{{\xrightarrow{\*13}}}62270208	
\dibbb{{\xrightarrow{\*14}}}871782912	
\dibbb{{\xrightarrow[/10]{\*15}}}1307674368	
\dibbb{{\xrightarrow[\bmod10^{10}]{\*16}}}922789888	
\dibbb{{\xrightarrow[\bmod10^{10}]{\*17}}}5687428096	
\dibbb{{\xrightarrow[\bmod10^{10}]{\*18}}}2373705728	
\dibbb{{\xrightarrow[\bmod10^{10}]{\*19}}}5100408832	
\dibbb{{\xrightarrow[/10,\quad\bmod10^{10}]{\*20}}}200817664	
\dibbb{{\xrightarrow{\*21}}}4217170944	
\dibbb{{\xrightarrow[\bmod10^{10}]{\*22}}}2777760768	
\dibbb{{\xrightarrow[\bmod10^{10}]{\*23}}}3888497664	
\dibbb{{\xrightarrow[\bmod10^{10}]{\*24}}}3323943936	
\dibbb{{\xrightarrow[/100]{\*25}}}830985984	
\dibbb{{\xrightarrow[\bmod10^{10}]{\*26}}}1605635584	
\dibbb{{\xrightarrow[\bmod10^{10}]{\*27}}}\dots$\linebreak[2]
Для\nolinebreak[3] ${N\,{<}\,5^{10}}$, \mbox{10--12} цифр водночас 
і\nolinebreak[3] досить багато, щоб не~втратити зразу всі ненульові цифри, і\nolinebreak[3] досить мало, щоб при черговому 
домноженні ніколи не~виникало переповнення \mbox{64-бі}\-то\-вого типу. 


\MyParagraph{Спосіб \textnumero$\,$2 отримати повні бали (крім Python).}
Вже відзначено, що класична модульна арифметика (зберігати % саму 
лише останню цифру) незастосовна, бо питають останню \emph{ненульову}, й потрібний розряд час від часу зсувається. Але % з урахуванням усього досі сказаного 
можна помітити, що раз кількість нулів наприкінці визначається кількістю п'ятірок, то, прибравши з\nolinebreak[2] $N!$ відповідну кількість п'ятірок і таку саму кількість двійок, отримаємо якраз факторіал крім нулів наприкінці; для цього факторіала крім нулів наприкінці \emph{можна} зберігати й обчислювати сам\'{у} лише останню цифру.

Конкретніше, будемо 
% при послідовних домноженнях 
при обчисленні
$1\dib{{\times}}2\dib{{\times}}\dots\dib{{\times}}i\dib{{\times}}\dots$ <<виймати>> з кожного % чергового 
числ\'{а} всі множники\nolinebreak[3] 2\nolinebreak[2] та~5, і рахувати окремо останню цифру добутку \emph{всього, крім} <<вийнятих>> двійок та п'ятірок, окремо різницю кількостей <<вийнятих>> (на~скільки двійок % було 
більше, ніж п'ятірок). Якось так: 

\vspace{-0.5\baselineskip}

{
\ifAfour
\def\widestColumn{0.65\textwidth}
\else
\def\widestColumn{0.65\textwidth}
\fi

% % % \ifAfour
% % % \begin{longtable}{@{}c|c|l|p{0.65\textwidth}|c|c@{}}
% % % \else
% % % \begin{longtable}{@{}c|c|l|p{0.6\textwidth}|c|c@{}}
% % % \fi
\begin{longtable}{@{}c|c|l|p{\widestColumn}|c|c@{}}
\multicolumn{2}{@{}c|}{Було}
&
Множимо на
&
\multicolumn{1}{c|}{Коментар}
&
\multicolumn{2}{c}{Стає}
\\\hline
\endhead
1 & 0 &
~$2=2^1\cdot5^0\cdot1$
&
\begin{minipage}{\widestColumn}\begin{small}\setstretch{0.75}
двійок стає на одну більше
\par\end{small}\end{minipage}
&
1 & 1 \\\hline
1 & 1 &
~$3=2^0\cdot5^0\cdot3$
&
\begin{minipage}{\widestColumn}\begin{small}\setstretch{0.75}
кількості двійок та п'ятірок незмінні, добуток решти домножується на~3
\par\end{small}\end{minipage}
&
3 & 1 \\\hline
3 & 1 &
~$4=2^2\cdot5^0\cdot1$
&
\begin{minipage}{\widestColumn}\begin{small}\setstretch{0.75}
двійок стає на дві більше
\par\end{small}\end{minipage}
&
3 & 3 \\\hline
3 & 3 &
~$5=2^0\cdot5^1\cdot1$
&
\begin{minipage}{\widestColumn}\begin{small}\setstretch{0.75}
<<зайвих>> двійок стає на одну менше, бо одна двійка <<зв'язується>> п'ятіркою
\par\end{small}\end{minipage}
&
3 & 2 \\\hline
3 & 2 &
~$6=2^1\cdot5^0\cdot3$
&
\begin{minipage}{\widestColumn}\begin{small}\setstretch{0.75}
і~двійок стає на одну більше, і~добуток решти домножується на~3
\par\end{small}\end{minipage}
&
9 & 3 \\\hline
9 & 3 &
~$7=2^1\cdot5^0\cdot7$
&
\begin{minipage}{\widestColumn}\begin{small}\setstretch{0.75}
кількості двійок та п'ятірок незмінні, добуток решти домножується на~7, зберігаємо з~63 лише останню цифру~3
\par\end{small}\end{minipage}
&
3 & 3 \\\hline
3 & 3 &
~$8=2^3\cdot5^0\cdot1$
&
\begin{minipage}{\widestColumn}\begin{small}\setstretch{0.75}
двійок стає на три більше
\par\end{small}\end{minipage}
&
3 & 6 \\\hline
3 & 6 &
~$9=2^0\cdot5^0\cdot9$
& &
7 & 6 \\\hline
7 & 6 &
$10=2^1\cdot5^1\cdot1$
&
\begin{minipage}{\widestColumn}\begin{small}\setstretch{0.75}
додаються одна двійка й одна п'ятірка, \emph{різниця} кількостей незмінна
\par\end{small}\end{minipage}
&
7 & 6 \\\hline
7 & 6 &
$11=2^0\cdot5^0\cdot11$
& &
7 & 6 \\\hline
7 & 6 &
$12=2^2\cdot5^0\cdot3$
& &
1 & 8 \\\hline
1 & 8 &
$13=2^0\cdot5^0\cdot13$
& &
3 & 8 \\\hline
3 & 8 &
$14=2^1\cdot5^0\cdot7$
& &
1 & 9 \\\hline
1 & 9 &
$15=2^0\cdot5^1\cdot3$
&
\begin{minipage}{\widestColumn}\begin{small}\setstretch{0.75}
<<зайвих>> двійок стає на одну менше, бо одна двійка <<зв'язується>> п'ятіркою;
крім того, добуток решти домножується на~3
\par\end{small}\end{minipage}
&
3 & 8 
% % % % % % \\\hline
% % % % % % % % % 3 & 8 &
% % % % % % % % % $16=2^4\cdot5^0\cdot1$
% % % % % % % % % & &
% % % % % % % % % 3 & 12 \\\hline
% % % % % % % % % 3 & 12 &
% % % % % % % % % $17=2^0\cdot5^0\cdot17$
% % % % % % % % % & &
% % % % % % % % % 1 & 12 \\\hline
% % % % % % % % % 1 & 12 &
% % % % % % % % % $18=2^1\cdot5^0\cdot9$
% % % % % % % % % & &
% % % % % % % % % 9 & 13
\end{longtable}

\vspace{-\baselineskip}

І так далі. 
%
Дорахувавши до % останнього 
множника~$N$ (включно), знаємо останню цифру <<$N!$~без\nolinebreak[2] двійок та п'ятірок>> (позначимо як~$d$) та кількість <<вільних>> (не~<<зв'язаних>> п'ятірками) двійок (позначимо як~$k$). Остат\'{о}чна відповідь може бути виражена як 
$(d\cdot2^k)\bmod10
\dib{{=}}
\bigl(d\cdot(2^k\bmod10)\bigr)\bmod10$. Враховуючи, що
${2^0\,{=}\,1}$,
${2^1\,{=}\,2}$,
${2^2\,{=}\,4}$,
${2^3\,{=}\,8}$,
\raisebox{0pt}[1ex][0pt]{${2^4\,{=}\,1\underline{\underline{6}}}$},
\raisebox{0pt}[1ex][0pt]{${2^5\,{=}\,3\underline{\underline{2}}}$},
\raisebox{0pt}[1ex][0pt]{${2^6\,{=}\,6\underline{\underline{4}}}$},
\raisebox{0pt}[1ex][0pt]{${2^7\,{=}\,12\underline{\underline{8}}}$},
\raisebox{0pt}[1ex][0pt]{${2^8\,{=}\,25\underline{\underline{6}}}$},~\dots,
тобто при ${k\,{>}\,0}$ останні цифри циклічно повторюються,
велике~$k$ можна замінити на\nolinebreak[2] ${(k\bmod4)}\dib{{+}}4$, 
що\nolinebreak[3] дає можливість обчислювати $2^k\bmod10$ за~$\Theta(1)$.
Втім, це мало на що впливає, бо основний процес домножень (із~<<вийманням>> з кожного % чергового 
числ\'{а} двійок та п'ятірок) значно довший.

% % % Якщо умовно, суто щоб навести в цьому абзаці приклад, писати останню цифру добутку всього, крім двійок та п'ятірок, ліворуч угорі, а різницю кількостей двійок і п'ятірок праворуч унизу, то процес можна зобразити якось так:
% % % ${}^{1}{}_{0}
% % % \dibbb{{\xrightarrow{\*2=2^1\cdot5^0\cdot1}}}{}^{1}{}_{1}$%
% % % (двійок стало на одну більше)%
% % % ${}\dibbb{{\xrightarrow{\*3=2^0\cdot5^0\cdot3}}}{}^{3}{}_{1}$%
% % % (кількості двійок та п'ятірок незмінні, добуток решти домножується на~3)%
% % % $\dibbb{{\xrightarrow{\*4=2^2\cdot5^0\cdot1}}}{}^{3}{}_{3}$%
% % % (двійок стало на дві більше)%
% % % $\dibbb{{\xrightarrow{\*5=2^0\cdot5^1\cdot1}}}{}^{3}{}_{2}$%
% % % (<<зайвих>> двійок стало на одну менше, бо одна двійка <<зв'язується>> п'ятіркою)$
% % % $\dibbb{{\xrightarrow{\*6=2^1\cdot5^0\cdot3}}}{}^{9}{}_{3}$%
% % % (і~двійок стало на одну більше, і~добуток решти домножується на~3)%
% % % $\dibbb{{\xrightarrow{\*7=2^1\cdot5^0\cdot7}}}{}^{3}{}_{3}$%
% % % (кількості двійок та п'ятірок незмінні, добуток решти 9 домножується на~7, зберігаємо з~63 лише останню цифру~3)%
% % % $\dibbb{{\xrightarrow{\*8=2^3\cdot5^0\cdot1}}}{}^{3}{}_{6}$%
% % % (двійок стало на три більше)%
% % % $\dibbb{{\xrightarrow{\*9=2^0\cdot5^0\cdot9}}}{}^{7}{}_{6}
% % % \dibbb{{\xrightarrow{\*10=2^1\cdot5^1\cdot1}}}{}^{7}{}_{6}$%
% % % (додаються одна двійка й одна п'ятірка, \emph{різниця} кількостей незмінна)%
% % % $
% % % \dibbb{{\xrightarrow{\*11=2^0\cdot5^0\cdot11}}}{}^{7}{}_{6}
% % % \dibbb{{\xrightarrow{\*12=2^2\cdot5^0\cdot3}}}{}^{1}{}_{8}
% % % \dibbb{{\xrightarrow{\*13=2^0\cdot5^0\cdot13}}}{}^{3}{}_{8}
% % % \dibbb{{\xrightarrow{\*14=2^1\cdot5^0\cdot7}}}{}^{1}{}_{9}
% % % \dibbb{{\xrightarrow{\*15=2^0\cdot5^1\cdot3}}}{}^{3}{}_{8}
% % % \dibbb{{\xrightarrow{\*16=2^4\cdot5^0\cdot1}}}{}^{3}{}_{12}
% % % \dibbb{{\xrightarrow{\*17=2^0\cdot5^0\cdot17}}}{}^{1}{}_{12}
% % % \dibbb{{\xrightarrow{\*18=2^1\cdot5^0\cdot9}}}{}^{9}{}_{13}\dots
% % % $

}


\MyParagraph{Який з цих способів кращий?} Важко сказати. Вони обидва мають складність $\Theta(N)$. (\mbox{Може} здатися, ніби для др\'{у}гого % забули, що 
є ще внутрішні цикли <<викидання>> двійок і п'ятірок, які дають додатковий множник $\log{}N$; але % при глибшому розгляді 
це\nolinebreak[3] не~так. Наприклад, сумарну кількість ітерацій, які шукають двійки, можна виразити як 
${N\mathbin{\mathrm{div}}2}\dibbb{{+}}
{N\mathbin{\mathrm{div}}2^2}\dib{{+}}
\dots\dibbb{{+}}
{N\mathbin{\mathrm{div}}2^k}$,
де ${k\,{=}\,\lfloor\log_2{N}\rfloor}$;
див.\nolinebreak[2] аргументацію формули~(\ref{eq:201213-2-C-binary}) на\nolinebreak[2] стор.~\pageref{eq:201213-2-C-binary}. Формула суми геометричної прогресії каже, що ця сума менша~$N$.)
При вказаних обмеженнях, обидва ці способи гарантовано проходять усі тести (всіма доступними на олімпіаді мовами програмування, \emph{крім Python}; див.\nolinebreak[3] наступний абзац). Теоретичний недолік першого способу\nolinebreak[3] --- якби дозволялися значення $N\dib{{\>}}5^{11}\dib{{\approx}}{}$48,8~млн, то \mbox{64-бі}\-то\-вого типу вже було~б недостатньо, щоб задовольнити обидві вимоги <<не~втратити зразу всі ненульові цифри>> та <<не~утворювати переповнень>>; др\'{у}гому ж способу цілком достатньо як \mbox{64-бі}\-то\-вого, так і навіть \mbox{32-бі}\-то\-вого типу аж до $N\dib{{\approx}}\frac{2^{31}}{9}\dib{{\approx}}$238~млн (але при ${N\,{\approx}\,10^8}$ обчислення вже тривали~б десятки секунд).
% % % Теоретичний недолік др\'{у}гого\nolinebreak[3] --- він потреб\'{у}є % помітно 
% % % більше % досить 
% % % повільних операцій\nolinebreak[3] \texttt{mod}; але експерименти показують, що це не~робить його помітно повільнішим за перший.

\MyParagraph{Навіщо скривдили Python?}
Він сам себе скривдив.
Програмам усіма мовами програмування виділявся однаковий ліміт часу.
Але для розрахунків, вжитих в обох цих способах, Python виявився значно повільнішим не~лише за традиційно швидкі мови \verb"g++" та\nolinebreak[3] \verb"fpc", а\nolinebreak[3] також і за відносно повільні \verb"java", \verb"pasabc-"\nolinebreak[3]\verb"linux" та\nolinebreak[3] \verb"mcs". Враховуючи цю об'єктивну і значну різницю швидкодії, перед туром було прийняте свідоме рішення змиритися з тим, що реалізації одного алгоритму різними мовами набирають різні бали. З~точки зору правил та традицій Всеукраїнської олімпіади з інформатики, це~погано, але в такого роду виключних випадках допускається. До~того~ж, у~цій самій задачі Python дозволяв отримати бали за тести вигляду $25\,{\<}\,N\,{\<}\,555$, пишучи <<лобовий>> розв'язок і взагалі не~замислюючись над тим, що така програма використовує довгу арифметику. Учасники, що вміють писати лише на Pascal чи~C++, такої можливості не~мали, і~їм це ніяк не~компенсувалося. Тож\nolinebreak[2] Python вигра\'{є} в\nolinebreak[3] одному, програ\'{є} в\nolinebreak[3] іншому, й навряд\nolinebreak[3] чи було~б справедливішим надавати йому додаткові переваги, збільшуючи ліміт часу. Тим\nolinebreak[3] п\'{а}че, що існує ще наступний спосіб.

\MyParagraph{Спосіб \textnumero$\,$3 отримати повні бали (включно з Python).}%%%\phantomsection\label{text:201819-2-C-contr-example-for-div-in-ring}
Обмеження ${N\,{\<}\,2\,018\,019}$ підбиралося під те, щоб переважною більшістю мов працювали й набирали повний бал способи 1--2. 
Але задачу \emph{можна} розв'язати ще значно ефективніше, якщо правильно модифікувати початий у способі~\textnumero$\,$2 підхід <<двійки компенсують п'ятірки, для решти рахуємо останню цифру>>.
Вважаємо гарантованим ${N\,{\>}\,10}$, бо для ${N\,{<}\,10}$ можна написати окремий\nolinebreak[3] \texttt{if}, щоб рахувати <<в~лоб>>.
Виділимо окремо множники, кратні~5, окремо решту, причому решту розіб'ємо на групи згідно десятків.
Наприклад, для 37! буде так: $
{(1{\times}2{\times}3{\times}4{\times}6{\times}7{\times}8{\times}9)}\dibbb{{\times}}(5{\times}10)\dibbb{{\times}}
{(11{\times}12{\times}13{\times}14{\times}16{\times}17{\times}18{\times}19)}\dibbb{{\times}}(15{\times}20)\dibbb{{\times}}
{(21{\times}22{\times}23{\times}24{\times}26{\times}27{\times}28{\times}29)}\dibbb{{\times}}(25{\times}30)\dibbb{{\times}}
{(31{\times}32{\times}33{\times}34{\times}36{\times}37)}\dibbb{{\times}}(35)$.
Добуток $(5{\times}10)\dibbb{{\times}}
(15{\times}20)\dibbb{{\times}}
(25{\times}30)\dibbb{{\times}}
(35)$ можна подати як $
((5{\cdot}1)\,{\times}\,
(5{\cdot}2))\dib{{\times}}
((5{\cdot}3)\,{\times}\,
(5{\cdot}4))\dib{{\times}}
((5{\cdot}5)\,{\times}\,
(5{\cdot}6))\dib{{\times}}
(5{\cdot}7)
\dibbb{{=}}
{5^7\,{\times}\,{7!}}$.
(Важливо, що з~кожного числ\'{а}, кратного~5, виділяється рівно одна п'ятірка, навіть якщо, як~з~25, їх можна було~б виділити більше.) %%% Можна, але треба одну.)
Так можна робити й при інших~$N$, отримуючи $5^{N\mathbin{\mathrm{div}}5}\dib{{\times}}{(N\mathbin{\mathrm{div}}5)!}$. Тобто, з початкової задачі для $N!$ можна виділити таку саму задачу для ${(N\mathbin{\mathrm{div}}5)!}$, і\nolinebreak[2] при\nolinebreak[2] цьому решта множників поділені на більш-менш зручні групи по~8 з~кожного десятку (в~останній групі, й\nolinebreak[2] лише у\nolinebreak[3] ній, може бути менше).
Тож будемо окремо заново розв'язувати всю задачу для ${(N\mathbin{\mathrm{div}}5)!}$, і окремо компенсувати п'ятірки, зібрані\nolinebreak[3] в~$5^{N\mathbin{\mathrm{div}}5}$, двійками з решти добутків.

Тобто, ${(N\mathbin{\mathrm{div}}5)}$ двійок треба <<вийняти>> з тих решти добутків. 
І~міркувати <<щоб вийняти множник з добутку, поділимо цей добуток на цей множник>> тут не~можна, бо в модульній арифметиці це часто\nolinebreak[2] не~так. 
% % % (Навіть для звичайних чисел, \mbox{як-то}~$\mathbb{R}$, це пра\-вильно не~абсолютно завжди, а~крім~0. 
% % % А~для\nolinebreak[2] остач виявляється ще складнішим.) 
Наприклад, ${26\bmod10}\dibbb{{=}}6\dibbb{{=}}{56\bmod10}$, тобто з~точки зору остач за модулем~10 ч\'{и}сла 26 та~56 однаковісінькі, але
${\tfrac{26}{2}\bmod10}\dibbb{{=}}{13\bmod10}\dibbb{{=}}3\dibbb{{\neq}}8\dibbb{{=}}{28\bmod10}\dibbb{{=}}{\tfrac{56}{2}\bmod10}$).

Спробуємо компенсувати 
дві п'ятірки з $(5{\times}10)$ двома двійками, <<вийнятими>> з ${(1{\times}2{\times}3{\times}4}\dib{{\times}}{6{\times}7{\times}8{\times}9)}$,
дві п'ятірки з $(15{\times}20)$ двома двійками з ${(11{\times}12{\times}13{\times}14}\dib{{\times}}{16{\times}17{\times}18{\times}19)}$, 
% і~т.~д.\linebreak[2]
тощо.\linebreak[2]
До\nolinebreak[3] \mbox{того}, що остання група може бути неповна, повернемося пізніше. Тим,\nolinebreak[2] що\nolinebreak[2] в\nolinebreak[3] кожних таких \mbox{8-}\nolinebreak[3]множ\-ни\-ко\-вих дужках можна знайти більше, ніж дві, двійки, знехтуємо: 
% знайти можна більше, але 
їх\nolinebreak[3] більше, але 
% шукатимемо рівно\nolinebreak[3] дві. 
<<вийматимемо>>\nolinebreak[3] дві. 

Візьмемо число, що закінчується цифрою~2, та число, що закінчується цифрою~4, й <<виймемо>> з кожного по~одній двійці, поділивши кожне окремо на~2 (ділимо сам\'{і} ч\'{и}сла, не~остачі). Можуть бути рівно два випадки: або ці ч\'{и}сла являють собою $20k\,{+}\,2$ та $20k\,{+}\,4$, або $20k\,{+}\,12$ та $20k\,{+}\,14$;\linebreak[2]
у~першому випадку після ділень виходить $10k\,{+}\,1$ та\nolinebreak[3] $10k\,{+}\,2$,\linebreak[2]
у~др\'{у}гому $10k\,{+}\,6$ та\nolinebreak[3] $10k\,{+}\,7$.\linebreak[2]
Оскільки в~межах цих \mbox{8-}\nolinebreak[3]множ\-ни\-ко\-вих дужок більше не~треба ділень, можна вертатися до % звичайної
модульної арифметики, відкидати $10k$ і казати, що ${1{\times}2\,{=}\,2}$ та ${6{\times}7\,{=}\,42}$ закінчуються на однакову \mbox{цифру}~2.
Лишається домножити це~2 на останні цифри решти шести множників у дужках (1,~3,\nolinebreak[2] 6,\nolinebreak[2] 7,\nolinebreak[2] 8,~9), і отримати, що остання цифра добутку (після <<виймання>> двох двійок) дорівнює  
${(
2\,{\cdot}\,
1\,{\cdot}\,
3\,{\cdot}\,
6\,{\cdot}\,
7\,{\cdot}\,
8\,{\cdot}\,
9)\bmod10}
% % % \dibbb{{=}}
% % % {18144\bmod10}
\dib{{=}}4$.
% % % Таких дужок ${N\mathbin{\mathrm{div}}10}$ 
% % % % штук\nolinebreak[3] --- значить, 
% % % штук, тож
% % % остання цифра 
% % % загального добутку всіх дужок
Остання цифра загального добутку всіх ${N\mathbin{\mathrm{div}}10}$ штук таких дужок
рівна ${4^{N\mathbin{\mathrm{div}}10}\bmod10}$. 
Оскільки 
${4^1\,{=}\,4}$,
\raisebox{0pt}[1ex][0pt]{${4^2\,{=}\,1\underline{\underline{6}}}$},
\raisebox{0pt}[1ex][0pt]{${4^3\,{=}\,6\underline{\underline{4}}}$},
\raisebox{0pt}[1ex][0pt]{${4^4\,{=}\,25\underline{\underline{6}}}$},~\dots,
% враховуючи, що розглядаємо лише 
а розглядаємо випадок
${N\,{\>}\,10}$, тобто ${(N\mathbin{\mathrm{div}}10)\,{\>}\,1}$,
значення ${4^{N\mathbin{\mathrm{div}}10}\bmod10}$ можна знайти % за~$\Theta(1)$, 
якось у стилі \texttt{if\nolinebreak[3] \mbox{((N$\,\,$div$\,\,$10)$\,$mod$\,\,$2)$\,$=$\,$1} \mbox{then} \mbox{a:=4} \mbox{else} \mbox{a:=6}}.

Повернемось до врахування того, що при ${N\bmod10}\dib{{\neq}}0$ існує ще остання дужка, яка рахується не~за~тими правилами, що решта дужок. Якщо $1\dib{{\<}}{N\bmod10}\dib{{\<}}4$, то цій дужці не~треба компенсовувати ніяку п'ятірку, можна просто перемножити в~модульній арифметиці останні цифри; інакше кажучи, ${(N\bmod10)!\bmod10}$. Якщо~ж $5\dib{{\<}}{N\bmod10}\dib{{\<}}9$, то треба компенсувати рівно одну п'ятірку. (Навіть якщо число останнього неповного десятку, що закінчується на~5, кратне деякому~$5^k$ при більшому~$k$, то всі п'ятірки, крім однієї, перейшли у ${(N\mathbin{\mathrm{div}}5)!}$.) Поділимо на~2 число, що закінчується цифрою~2; можливі рівно два випадки: або воно вигляду $20k\,{+}\,2$, або $20k\,{+}\,12$; у~першому отримуємо $10k\,{+}\,1$, у~др\'{у}гому $10k\,{+}\,6$; більше ділень не~треба, можна вертатися до модульної арифметики. Точно треба домножити на 3 та~4 (бо\nolinebreak[3] розглядаємо випадок $5\dib{{\<}}{N\bmod10}\dib{{\<}}9$); отримуємо 
або ${(1{\times}3{\times}4)\bmod10}\dibbb{{=}}{12\bmod10}\dibbb{{=}}2$,
або ${(6{\times}3{\times}4)\bmod10}\dibbb{{=}}{72\bmod10}\dibbb{{=}}2$,
тобто в обох випадках~2. Множити на~5 не~треба й не~можна, ця п'ятірка вже врахована іншими засобами.
Лишається тільки домножити в модульній арифметиці на 6,~7,~\dots,\nolinebreak[2] ${N\bmod10}$ (зокрема, при ${N\bmod10\,{=}\,5}$, не~домножувати ні~на~що).

Дії попереднього абзацу насправді рівносильні таким: <<порахувати ``в~лоб'' ${(N\bmod10)!}$; якщо кратний~10, поділити на~10; взяти останню цифру>>. Це~навіть простіше писати. Але\nolinebreak[2] неясно, як строго доводити правильність цього, не~спираючись на попередній абзац.

Само собою, треба помножити (в~модульній арифметиці) результат (поза)минулого абзаца на раніше отриманий (рівний або~4, або~6) результат ${4^{N\mathbin{\mathrm{div}}10}\bmod10}$, а\nolinebreak[3] також домножити (в~модульній арифметиці) цей добуток на ${(N\mathbin{\mathrm{div}}5)!\bmod10}$, розв'язавши всю задачу заново з аргументом ${N\mathbin{\mathrm{div}}5}$ замість~$N$. Це~повторюватиметься приблизно $\log_{5}{N}$ разів. Що й задає асимптотичну оцінку всього р\'{о}зв'язку $\Theta(\log{N})$, бо все інше виконується за~$\Theta(1)$ (щось \mbox{\texttt{if}-ом}, щось циклом, але з дуже малою кількістю ітерацій). Так що задачу в~принципі можна було давати і для значно більших~$N$. Втім, вона і з обмеженням ${N\,{\<}\,2\,018\,019}$ виявилася дещо заскладною для переважної більшості учасників.
