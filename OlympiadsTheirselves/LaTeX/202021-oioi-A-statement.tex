{

\PrintEjudgeConstraintsfalse

\begin{problemAllDefault}{Василько та циркуль--1}
\label{problem:2020oioi-circles-in-room}

Василько взяв великого циркуля та зайшов до порожньої кімнати, підлога якої являє собою прямокутник, вершини якого мають координати $(0;0)$, $(A; 0)$, $(A; B)$, $(0; B)$. Поставивши циркуль на підлозі цієї кімнати в точці з координатами $(x;y)$ (ця точка розміщена всер\'{е}дині (внутри, inside) цієї кімнати, тобто ${0\,{<}\,x\,{<}\,A}$, ${0\,{<}\,y\,{<}\,B}$), він почав будувати к\'{о}ла (окружности, circles), радіусами $r$, $2r$, $3r$,~\dots{} (радіус кожного наступного на~$r$ більший за радіус попереднього). 

Скільки повних кіл може побудувати в цій кімнаті Василько?

Всі згадані велич\'{и}ни $A$, $B$, $x$, $y$, $r$ гарантовано натуральні (цілі додатні). 

Якщо для побудови чергового к\'{о}ла потрібно чиркати циркулем об стіну\nolinebreak[3] --- вважати, що побудувати відповідне коло вже неможливо.

У~цій задачі треба здати не~програму, а~вираз: 
вписати його (сам вираз, не~назву файлу) у~відповідне поле перевіряючої системи
і відправити на~перевірку. Правила запису виразу:
можна використовувати десяткові ч\'{и}сла, арифметичні дії ``\verb"+"''~(плюс), 
``\verb"-"''~(мінус), ``\verb"*"''~(множення), ``\verb"/"''~(ділення дробове, наприклад, \verb"17/5"${=}3{,}4$), ``\verb"//"''~(ділення цілочисельне, наприклад, \verb"17//5"${=}3$), круглі дужки ``\verb"("''\nolinebreak[2] та~``\verb")"'' 
для групування та зміни порядку дій, а~також функції \verb"min" та \verb"max".
Формат запису функцій: після \verb"min" або \verb"max" відкривна кругла дужка, потім перелік (через кому) виразів, від яких береться мінімум або максимум, потім закривна кругла дужка; кількість виразів, з яких береться мінімум або максимум, може бути довільна (можна хоч з \mbox{2-х}, хоч з \mbox{3-х}, хоч з \mbox{4-х}, хоч з \mbox{10-и}, хоч з \mbox{1-го})
Дозволяються пропуски (пробіли), 
але, звісно, не~всер\'{е}дині чисел і не~всер\'{е}дині імен \verb"min" та \verb"max".
Згадані в умові $A$, $B$, $x$, $y$, $r$ повинні називатися с\'{а}ме 
\texttt{A},
\texttt{B},
\texttt{x},
\texttt{y},
\texttt{r},
а~не~якось інакше. Всі ці букви повинні бути латинські (англійські), причому \texttt{A} та \texttt{B}\nolinebreak[3] --- великі, \texttt{x}, \texttt{y} та \texttt{r}\nolinebreak[3] --- маленькі. Функції \verb"min" та/або \verb"max" слід писати маленькими латинськими (англійськими) буквами.

\myflfigaw{\begin{mfpic}[1.75]{-5}{125}{-5}{95}
\axes
\dashed\lines{( -5, 10),(125, 10)}
\dashed\lines{( -5, 20),(125, 20)}
\dashed\lines{( -5, 30),(125, 30)}
\dashed\lines{( -5, 40),(125, 40)}
\dashed\lines{( -5, 50),(125, 50)}
\dashed\lines{( -5, 60),(125, 60)}
\dashed\lines{( -5, 70),(125, 70)}
\dashed\lines{( -5, 80),(125, 80)}
\dashed\lines{( -5, 90),(125, 90)}
%
\dashed\lines{( 10, -5),( 10, 95)}
\dashed\lines{( 20, -5),( 20, 95)}
\dashed\lines{( 30, -5),( 30, 95)}
\dashed\lines{( 40, -5),( 40, 95)}
\dashed\lines{( 50, -5),( 50, 95)}
\dashed\lines{( 60, -5),( 60, 95)}
\dashed\lines{( 70, -5),( 70, 95)}
\dashed\lines{( 80, -5),( 80, 95)}
\dashed\lines{( 90, -5),( 90, 95)}
\dashed\lines{(100, -5),(100, 95)}
\dashed\lines{(110, -5),(110, 95)}
\dashed\lines{(120, -5),(120, 95)}
%
\pen{1.5pt}
\polygon{(0,0),(120,0),(120,90),(0,90)}
\point{(40,45)}
\circle{(40,45), 8}
\circle{(40,45),16}
\circle{(40,45),24}
\circle{(40,45),32}
%
\tlabel[tc](0,0){0}
\tlabel[tc](40,0){40}
\tlabel[tc](120,0){120}
%
\tlabel[cr](0,0){0}
\tlabel[cr](0,45){45}
\tlabel[cr](0,90){90}
\end{mfpic}}


% \begin{tabular}[b]{p{0.3\textwidth}c}
% Для кращого розуміння умови задачі наведено також рисунок. На ньому 
Наприклад, на цьому рисунку
${A\,{=}\,120}$,
${B\,{=}\,90}$,
${x\,{=}\,40}$,
${y\,{=}\,45}$,
${r\,{=}\,8}$,
а\nolinebreak[3] пунктирні лінії (яких нема ні в кімнаті, ні в решті умови задачі, лише на рисунку) проведені там, де координата кратна~10. 
Як~бачимо, правильна числова відповідь (кількість кіл) для цих 
\texttt{A},
\texttt{B},
\texttt{x},
\texttt{y},
\texttt{r}
дорівнює~4 (п'яте коло можна було~б намалювати, подряпавши стіну, але Василько цього не~робитиме).
% &
% \end{tabular}

\end{problemAllDefault}

}