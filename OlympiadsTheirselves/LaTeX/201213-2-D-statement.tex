\begin{problem}{Лабіринт}{maze.in}{maze.out}{1 сек}{64 Мб}

Тезею з лабіринту Мінотавра допоміг вийти клубок ниток Аріадни. Ви можете замість
клубка використовувати персональний комп'ютер.

Потрібно написати програму, яка вводить маршрут Тезея в лабіринті і знаходить
найкоротший зворотний шлях, по якому Тезей зможе вийти з лабіринту, не\nolinebreak[3] заходячи в\nolinebreak[2]
тупики і не\nolinebreak[3] роблячи кругів.

\Task 
Напишіть програму \texttt{maze}, яка читає один рядок тексту (шлях Тезея в лабіринті)
та знаходить найкоротший зворотний шлях.

\InputFile
Файл \texttt{maze.in} містить маршрут Тезея, який представлений одним рядком, що
складається з букв: \texttt{N}, \texttt{S}, \texttt{W}, \texttt{Е}. Довжина рядка не~більше 200 символів.
Букви позначають:

\vspace{-0.5\baselineskip}

\begin{multicols}{2}
\par\texttt{N} --- один крок на північ,
\par\texttt{S} --- один крок на південь,
\par\texttt{W} --- один крок на захід,
\par\texttt{E} --- один крок на схід.
\end{multicols}

\vspace{-0.5\baselineskip}

\OutputFile
Ваша програма повинна створити текстовий файл \texttt{maze.out} і вивести туди
обернений шлях мінімальної довжини. Якщо відповідей декілька, виведіть будь-яку.
Зверніть увагу, що можна іти лише по слідам Тезея і не~можна скорочувати шлях, ідучи
навпростець.

\Example
\begin{exampleWidthsAndDefaultFileNames}{5em}{3em}%
\exmp{EENNESWSSWE}{NWW}%
\end{exampleWidthsAndDefaultFileNames}

\end{problem}
