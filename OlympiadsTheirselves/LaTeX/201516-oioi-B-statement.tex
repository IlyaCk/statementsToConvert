\begin{problemAllDefault}{Фасування олії}

На виробництві по переробці насіння в день виробляють $N$ літрів олії. Для її фасування використовують тару об’ємом 1, 2, 3, 4, 5, 6 літрів.

Напишіть програму \texttt{oil}, яка б давала можливість з’ясувати, яку мінімальну кількість тари необхідно для фасування.

\InputFile	
Єдине ціле число~$N$\nolinebreak[3] --- кількість літрів олії.

\OutputFile	
Програма повинна вивести єдине ціле число\nolinebreak[3] --- мінімальну кількість тари.

\ifAfour\else
\vspace{-\baselineskip}
\fi

\Examples
\ifAfour
\par
\fi
\begin{exampleSimple}{5em}{3em}%
\exmp{7}{2}%
\ifAfour%
\end{exampleSimple}%
\hspace{-1em}%
\begin{exampleSimple}{5em}{3em}%
\fi%
\exmp{12}{2}%
\ifAfour%
\end{exampleSimple}%
\hspace{-1em}%
\begin{exampleSimple}{5em}{3em}%
\fi%
\exmp{19}{4}%
\end{exampleSimple}

\end{problemAllDefault}