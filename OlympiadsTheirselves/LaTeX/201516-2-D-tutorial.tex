{

\def\tabbb{\hspace*{1em}}

\myflfigaw{\ifAfour\begin{minipage}{9em}\else\begin{minipage}{8.5em}\fi\begin{small}\renewcommand{\baselinestretch}{0.875}\begin{alltt}res:=0;\\
for c:=a to b do\\
\tabbb{}for p:=1 to c do\\
\tabbb\tabbb{}if c mod p = 0\\
\tabbb\tabbb\tabbb{}then\\
\tabbb\tabbb\tabbb{}\tabbb{}res:=res+1;\end{alltt}\end{small}\end{minipage}}
\Tutorial \MyParagraph{Розв'язок на 40--50\%\%.}
Для отримання 40\% балів досить реалізувати те, що написано в умові. Наприклад, див. праворуч. Але складність такого алгоритма ${\Theta((B\,{-}\,A\,{+}\,1)\times{B})}$, тож він отримує вердикт <<Перевищено час роботи>> на тестах усіх блоків, крім \mbox{1-го} (${1\,{\<}\,A\,{\<}\,B\,{\<}\,1000}$).

}

Враховуючи міркування зі\nolinebreak[3] стор.~\pageref{text:about-sqrt-n-in-divisors-list} про те, чому для знаходження дільників~$N$ достатньо перебирати <<претендентів>> до~$\sqrt{N}$ (а~не~до~$N$), цей алгоритм можна модифікувати, лишивши незмінним зовнішній цикл, % \verb"for c:=a to b do", 
але зменшивши діапазон % перебору 
внутрішнього % з~$c$ 
до~$\sqrt{c}$. Це\nolinebreak[3] дає складність $\Theta((B\,{-}\,A\,{+}\,1)\times\sqrt{B})$, що допуст\'{и}мо 
% як при обмеженнях \mbox{1-го} блоку <<$1\,{\<}\,A\,{\<}\,B\,{\<}\,1000$>>, так і \mbox{3-го} блоку <<$10^8\,{\<}\,B\,{\<}\,10^9$, але ${B\,{-}\,100}\,{\<}\,A\,{\<}\,B$>>.%
як для \mbox{1-го} блоку, так і \mbox{3-го}.%
\label{label:sum-num-divs-end-1st-and-3rd-blocks-algorithm}

\MyParagraph{Повний розв'язок.}\phantomsection\label{text:num-divisors-in-range-best-algo-begin}
Не~варто дотримуватися (нав'язуваної приміткою з~умови) тактики <<знайти кількості дільників кожного числ\'{а} й додати>>. Кількості дільників окремих чисел не\nolinebreak[3] питають, тож можна  
% порахувати всі ті\nolinebreak[3] самі дільники 
рахувати відповідь
якось інакше. Наприклад, у\nolinebreak[3] іншому порядку.
%
В усіх подальших алгоритмах, кількості дільників легше не\nolinebreak[3] шукати на\nolinebreak[3] проміжку від\nolinebreak[2] $A$ до~$B$, а знайти один раз на\nolinebreak[3] проміжку від\nolinebreak[2] 1 до~$B$, інший\nolinebreak[3] --- % на\nolinebreak[3] проміжку 
від\nolinebreak[2] 1 до~$(A\,{-}\,1)$, й відняти др\'{у}гий результат з першого (див.\nolinebreak[2] також стор.~\pageref{text:201213-2-C-about-range-subtract}).

%%% Тим паче, що у\nolinebreak[3] програмуванні це означає, що такий пошук пишеться \emph{один} раз, оформляється як функція, і двічі викликається.



%\dots{} %, а\nolinebreak[3] ${B\,{-}\,A\,{+}\,1}$ ітерацій циклу <<\verb"for c:=a to b...">> при $A{\sim}1$, $B{\sim}10^{12}$ було~б забагато, навіть якби вдавалося знайти кількість дільників кожного числ\'{а} за $\Theta(1)$ (що\nolinebreak[3] нереально).
% Тобто, треба рахувати ту саму суму, але інакше. %Наприклад, у іншому порядку.

Число~1 є\nolinebreak[3] дільником усіх чисел, тож дільник~1 приносить у шукану суму стільки, скільки чисел у проміжку. Число~2\nolinebreak[3] є\nolinebreak[3] дільником усіх парних чисел, тож приносить у шукану суму стільки, скільки на проміжку парних чисел. І~т.~д. А\nolinebreak[3] серед усіх чисел від\nolinebreak[1] 1 до~$N$ є рівно \verb"N div k" чисел, кратних~$k$.\phantomsection\label{text:sum-num-divs-end-of-theta-b-algorithm}

На перший погляд, так треба перебирати всі дільники аж до~$N$ (яке один раз дорівнює $A\,{-}\,1$, інший раз~$B$), 
%%% і реалізація такого алгоритму легко пройде тести \mbox{1-го} та \mbox{2-го} блоків (отримавши таким чином 60\% балів), але буде не~досить ефективною ні для \mbox{4-го} блоку тестів, ні навіть для \mbox{3-го}. 
і алгоритм матиме складність ${\Theta(A\,{+}\,B)}\dibbb{{=}}\Theta(B)$, що не~досить ефективно ні для \mbox{4-го} блоку тестів, ні навіть для \mbox{3-го}.

Але це можна оптимізувати, 
використавши 
% факт (зі~згаданої стор.~\pageref{text:about-sqrt-n-in-divisors-list}) 
властивість
<<якщо~$p$ є дільником~$N$, то ${(N/p)}$ теж є дільником~$N$, а з чисел $p$ та $N/p$ хоча~б одне${\,\,{\<}\,\sqrt{N}}$>>.
Приблизно так: перебравши дільники лише до~$\sqrt{N}$, подвоїти результат, щоб урахувати дільники, більші~$\sqrt{N}$.
Це не~зовсім правда, бо враховує деякі дільники двічі;
але цю похибку можна компенсувати, й отримати правильний алгоритм підзадачі складності $\Theta(\sqrt{N})$ (отже, складність усієї задачі буде ${\Theta(\sqrt{A}\,{+}\,\sqrt{B})}\dibbb{{=}}\Theta(\sqrt{B})$).

\noindent
\begin{flushright}
\begin{mfpic}[\ifAfour 10\else 12\fi]{0.5}{33.5}{-20.000}{0}
\tlabel[cc](1, 0){${}_{1}$}
\tlabel[cc](33, -0.625){${}_{1}$}
\tlabel[cc](2, 0){${}_{2}$}
\tlabel[cc](33, -1.250){${}_{2}$}
\tlabel[cc](3, 0){${}_{3}$}
\tlabel[cc](33, -1.875){${}_{3}$}
\tlabel[cc](4, 0){${}_{4}$}
\tlabel[cc](33, -2.500){${}_{4}$}
\tlabel[cc](5, 0){${}_{5}$}
\tlabel[cc](33, -3.125){${}_{5}$}
\tlabel[cc](6, 0){${}_{6}$}
\tlabel[cc](33, -3.750){${}_{6}$}
\tlabel[cc](7, 0){${}_{7}$}
\tlabel[cc](33, -4.375){${}_{7}$}
\tlabel[cc](8, 0){${}_{8}$}
\tlabel[cc](33, -5.000){${}_{8}$}
\tlabel[cc](9, 0){${}_{9}$}
\tlabel[cc](33, -5.625){${}_{9}$}
\tlabel[cc](10, 0){${}_{10}$}
\tlabel[cc](33, -6.250){${}_{10}$}
\tlabel[cc](11, 0){${}_{11}$}
\tlabel[cc](33, -6.875){${}_{11}$}
\tlabel[cc](12, 0){${}_{12}$}
\tlabel[cc](33, -7.500){${}_{12}$}
\tlabel[cc](13, 0){${}_{13}$}
\tlabel[cc](33, -8.125){${}_{13}$}
\tlabel[cc](14, 0){${}_{14}$}
\tlabel[cc](33, -8.750){${}_{14}$}
\tlabel[cc](15, 0){${}_{15}$}
\tlabel[cc](33, -9.375){${}_{15}$}
\tlabel[cc](16, 0){${}_{16}$}
\tlabel[cc](33, -10.000){${}_{16}$}
\tlabel[cc](17, 0){${}_{17}$}
\tlabel[cc](33, -10.625){${}_{17}$}
\tlabel[cc](18, 0){${}_{18}$}
\tlabel[cc](33, -11.250){${}_{18}$}
\tlabel[cc](19, 0){${}_{19}$}
\tlabel[cc](33, -11.875){${}_{19}$}
\tlabel[cc](20, 0){${}_{20}$}
\tlabel[cc](33, -12.500){${}_{20}$}
\tlabel[cc](21, 0){${}_{21}$}
\tlabel[cc](33, -13.125){${}_{21}$}
\tlabel[cc](22, 0){${}_{22}$}
\tlabel[cc](33, -13.750){${}_{22}$}
\tlabel[cc](23, 0){${}_{23}$}
\tlabel[cc](33, -14.375){${}_{23}$}
\tlabel[cc](24, 0){${}_{24}$}
\tlabel[cc](33, -15.000){${}_{24}$}
\tlabel[cc](25, 0){${}_{25}$}
\tlabel[cc](33, -15.625){${}_{25}$}
\tlabel[cc](26, 0){${}_{26}$}
\tlabel[cc](33, -16.250){${}_{26}$}
\tlabel[cc](27, 0){${}_{27}$}
\tlabel[cc](33, -16.875){${}_{27}$}
\tlabel[cc](28, 0){${}_{28}$}
\tlabel[cc](33, -17.500){${}_{28}$}
\tlabel[cc](29, 0){${}_{29}$}
\tlabel[cc](33, -18.125){${}_{29}$}
\tlabel[cc](30, 0){${}_{30}$}
\tlabel[cc](33, -18.750){${}_{30}$}
\tlabel[cc](31, 0){${}_{31}$}
\tlabel[cc](33, -19.375){${}_{31}$}
\tlabel[cc](32, 0){${}_{32}$}
\tlabel[cc](33, -20.000){${}_{32}$}
\lines{(0.75, -3.438), (1, -3.438)}
\lines{(1.25, -3.438), (1.5, -3.438)}
\lines{(1.75, -3.438), (2, -3.438)}
\lines{(2.25, -3.438), (2.5, -3.438)}
\lines{(2.75, -3.438), (3, -3.438)}
\lines{(3.25, -3.438), (3.5, -3.438)}
\lines{(3.75, -3.438), (4, -3.438)}
\lines{(4.25, -3.438), (4.5, -3.438)}
\lines{(4.75, -3.438), (5, -3.438)}
\lines{(5.25, -3.438), (5.5, -3.438)}
\lines{(5.75, -3.438), (6, -3.438)}
\lines{(6.25, -3.438), (6.5, -3.438)}
\lines{(6.75, -3.438), (7, -3.438)}
\lines{(7.25, -3.438), (7.5, -3.438)}
\lines{(7.75, -3.438), (8, -3.438)}
\lines{(8.25, -3.438), (8.5, -3.438)}
\lines{(8.75, -3.438), (9, -3.438)}
\lines{(9.25, -3.438), (9.5, -3.438)}
\lines{(9.75, -3.438), (10, -3.438)}
\lines{(10.25, -3.438), (10.5, -3.438)}
\lines{(10.75, -3.438), (11, -3.438)}
\lines{(11.25, -3.438), (11.5, -3.438)}
\lines{(11.75, -3.438), (12, -3.438)}
\lines{(12.25, -3.438), (12.5, -3.438)}
\lines{(12.75, -3.438), (13, -3.438)}
\lines{(13.25, -3.438), (13.5, -3.438)}
\lines{(13.75, -3.438), (14, -3.438)}
\lines{(14.25, -3.438), (14.5, -3.438)}
\lines{(14.75, -3.438), (15, -3.438)}
\lines{(15.25, -3.438), (15.5, -3.438)}
\lines{(15.75, -3.438), (16, -3.438)}
\lines{(16.25, -3.438), (16.5, -3.438)}
\lines{(16.75, -3.438), (17, -3.438)}
\lines{(17.25, -3.438), (17.5, -3.438)}
\lines{(17.75, -3.438), (18, -3.438)}
\lines{(18.25, -3.438), (18.5, -3.438)}
\lines{(18.75, -3.438), (19, -3.438)}
\lines{(19.25, -3.438), (19.5, -3.438)}
\lines{(19.75, -3.438), (20, -3.438)}
\lines{(20.25, -3.438), (20.5, -3.438)}
\lines{(20.75, -3.438), (21, -3.438)}
\lines{(21.25, -3.438), (21.5, -3.438)}
\lines{(21.75, -3.438), (22, -3.438)}
\lines{(22.25, -3.438), (22.5, -3.438)}
\lines{(22.75, -3.438), (23, -3.438)}
\lines{(23.25, -3.438), (23.5, -3.438)}
\lines{(23.75, -3.438), (24, -3.438)}
\lines{(24.25, -3.438), (24.5, -3.438)}
\lines{(24.75, -3.438), (25, -3.438)}
\lines{(25.25, -3.438), (25.5, -3.438)}
\lines{(25.75, -3.438), (26, -3.438)}
\lines{(26.25, -3.438), (26.5, -3.438)}
\lines{(26.75, -3.438), (27, -3.438)}
\lines{(27.25, -3.438), (27.5, -3.438)}
\lines{(27.75, -3.438), (28, -3.438)}
\lines{(28.25, -3.438), (28.5, -3.438)}
\lines{(28.75, -3.438), (29, -3.438)}
\lines{(29.25, -3.438), (29.5, -3.438)}
\lines{(29.75, -3.438), (30, -3.438)}
\lines{(30.25, -3.438), (30.5, -3.438)}
\lines{(30.75, -3.438), (31, -3.438)}
\lines{(31.25, -3.438), (31.5, -3.438)}
\lines{(31.75, -3.438), (32, -3.438)}
\lines{(32.25, -3.438), (32.5, -3.438)}
\gfill\lclosed\arc[p]{(1, -0.625), -45, 135, 0.2}
\circle{(1, -0.625), 0.2}
\gfill\circle{(2, -0.625), 0.2}
\circle{(2, -1.250), 0.2}
\lines{(1.8, -0.625),  (1.600, -0.625),  (1.600, -1.250),  (1.8, -1.250)}
\gfill\circle{(3, -0.625), 0.2}
\circle{(3, -1.875), 0.2}
\lines{(2.8, -0.625),  (2.600, -0.625),  (2.600, -1.875),  (2.8, -1.875)}
\gfill\lclosed\arc[p]{(4, -1.250), -45, 135, 0.2}
\circle{(4, -1.250), 0.2}
\gfill\circle{(4, -0.625), 0.2}
\circle{(4, -2.500), 0.2}
\lines{(3.8, -0.625),  (3.600, -0.625),  (3.600, -2.500),  (3.8, -2.500)}
\gfill\circle{(5, -0.625), 0.2}
\circle{(5, -3.125), 0.2}
\lines{(4.8, -0.625),  (4.600, -0.625),  (4.600, -3.125),  (4.8, -3.125)}
\gfill\circle{(6, -1.250), 0.2}
\circle{(6, -1.875), 0.2}
\lines{(5.8, -1.250),  (5.700, -1.250),  (5.700, -1.875),  (5.8, -1.875)}
\gfill\circle{(6, -0.625), 0.2}
\circle{(6, -3.750), 0.2}
\lines{(5.8, -0.625),  (5.500, -0.625),  (5.500, -3.750),  (5.8, -3.750)}
\gfill\circle{(7, -0.625), 0.2}
\circle{(7, -4.375), 0.2}
\lines{(6.8, -0.625),  (6.600, -0.625),  (6.600, -4.375),  (6.8, -4.375)}
\gfill\circle{(8, -1.250), 0.2}
\circle{(8, -2.500), 0.2}
\lines{(7.8, -1.250),  (7.700, -1.250),  (7.700, -2.500),  (7.8, -2.500)}
\gfill\circle{(8, -0.625), 0.2}
\circle{(8, -5.000), 0.2}
\lines{(7.8, -0.625),  (7.500, -0.625),  (7.500, -5.000),  (7.8, -5.000)}
\gfill\lclosed\arc[p]{(9, -1.875), -45, 135, 0.2}
\circle{(9, -1.875), 0.2}
\gfill\circle{(9, -0.625), 0.2}
\circle{(9, -5.625), 0.2}
\lines{(8.8, -0.625),  (8.600, -0.625),  (8.600, -5.625),  (8.8, -5.625)}
\gfill\circle{(10, -1.250), 0.2}
\circle{(10, -3.125), 0.2}
\lines{(9.8, -1.250),  (9.700, -1.250),  (9.700, -3.125),  (9.8, -3.125)}
\gfill\circle{(10, -0.625), 0.2}
\circle{(10, -6.250), 0.2}
\lines{(9.8, -0.625),  (9.500, -0.625),  (9.500, -6.250),  (9.8, -6.250)}
\gfill\circle{(11, -0.625), 0.2}
\circle{(11, -6.875), 0.2}
\lines{(10.8, -0.625),  (10.600, -0.625),  (10.600, -6.875),  (10.8, -6.875)}
\gfill\circle{(12, -1.875), 0.2}
\circle{(12, -2.500), 0.2}
\lines{(11.8, -1.875),  (11.733, -1.875),  (11.733, -2.500),  (11.8, -2.500)}
\gfill\circle{(12, -1.250), 0.2}
\circle{(12, -3.750), 0.2}
\lines{(11.8, -1.250),  (11.600, -1.250),  (11.600, -3.750),  (11.8, -3.750)}
\gfill\circle{(12, -0.625), 0.2}
\circle{(12, -7.500), 0.2}
\lines{(11.8, -0.625),  (11.467, -0.625),  (11.467, -7.500),  (11.8, -7.500)}
\gfill\circle{(13, -0.625), 0.2}
\circle{(13, -8.125), 0.2}
\lines{(12.8, -0.625),  (12.600, -0.625),  (12.600, -8.125),  (12.8, -8.125)}
\gfill\circle{(14, -1.250), 0.2}
\circle{(14, -4.375), 0.2}
\lines{(13.8, -1.250),  (13.700, -1.250),  (13.700, -4.375),  (13.8, -4.375)}
\gfill\circle{(14, -0.625), 0.2}
\circle{(14, -8.750), 0.2}
\lines{(13.8, -0.625),  (13.500, -0.625),  (13.500, -8.750),  (13.8, -8.750)}
\gfill\circle{(15, -1.875), 0.2}
\circle{(15, -3.125), 0.2}
\lines{(14.8, -1.875),  (14.700, -1.875),  (14.700, -3.125),  (14.8, -3.125)}
\gfill\circle{(15, -0.625), 0.2}
\circle{(15, -9.375), 0.2}
\lines{(14.8, -0.625),  (14.500, -0.625),  (14.500, -9.375),  (14.8, -9.375)}
\gfill\lclosed\arc[p]{(16, -2.500), -45, 135, 0.2}
\circle{(16, -2.500), 0.2}
\gfill\circle{(16, -1.250), 0.2}
\circle{(16, -5.000), 0.2}
\lines{(15.8, -1.250),  (15.700, -1.250),  (15.700, -5.000),  (15.8, -5.000)}
\gfill\circle{(16, -0.625), 0.2}
\circle{(16, -10.000), 0.2}
\lines{(15.8, -0.625),  (15.500, -0.625),  (15.500, -10.000),  (15.8, -10.000)}
\gfill\circle{(17, -0.625), 0.2}
\circle{(17, -10.625), 0.2}
\lines{(16.8, -0.625),  (16.600, -0.625),  (16.600, -10.625),  (16.8, -10.625)}
\gfill\circle{(18, -1.875), 0.2}
\circle{(18, -3.750), 0.2}
\lines{(17.8, -1.875),  (17.733, -1.875),  (17.733, -3.750),  (17.8, -3.750)}
\gfill\circle{(18, -1.250), 0.2}
\circle{(18, -5.625), 0.2}
\lines{(17.8, -1.250),  (17.600, -1.250),  (17.600, -5.625),  (17.8, -5.625)}
\gfill\circle{(18, -0.625), 0.2}
\circle{(18, -11.250), 0.2}
\lines{(17.8, -0.625),  (17.467, -0.625),  (17.467, -11.250),  (17.8, -11.250)}
\gfill\circle{(19, -0.625), 0.2}
\circle{(19, -11.875), 0.2}
\lines{(18.8, -0.625),  (18.600, -0.625),  (18.600, -11.875),  (18.8, -11.875)}
\gfill\circle{(20, -2.500), 0.2}
\circle{(20, -3.125), 0.2}
\lines{(19.8, -2.500),  (19.733, -2.500),  (19.733, -3.125),  (19.8, -3.125)}
\gfill\circle{(20, -1.250), 0.2}
\circle{(20, -6.250), 0.2}
\lines{(19.8, -1.250),  (19.600, -1.250),  (19.600, -6.250),  (19.8, -6.250)}
\gfill\circle{(20, -0.625), 0.2}
\circle{(20, -12.500), 0.2}
\lines{(19.8, -0.625),  (19.467, -0.625),  (19.467, -12.500),  (19.8, -12.500)}
\gfill\circle{(21, -1.875), 0.2}
\circle{(21, -4.375), 0.2}
\lines{(20.8, -1.875),  (20.700, -1.875),  (20.700, -4.375),  (20.8, -4.375)}
\gfill\circle{(21, -0.625), 0.2}
\circle{(21, -13.125), 0.2}
\lines{(20.8, -0.625),  (20.500, -0.625),  (20.500, -13.125),  (20.8, -13.125)}
\gfill\circle{(22, -1.250), 0.2}
\circle{(22, -6.875), 0.2}
\lines{(21.8, -1.250),  (21.700, -1.250),  (21.700, -6.875),  (21.8, -6.875)}
\gfill\circle{(22, -0.625), 0.2}
\circle{(22, -13.750), 0.2}
\lines{(21.8, -0.625),  (21.500, -0.625),  (21.500, -13.750),  (21.8, -13.750)}
\gfill\circle{(23, -0.625), 0.2}
\circle{(23, -14.375), 0.2}
\lines{(22.8, -0.625),  (22.600, -0.625),  (22.600, -14.375),  (22.8, -14.375)}
\gfill\circle{(24, -2.500), 0.2}
\circle{(24, -3.750), 0.2}
\lines{(23.8, -2.500),  (23.750, -2.500),  (23.750, -3.750),  (23.8, -3.750)}
\gfill\circle{(24, -1.875), 0.2}
\circle{(24, -5.000), 0.2}
\lines{(23.8, -1.875),  (23.650, -1.875),  (23.650, -5.000),  (23.8, -5.000)}
\gfill\circle{(24, -1.250), 0.2}
\circle{(24, -7.500), 0.2}
\lines{(23.8, -1.250),  (23.550, -1.250),  (23.550, -7.500),  (23.8, -7.500)}
\gfill\circle{(24, -0.625), 0.2}
\circle{(24, -15.000), 0.2}
\lines{(23.8, -0.625),  (23.450, -0.625),  (23.450, -15.000),  (23.8, -15.000)}
\gfill\lclosed\arc[p]{(25, -3.125), -45, 135, 0.2}
\circle{(25, -3.125), 0.2}
\gfill\circle{(25, -0.625), 0.2}
\circle{(25, -15.625), 0.2}
\lines{(24.8, -0.625),  (24.600, -0.625),  (24.600, -15.625),  (24.8, -15.625)}
\gfill\circle{(26, -1.250), 0.2}
\circle{(26, -8.125), 0.2}
\lines{(25.8, -1.250),  (25.700, -1.250),  (25.700, -8.125),  (25.8, -8.125)}
\gfill\circle{(26, -0.625), 0.2}
\circle{(26, -16.250), 0.2}
\lines{(25.8, -0.625),  (25.500, -0.625),  (25.500, -16.250),  (25.8, -16.250)}
\gfill\circle{(27, -1.875), 0.2}
\circle{(27, -5.625), 0.2}
\lines{(26.8, -1.875),  (26.700, -1.875),  (26.700, -5.625),  (26.8, -5.625)}
\gfill\circle{(27, -0.625), 0.2}
\circle{(27, -16.875), 0.2}
\lines{(26.8, -0.625),  (26.500, -0.625),  (26.500, -16.875),  (26.8, -16.875)}
\gfill\circle{(28, -2.500), 0.2}
\circle{(28, -4.375), 0.2}
\lines{(27.8, -2.500),  (27.733, -2.500),  (27.733, -4.375),  (27.8, -4.375)}
\gfill\circle{(28, -1.250), 0.2}
\circle{(28, -8.750), 0.2}
\lines{(27.8, -1.250),  (27.600, -1.250),  (27.600, -8.750),  (27.8, -8.750)}
\gfill\circle{(28, -0.625), 0.2}
\circle{(28, -17.500), 0.2}
\lines{(27.8, -0.625),  (27.467, -0.625),  (27.467, -17.500),  (27.8, -17.500)}
\gfill\circle{(29, -0.625), 0.2}
\circle{(29, -18.125), 0.2}
\lines{(28.8, -0.625),  (28.600, -0.625),  (28.600, -18.125),  (28.8, -18.125)}
\gfill\circle{(30, -3.125), 0.2}
\circle{(30, -3.750), 0.2}
\lines{(29.8, -3.125),  (29.750, -3.125),  (29.750, -3.750),  (29.8, -3.750)}
\gfill\circle{(30, -1.875), 0.2}
\circle{(30, -6.250), 0.2}
\lines{(29.8, -1.875),  (29.650, -1.875),  (29.650, -6.250),  (29.8, -6.250)}
\gfill\circle{(30, -1.250), 0.2}
\circle{(30, -9.375), 0.2}
\lines{(29.8, -1.250),  (29.550, -1.250),  (29.550, -9.375),  (29.8, -9.375)}
\gfill\circle{(30, -0.625), 0.2}
\circle{(30, -18.750), 0.2}
\lines{(29.8, -0.625),  (29.450, -0.625),  (29.450, -18.750),  (29.8, -18.750)}
\gfill\circle{(31, -0.625), 0.2}
\circle{(31, -19.375), 0.2}
\lines{(30.8, -0.625),  (30.600, -0.625),  (30.600, -19.375),  (30.8, -19.375)}
\gfill\circle{(32, -2.500), 0.2}
\circle{(32, -5.000), 0.2}
\lines{(31.8, -2.500),  (31.733, -2.500),  (31.733, -5.000),  (31.8, -5.000)}
\gfill\circle{(32, -1.250), 0.2}
\circle{(32, -10.000), 0.2}
\lines{(31.8, -1.250),  (31.600, -1.250),  (31.600, -10.000),  (31.8, -10.000)}
\gfill\circle{(32, -0.625), 0.2}
\circle{(32, -20.000), 0.2}
\lines{(31.8, -0.625),  (31.467, -0.625),  (31.467, -20.000),  (31.8, -20.000)}
\end{mfpic}

\end{flushright}

\ifAfour
\ifBigStretch
\TODOlater
\else
\vspace{-16\baselineskip}
\parshape=17
0em 0.25\textwidth
0em 0.25\textwidth
0em 0.25\textwidth
0em 0.25\textwidth
0em 0.40\textwidth
0em 0.45\textwidth
0em 0.50\textwidth
0em 0.55\textwidth
0em 0.60\textwidth
0em 0.65\textwidth
0em 0.70\textwidth
0em 0.75\textwidth
0em 0.80\textwidth
0em 0.85\textwidth
0em 0.90\textwidth
0em 0.95\textwidth
0em \textwidth
\fi
\else
\ifBigStretch
\vspace{-12.5\baselineskip}
\parshape=13
0em 0.16\textwidth
0em 0.16\textwidth
0em 0.16\textwidth
0em 0.32\textwidth
0em 0.39\textwidth
0em 0.46\textwidth
0em 0.53\textwidth
0em 0.6\textwidth
0em 0.67\textwidth
0em 0.74\textwidth
0em 0.81\textwidth
0em 0.88\textwidth
0em \textwidth
\else
\TODOlater
\fi
\fi
\noindent
Роз\-г\-ля\-не\-мо всі ді\-ль\-ни\-ки всіх чи\-сел до\nolinebreak[1] 32.
Кру\-ж\'{е}ч\-ки по\-зна\-ча\-ють, що число <<\textnumero~рядка>> є дільником числ\'{а} <<\textnumero~стовпчика>>. Чорний\nolinebreak[3] \mbox{(\textbullet)} кружечок\nolinebreak[3] --- дільник менший кореня відповідного числа, білий\nolinebreak[3] \mbox{(\textopenbullet)}\nolinebreak[3] --- більший, напівзаповнений\nolinebreak[3] --- рівний. Вертикальні лінії виражають попарний зв'язок між $p$ та~$N/p$. Штрихова горизонтальна лінія між 5 та~6 виражає перебір дільників до $\lfloor\sqrt{N}\rfloor$ (воно ж \verb"trunc(sqrt(N))" чи \verb"floor(sqrt(1.0*N))"). 
Очевидно, cумарну кількість дільників можна виразити як кількість чорних кружечків, помножену на~2, плюс кількість напівзаповнених. А\nolinebreak[3] для пошуку цих кількостей достатньо перебрати рядки з \mbox{1-го} по~\mbox{$\lfloor\sqrt{N}\rfloor$-й}, відзначаючи, що кількість усіх кружечків у\nolinebreak[2] рядку~\textnumero$\,$\texttt{i} (при $1\dib{{\<}}\texttt{i}\dib{{\<}}{\lfloor\sqrt{N}\rfloor}$) становить \verb|N div i|, і\nolinebreak[3] серед них є рівно~1 напівзаповнений та рівно\nolinebreak[2] $(\texttt{i}\,{-}\,1)$\nolinebreak[2] білий (що\nolinebreak[2] дає кількість чорних \verb|(N div i) - i|). 
%
Завершіть ці міркування самостійно та запрограмуйте їх.\phantomsection\label{text:num-divisors-in-range-best-algo-end}

\vspace{0.125\baselineskip}
\myhrulefill
\vspace{0.125\baselineskip}

Насамкінець, можлива ситуація, коли учасник олімпіади придумав і реалізував як алгоритм, що\nolinebreak[3] проходить \mbox{1-й} та \mbox{2-й} блоки тестів, так і алгоритм, що\nolinebreak[3] проходить \mbox{1-й} та \mbox{3-й} блоки, але не\nolinebreak[3] вміє розв'язати задачу повністю. Чи\nolinebreak[3] \mbox{може} такий учасник гарантувати собі 70\%\nolinebreak[2] балів, узявши однією програмою бали і\nolinebreak[3] за\nolinebreak[3] \mbox{2-й}, і\nolinebreak[3] за\nolinebreak[3] \mbox{3-й}\nolinebreak[3] блоки? Запросто, якщо зробить так: прочитавши $A$ та~$B$, знайде, яке зі значень ${(A\,{+}\,B)}$ чи ${((B\,{-}\,A)\times\sqrt{B})}$ менше, і залежно від цього викличе одну чи іншу підпрограму, що\nolinebreak[3] реалізує відповідний алгоритм.
