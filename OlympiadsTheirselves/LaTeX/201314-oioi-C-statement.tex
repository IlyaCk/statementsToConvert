\begin{problemAllDefault}{Сума квадратів}

Для заданого натурального числ\'{а}~$N$, визначити, скількома різними способами можна розкласти його в~суму двох точних додатних квадратів. 

Іншими словами: для заданого~$N$, з’ясувати, скільки є різних способів подати його як $N\dib{{=}}x^2\dib{{+}}y^2$, причому $x$ та\nolinebreak[3] $y$ являють собою цілі строго додатні числа, а\nolinebreak[3] розкладення, в\nolinebreak[3] яких значення $x$ та\nolinebreak[3] $y$ лише обміняні місцями, вважаються однаковими.


\InputFile Вхідні дані --- натуральне число\nolinebreak[3] $N$\nolinebreak[3] --- слід прочитати зі\nolinebreak[3] стандартного входу (клавіатури).

\OutputFile	Результат --- знайдену кількість способів\nolinebreak[3] --- слід вивести на\nolinebreak[3] стандартний вихід (екран).


\Examples

\begin{exampleSimpleThree}{5em}{3em}{\ifAfour31.5em\else25.5em\fi}{Примітка}%
\exmp{16}{0}{\begin{\mainFontFamily}Розкласти у суму \emph{додатних} точних квадратів неможливо\end{\mainFontFamily}}%
\exmp{10}{1}{\begin{\mainFontFamily}Єдине розкладення $10\dib{{=}}1^2{+}3^2$\end{\mainFontFamily}}%
\exmp{4225}{4}{\begin{\mainFontFamily}Чотири різні розкладення: $4225\dib{{=}}16^2{+}63^2\dib{{=}}25^2{+}60^2\dib{{=}}33^2{+}56^2\dib{{=}}39^2{+}52^2$\end{\mainFontFamily}}%
\end{exampleSimpleThree}

% TODO звертати увагу на rmfamily чи sffamily

% \Scoring	
100~балів (з~250) припадатиме на\nolinebreak[3] тести, в\nolinebreak[3] яких $1{\<}N{\<}1234$.

Ще\nolinebreak[3] 50~балів\nolinebreak[3] --- на тести, в яких $12345{\<}N{\<}123456$.

Решта\nolinebreak[2] 100~балів\nolinebreak[3] --- на тести, в яких $12345678{\<}N{\<}123456789$.

Здати потрібно одну програму, а не окремі для трьох випадків; різні обмеження вводяться виключно для того, щоб дати приблизне уявлення, скільки балів можна отримати, розв’язавши задачу не повністю.

\end{problemAllDefault}