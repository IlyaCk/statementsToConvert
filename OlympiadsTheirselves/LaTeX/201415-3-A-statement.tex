\begin{problemAllDefault}{Гірлянда --- garland}

\hyphenpenalty=500

На Новорічні свята Василь разом з батьком вирішили зробити електричну гірлянду із лампочок. При створенні гірлянди лапочки включаються в коло послідовно та деякі паралельно так, що в паралельному з’єднанні може знаходитись лише дві лампочки. Для з’ясування можливості підключення гірлянди до джерела живлення, Василю треба визначити загальний опір створеної гірлянди. З~фізики йому відомо, що при послідовному з’єднанні загальний опір кола розраховується за формулою: $R_1\dib{{+}}R_2\dib{{+}}\cdots\dib{{+}}R_n$, а при паралельному: $\frac{1}{R} \dib{{=}} \frac{1}{R_1} \dib{{+}} \frac{1}{R_2} \dib{{+}} \cdots \dib{{+}} \frac{1}{R_n}$.

%%%\Task 
Напишіть програму \texttt{garland}, яка б обчислювала загальний опір гірлянди.

\InputFile У першому рядку задано загальну кількість лампочок~$N$, в~другому\nolinebreak[3] --- через пробіл опір кожної з лампочок (значення опору є цілим числом і знаходиться в межах від 1 до~1000). В~третьому рядку знаходиться ціле число~$K$~--- кількість пар лампочок ($0\dib{{\<}}K\dib{{\<}}{N/2}$), які включені в\nolinebreak[3] коло паралельно. Кожен наступний рядок містить по\nolinebreak[2] два порядкових номери лампочок, що включені паралельно.

\OutputFile Загальний опір гірлянди (як дійсне число без заокруглень).

\savebox{\mypictbox}{\mbox{\def\resistor#1#2#3#4{ %% x y dy caption
\gclear\rect{(#1-0.8,#2-0.2),(#1+0.8,#2+0.2)}
\rect{(#1-0.8,#2-0.2),(#1+0.8,#2+0.2)}
\ifnum #3 < 0
\tlabel[tc](#1,#2-0.25){#4}
\else
\tlabel[bc](#1,#2+0.25){#4}
\fi}
\raisebox{-36pt}{\begin{mfpic}[21]{-0.125}{8.25}{0.25}{3}
\lines{(-0.125,2),(8.25,2)}
\lines{(2,2),(2,1.25),(4,1.25),(4,2)}
\point{(2,2)}
\point{(4,2)}
\lines{(6,2),(6,1.25),(8,1.25),(8,2)}
\point{(6,2)}
\point{(8,2)}
\resistor{1}{2}{+1}{$R_1{=}2$}
\resistor{3}{2}{+1}{$R_2{=}4$}
\resistor{3}{1.25}{-1}{$R_3{=}6$}
\resistor{5}{2}{+1}{$R_4{=}7$}
\resistor{7}{2}{+1}{$R_5{=}3$}
\resistor{7}{1.25}{-1}{$R_6{=}1$}
\gclear\circle{(-0.125,2),0.0625}
\circle{(-0.125,2),0.0625}
\gclear\circle{(8.25,2),0.0625}
\circle{(8.25,2),0.0625}
\end{mfpic}}}}

%\vspace{0.125\baselineskip}
%\begin{minipage}{\textwidth}{
\Example
%
%\vspace{-\baselineskip}
%\makeTableLongtrue
%
\begin{exampleSimpleThree}{6em}{3em}{\ifAfour 18em\else 14.5em\fi}{Схема прикладу}
\exmp{6
2 4 6 7 3 1
2
2 3
5 6}{12.15}{\usebox{\mypictbox}}%
\end{exampleSimpleThree}

% }\end{minipage}


\Note
\ifAfour
Відповідь можна
\else
Можна 
\fi
виводити також і у форматі \texttt{1.2150000000000E+01}.

\end{problemAllDefault}
