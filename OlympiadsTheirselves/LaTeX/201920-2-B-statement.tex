\begin{problemAllDefault}{Спільні дотичні}

Як відомо, дотичною до кола є пряма, що має рівно одну спільну точ\-ку з цим колом. 
Можлива ситуація, коли одна й та сама пряма є дотичною  відразу до двох кіл. 
Тоді вона називається спільною дотич\-ною. 
Напишіть програму, яка знаходитиме кількість різних спільних дотичних для заданих двох кіл. 
При виведенні \mbox{врахуйте} стародавню традицію приписувати числу~7 значення <<багато>>. 
Тобто, виводьте~7 для всіх кількостей спільних дотичних~$\>$~7.

\InputFile
Шість цілих чисел $X1$, $Y1$, $R1$, $X2$, $Y2$, $R2$\nolinebreak[3] --- координати центра і радіуси \mbox{1-го} і \mbox{2-го} кола,
записані саме в такому порядку, кожне число в окремому рядку. 
Абсолютні величини (модулі) координат не~перевищують мільйон. Для радіусів значення у межах від~1 до мільйона.

\OutputFile
Ваша програма повинна вивести єдине число\nolinebreak[3] --- % шукану 
кількість спільних дотичних, з~урахуванням згаданої стародавньої традиції.

\savebox{\mypictbox}{\begin{mfpic}[2.4]{-1}{61}{-13}{13}
\axes
\tlabel[cl](0.5,+10){${}_{+10}$}
\tlabel[cl](0.5,-10){${}_{-10}$}
\tlabel[bc](10,0.5){${}_{10}$}
\tlabel[bc](20,0.5){${}_{20}$}
\tlabel[bc](30,0.5){${}_{30}$}
\tlabel[bc](40,0.5){${}_{40}$}
\tlabel[bc](50,0.5){${}_{50}$}
\tlabel[bc](60,0.5){${}_{60}$}
\dotted\lines{(-1,-10),(61,-10)}
\dotted\lines{(-1,+10),(61,+10)}
\dotted\lines{(10,-12),(10,+12)}
\dotted\lines{(20,-12),(20,+12)}
\dotted\lines{(30,-12),(30,+12)}
\dotted\lines{(40,-12),(40,+12)}
\dotted\lines{(50,-12),(50,+12)}
\dotted\lines{(60,-12),(60,+12)}
\pen{1pt}
\circle{(20,0),4}
\circle{(50,0),10}
\lines{(58.8,-12.002),(-0.96,0.196)}
\lines{(58.8,12.002),(-0.96,-0.196)}
\lines{(51.2,-11.94),(5.733,12.05)}
\lines{(51.2,11.94),(5.733,-12.05)}
\end{mfpic}}

\Example

% \begin{exampleSimpleThree}{5em}{5em}{175pt}{Зображення цього тесту}
\begin{exampleSimpleThree}{5em}{5em}{144pt}{Зображення цього тесту}
\exmp{20
0
4
50
0
10}{4}{\raisebox{-56pt}[0pt][0pt]{\usebox{\mypictbox}}}%
\end{exampleSimpleThree}

\Note
Зображення наведене суто для кращого розуміння прикладу, Ваша програма малювати його не~повинна.

\Scoring
У цій задачі оцінювання потестове (кожен тест перевіряється й оцінюється незалежно від решти, 
отримані бали додаються).
% оцінка за розв'язок задачі є сумою оцінок за проходження цих окремих тестів).

\end{problemAllDefault}
