{

\begin{problemAllDefault}{Дуже важливі числа}

Марічка знає два Дуже Важливі Числ\'{а}\nolinebreak[3] --- $A$ та $B$. Усі ч\'{и}сла між $A$ та $B$ теж дуже важливі для Марічки. 
Сашко пообіцяв Марічці надрукувати всі дуже важливі для Марічки ч\'{и}сла, а\nolinebreak[3] також порахувати їх кількість. 
Причому, якщо ${A < B}$, то Сашко надрукує ці ч\'{и}сла в порядку зростання; інакше\nolinebreak[3] --- в порядку спадання.

\myflfigaw{%
\Examples
% \begin{exampleWidthsAndDefaultFileNames}{12em}{13em}
\begin{exampleSimple}{5em}{13em}
\exmp{1\\
5}{5 very important numbers:\\
1\\
2\\
3\\
4\\
5}%
\exmp{5\\
2}{4 very important numbers:\\
5\\
4\\
3\\
2}%
% \end{exampleWidthsAndDefaultFileNames}
\end{exampleSimple}}

\InputFile
Програмі на вхід подаються два Дуже Важливі Числ\'{а} $A$ та $B$ 
($-100\dib{{\<}}A\dib{{\<}}100$, $-100\dib{{\<}}B\dib{{\<}}100$, кожне число у своєму рядку).

\OutputFile
У перший рядок виведіть повідомлення про кількість дуже важливих для Марічки чисел (див.\nolinebreak[2] приклади; перший рядок завжди виводити символ-у-символ, як у прикладі, замінюючи лише число). 
Далі виведіть усі дуже важливі для Марічки ч\'{и}сла так, як це пообіцяв надрукувати Сашко.

\end{problemAllDefault}

}