{

%%% \makeTableLongtrue

\makeTableLongfalse

\begin{problemAllDefault}{Квантифікація предиката}

Задано прямокутну таблицю, що складається лише з нулів та/\nolinebreak[3]або одиниць.
До\nolinebreak[3] цієї таблиці треба застосувати одну з таких дій:

\begin{itemize}

\item
\texttt{AiAj} або \texttt{AjAi} (дія може бути позначена будь-яким з цих двох способів, при цьому смисл цих записів однаковий) --- якщо геть усі комірки таблиці містять одиниці, то результат дії одиниця, в усіх інших випадках результат нуль;

\item
\texttt{EiEj} або \texttt{EjEi} (дія може бути позначена будь-яким з цих двох способів, при цьому смисл цих записів однаковий) --- якщо хоча~б десь у таблиці є хоча~б одна одиниця, то результат дії одиниця, інакше результат нуль;

\item
\texttt{EiAj} --- якщо у~таблиці є хоча~б один рядок, що складається з самих лише одиниць, то результат дії одиниця, інакше результат нуль;

\item
\texttt{EjAi} --- якщо у~таблиці є хоча~б один стовпчик, що складається з самих лише одиниць, то результат дії одиниця, інакше результат нуль;

\item
\texttt{AiEj} --- якщо у кожному рядку таблиці є хоча~б одна одиниця, то результат дії одиниця, інакше результат нуль;

\item
\texttt{AjEi} --- якщо у кожному стовпчику таблиці є хоча~б одна одиниця, то результат дії одиниця, інакше результат нуль.


\end{itemize}

<<Хоча~б один/одна>> скрізь означає, що можна~1, а~можна й більше.

Напишіть програму, яка застосуватиме вказані дії до таблиці.

\InputFile
Перший рядок містить два цілі числа~$N$ та $M$ (кожне від~1 до~12)\nolinebreak[3] --- кількість рядків та кількість стовпчиків відповідно. 
Кожен з подальших $N$ рядків містить по рівно~$M$ нулів та/або одиниць, розділених одинарними пробілами.
Останній $(N{+}2)$-й рядок містить потрібну дію, тобто один із записів 
\texttt{AiAj}, або
\texttt{AjAi}, або
\texttt{EiEj}, або
\texttt{EjEi}, або
\texttt{EiAj}, або
\texttt{EjAi}, або
\texttt{AiEj}, або\nolinebreak[3]
\texttt{AjEi}.


\OutputFile
Виведіть єдину цифру 0~або~1\nolinebreak[3] --- результат дії.

\Examples

\ifAfour\vspace*{-\baselineskip}\par\mbox{}\hspace*{5em}\fi
\begin{exampleSimpleExtraNarrow}{2.5em}{3em}
\exmp{2 3
1 0 1
1 1 0
AiAj}{0}%
\exmp{2 3
1 0 1
1 1 0
AjAi}{0}%
\end{exampleSimpleExtraNarrow}%
\ifAfour\hspace{-0.5em}\fi
\begin{exampleSimpleExtraNarrow}{2.5em}{3em}
\exmp{2 3
1 0 1
1 1 0
EiEj}{1}%
\exmp{2 3
1 0 1
1 1 0
EjEi}{1}%
\end{exampleSimpleExtraNarrow}%
\ifAfour\hspace{-0.5em}\fi
\begin{exampleSimpleExtraNarrow}{2.5em}{3em}
\exmp{2 3
1 0 1
1 1 0
EiAj}{0}%
\exmp{2 3
1 0 1
1 1 0
EjAi}{1}%
\end{exampleSimpleExtraNarrow}%
\ifAfour\hspace{-0.5em}\fi
\begin{exampleSimpleExtraNarrow}{2.5em}{3em}
\exmp{2 3
1 0 1
1 1 0
AiEj}{1}%
\exmp{2 3
1 0 1
1 1 0
AjEi}{1}%
\end{exampleSimpleExtraNarrow}

% % % \ifAfour\end{small}\fi


\Scoring\phantomsection\label{text:scoring-by-blocks-in-quantification-problem-begin}
Тести 1--8 є тестами з умови; вони перевіряються, але самі по собі не~приносять балів.
Решта тестів утворюють такі 14 блоків:

{

\begin{small}

\def\tests#1{\mbox{тести #1;\hspace{0.25em plus 2em}}}
\def\points#1{\mbox{#1 балів;\hspace{0.25em plus 2em}}}
\def\whenChecked#1{перевіряється й оцінюється #1}
\def\checkedAlways{\whenChecked{завжди}}
\def\checkReq#1{\whenChecked{лише в разі успішного проходження тестів #1}}

\noindent
\ifAfour
\begin{tabular}{@{}p{0.04\textwidth}|p{0.31\textwidth}|p{0.59\textwidth}@{}}
\else
\begin{tabular}{@{}p{0.045\textwidth}|p{0.325\textwidth}|p{0.57\textwidth}@{}}
\fi
дія
&
$M=N=2$
&
$1\<N\<12$, ~ ~ $1\<M\<12$
\\\hline%\endhead
\ifAfour\begin{scriptsize}\fi
\texttt{AiAj} або \texttt{AjAi}
\ifAfour\par\vspace*{-\baselineskip}\end{scriptsize}\fi
&
\points{5}
\tests{9--40}
\checkedAlways
&
\points{20}
\tests{145--148}
\checkReq{9--40}
\\
\hline
\ifAfour\begin{scriptsize}\fi
\texttt{EiEj} або \texttt{EjEi} 
\ifAfour\par\vspace*{-\baselineskip}\end{scriptsize}\fi
&
\points{5}
\tests{41--72}
\checkedAlways
&
\points{20}
\tests{149--152}
\checkReq{41--72}
\\
\hline
\texttt{EiAj}
&
\points{5}
\tests{73--88}
\checkedAlways
&
\points{20}
\tests{153--168}
\checkReq{73--88}
\\
\hline
\texttt{EjAi}
&
\points{5}
\tests{89--104}
\checkedAlways
&
\points{20}
\tests{169--184}
\checkReq{89--104}
\\
\hline
\texttt{AiEj}
&
\points{5}
\tests{105--120}
\checkedAlways
&
\points{20}
\tests{185--200}
\checkReq{105--120}
\\
\hline
\texttt{AjEi}
&
\points{5}
\tests{121--136}
\checkedAlways
&
\points{20}
\tests{201--216}
\checkReq{121--136}
\\
\hline
будь-яка
&
\begin{footnotesize}%
\points{20}
\tests{137--144}
\checkReq{9--136}
\par\vspace*{-\baselineskip}\end{footnotesize}%
&
\points{80}
\tests{217--232}
\checkReq{1--216}
\\
\end{tabular}
% \end{longtable}

\end{small}

Скрізь мається на увазі, що для нарахування балів за блок Ваша програма повинна успішно пройти \emph{всі} тести цього блоку.\phantomsection\label{text:scoring-by-blocks-in-quantification-problem-end}

}

% % % У тестах 9--40 розмір таблиці гарантовано $2{\times}2$, дія гарантовано \texttt{AiAj} або \texttt{AjAi}; якщо \emph{всі} тести групи пройдені правильно, то нараховується 5~балів.

% % % У тестах 41--72 розмір таблиці гарантовано $2{\times}2$, дія гарантовано \texttt{EiEj} або \texttt{EjEi}; якщо \emph{всі} тести групи пройдені правильно, то нараховується 5~балів.

% % % У тестах 73--88 розмір таблиці гарантовано $2{\times}2$, дія гарантовано \texttt{EiAj}; якщо \emph{всі} тести групи пройдені правильно, то нараховується 5~балів.

% % % У тестах 89--104 розмір таблиці гарантовано $2{\times}2$, дія гарантовано \texttt{EjAi}; якщо \emph{всі} тести групи пройдені правильно, то нараховується 5~балів.

% % % У тестах 105--120 розмір таблиці гарантовано $2{\times}2$, дія гарантовано \texttt{AiEj}; якщо \emph{всі} тести групи пройдені правильно, то нараховується 5~балів.

% % % У тестах 121--136 розмір таблиці гарантовано $2{\times}2$, дія гарантовано \texttt{AjEi}; якщо \emph{всі} тести групи пройдені правильно, то нараховується 5~балів.

% % % У тестах 137--168 розмір таблиці гарантовано $2{\times}2$, дія може бути довільна; якщо \emph{всі} тести групи пройдені правильно, то нараховується 10~балів; цей блок тестів перевіряється й оцінюється \emph{лише} за умови, що пройдені абсолютно всі попередні тести 1--136.


\end{problemAllDefault}

}