\begin{problemAllDefault}{Дірка і Пробка}

Пан Дивак не~пожалів зусиль, щоб виготовити Дуже Дивну Дірку та Дуже Дивну Пробку. Тепер його цікавить, чи можна цією Пробкою затикати цю Дірку, і якщо можна, то наскільки глибоко вона заходитиме всер\'{е}дину Дірки.

{

\looseness=-1
Дуже Дивна Пробка являє собою послідовність нанизаних на спільну вісь циліндрів однакової висот\'{и}~1, але, як~правило, різних радіусів. Аналогічно, Дуже Дивна Дірка являє собою послідовність циліндричних порожнин зі спільною віссю, однаковими висотами~1, але, як~правило, різними радіусами.
Циліндри, які утворюють Пробку, пан Дивак рахує знизу догори, а\nolinebreak[3] циліндри, які утворюють Пробку\nolinebreak[3] --- згори донизу. Затикати Дірку Пробкою пан Дивак планує лише вставляючи Пробку в\nolinebreak[2] Дірку згори донизу, так, щоб спільна вісь усіх циліндрів Пробки збігалася зі спільною віссю усіх циліндрів Дірки. 

}

\begin{small}

\input 201718-2-D-statement-picture

\noindent
\begin{tabular}{@{}p{0.28\textwidth}|p{0.33\textwidth}|p{0.33\textwidth}@{}}
\begin{mfpic}[6.5]{-10}{10}{0}{8}
\Bung{5}{4}{2}{4}{9}{7}%{\hatch}
\Axe{-0.25}{6.5}
\end{mfpic}
&
\begin{mfpic}[6.5]{-13}{13}{-8}{1}
\Hole{10}{7}{5}{6}{8}{11}{3}{12}%{\lhatch}
\Axe{-7.25}{0.5}
\end{mfpic}
&
\begin{mfpic}[6.5]{-10}{10}{0}{8}
\Bung{5}{4}{2}{4}{9}{7}%{\hatch}
\begin{coords}
\shift{(0,5)}
\Hole{10}{7}{5}{6}{8}{11}{3}{12}%{\lhatch}
\end{coords}
\Axe{-2.25}{6.5}
\end{mfpic}
\\
Пробка з послідовністю радіусів
5, 4, 2, 4, 9,~7.
&
Дірка з послідовністю радіусів
10, 7, 5, 6, 8, 11,~3.
&
Пробка з указаними радіусами у Дірці з указаними радіусами
\\
\end{tabular}

\end{small}

Наприклад, на рисунку зображено окремо Пробку, окремо Дірку, та ситуацію, коли ця Пробка заткнула цю Дірку; рух Пробки донизу скінчився тоді, коли частина Пробки радіусом~9 лягла поверх частини Дірки радіусом~7. 
Зверніть увагу, що частина Пробки радіусом~5 пройшла крізь частину Дірки радіусом~5, бо Пан Дивак натискає на Пробку зі значною силою (але не настільки великою, щоб пропхнути строго більше крізь строго менше).

% % % Що\nolinebreak[3] ж\nolinebreak[3] стосується однакових радіусів~5, то Пан Дивак завжди натискає на Пробку з силою, достатньою, щоб частини Пробки проходили через частини Дірки того самого радіуса, але\nolinebreak[2] не~настільки сильно, щоб якась частина Пробки пройшла через частину Дірки меншого радіуса.


Напишіть програму, яка за описами Пробки та Дірки з'ясовуватиме, як\nolinebreak[3] глибоко ця Пробка може зайти углиб цієї Дірки.

\InputFile
\mbox{1-й}\nolinebreak[3] рядок містить єдине число\nolinebreak[3] $N$\nolinebreak[3] --- кількість циліндрів у\nolinebreak[3] Пробці. \mbox{2-й}\nolinebreak[3] рядок містить (розділені пробілами) $N$\nolinebreak[3] радіусів циліндрів Пробки $a_1$,\nolinebreak[1] $a_2$,\nolinebreak[3] \dots,\nolinebreak[2] $a_N$, знизу догори. 
\mbox{3-й}\nolinebreak[3] рядок містить єдине число\nolinebreak[3] $M$\nolinebreak[3] --- кількість циліндрів у\nolinebreak[3] Дірці. \mbox{4-й}\nolinebreak[3] рядок містить (розділені пробілами) $M$\nolinebreak[3] радіусів циліндрів Дірки $b_1$,\nolinebreak[1] $b_2$,\nolinebreak[3] \dots,\nolinebreak[2] $b_M$, згори донизу. Обмеження на\nolinebreak[2] $N$,\nolinebreak[2] $M$\linebreak[2] див.\nolinebreak[2] у\nolinebreak[2] розділі <<Оцінювання>>;\linebreak[2] усі значення $a_i$ та\nolinebreak[3] $b_j$ є цілими з проміжку від~1 до~$10^9$.


\OutputFile
Виведіть єдине число\nolinebreak[3] --- глибина, на яку Пробка (рахуючи від її низу) увіходить всер\'{е}дину Дірки (рахуючи від її верху). У~випадку, коли Пробка не~заходить всередину Дірки, бо найнижчий циліндр Пробки ширший за найвищий отвір Дірки, слід виводити число~0. У~випадку, коли Пробка може пройти Дірку наскрізь, слід виводити число\nolinebreak[3] 777555777.

% % % \Examples
% % % \begin{exampleSimple}{9em}{5em}
% % % \exmp{6
% % % 5 4 2 4 9 7
% % % 7
% % % 10 7 5 6 8 11 3}{5}%
% % % \exmp{2
% % % 17 23
% % % 2
% % % 4 7}{0}%
% % % \exmp{2
% % % 4 7
% % % 2
% % % 17 23}{777555777}%
% % % \end{exampleSimple}

% % % \Examples

% % % \noindent
% % % \begin{exampleSimple}{7.5em}{5em}
% % % \exmp{6
% % % 5 4 2 4 9 7
% % % 7
% % % 10 7 5 6 8 11 3}{5}\end{exampleSimple}
% % % \begin{exampleSimple}{4.5em}{5em}
% % % \exmp{2
% % % 17 23
% % % 2
% % % 4 7}{0}\end{exampleSimple}
% % % \begin{exampleSimple}{4.5em}{5em}
% % % \exmp{2
% % % 4 7
% % % 2
% % % 17 23}{777555777}%
% % % \end{exampleSimple}

\Examples

\makeTableLongfalse
\begin{exampleSimpleThree}{7.5em}{5em}{\ifAfour 11cm\else 10.25cm\fi}{Примітки}
%%%\noindent\hspace*{-0.5cm}\begin{exampleThreeWithSpecNameColTwoAndLineStretch}{7.5em}{5em}{9.75cm}{Результати}{Примітки}{-3pt}{-3pt}{0.625}
\exmp{6
5 4 2 4 9 7
7
10 7 5 6 8 11 3}{5}{Приклад з рисунків; нижче за верхній край Дірки зайшли 5 з 6 циліндрів Пробки}%
\exmp{2
17 23
2
4 7}{0}{Самий низ Пробки лежить згори на самому верху Дірки}%
\exmp{2
9 23
2
14 7}{1}{Самий низ (9) Пробки зайшов всередину самого верху (14) Дірки, а потім Пробка обома циліндрами вперлася в Дірку (9~у~7, 23~у~14)}%
\exmp{2
4 7
2
17 23}{777555777}{Пробка може пройти крізь Дірку наскрізь}%
\exmp{5
2 4 7 17 23
4
36 30 24 18}{7}{Пробка майже пройшла крізь Дірку, але самий верх Пробки (23) все-таки вперся у самий низ Дірки (18)}%
%%%\end{exampleThreeWithSpecNameColTwoAndLineStretch}
\end{exampleSimpleThree}

% % % \begin{mfpic}[5]{-10}{10}{0}{8}
% % % \Bung{2}{4}{7}{17}{23}{0}%{\hatch}
% % % \begin{coords}
% % % \shift{(0,7)}
% % % \Hole{36}{30}{24}{18}{40}{40}{40}{40}%{\lhatch}
% % % \end{coords}
% % % \Axe{-2.25}{6.5}
% % % \end{mfpic}


% % % \Note
% % % \mbox{1-й} приклад входу/виходу відповідає наведеним рисункам.

\Scoring
20\%\nolinebreak[3] балів припадає на тести, в\nolinebreak[3] яких кількість циліндрів Пробки ${N\,{=}\,1}$, Дірки ${2\,{\<}\,M\,{\<}\,12345}$;
ще\nolinebreak[3] 20\%\nolinebreak[3] балів\nolinebreak[3] --- на тести, в\nolinebreak[3] яких ${2\,{\<}\,N\,{\<}\,12345}$, ${M\,{=}\,1}$;
ще\nolinebreak[3] 20\%\nolinebreak[3] балів\nolinebreak[3] --- на тести, в\nolinebreak[3] яких $12\dib{{\<}}N\dib{{\<}}123$, $12\dib{{\<}}M\dib{{\<}}123$; 
решта\nolinebreak[3] 40\%\nolinebreak[3] балів\nolinebreak[3] --- на тести, в\nolinebreak[3] яких $12\dib{{\<}}N\dib{{\<}}123456$, $12\dib{{\<}}M\dib{{\<}}123456$.
%
Писати треба одну програму, а не різні програми для різних випадків; єдина мета цього переліку різних блоків обмежень\nolinebreak[3] --- дати уявлення про те, скільки балів можна отримати, якщо розв’язати задачу частково.




\end{problemAllDefault}