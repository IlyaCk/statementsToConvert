\begin{problem}{Гра ``Вгадай число''}{Клавіатура (stdin)}{Екран (stdout)}{1 сек}{256 мегабайтів}
\label{prob:201617-oioi-C-guess-number}

Напишіть програму, яка гратиме у~гру ``Вгадай число'': 
визначатиме загадане суперником ціле число із заданого діапазону
на~основі запитів до суперника. 

У~кожному запиті Ваша програма повинна виводити 
слово \texttt{try} та одне ціле число, 
що трактується як запитання \textsl{<<Загадане число~\dots?>>}.
Суперник на це відповідає (гарантовано чесно) одне з~трьох:
або, що~с\'{а}ме~це число й загадане,
або, що~загадане число більше,
або, що~загадане число менше.
Ваша програма повинна продовжувати (або завершити) гру, 
враховуючи отримані відповіді.
Кількість спроб вгадування (незалежно від величини проміжку)\nolinebreak[3] --- 
до~50~(п'ятдесяти), включно.
При перевищенні цього ліміту, Вашій програмі присуджується технічна поразка
(вона не~отримує балів за відповідний тест).

Тобто, ця задача є інтерактивною: 
Ваша програма не~отримає всіх вхідних даних на~початку,
а~отримуватиме по~мірі виконання доуточнення, 
котрі залежатимуть від попередніх дій Вашої програми. 
Тим~не~менш, \begin{slshape}її перевірка теж відбувається 
\underline{автоматично}\end{slshape}. 
Тому, у~цій Вашій програмі, як і в програмах--розв'язках інших задач, 
теж\nolinebreak[3] слід 
\emph{не}~<<організовувати діалог інтуїтивно зрозумілим чином>>,
а~чітко дотримуватися формату. Тільки це не~формат вхідного та вихідного файлів,
а~формат спілкування з~програмою, котра грає роль суперника.

\InputFile
На~початку, один раз, на вхід Вашій програмі подаються 
записані в один рядок через пропуск (пробіл) два цілі числ\'{а} $a$, $b$ 
($-2\cdot10^9\dib{{\<}}a\dib{{\<}}b\dib{{\<}}2\cdot10^9$) --- межі проміжку.
Це~означає, що загадане ціле число гарантовано перебуває в~межах 
$a\dib{{\<}}x\dib{{\<}}b$.
На~кожному наступному кроці, на вхід Вашій програмі подається 
рядок, що містить рівно один з~трьох символів:
\begin{itemize}
\item
~ \verb"=" ~ означає, що останнє виведене Вашою програмою число і є загаданим;
\item
~ \verb"+" ~ означає, що загадане число більше, ніж виведене Вашою програмою;
\item
~ \verb"-" ~ означає, що загадане число менше, ніж виведене Вашою програмою.
\end{itemize}

\OutputFile
Поки Ваша програма <<не~впевнена>>, яке число загадане,  
вона повинна повторювати виведення 
в\nolinebreak[2] один рядок через пропуск (пробіл або табуляцію) 
сл\'{о}ва ``\texttt{try}'' (маленькими латинськими буквами, без лапок)
та одного цілого числ\'{а}\nolinebreak[3] --- 
чергової спроби вгадування\nolinebreak[3].
Кожне число-спроба повинно бути в~межах $a\dib{{\<}}x\dib{{\<}}b$ (див.~вище).
Настійливо рекомендується після кожного такого виведення 
робити дію \verb"flush(output)"\nolinebreak[2] (Pascal), 
вона~ж \verb"cout.flush()"\nolinebreak[2] (C++), 
вона~ж \verb"fflush(stdout)"\nolinebreak[2] (C), 
вона~ж \verb"sys.stdout.flush()"\nolinebreak[2] (Python),
вона~ж \verb"System.out.flush()"\nolinebreak[2] (Java).
Це~істотно зменшує ризик, 
що~проміжна відповідь <<застряне>> десь по~дорозі, 
не~дійшовши до програми-суперника.

Коли Вашій програмі <<стає абсолютно очевидно>>, яке число загадане,
вона повинна припинити основну частину гри і~вивести 
в\nolinebreak[2] один рядок через пропуск (пробіл або табуляцію) 
слово ``\texttt{answer}'' (маленькими латинськими буквами, без лапок)
та вгадане число. Після цього слід остат\'{о}чно завершити роботу, 
не~виводячи жодного іншого символа.

\Example

\noindent\hspace*{\ifAfour -0.4cm\else -0.5cm\fi}\begin{exampleThreeWithSpecNameColTwoAndLineStretch}{5em}{5em}{\ifAfour 11.8cm\else 11cm\fi}{Результати}{Примітки}{\ifBigStretch -7pt\else -5.5pt\fi}{\ifBigStretch -4pt\else -2pt\fi}{0.625}
\exmp{1 100
~
+
~
-
~
-
~
-
~
=}{~
try 10
~
try 20
~
try 19
~
try 18
~
try 17
~
answer 17}{\noindent\begin{minipage}[t]{\ifAfour 11.8cm\else 11cm\fi}
\setstretch{0.875}
1)~Програма-суперник з самого початку загадала~17.

2)~Щоб було видніше, в якому порядку відбуваються введення та виведення, у~прикладі використано розділення порожніми рядками. При фактичній перевірці програма-суперник не~виво\-ди\-тиме ніяких порожніх рядків і не~чекатиме їх від Вашої програми.

3)~Ваша програма не~зобов'язана грати за~стратегією, що наведена у~цьому прикладі. Це лише приклад можливої роботи програми, що успішно завершує конкретно~цю~гру.
\end{minipage}}%
\end{exampleThreeWithSpecNameColTwoAndLineStretch}

\end{problem}