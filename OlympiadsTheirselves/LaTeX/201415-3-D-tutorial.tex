\Tutorial	
% 
\MyParagraph{Як легко набрати частину балів.}
Для самої лише підзадачі~D1 і часткового випадку <<всі кімнати мають прямокутну форму>>в якості розв'язку можна запропонувати програму \IdeOne{HHr9a3}.
Вона спирається на те, що у випадку прямокутності кімнат кожна кімната однозначно задається лівим верхнім кутом, а~перевіряти, чи~справді клітинка є таким кутом, можна умовою \texttt{\mbox{(data[i][j]='0')} \mbox{and} \mbox{(data[i-1][j]='1')} \mbox{and} \mbox{(data[i][j-1]='1')}}, тобто сама клітинка вільна, а ліворуч і згори стіни.
\ifAfour\else\par\fi
Очевидно (в~т.~ч. з\nolinebreak[3] \mbox{2-го}\nolinebreak[3] тесту з\nolinebreak[3] умови), що для <<закручених>> кімнат це може й не\nolinebreak[3] бути правдою. Але\nolinebreak[2] в\nolinebreak[3] умові обіцяно значну частину тестів з кімнатами прямокутної форми, тож при відсутності кращих ідей можна написати хоча~б такий розв’язок. Він набирає 26~балів (з~50 за усю~D1, зі~100 за усю~D).



\MyParagraph{Повний розв'язок підзадачі~D1 (лише ідеї).}
Для відстеження (як~завгодно <<закручених>>) кімнат можна реалізувати будь-який з алгоритмів:

\begin{enumerate}

\item	
\emph{Пошук ушир} (\emph{у~ширину}), англ. \emph{breadth first search} (\emph{BFS});

% % % \vspace{0.125pt plus 2pt}
\item
\emph{Пошук углиб} (\emph{у~глибину}), англ. \emph{depth first search} (\emph{DFS});

% % % \vspace{0.125pt plus 2pt}
\item
Різноманітні алгоритми графічної \emph{заливки} (\emph{flood fill}).

\end{enumerate}


Усі ці алгоритми легко знайти в літературі чи Інтернеті, й усі вони надто громіздкі, щоб пояснювати їх тут. Кожним із них можна реалізувати задачу зі складністю $\Theta(N{\cdot}M)$. Для цього треба:
\begin{enumerate}
\item
Просто перебирати усі клітинки з\'{а}мку, і щоразу, знайшовши~0, запускати BFS/DFS/заливку, щоб повністю виділити відповідну кімнату, позамінявши її нулі на інші значення, та обчислити її розмір.

\vspace{0.125pt plus 2pt}
\item
Забезпечити, щоб кожен такий виклик BFS/DFS/заливки, якому вказується, починаючи звідки дослідити кімнату, працював за час, пропорційний розміру цієї кімнати, а не усього з\'{а}мку.

\end{enumerate}


%%% \ifallIdeOneLinksCopiedHere\myhrulefill\fi % TODO: check!!!
% % % \vspace{0.5\baselineskip plus 0.25ex}
% % % \myhrulefill
% % % \vspace{0.5\baselineskip plus 0.25ex}

Що з DFS/BFS/заливки тут простіше й доречніше\nolinebreak[3] --- важко сказати, сильно залежить від умінь конкретного учасника. Часто стверджують, ніби найпростіше реалізувати DFS, але ця думка сумнівна, бо сформована під впливом мови Паскаль, у~якій є рекурсія й нема стандартної бібліотечної черги. 
Згадуючи DFS, варто згадати й про проблему переповнення стеку (див., зокрема, стор.~\mbox{\pageref{text:static-dynamic-and-stack-memory-begin}--\pageref{text:static-dynamic-and-stack-memory-end}}).
Конкретно на цій обласній олімпіаді забезпечено великий стек, тож \emph{можна} без проблем писати рекурсивний DFS; але, враховуючи великий розмір з\'{а}мку $1000{\*}1000$, при інших налаштуваннях ejudge проблеми з переповненням стеку рекурсивним DFS цілком можливі.

\MyParagraph{Підзадача~D2.}
Частину балів (орієнтовно до~20 з~50) можна отримати, розв’язуючи підзадачу~D1 багатократно (для абсолютно кожної <<1>> у внутрішній стіні, замінимо її на <<0>> і заново розв’яжемо~D1; серед усіх таких відповідей виберемо максимальну). Але такий алгоритм має складність $O(M^2{\cdot}N^2)$, тож ні\'{я}к\nolinebreak[2] не\nolinebreak[3] може бути ефективним розв’язком для ${M\,{\approx}\,N\,{\approx}\,1000}$.

{
\def\castleSubProblemTwoFirstPhrase{Перепишемо про\-гра\-му-роз\-в'\-я\-зок під\-зада\-чі~D1 так, щоб при підрахунку кількостей та розмірів кімнати не~просто виділялися, а \emph{різні кімнати} виділялися \emph{різними значеннями} (а~клітинки однієї кімнати\nolinebreak[3] --- однаковими). Якщо це робити прямо у\nolinebreak[3] масиві з позначками <<0\nolinebreak[3] --- прохід, 1\nolinebreak[3] --- стіна>>, виділяючи кімнати позначками 2, 3, 4,~\dots, може вийти, наприклад, так\ifAfour, як праворуч.\else:\fi}
\def\castleSubProblemTwoFirstExampleInExplain{\begin{exampleSimpleThreeWithSpecNameColTwo}{6em}{11.5em}{10em}{Виділені кімнати}{Розміри кімнат}%
\exmp{9 12
111111111111
101001000001
111001011111
100101000001
100011111101
100001000101
111111010101
100000010001
111111111111}{~\\
1 1 1 1 1 1 1 1 1 1 1 1\\
1 2 1 3 3 1 4 4 4 4 4 1\\
1 1 1 3 3 1 4 1 1 1 1 1\\
1 5 5 1 3 1 4 4 4 4 4 1\\
1 5 5 5 1 1 1 1 1 1 4 1\\
1 5 5 5 5 1 4 4 4 1 4 1\\
1 1 1 1 1 1 4 1 4 1 4 1\\
1 4 4 4 4 4 4 1 4 4 4 1\\
1 1 1 1 1 1 1 1 1 1 1 1}{\raisebox{-12pt}{\begin{\mainFontFamily}\begin{tabular}{@{}r|c@{}c@{$\,\,\,\,$}c@{$\,\,$}c@{$\,\,$}c}індекс & \dots &2&3&4&5\\\hline{}значення & \dots&1&5&28&9\end{tabular}\end{\mainFontFamily}}}%
\end{exampleSimpleThreeWithSpecNameColTwo}}

\ifAfour
\myflfigaw{\hspace*{-2em}\castleSubProblemTwoFirstExampleInExplain\hspace*{-1em}}
\hyphenpenalty=-1
\tolerance=9999
\castleSubProblemTwoFirstPhrase
\else
\castleSubProblemTwoFirstPhrase
\par
\vspace{-0.75\baselineskip}
\par
\begin{center}
\castleSubProblemTwoFirstExampleInExplain
\end{center}
\par
\vspace{-0.75\baselineskip}
\fi

Якщо зберігати розміри кімнат у масиві так, щоб індексами масиву були ті самі ч\'{и}сла, якими позначено кімнати, то для визначення, яка вийде площа кімнати після руйнування стіни, можна просто додати до одинички (площі самої зруйнованої стіни) площі кімнат-сусідів. 

Тому перебір усіх можливих <<1>> (у~внутрішніх стінах) можна залишити, бо тепер для кожної такої <<1>> треба робити значно менше дій 
(не~виділяти заново відповідні кімнати, а\nolinebreak[3] бр\'{а}ти готові, знайдені один раз, дані про них).

Здається <<логічним>> (і~приклади з умови це <<підтверджують>>), ніби максимальна кімната буде утворена за рахунок об’єднання двох кімнат. Але % насправді 
це лише поширений випадок, а\nolinebreak[3] не\nolinebreak[3] обов’язкова властивість: замість \mbox{2-х} може бути будь-яке число від~1 до~4. 
%% (див.\nolinebreak[2] приклади). 
%% Тому  краще, не~роблячи необґрунтованих припущень, акуратно реалізувати для кожної не~зовнішньої\nolinebreak[3] <<1>> перегляд усіх сусідів-кімнат, додаючи площі усіх різних.

\def\castleTestIXcomment{Тест \textnumero$\,$9\ifAfour{} (найлівіший з прикладів праворуч)\fi. Більшість внутрішніх стін <<товсті>>, тож вилучення \emph{однієї} <<1>> зазвичай не\nolinebreak[3] призводить до з'єднання кімнат. А\nolinebreak[3] в\nolinebreak[3] тому єдиному місці, де призводить (\mbox{2-й}\nolinebreak[3] знизу рядок)\nolinebreak[3] --- утворюється кімната площею всього-навсього $2\dib{{+}}1\dib{{+}}1\dib{{=}}4$.
Набагато більшу площу $44\dib{{+}}1\dib{{=}}45$ можна отримати, зруйнувавши будь-яку (не~зовнішню) стіну кімнати площі~44, утворивши <<нішу>> замість <<проходу>>.}
\def\castleTestIXcontent{\begin{minipage}{11em}\ifAfour\begin{scriptsize}\else\begin{small}\fi\renewcommand{\baselinestretch}{0.875}\begin{alltt}13 25\\
1111111111111111111111111\\
1000000110000110000110001\\
1000000110000110000110001\\
1000000110000110000110001\\
1000000110000110000110001\\
1000000110000111111110001\\
1000000110000111111110001\\
1111111110000110000110001\\
1111111110000110000110001\\
1000000110000111111110001\\
1000000110000111111110001\\
1000000110000110010110001\\
1111111111111111111111111\end{alltt}\ifAfour\end{scriptsize}\else\end{small}\fi\end{minipage}}
\def\castleTestXVIcomment{Тест \textnumero$\,$16 \ifAfour(середній, він же лівіший з майже однакових прикладів)\else(лівіший з прикладів)\fi. Руйнування однієї <<1>> по центру призводить до з’єднання \emph{відразу 4-х} кімнат. Переконавшись у можливості такої ситуації, легко уявити і вхідні дані \ifAfour{}най\-\fi{}пра\-ві\-шо\-го прикладу, де різні сусіди центральної одинички до того ж ще й десь далеко з'єднані в одну кімнату.}
\def\castleTestXVIcontent{\fbox{\begin{minipage}{5.5em}\begin{small}\renewcommand{\baselinestretch}{0.875}\begin{alltt}9 11\\
11111111111\\
10011011001\\
11001010011\\
11101010111\\
10100100101\\
11101010111\\
11001010011\\
10011011001\\
11111111111\end{alltt}\end{small}\end{minipage}}\fbox{\begin{minipage}{5.5em}\begin{small}\renewcommand{\baselinestretch}{0.875}\begin{alltt}9 11\\
11111111111\\
10011011001\\
11000010011\\
11101010111\\
10100100101\\
11101010111\\
11001010011\\
10011011001\\
11111111111\end{alltt}\end{small}\end{minipage}}}

\ifAfour
\myflfigaw{\castleTestIXcontent\castleTestXVIcontent}
\castleTestIXcomment
\par
\castleTestXVIcomment
\else
\par
\vspace{0.375\baselineskip}
\par
\mytextandpicture{\castleTestIXcomment}{\castleTestIXcontent}
\par
\vspace{0.375\baselineskip}
\par
\mytextandpicture{\castleTestXVIcomment}{\castleTestXVIcontent}
\fi

Отже\nolinebreak[3] --- акуратно переглянути для кожної не~зовнішньої <<1>> \emph{усі} кімнати-сусіди, додаючи площі \emph{усіх різних}.
Складність цього алгоритму теж (як\nolinebreak[3] і\nolinebreak[3] для\nolinebreak[2] під\-зада\-чі~D1) $\Theta(N{\cdot}M)$, але множник, яким нехтують у асимптотичних позначеннях, тепер значно більший.

}