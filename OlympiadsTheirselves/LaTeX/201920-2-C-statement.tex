\begin{problemAllDefault}{Прямокутні суми}

Дано прямокутну таблицю розмірами $N$ рядків на $M$ стовпчиків, елементи якої є натуральними числами. Потрібно багатократно знаходити суми всіх чисел, що потрапляють у деякі прямокутники цієї таблиці, утворені рядками з \mbox{$i_1$-го} по \mbox{$i_2$-й} та стовпчиками з \mbox{$j_1$-го} по \mbox{$j_2$-й} (нумерація\nolinebreak[2] з~1, м\'{е}жі включно, ${1\,{\<}\,i_1\,{\<}\,i_2\,{\<}\,N}$, нумерація рядків згори донизу, ${1\,{\<}\,j_1\,{\<}j_2\,{\<}\,M}$, нумерація стовпчиків зліва направо).

\InputFile  
У~\mbox{1-му}~рядку через одинарний пробіл записано ч\'{и}сла $N$ та~$M$, далі $N$~рядків по $M$ чисел у кожному містять значення комірок таблиці. 
Далі в окремому рядку записано число~$T$\nolinebreak[3] --- кількість подальших запитів, потім ще\nolinebreak[3] $T$\nolinebreak[3] рядків\nolinebreak[3] --- сам\'{і} запити, кожен у вигляді ${i_1\,\,i_2\,\,j_1\,\,j_2}$ (через одинарні пробіли, ${1\,{\<}\,i_1\,{\<}\,i_2\,{\<}\,N}$,\hspace{0.125em plus 1em} ${1\,{\<}\,j_1\,{\<}j_2\,{\<}\,M}$).

\OutputFile 
Рівно $T$ рядків, кожен з яких містить єдине число\nolinebreak[3] --- суму, що є відповіддю чергового запиту.

\savebox{\mypictbox}{\begin{tabular}[t]{@{}p{9em}c@{}}<<1~4~1~5>> задає всю таблицю;\hspace*{0pt plus 1em}\linebreak[3] <<1~2~3~4>>\nolinebreak[3] --- правий-верхній з виділених прямокутників;\hspace*{0pt plus 1em}\linebreak[3] <<2~4~2~2>>\nolinebreak[3] --- лівий.
&
\begin{tabular}[t]{ccccc}
\\\cline{3-4}
1 & 2 & \multicolumn{1}{|c}{3} & \multicolumn{1}{c|}{4} & 5 \\\cline{2-2}
2 & \multicolumn{1}{|c|}{3} & \multicolumn{1}{|c}{4} & \multicolumn{1}{c|}{5} & 6 \\\cline{3-4}
3 & \multicolumn{1}{|c|}{4} & 5 & 6 & 7 \\
4 & \multicolumn{1}{|c|}{5} & 6 & 7 & 8 \\\cline{2-2}
\end{tabular}
\end{tabular}}

\Example

\begin{exampleSimpleThree}{5em}{5em}{19em}{Коментар}
\exmp{4 5
1 2 3 4 5
2 3 4 5 6
3 4 5 6 7
4 5 6 7 8
3
1 4 1 5
1 2 3 4
2 4 2 2}{90
16
12}{\usebox{\mypictbox}}%
\end{exampleSimpleThree}

\Scoring
Оцінювання поблокове (бали за блок нараховуються лише в разі успішного проходження всіх тестів блоку); тест з умови є \mbox{1-м} і входить у блок~1.
Стовпчик <<передумови>> означає, що розв'язок запускатиметься і перевірятиметься на поточному блоці, лише якщо цей самий розв'язок успішно пройшов блоки, згадані в передумовах.
Нерівності через кому (\mbox{як-то} <<${100\,{\<}\,N,M\,{\<}\,400}$>>) означають, що всі перелічені через кому змінні перебувають в указаному діапазоні. Обмеження на ``$a_{ij}$'' вказують можливі 
значення % сам\'{и}х 
елементів таблиці.
% % % %
% % % Кожен з р\'{о}зв'язків, які Ви здасте, перевірятиметься, починаючи з першого блоку, на всіх, для яких виконані передумови; з~різних спроб вибиратиметься максимальний результат, але це буде результат \emph{однієї} найкращої Вашої програми.


\begin{small}

\begin{longtable}{@{}c|c|p{0.57\textwidth}|p{0.08\textwidth}|c@{}}
блок\hspace*{-0.01\textwidth}
&
тести
&
\multicolumn{1}{c|}{обмеження}
&\begin{minipage}{0.08\textwidth}\renewcommand\baselinestretch{0.75}\begin{scriptsize}
перед\-умови\par
\end{scriptsize}\end{minipage}
&
бали
\\\hline\hline\endhead
1 & 
1--5 & 
${2\,{\<}\,N,M,T,a_{ij}\,{\<}\,9}$ & 
нема & 
15\% 
\\\hline
2 & 
6--10 & 
${1\,{\<}\,N,M,T\,{\<}\,100}$; ${1\,{\<}\,a_{ij}\,{\<}\,10^4}$ &
блок~1 & 
15\% \\\hline
3 & 
11--15 & 
${1\,{\<}\,N\,{\<}\,25}$; ${100\,{\<}\,M\,{\<}\,2019}$; ${10^3\,{\<}\,T,a_{ij}\,{\<}\,10^5}$ &
блок~1 & 
15\% \\\hline
4 & 
16--20 & 
${100\,{\<}\,N\,{\<}\,2019}$; ${1\,{\<}\,M\,{\<}\,25}$; ${10^3\,{\<}\,T,a_{ij}\,{\<}\,10^5}$ &
блок~1 & 
15\% \\\hline
5 &
21--27 & 
${100\,{\<}\,N,M\,{\<}\,400}$; ${123\,{\<}\,T,a_{ij}\,{\<}\,12345}$ &
бл.\mbox{1--2} & 
15\% \\\hline
6 &
28--35 & 
${400\,{\<}\,N,M\,{\<}\,700}$; ${12345\,{\<}\,T\,{\<}\,222555}$;\hfill~\linebreak ${1\,{\<}\,a_{ij}\,{\<}\,10^9}$ &
блоки \mbox{1--2},~5 & 
25\% \\
\end{longtable}

\end{small}

\end{problemAllDefault}
