\Tutorial

Задача, насправді, досить проста й нудна, в ній нема чого пояснювати. Треба просто перевірити, 
чи\nolinebreak[2] ${A\,{\<}\,B}$,
чи\nolinebreak[2] ${A\,{>}\,B}$,
й залежно від цього виразити кількість чисел як 
${B\,{-}\,A\,{+}\,1}$ чи
${A\,{-}\,B\,{+}\,1}$ відповідно,
та пустити цикл 
в будь-якому разі від~$A$\nolinebreak[2] до~$B$,
але (залежно від все\nolinebreak[2] тієї~ж умови)
чи\nolinebreak[3] то\nolinebreak[2] у\nolinebreak[3] бік збільшення, 
чи\nolinebreak[3] то\nolinebreak[2] у\nolinebreak[3] бік зменшення відповідно.
Головне, на чому можна було помилитися\nolinebreak[3] --- випадок ${A\,{=}\,B}$.
Його цілком можна, як тут і запропоновано, включити до випадку ${A\,{\<}\,B}$,
але треба не~забути про\nolinebreak[3] це і перевірити, чи\nolinebreak[3] це\nolinebreak[2] зроблено правильно.
 
