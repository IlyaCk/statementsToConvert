{

\begin{problemAllDefault}{Файлова система}

У багатьох файлових системах дисковий об’єм, зайнятий файлом, не~завжди дорівнює розміру файла. Це\nolinebreak[2] пов’язане з тим, що диск поділяється на кластери однакового розміру, і кожен кластер може бути або вільним, або використаним одним файлом (але\nolinebreak[2] не\nolinebreak[2] поділеним між кількома файлами). Для\nolinebreak[2] прикладу, на диску з розміром кластера 512\nolinebreak[3] байтів зберігається файл розміром 600\nolinebreak[3] байтів. В~такому випадку файл зберігатиметься у двох кластерах, займаючи 1024\nolinebreak[3] байти.

Напишіть програму, яка, прочитавши розмір кластера, кількість файлів і їхні розміри, обчислюватиме сумарний об’єм, який займають ці файли на диску.

\InputFile
Ваша програма має прочитати спочатку розмір кластера, потім (у\nolinebreak[3] наступному рядку) кількість файлів\nolinebreak[3] $N$, потім (у\nolinebreak[3] наступному рядку) $N$\nolinebreak[3] чисел, розділених пропусками (пробілами)\nolinebreak[3] --- розмір кожного з цих файлів.
%
Розмір кластера є натуральним числом від 1 до\nolinebreak[2] 1048576; кількість файлів є цілим числом від 0 до~10; розмір кожного файлу є цілим числом від 0 до\nolinebreak[2] 1048576.

\OutputFile
Ваша програма має вивести єдине ціле число в єдиному рядку\nolinebreak[3] --- знайдений сумарний об’єм, зайнятий файлами на\nolinebreak[3] диску.

\Examples
\ifAfour
\hspace*{-1.125em}
\else
\par\noindent\hspace*{-0.25em}
\fi
\begin{exampleSimple}{\ifAfour 8em\else 8.5em\fi}{5em}
\exmp{512
1
600}{1024}%
\exmp{1024
3
4096 33792 76800}{114688}%
\end{exampleSimple}%
\ifAfour%
\hspace{-1.125em}%
\else%
\hspace{-0.5em}%
\fi%
\begin{exampleSimple}{\ifAfour 13em\else 13.5em\fi}{5em}
\exmp{92651
6
6767 8170 6899 4827 1 9019}{555906}%
\exmp{32768
5
16 32 128 128 0}{131072}%
\end{exampleSimple}

\Note
Під час перевірки в єджаджі, саме ці чотири приклади, саме в такому порядку, й будуть тестами 1--4.

\end{problemAllDefault}

}