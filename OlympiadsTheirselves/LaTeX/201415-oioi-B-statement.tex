\begin{problemAllDefault}{Перевезення вантажу}

Перевізник транспортує вантаж залізницею, а потім річковим транспортом.

Товар транспортується в потязі, який складається з $N$ (${1\,{\<}\,N\,{\<}\,100}$) вагонів, в кожному з яких $k_1$,~\dots, $k_N$ (${1\,{\<}\,k_i\,{\<}\,1000}$) коробок вантажу.

У річковому порті вантаж з потягу перевантажують на корабель, який може взяти не більше ніж $P$ (${1\,{\<}\,P\,{\<}\,10000}$) коробок. Якщо якісь коробки не\nolinebreak[3] помістяться на цей корабель, вони дуже довго чекатимуть наступного. Причому чекатимуть обов'язково у тих вагонах, в\nolinebreak[3] яких прибули у~порт.

Для ефективного транспортування диспетчеру необхідно розвантажити якомога більше вагонів, завантаживши корабель, та відправити звільнені вагони на наступне завантаження.

\Task	Напишіть програму \texttt{transport}, яка визначала\nolinebreak[3] б максимальну кількість вагонів, які можна розвантажити, не~перевищуючи вантажопідйомність корабля.

\InputFile  1-й рядок: єдине число $N$ --- кількість вагонів в потязі.

\ifAfour\else
\noindent
\fi
2-й рядок: $k_1$~\dots{} $k_N$ через пропуски (пробіли)\nolinebreak[3] --- кількості коробок у вагонах.

\ifAfour\else
\noindent
\fi
3-й рядок: $P$\nolinebreak[3] --- кількість коробок, яку може взяти на~борт корабель.


\OutputFile Максимальна кількість вагонів потягу, які вдасться розвантажити.

\ifAfour\else
\vspace{-0.875\baselineskip plus 0.25 ex}
\fi

\Example
\begin{exampleSimple}{5em}{3em}%
\exmp{3\par
5 7 3\par
9}{2}%
\end{exampleSimple}

\end{problemAllDefault}
