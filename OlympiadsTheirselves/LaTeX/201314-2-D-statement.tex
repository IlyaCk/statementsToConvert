\begin{problem}{Всюдисущі ч\'{и}сла}{Клавіатура (stdin)}{Екран (stdout)}{3 сек}{64 мегабайти}\label{sec:omnipresent-numbers}

Дано прямокутну таблицю $N{\*}M$\nolinebreak[3] чисел. Гарантовано, що\nolinebreak[2] у\nolinebreak[3] кожному окремо взятому рядку всі ч\'{и}сла різні й монотонно зростають.

Напишіть програму, яка шукатиме перелік (також у\nolinebreak[3] порядку зростання) всіх тих чисел, які зустрічаються\linebreak[1] в\nolinebreak[3] усіх $N$ рядках.

\InputFile Вхідні дані слід прочитати зі стандартного входу (клавіатури). У~першому рядку задано два числ\'{а} $N$ та~$M$. Далі йдуть $N$ рядків, кожен з яких містить рівно~$M$ розділених пропусками чисел (гарантовано у~порядку зростання). 

\OutputFile	Результати виведіть на стандартний вихід (екран). Програма має вивести в один рядок через пробіли у порядку зростання всі ті числа, які зустрілися абсолютно в усіх рядках. Кількість чисел виводити не~треба. Після виведення всіх чисел потрібно зробити одне переведення рядка. Якщо нема жодного числа, що зустрілося в усіх рядках, виведення повинно не~містити жодного видимого символу, але містити переведення рядка.

{
\setlength\mytemplen{\parindent}

\def\lastScoringPhrase{Здавати потрібно одну програму, а~не~чотири; різні обмеження вказані, щоб пояснити, скільки балів можна отримати, розв’язавши задачу не~повністю.}

\noindent
\ifAfour
\begin{minipage}{0.3\textwidth}
\else
\begin{minipage}{0.35\textwidth}
\fi
\hspace*{\mytemplen}\Example\par\ifBigStretch\vspace{0.5\baselineskip}\fi\noindent
\begin{exampleSimple}{6em}{3em}%
\exmp{4 5
6 8 10 13 19
8 9 13 16 19
6 8 12 13 15
3 8 13 17 19}{8 13}%
\end{exampleSimple}
\end{minipage}
~
\ifAfour
\begin{minipage}{0.68\textwidth}
\else
\begin{minipage}{0.62\textwidth}
\fi
20\%\nolinebreak[3] балів припадатиме на тести, в~яких $3\dib{{\<}}{N,M}\dib{{\<}}20$, значення чисел від~0 до~100.
%
Ще\nolinebreak[3] 20\%\nolinebreak[3] --- на тести, в~яких $3\dib{{\<}}{N,M}\dib{{\<}}20$, значення чисел від $-10^9$ до~$+10^9$.
%
Ще\nolinebreak[3] 20\%\nolinebreak[3] --- на тести, в~яких $1000\dib{{\<}}{N,M}\dib{{\<}}1234$, значення від 0 до 12345.
%
Решта 40\%\nolinebreak[3] --- на тести, в~яких $1000\dib{{\<}}{N,M}\dib{{\<}}1234$, значення від $-10^9$ до~$+10^9$.
\ifAfour
\lastScoringPhrase
\fi
\end{minipage}

% % % \noindent

\ifAfour\else
\lastScoringPhrase
\fi

}

\end{problem}
