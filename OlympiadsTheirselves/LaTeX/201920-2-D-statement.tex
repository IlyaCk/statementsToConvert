\begin{problemAllDefault}{Прямокутні максимуми}

\myflfigaw{\begin{minipage}{13em}
\Example\\
\begin{exampleSimple}{5em}{5em}
\exmp{4 5\\
1 2 3 4 5\\
2 3 4 5 6\\
3 4 5 6 7\\
4 5 6 7 8\\
3\\
1 4 1 5\\
1 2 3 4\\
2 4 2 2}{8\\
5\\
5}%
\end{exampleSimple}
\end{minipage}}

\underline{\emph{Єдина}} відмінність 
% постановки (формулювання) від попередньої задачі\nolinebreak[3] --- 
% постановки задачі від попередньої\nolinebreak[3] --- 
% потрібно знаходити максимальне число прямокутника, а~не~суму його чисел. 
від попередньої задачі\nolinebreak[3] --- 
потрібно знаходити максимальне число прямокутника, а~не~суму. 

\InputFile
Формат повністю відповідає попередній задачі.

\OutputFile
Формат відрізняється від попередньої задачі \underline{\emph{лише}} тим, що для кожного з $T$ прямокутників треба вивести максимальне число, а~не~суму.

\Scoring
Аналогічно попередній задачі, але блоки % та розподіл балів між ними 
трохи інші:

% \vspace*{-0.5\baselineskip}

\begin{small}

\noindent\begin{tabular}{@{}c|c|p{0.35\textwidth}|c|c@{}}
блок
&
тести
&
\multicolumn{1}{c|}{обмеження}
&
перед\-умови
&
бали
\\\hline\hline%\endhead
1 & 
1--5 & 
${2\,{\<}\,N,M,T,a_{ij}\,{\<}\,9}$ & 
нема & 
20\% 
\\\hline
2 & 
6--10 & 
${1\,{\<}\,N,M,T\,{\<}\,100}$; ${1\,{\<}\,a_{ij}\,{\<}\,10^4}$ &
блок~1 & 
20\% \\\hline
3 &
11--19 & 
${100\,{\<}\,N,M\,{\<}\,400}$; ${123\,{\<}\,T,a_{ij}\,{\<}\,43210}$ &
блоки \mbox{1--2} & 
40\% \\\hline
4 &
20--25 & 
${400\,{\<}\,N,M\,{\<}\,500}$; ${54321\,{\<}\,T\,{\<}\,222555}$;\hfill~\linebreak ${1\,{\<}\,a_{ij}\,{\<}\,10^9}$ &
блоки \mbox{1--3} & 
20\% \\
\end{tabular}

\end{small}

% \vspace{-0.75\baselineskip}

\begin{footnotesize}

\Note
Для отримання значних балів за ці дві останні задачі потрібно вибрати й реалізувати не~просто правильні, а~ефективні правильні алгоритми. Крім того, якщо мова програмування має багато різних способів введення\&виведення, вибір швидкого способу читання вхідних даних і швидкого способу виведення результатів теж може вплинути на результат.
Див.~також \verb"https://ejudge.ckipo.edu.ua/io.pdf"

\end{footnotesize}

% \vspace{-1.5\baselineskip}

% \begin{small}

% \Note
% Само собою, для отримання значних балів за ці дві останні задачі потрібно вибрати не~просто правильні, а~ефективні правильні  алгоритми. Але ці задачі ще й мають вхідні дані дуже великих розмірів. Для багатьох сучасних мов програмування, що мають різні способи читання вхідних та виведення вихідних даних, ці способи можуть помітно відрізнятися за швидкістю, й вибір способів читання/\nolinebreak[2]виведення теж може істотно вплинути на швидкодію програми в цілому.
% %
% Не~забувайте також, що ви можете читати вхідні дані хоч зі стандартного входу (клавіатури), хоч з текстового файлу \texttt{input.txt} в поточній папці (але з чогось одного, не~поперемінно), та, аналогічно, виводити хоч на стандартний вихід (екран), хоч у текстовий файл \texttt{output.txt} поточної папки. Файлові введення/\nolinebreak[2]виведення часто (але не~завжди) швидші, ніж консольні.

% \end{small}

 
\end{problemAllDefault}
