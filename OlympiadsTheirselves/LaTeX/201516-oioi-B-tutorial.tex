\Tutorial	
\emph{Завдяки} тому, що в переліку об'ємів тари є \emph{всі} підряд об'єми 1, 2, 3, 4, 5 та 6~л, завжди можна використати \verb"N div 6" штук якнайбільших 6-літрових пляшок, а~решту (якщо така~є, тобто якщо \verb"N mod 6 > 0") розмістити у ще одну, останню, пляшку. Ще\nolinebreak[3] меншої кількості пляшок не~досить, бо сумарний об'єм меншої кількості пляшок точно строго менший потрібного.

Тобто, відповідь можна виразити як \texttt{res:=N~div~6;\linebreak[1] if\nolinebreak[3] N~mod~6>0 then inc(res)}, або навіть одним виразом $\lceil{}N/6\rceil$ (тобто $N/6$, заокруглене догори)\nolinebreak[3] --- це\nolinebreak[3] може бути записано як \verb"(N+5) div 6" (Pascal) чи \verb"ceil(N/6.0)" (C/C++).

Насамкінець, \emph{якби} у переліку об'ємів були не\nolinebreak[3] всі підряд числа від 1 до~6, а лише деякі, жадібний алгоритм <<взяти якнайбільшу кількість якнайбільших пляшок>> міг~би виявитися і неправильним (наприклад: якби існували лише пляшки 1~л, 5~л та 6~л, а пофасувати треба було 16~л, то оптимальний спосіб 
\mbox{6+5+5}
% ${6{+}5{+}5}$ 
отримувався~б \emph{усупереч} згаданій жадібній ідеї).
Тоді слід було~б використовувати деякий набагато складніший алгоритм; наприклад, оснований на псевдополіноміальному динамічному програмуванні (з~використанням масиву, індекси якого відповідають кількості олії, яку треба пофасувати, а\nolinebreak[3] значення\nolinebreak[3] --- відповідним мінімальним кількостям пляшок).