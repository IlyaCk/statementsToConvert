\Tutorial
Нехай у\nolinebreak[3] купці одна, дві або три палички. Тоді за один хід можна забрати їх усі й цим виграти. Від суперника тут нічого не~залежить.

Тепер нехай у\nolinebreak[3] купці чотири палички. Тоді 
%%% не~просто не~можна виграти за один хід, а\nolinebreak[2] ще\nolinebreak[2] й 
% будь-який допустимий правилами гри хід призводитиме до того, що супернику залишиться 
після будь-якого допустимого правилами гри ходу супернику залишиться 
або \mbox{4$\,$--$\,$1$\,$=$\,$3} палички,
або \mbox{4$\,$--$\,$2$\,$=$\,$2},
або \mbox{4$\,$--$\,$3$\,$=$\,$1}.
% % % або $4{-}1\dib{{=}}3$ палички,
% % % або $4{-}2\dib{{=}}2$,
% % % або $4{-}3\dib{{=}}1$.
Після чого, суперник зможе взяти усі палички одним ходом і тим виграти. Ми\nolinebreak[3] (\mbox{1-й}\nolinebreak[3] гравець) % ніяк 
не~можемо цьому перешкодити.

Таким чином, позиції 
<<1~паличка>>,
<<2~палички>> та
<<3~палички>> \emph{виграшні}:
починаючи з них, можна виграти, і суперник з цим нічого не~вдіє.
А\nolinebreak[3] позиція <<4~палички>> \emph{програшна}:
якщо суперник вміє грати, він точно виграє, і той, кому дісталося 4~палички, з цим нічого не~вдіє.

Якщо у~купці 5, 6 або~7 паличок, можна зробити такий хід, щоб після нього супернику дісталася купка з \mbox{4-х}\nolinebreak[2] паличок (що,\nolinebreak[3] як ми вже\nolinebreak[2] знаємо, означає, що він нічого не~зможе протиставити правильній грі свого суперника, тобто нашій правильній грі). Таким чином, позиції
<<5~паличок>>,
<<6~паличок>> та
<<7~паличок>> теж виграшні. 
Аналогічно показується, що позиція <<8~паличок>> програшна.
І~так далі: позиції, де кількість паличок кратна~4, програшні (починаючи з них, перемогти грамотного суперника неможливо), а\nolinebreak[3] позиції, де кількість паличок не~кратна~4, виграшні (треба забирати \texttt{n~mod~4} паличок, і цим або виграємо негайно (при\nolinebreak[2] ${1\,{\<}\,n\,{\<}\,3}$), або супернику дістанеться програшна позиція).

Аналіз % цієї 
гри часто цим і закінчують, а питання <<що\nolinebreak[2] робити, коли нам дісталася програшна позиція?>> або % взагалі 
ігнорують, або відповідають на нього абияк. Зокрема, часто 
% роблять висновок 
заявляють
<<раз\nolinebreak[2] при\nolinebreak[2] ході з\nolinebreak[2] програшної гарантовано виграти % все\nolinebreak[2] одно 
неможливо, будемо відтягувати кінець, беручи щоразу лише по одній паличці: чим довше триватиме гра, тим більше шансів, що неідеальний суперник 
% проявить свою неідеальність%
помилиться%
>>.

Але\nolinebreak[2] в умові не~дарма сказано, що у\nolinebreak[3] приблизно половині тестів треба грати проти безграмотних програм-суперниць, \emph{причому в~різних тестах різних}. Серед цих безграмотних програм-суперниць є така, яка завжди намагається забирати якнайбільше паличок, тобто або~3, або, якщо їх всього лишилося менш, ніж~3, то~всі. Правда, у\nolinebreak[2] \emph{такої} суперниці можна виграти, маючи на~початку програшну позицію? Але для цього треба, перебуваючи у програшній позиції, взяти не~одну паличку, 
% а дві або~три\dots{} 
а~2\nolinebreak[3] або~3\dots{} 
(Це~не~допомагає, якщо початкова кількість паличок$\,$=$\,$4: програма-суперниця <<не~встигає збитися>>; але при значно більших початкових кількостях паличок <<перехоплення ініціативи>> цілком реальне.)

Так\nolinebreak[2] само, як не~можна у\nolinebreak[3] програшних позиціях завжди виводити~1, не~можна й виводити завжди~2 чи завжди~3, бо серед інших програм-суперниць є та, що завжди намагається забирати дві палички (й~тільки якщо лишилася всього одна, то забере її одну), а~також та, що абсолютно завжди забирає одну паличку. Тому треба або робити хід із програшної позиції якимсь хитро залежним від того, з якої с\'{а}ме програшної позиції ходимо\dots

\phantomsection\label{text:random-in-simple-sticks-game}\dots{}або застосувати всемогутні випадкові ч\'{и}сла: з\nolinebreak[2] виграшних позицій ходити, згідно з основним аналізом, ходом \texttt{n~mod~4}, а\nolinebreak[3] з\nolinebreak[3] програшних\nolinebreak[3] --- суто випадково, від~1 до~3. Це,\nolinebreak[3] звісно, не~допоможе проти грамотного суперника, але ситуації, описані у двох попередніх абзацах, стануть неможливими, й у\nolinebreak[2] випадках, де\nolinebreak[3] це\nolinebreak[3] можливо, ініціатива перехоплюватиметься.
%
(Ще~раз: тут \emph{не}~пропонується \emph{завжди} ходити випадково; така програма майже завжди програвала~б <<ідеальному>> супернику, котрий не~помиляється. Пропонується у~першу чергу дивитися на виграш\-ність/\linebreak[2]про\-граш\-ність, а\nolinebreak[3] вже\nolinebreak[2] у\nolinebreak[3] випадку програшності, коли гарантій все\nolinebreak[3] одно нема, ходити випадково.)

У\nolinebreak[3] розв'язках цієї задачі якось трапилася кумедна помилка. Одна особа, бажаючи зробити % псевдовипадкові 
ч\'{и}сла якнайвипадковішими, писала перед кожним викликом \texttt{(rand()\%3)+1} (згенерувати чергове псевдовипадкове число від~1 до~3) новий виклик \texttt{srand(time(NULL))} (переналаштувати генератор псевдовипадкових чисел). Така реалізація (мовою~C++) програ\'{є} згаданим суперникам вигляду <<намагатися завжди бр\'{а}ти~\dots>>, причому при різних запусках\nolinebreak[3] --- різним суперникам (на\nolinebreak[2] різних тестах): то супернику, що~намагається завжди бр\'{а}ти~3, то тому, що намагається завжди бр\'{а}ти~2, то тому, що завжди бере~1. Причина в~тому, що % аргумент \texttt{srand}-а 
\texttt{time(NULL)} залежить від поточного системного часу, але не~наносекунд, які щоразу різні, а~цілих секунд, які, як~правило, однакові протягом усього виконання програми. Так і виходило, що генератор, щоразу однаково переініціалізований, видавав одне й те~ж.\phantomsection\label{text:sticks-simple-end}

% Насамкінець, 
Розглянута в цій задачі гра % взагалі-то 
має % власну 
назву: \emph{гра\nolinebreak[3] Баш\'{е}}. Втім, 
% це той випадок, коли задачу довго й 
цю задачу
потроху розв'язували різні люди, внесок К.$\,$Г.~Баше 
% не~можна вважати 
не~є
вирішальним, і назва <<гра\nolinebreak[3] Баше>> поширена, але не~%зовсім 
загальноприйнята.