\begin{problemAllDefault}{Паркет--1}

{\tolerance=9999

Щоб зобразити за допомогою паркету Супер-Креативний Візерунок, треба\linebreak[1]
$N_1$\nolinebreak[3] дощечок розмірами\nolinebreak[2] $1{\*}1$,\linebreak[1]
$N_2$\nolinebreak[3] дощечок розмірами\nolinebreak[2] $2{\*}1$,\linebreak[1]
$N_3$\nolinebreak[3] розмірами\nolinebreak[1] $3{\*}1$,\linebreak[1]
$N_4$\nolinebreak[4] розмірами\nolinebreak[1] $4{\*}1$ та 
$N_5$\nolinebreak[3] дощечок розмірами\nolinebreak[3] $5{\*}1$. 
Купити можна лише дощечки розмірами\nolinebreak[2] $5{\*}1$. Дощечки можна різати, але не~можна склеювати. Наприклад, коли потрібні п’ять дощечок $2{\*}1$, їх не~можна зробити з двох дощечок $5{\*}1$, але можна з трьох. Для цього дві з них розріжемо на три частини $2{\*}1$, $2{\*}1$ та $1{\*}1$ кожну, а третю\nolinebreak[3] --- на дві частини $2{\*}1$ та $3{\*}1$. Отримаємо потрібні п’ять дощечок $2{\*}1$, а дві дощечки $1{\*}1$ та одна $3{\*}1$ підуть у відходи.

}

Напишіть програму, яка, прочитавши кількості дощечок $N_1$, $N_2$, $N_3$, $N_4$ та\nolinebreak[3] $N_5$, знайде, яку мінімальну кількість дощечок $5{\*}1$ необхідно купити.

\InputFile	Вхідні дані слід прочитати зі стандартного входу (клавіатури). Це будуть п’ять чисел $N_1$, $N_2$, $N_3$, $N_4$ та $N_5$ (саме в такому порядку), розділені пропусками (пробілами).

\OutputFile	Єдине число (скільки дощечок треба купити) виведіть на стандартний вихід (екран).

\ifAfour\vspace{-0.5\baselineskip}\fi

\Examples
\begin{exampleSimple}{5em}{3em}%
\exmp{0 5 0 0 0}{3}%
\end{exampleSimple}
\begin{exampleSimple}{5em}{3em}%
\exmp{1 1 1 1 1}{3}%
\end{exampleSimple}

\Scoring	Усі кількості невід’ємні; 90~балів (з~250) припадатиме на\nolinebreak[3] тести, в~яких сумарна кількість $N_1\dib{{+}}N_2\dib{{+}}N_3\dib{{+}}N_4\dib{{+}}N_5$ перебуває в межах від 0 до~20, ще\nolinebreak[3] 80~балів\nolinebreak[3] --- від 100 до \mbox{10\hspace{0.125em}000}, решта 80~балів\nolinebreak[3] --- від \mbox{500\hspace{0.125em}000\hspace{0.125em}000} до \mbox{2\hspace{0.125em}000\hspace{0.125em}000\hspace{0.125em}000}. 

Здати потрібно одну програму, а не для кожного випадку окремо; різні обмеження вводяться виключно для того, щоб дати приблизне уявлення, скільки балів можна отримати, розв’язавши задачу не повністю.



\end{problemAllDefault}