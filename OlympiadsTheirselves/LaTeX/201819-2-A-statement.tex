{

\PrintEjudgeConstraintsfalse

\begin{problemAllDefault}{Кількість шляхів}

\def\waysQuantityFirstPhrase{На рисунку зображено, через якi промiжнi пункти можливо дiстатися з пункту~$A$ до пункту~$D$.
Пiдписи ребер означають <<мiж цими % населеними 
пунктами є стiльки-то рiзних рейсiв>>.
Скiльки всього є способiв дiстатися 
% з пункту~$A$ до пункту~$D$? 
з~$A$ до~$D$?}
\def\waysQuantityPicture{\begin{mfpic}[12]{0}{10}{0}{4}
\pointdef{A}(0,0)
\pointdef{B}(4,4)
\pointdef{C}(6,1)
\pointdef{D}(10,0)
\pointdef{AB}(0.5*\Ax+0.5*\Bx , 0.5*\Ay+0.5*\By)
\pointdef{AC}(0.5*\Ax+0.5*\Cx , 0.5*\Ay+0.5*\Cy)
% \pointdef{AD}(0.5*\Ax+0.5*\Dx , 0.5*\Ay+0.5*\Dy)
\pointdef{BC}(0.5*\Bx+0.5*\Cx , 0.5*\By+0.5*\Cy)
% \pointdef{BD}(0.5*\Bx+0.5*\Dx , 0.5*\By+0.5*\Dy)
\pointdef{CD}(0.5*\Cx+0.5*\Dx , 0.5*\Cy+0.5*\Dy)
\tlabel[tr](\Ax,\Ay){$A$}
\tlabel[br](\Bx,\By){$B$}
\tlabel[bl](\Cx,\Cy){$C$}
\tlabel[tl](\Dx,\Dy){$D$}
\lines{\A,\B}
\lines{\A,\C}
% \lines{\A,\D}
\lines{\B,\C}
% \lines{\B,\D}
\lines{\C,\D}
\arrow[l5]\lines{\A,\AB}
\arrow[l5]\lines{\A,\AC}
% \arrow[l5]\lines{\A,\AD}
\arrow[l5]\lines{\B,\BC}
% \arrow[l5]\lines{\B,\BD}
\arrow[l5]\lines{\C,\CD}
\tlabel[br](\ABx,\ABy){$k1$}
\tlabel[bl](\BCx,\BCy){$k2$}
\tlabel[tl](\ACx,\ACy){$k3$}
\tlabel[bl](\CDx,\CDy){$k4$}
\point{\A}
\point{\B}
\point{\C}
\point{\D}
\end{mfpic}}

\ifAfour
\myflfigaw{\waysQuantityPicture}
\waysQuantityFirstPhrase
\else
\noindent
\begin{tabular}{@{}p{0.7\textwidth}c@{}}
\waysQuantityFirstPhrase
&
\raisebox{-48pt}{\waysQuantityPicture}
\end{tabular}
\fi

Рахувати треба всi способи, не~намагаючись вибирати найкоротшi чи ще якiсь; 
переходи дозволенi лише у вiдповiдностi з наведеними напрямками ребер;
вiдповiдь виразити як формулу вiд~$k1$,~$k2$,~$k3$,~$k4$.

Напишіть \emph{вираз}, котрий знаходитиме цю кількість способів. 

Змінні, від яких залежить вираз, обов'язково повинні мати с\'{а}ме описаний смисл, 
і називатися вони повинні с\'{а}ме \texttt{k1}, \texttt{k2}, \texttt{k3}, \texttt{k4} 
(``\texttt{k}''~--- маленькі латинські).

У~цій задачі треба здати не~програму, а~вираз: 
вписати його (сам вираз, не~назву файлу) у~відповідне поле відповідної сторінки ejudge і відправити на~перевірку. Правила запису виразу:
можна використовувати цілі десяткові ч\'{и}сла, арифметичні дії ``\verb"+"''~(плюс), 
``\verb"-"''~(мінус), ``\verb"*"''~(множення), ``\verb"/"''~(ділення дробове, наприклад, \verb"17/5"${=}3{,}4$), ``\verb"//"''~(ділення цілочисельне, наприклад, \verb"17//5"${=}3$), круглі дужки ``\verb"("''\nolinebreak[2] та~``\verb")"'' 
для групування та зміни порядку дій.
Дозволяються пропуски (пробіли), але\nolinebreak[3] не~всер\'{е}дині чисел.
Множення треба писати явною зірочкою.

Наприклад: можна здати вираз 
\verb"k1+k2+k3*k4"
і отримати 20~балів з~200, бо він неправильний,
але все~ж іноді відповідь випадково збігається з~правильною. 
А~<<цілком аналогічні>> вирази 
\verb"a+b+c*d"
та
\verb"k1+k2+k3k4"
будуть оцінені на~0~балів кожен, 
бо~змінні повинні називатися так, як вказано,
а~множення треба писати зірочкою, не~пропускаючи.

Вкажемо також, що при 
\texttt{k1}$\dib{{=}}$\texttt{k2}$\dib{{=}}$\texttt{k3}$\dib{{=}}$2, 
\texttt{k4}$\dib{{=}}$1
кількість способів дорівнює~6,
і цими шістьма способами~є:
\begin{itemize}
\item
\mbox{1-й} зі способів $A{\to}C$, єдиний спосіб $C{\to}D$;
\item
\mbox{2-й} зі способів $A{\to}C$, єдиний спосіб $C{\to}D$;
\item
\mbox{1-й} зі способів $A{\to}B$, \mbox{1-й} зі способів $B{\to}C$, єдиний спосіб $C{\to}D$;
\item
\mbox{1-й} зі способів $A{\to}B$, \mbox{2-й} зі способів $B{\to}C$, єдиний спосіб $C{\to}D$;
\item                                                        
\mbox{2-й} зі способів $A{\to}B$, \mbox{1-й} зі способів $B{\to}C$, єдиний спосіб $C{\to}D$;
\item                                                                 
\mbox{2-й} зі способів $A{\to}B$, \mbox{2-й} зі способів $B{\to}C$, єдиний спосіб $C{\to}D$.

\end{itemize}

\ifStatementOnly
\vfill\par
Всі задачі цього змагання, включно з цією задачею, заборонено повторно здавати в ejudge після того, як вже здано повнобальний розв'язок.
\fi

\end{problemAllDefault}

}