{

\begin{problemAllDefault}{Preview}

Preview (попередній перегляд) графічного зображення\nolinebreak[3] --- це спеціально зменшена його версія, призначена, щоб могти показати його загальний вигляд швидко й не~займаючи багато пам'яті. Таке зменшення розмірів зазвичай призводить до втрати якості. В~цій задачі розглянемо аж надто простий (та досить поганий за якістю) спосіб створення preview-зображення.

Нехай початкове зображення має форму квадрата $2^k{\times}2^k$ пікселів, де ${2\,{\<}\,k\,{\<}\,9}$. Інакше кажучи, ст\'{о}рони цього квадрата рівні 
або~4, 
або~8,
або~16,
або~32,
або~64,
або~128,
або~256,
або~512 піксел(і/ів).
Будемо будувати preview-зображення розмірами 
$2^{k-1}{\times}2^{k-1}$,
$2^{k-2}{\times}2^{k-2}$,~\dots,
$2^{1}{\times}2^{1}$ (інакше кажучи, зі сторонами квадрата удвічі меншими, вчетверо меншими, і~т.~д., до розміру $2{\times}2$ включно). Таким чином, кожен піксель утвореного preview-зображення відповідає квадрату 
$2{\times}2$,
чи\nolinebreak[3] $4{\times}4$,~\dots,
чи\nolinebreak[3] $2^{k-1}{\times}2^{k-1}$
початкового зображення.
От\nolinebreak[3] і\nolinebreak[3] робитимемо колір відповідного пікселя preview-зображення рівним середньому арифметичному кольорів пікселів відповідного квадрата.

Як розуміти <<середнє арифметичне кольорів>>? 
% У~цій задачі вимагається 
Слід
розглянути дві версії:
\begin{enumerate}
\item
Зображення подане в сірих кольорах різної інтенсивності, й колір можна задати цілим числом від~0 до~9. Тоді середнє арифметичне кольорів зводиться просто до середнього арифметичного відповідних чисел.
\item
Зображення подане у форматі RGB, тобто для кожного пікселя окремо вказано яскравіть 
\mbox{R-}\nolinebreak[3]каналу (червоного), 
окремо \mbox{G-}\nolinebreak[3]каналу (зеленого), 
окремо \mbox{B-}\nolinebreak[3]каналу (синього).
Кожна яскравіть подається рівно двома шістнадцятковими цифрами, тобто в межах від\nolinebreak[2] 00\nolinebreak[2] до~FF (де\nolinebreak[2] FF$_{Hex}\,{=}\,255_{Dec}$; обидві меж\'{і} включно). Треба порахувати середнє за кожним каналом окремо, й зібрати три результати в колір пікселю. Значення трьох 
% каналів
яскравостей 
кожного пікселю записані у~вхідних даних та повинні бути записані у~знайденій відповіді підряд (без\nolinebreak[3] пропусків) і з символом~``\verb"#"'' (решітка, ASCII-код~35) спереду.
\end{enumerate}

\noindent
В~обох випадках, вимагається рахувати середнє арифметичне з округленням до найближчого цілого (якщо відстані до цілих однакові, то до більшого з~них). Наприклад:
\mbox{ave(1,$\,$1,$\,$1,$\,$2)$\,$=$\,$1};\hspace{0.25em plus 0.125em}
\mbox{ave(2,$\,$2,$\,$1,$\,$2)$\,$=$\,$2};\hspace{0.25em plus 0.125em}
\mbox{ave(1,$\,$2,$\,$1,$\,$2)$\,$=$\,$2}.

% % % \Examples

% % % \noindent\hspace*{-1em}
% % % \begin{exampleSimple}{15.5em}{7.5em}
% % % \exmp{rgb
% % % 4
% % % \#a4e794 \#b0c1fa \#c4d243 \#3acf90
% % % \#b09726 \#6efde5 \#28c477 \#3d2299
% % % \#55b48f \#b76573 \#6883a1 \#c2bf34
% % % \#d3951d \#601601 \#2dcbaa \#a3fe75}{\#9dcfa6 \#59a279
% % % \#907148 \#7fc37d}%
% % % \end{exampleSimple}
% % % \hspace*{-1.25em}
% % % \begin{exampleSimpleExtraNarrow}{7.5em}{3.5em}
% % % \exmp{gray
% % % 8
% % % 1	2	3	2	1	2	5	1
% % % 2	1	2	3	0	3	9	1
% % % 3	2	1	2	2	9	7	5
% % % 2	3	2	1	8	7	7	8
% % % 1	5	3	0	3	5	9	4
% % % 9	6	9	5	4	6	6	5
% % % 2	4	4	3	8	4	5	3
% % % 1	8	4	5	9	2	8	3}{2		3		2		4
% % % 3		2		7		7
% % % 5		4		5		6
% % % 4		4		6		5
% % % ~
% % % 2				5
% % % 4				5}%
% % % \end{exampleSimpleExtraNarrow}
% % % \hspace*{-1.25em}


\myflfigaw{\hspace*{-1em}\begin{tabular}{@{}c@{}}
\Examples
\\
\begin{exampleSimple}{15.5em}{7.5em}
\exmp{gray\\
8\\
1	2	3	2	1	2	5	1\\
2	1	2	3	0	3	9	1\\
3	2	1	2	2	9	7	5\\
2	3	2	1	8	7	7	8\\
1	5	3	0	3	5	9	5\\
9	6	9	5	4	6	6	5\\
2	4	4	3	8	4	5	4\\
1	8	4	5	9	4	8	4}{2		3		2		4\\
3		2		7		7\\
5		4		5		6\\
4		4		6		5\\
~\\
2				5\\
4				6}%
\exmp{rgb\\
4\\
\#a4e794 \#b0c1fa \#c4d243 \#3acf90\\
\#b09726 \#6efde5 \#28c477 \#3d2299\\
\#55b48f \#b76573 \#6883a1 \#c2bf34\\
\#d3951d \#601601 \#2dcbaa \#a3fe75}{\#9dcfa6 \#59a279\\
\#907148 \#7fc37d}%
\end{exampleSimple}\end{tabular}\hspace*{-0.5em}}

{\hyphenpenalty=0

\InputFile
Перший рядок містить одне з двох слів ``\texttt{gray}'' чи ``\texttt{rgb}'' (без\nolinebreak[3] лапок, маленькими латинськими буквами) на позначення того, чи в цьому тесті кольори сірі (як~у~пункті~1), чи RGB (як~у~пункті~2).
Другий рядок містить розмір~$N$ (сторону квадрата), яка гарантовано є однією з вищеперелічених.
Далі слідують $N$ рядків, кожен з яких містить кольори $N$ пікселів (якщо перший рядок тесту був\nolinebreak[2] ``\texttt{gray}'',\linebreak[2] то\nolinebreak[2] одно\-цифро\-вих чисел від~0 до~9, а\nolinebreak[3] якщо\nolinebreak[3] ``\texttt{rgb}'',\linebreak[2] то\nolinebreak[2] описаних у пункті~2 послідовностей з ``\verb"#"'' та шести \mbox{16-вих} цифр; для цифр зі значеннями від~10 до~15 використані \emph{маленькі} букви \mbox{\texttt{a}--\texttt{f}}). У~кожному з випадків, кольори різних пікселів розділяються одинарними пробілами (ASCII-код~32).

\OutputFile
От\-ри\-мані pre\-view-зо\-бра\-же\-ння, аналогічно формату вхідних даних, але без перших двох рядків (зразу кольори пікселів). Спочатку $N/2$ рядків, що містять preview розміру ${(N/2)}\dibbb{{\times}}{(N/2)}$; якщо ${N/4\,{\>}\,2}$, то один порожній рядок і $N/4$ рядків, що містять preview розміру ${(N/4)}\dibbb{{\times}}{(N/4)}$; і\nolinebreak[3] так\nolinebreak[3] далі (до\nolinebreak[3] preview розміру $2{\times}2$ включно, розділяючи різні preview одинарними\nolinebreak[2] по\-рож\-німи рядками).

\Notes 
Зверніть увагу на перше число передостаннього рядка відповіді першого тесту та на передостанній символ відповіді др\'{у}гого тесту.

\Scoring
У~цій задачі оцінювання поблокове, тобто тести розбиті на блоки, і бали за блок нараховуються лише в разі успішного проходження \emph{всіх} тестів блоку. Блоки 1--2 складаються кожен з єдиного тесту з умови, й результати їх перевірки показуються, але ні\'{я}к не~впливають на бали. Далі є ще~10~блоків:

}

\def\block#1#2#3#4{\begin{small}\begin{tabular}{@{}c}блок #1\\\begin{scriptsize}{тести~}\end{scriptsize}#2\\\begin{large}#3\%\end{large}\scriptsize{~балів}\end{tabular}\end{small}\begin{footnotesize}\begin{minipage}{0.125\textwidth}#4\end{minipage}\end{footnotesize}}

\noindent
\begin{tabular}{@{}c||c|c|c@{}}
 &
\multicolumn{3}{c}{Перший рядок (\texttt{gray}/\texttt{rgb})} 
 \\
 & лише~\texttt{gray} & лише~\texttt{rgb} & будь-який з двох\\
\hline\hline
$N=4$
&
\block{3}{3--6}{5}{перевіряється завжди}
& 
\block{4}{7--10}{5}{перевіряється завжди}
& 
\block{5}{11--14}{5}{перевіряється лише в разі зарахування блоків 3--4}
\\
\hline
$N=8$
&
\block{6}{15--18}{5}{перевіряється завжди}
& 
\block{7}{19--22}{5}{перевіряється завжди}
& 
\block{8}{23--26}{5}{перевіряється лише в разі зарахування блоків 6--7}
% 8\% 
\\
\hline
\begin{tabular}{@{}c@{}}
$N$ або~4,\\
або~8, або~16
\end{tabular}
&
\block{9}{27--30}{10}{перевіряється лише в разі зарахування блоків 3,~6}
& 
\block{10}{31--34}{10}{перевіряється лише в разі зарахування блоків 4,~7}
& 
такого блоку нема
\\
\hline
\begin{tabular}{@{}c@{}}
$N$ \small{будь-яке} \\
\small{з вищезгаданих}
\end{tabular}
&
\block{11}{35--45}{20}{перевіряється лише в разі зарахування блоків 3,~6,~9}
& 
такого блоку нема
& 
\block{12}{46--60}{30}{перевіряється лише в разі зарахування блоків 3--11}
\\
\end{tabular}

\end{problemAllDefault}

}