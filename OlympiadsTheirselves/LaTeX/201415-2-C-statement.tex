\begin{problemAllDefault}{Ко-анаграмічно-прості}

Число називається \emph{простим}, якщо воно має рівно два різні дільники\nolinebreak[3] --- себе і одиницю. Наприклад: 23\nolinebreak[3] --- просте число, а\nolinebreak[3] 35 не~просте, бо $5{\*}7\dib{{=}}35$. Число~1 теж не~просте (лише один дільник).

Якщо у числі змінити порядок цифр, властивість простоти може змінитися: наприклад, 35 --- не~просте число, а\nolinebreak[3] 53\nolinebreak[3] --- просте.

Будемо називати число \emph{ко-анаграмічно-простим}, коли при хоча\nolinebreak[3] б одному можливому порядку цифр утворюється просте число. Наприклад, 35 є ко-анаграмічно-простим (бо\nolinebreak[3] 53\nolinebreak[3] просте), а\nolinebreak[2] 225\nolinebreak[3] --- не~є, бо жодне з чисел 225, 252 та 522 не~є~простим.

Напишіть програму, яка читає одне ціле додатне значення $n$ ($10\dib{{\<}}n\dib{{\<}}9999$) і виводить у першому рядку мінімальне можливе ко-анаграмічно-просте число, більше-рівне за~$n$, а~у~др\'{у}гому рядку\nolinebreak[3] --- ту перестановку цифр, яка робить його простим. Якщо при перестановці з’являються нулі спереду числа --- слід вважати, що вони не~впливають на значення числа (05 дорівнює~5), але виводити слід обов’язково із цими нулями (якщо правильно~05, то вивести 5 неправильно). Разом з тим, перше число відповіді починати з нуля не~можна.

Якщо можливі різні правильні відповіді --- виводьте будь-яку одну з них.

% \begin{multicols}{2}
\Examples

\noindent
\hspace*{-0.75em}
\begin{exampleSimple}{4em}{4em}%
\exmp{35}{35
53}%
\end{exampleSimple}
\hspace*{-2em}
\begin{exampleSimple}{4em}{4em}%
\exmp{49}{50
05}%
\end{exampleSimple}
\hspace*{-2em}
\begin{exampleSimple}{4em}{4em}%
\exmp{225}{227
227}%
\end{exampleSimple}
\hspace*{-1.5em}
% \end{multicols}

\end{problemAllDefault}
