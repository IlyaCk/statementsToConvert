\begin{problemAllDefault}{Ложбан}

\emph{Ложбан} (англ.\nolinebreak[3] \emph{Lojban})\nolinebreak[3] --- це штучна мова, яка була створена у\nolinebreak[3] 1987~році Групою Логічних Мов та базується на логлані (логічна мова). При створенні основною ціллю була повніша, вільно доступна, зручна для використання мова. Вона є експериментальною і сконструйована для перевірки гіпотези Сепіра-Ворфа. Ця\nolinebreak[2] гіпотеза припускає, що люди, які говорять різними мовами, по-різному сприймають світ і по-різному мислять.

Дана мова є однією з найпростіших штучних мов. Наприклад цифри від\nolinebreak[2] 0\nolinebreak[1] до\nolinebreak[3] 9 записуються наступним чином:

\vspace{\ifAfour-1.5\baselineskip\else-0.5\baselineskip\fi plus 6pt minus 6pt}

\begin{center}
\begin{ttfamily}
\begin{tabular}{cc@{ ~ ~ ~ }cc@{ ~ ~ ~ }cc@{ ~ ~ ~ }cc}
1 & pa	&	4 & vo	&	7 & ze				\\
2 & re	&	5 & mu	&	8 & bi	&	0 & no	\\
3 & ci	&	6 & xa	&	9 & so				
\end{tabular}
\end{ttfamily}

\end{center}

\vspace{\ifAfour-0.75\baselineskip\else-0.5\baselineskip\fi plus 6pt minus 6pt}

Великі числа утворюються склеюванням цифр разом. Наприклад, число~123 це \texttt{pareci}.

\Task	\hspace{0pt minus 0.5em}Напишіть програму, яка зчитує рядок на\nolinebreak[2] Ложбані (що\nolinebreak[3] представляє собою \mbox{число${}\<{}$1\hspace{0.125em}000\hspace{0.125em}000}) та виводить цей рядок у цифровому вигляді.

\Example
\begin{exampleSimple}{5em}{3em}%
\exmp{renopavo}{2014}%
\end{exampleSimple}

\end{problemAllDefault}
