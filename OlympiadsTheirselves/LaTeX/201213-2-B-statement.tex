\begin{problem}{Матриця}{matrix.in}{matrix.out}{1 сек}{64 Мб}

Вам дана таблиця розміром $N$ рядків на $M$ стовпців. У\nolinebreak[3] кожній клітинці таблиці записано
число\nolinebreak[2] 0\nolinebreak[2] або~1. За\nolinebreak[2] один хід можна вибрати один із рядків таблиці і циклічно зсунути
значення в ній на одну клітинку або вліво, або вправо.

Циклічно зсунути рядок таблиці на одну клітинку вправо означає перемістити значення
кожної комірки цього рядка, крім останньої, в\nolinebreak[2] сусідню комірку праворуч, а\nolinebreak[2] значення
останньої комірки перемістити в першу комірку. Аналогічним чином, але у зворотний бік
виконується циклічний зсув рядка таблиці вліво. Наприклад, якщо циклічно зсунути рядок
<<00101>> на одну клітинку вправо\nolinebreak[3] --- вийде рядок <<10010>>, якщо ж зрушити рядок <<00101>> на
одну комірку вліво\nolinebreak[3] --- вийде рядок <<01010>>.

\Task 
Напишіть програму \texttt{matrix}, яка читає таблицю чисел (матрицю) та визначає
найменшу кількість ходів при яких в матриці утвориться стовпчик, що містить лише
одиниці.

\InputFile
Перший рядок файлу \texttt{matrix.іn} містить два цілих числа, розділених
пробілом: $N$ ($1\dib{{\<}}N\dib{{\<}}100$)\nolinebreak[3] --- кількість рядків в таблиці і $M$ ($1\dib{{\<}}M\dib{{\<}}100$)\nolinebreak[3] --- кількість стовпців в
таблиці. Далі слідують $N$ рядків, кожна з яких містить по $M$ символів <<0>> або\nolinebreak[3] <<1>>: \mbox{$j$-ий}\nolinebreak[3]
символ \mbox{$i$-ого} рядка описує вміст комірки в \mbox{$i$-тому} рядку і \mbox{$j$-ому} стовпці таблиці.
Гарантується, що в описі таблиці не\nolinebreak[3] зустрічається ніяких символів крім <<0>>~і~<<1>>.

\OutputFile
Ваша програма повинна створити текстовий файл \texttt{matrix.out} і вивести туди
єдине число: найменшу кількість ходів, за які можна в якому-небудь із стовпців таблиці
отримати лише одиниці. Якщо цього зробити неможливо, виведіть число~$-1$.

\Examples
\begin{exampleWidthsAndDefaultFileNames}{2em}{2em}%
\exmp{3 6
101010
000100
100000}{3}%
\end{exampleWidthsAndDefaultFileNames}
\begin{exampleWidthsAndDefaultFileNames}{2em}{2em}%
\exmp{2 3
111
000}{-1}%
\end{exampleWidthsAndDefaultFileNames}

\end{problem}
\pagebreak[2]