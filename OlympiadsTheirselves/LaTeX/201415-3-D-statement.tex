\begin{problem}{З\'{а}мок --- castle}{\stdinOrInputTxt}{\stdoutOrOutputTxt}{D1 --- 1 с, D2 --- 3 с}{128 мегабайтів}


Стародавній з\'{а}мок має прямокутну форму. Замок містить щонайменше дві кімнати. Підлогу замка можна умовно поділити на $M{\*}N$ клітин. Кожна така клітинка містить <<0>> або <<1>>, які задають порожні ділянки та стіни замку відповідно.

\looseness=-2
\Task Напишіть програму \texttt{castle}, яка б знаходила кількість кімнат у зам\-ку, площу найбільшої кімнати (яка вимірюється кількістю клітинок) та площу найбільшої кімнати, яку можна утворити шляхом видалення стіни або її частини, тобто, замінивши лише одну <<1>> на <<0>>. Видаляти зовнішні стіни заборонено.

\InputFile План замку задається у вигляді послідовності чисел, по одному числу, яке характеризує кожну клітинку. Перший рядок містить два цілих числа $M$ та\nolinebreak[2] $N$\nolinebreak[3] --- кількість рядків та кількість стовпчиків (${3\,{\<}\,M\,{\<}\,1000}$, ${3\,{\<}\,N\,{\<}\,1000}$). $M$~наступних рядків містить по\nolinebreak[3] $N$\nolinebreak[3] нулів або одиниць, що\nolinebreak[3] йдуть поспіль (без\nolinebreak[2] пробілів). Перший та останній рядок, а\nolinebreak[3] також перший та останній стовпчик формують зовнішні стіни замку і складаються лише з одиниць.

\OutputFile Дана задача розділена в системі ejudge на дві підзадачі. У~підзадачі\nolinebreak[3] \texttt{D1} треба здати програму, що знаходить кількість кімнат та площу найбільшої кімнати замку (по\nolinebreak[3] одному числу в\nolinebreak[3] рядку), у~підзадачі\nolinebreak[2] \texttt{D2}\nolinebreak[3] --- площу найбільшої кімнати, яка утвориться в\nolinebreak[3] разі видалення внутрішньої стіни.\pagebreak[1]

% % % \vspace*{5\baselineskip}

% % % ~

% % % \vspace{-6\baselineskip}

\ifAfour\else\noindent\fi
\Examples

\ifAfour
\vspace{-2.25\baselineskip}
\else
\vspace{-1.25\baselineskip}
\fi

{\makeTableLongtrue % TODO: перевірити верстку
\begin{exampleSimpleThreeWithSpecNameColTwo}{7em}{3em}{3em}{Результати (D1)}{Результати (D2)}%
\exmp{6 8
11111111
10011001
10011001
11111001
10101001
11111111}{4
8}{10}%
\exmp{9 12
111111111111
101001000001
111001011111
100101000001
100011111101
100001000101
111111010101
100000010001
111111111111}{4
28}{38}%
\end{exampleSimpleThreeWithSpecNameColTwo}}

\Scoring Значна частина тестів буде містити план замку з кімнатами лише прямокутної форми. Не~менше половини тестів такі, що ${3\,{\<}\,M\,{\<}\,20}$, ${3\,{\<}\,N\,{\<}\,50}$.

\end{problem}

%%%\pagebreak % TODO: check!!!