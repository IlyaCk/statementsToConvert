{

\PrintEjudgeConstraintsfalse

\begin{problemAllDefault}{Змії}

Із тераріуму втекли \texttt{x}~гадюк, \texttt{y}~кобр та \texttt{z}~гюрз. 
Довжина кожної гадюки~--- 1$\,$м,
кожної кобри~--- 1$\,$м~30$\,$см,
а~гюрзи~--- 1$\,$м~15$\,$см.

Напишіть \emph{вираз}, який знаходитиме, скільки повних метрів змій втекло з тераріуму. 
У~цій задачі треба здати не~програму, а~вираз: 
вписати його (сам вираз, не~назву файлу) у~відповідне поле перевіряючої системи
і відправити на~перевірку. Правила запису виразу:
можна використовувати цілі десяткові ч\'{и}сла, арифметичні дії ``\verb"+"''~(плюс), 
``\verb"-"''~(мінус), ``\verb"*"''~(множення), ``\verb"/"''~(ділення дробове, наприклад, \verb"18/5"$\,{=}\,3{,}6$), ``\verb"//"''~(ділення цілочисельне, наприклад, \verb"18//5"$\,{=}\,3$), круглі дужки ``\verb"("''\nolinebreak[2] та~``\verb")"'' 
для групування та зміни порядку дій.
Дозволяються пропуски (пробіли), але\nolinebreak[3] не~всер\'{е}дині чисел.

Змінні, від яких залежить вираз, обов'язково повинні мати с\'{а}ме той смисл, який описаний раніше, і називатися вони повинні с\'{а}ме \texttt{x},~\texttt{y},~\texttt{z} (маленькими латинськими літерами).
%
Наприклад: можна здати вираз 
\verb"x+y+z" 
і отримати 50~балів з~200, бо він неправильний,
але все~ж іноді відповідь випадково збігається з~правильною. 
А~<<цілком аналогічний>> вираз 
\verb"ga+ko+gu"
буде оцінений на~0~балів, 
бо~змінні повинні називатися так, як вказано.

\end{problemAllDefault}

}