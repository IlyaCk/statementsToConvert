\begin{problemAllDefault}{Остання ненульова цифра}

Для натуральних (цілих строго додатних) чисел, факторіал можна отримати як добуток 
\ifStatementOnly
$$
\else
$
\fi
N!%
\ifStatementOnly
=
\else
\dibbb{{=}}%
\fi
1\times2\times\dots\times{}N.
\ifStatementOnly
$$\par
\else
$
\fi
Напишіть програму, що знаходитиме останню ненульову цифру факторіала введеного натурального числ\'{а}.

\InputFile
Єдине натуральне число~$N$ ($1\dib{{\<}}N\dib{{\<}}2018019$).

\OutputFile
Остання ненульова цифра числа $N!$.

\ifAfour
\Examples
\begin{exampleSimpleExtraNarrow}{2.5em}{3em}
\exmp{4}{4}\end{exampleSimpleExtraNarrow}
\begin{exampleSimpleExtraNarrow}{2.5em}{3em}
\exmp{6}{2}\end{exampleSimpleExtraNarrow}
\begin{exampleSimpleExtraNarrow}{2.5em}{3em}
\exmp{7}{4}\end{exampleSimpleExtraNarrow}
\else
\Examples
\par\noindent\hspace*{-1em}
\begin{exampleSimple}{5em}{5em}
\exmp{4}{4}\end{exampleSimple}\hspace*{-1.75em}
\begin{exampleSimple}{5em}{5em}
\exmp{6}{2}\end{exampleSimple}\hspace*{-1.75em}
\begin{exampleSimple}{5em}{5em}
\exmp{7}{4}\end{exampleSimple}
\fi

% % % \Examples
% % % \begin{exampleSimple}{5em}{5em}
% % % \exmp{4}{4}
% % % \exmp{6}{2}
% % % \exmp{7}{4}\end{exampleSimple}

\Notes
$4! = {1{\times}2{\times}3{\times}4} = 24$, 
останньою ненульовою цифрою є~4.

$6! = {1{\times}2{\times}3{\times}4{\times}5{\times}6} = 720$, 
останньою ненульовою цифрою є~2.

$7! = {1{\times}2{\times}3{\times}4{\times}5{\times}6{\times}7} = 5040$, 
останньою ненульовою цифрою є~4.

\Scoring
Тести з умови не~приносять балів. Тести з рештою\nolinebreak[3] 20\nolinebreak[2] 
%значеннями\nolinebreak[3] $N$ з\nolinebreak[3] діапазону 
значеннями з\nolinebreak[3] проміжку
$1\dib{{\<}}N\dib{{\<}}23$ приносять по\nolinebreak[3] 2\%\nolinebreak[3] балів кожен (сумарно\nolinebreak[3] 40\%).
Решта\nolinebreak[3] 60\%\nolinebreak[3] балів поділені порівну (по~10\%) між такими шістьма блоками: 
два блоки з обмеженнями $25\dib{{\<}}N\dib{{\<}}555$,
два з обмеженнями $1234\dib{{\<}}N\dib{{\<}}54321$ та
два з обмеженнями $123456\dib{{\<}}N\dib{{\<}}2018019$.
Для\nolinebreak[3] кожного з блоків, бали нараховуються, лише якщо успішно пройдено \emph{всі} тести цього блоку. Блоки оцінюються незалежно 
один від одного.
% (нема залежностей вигляду <<блок перевіряється й оцінюється лише за умови успішного проходження таких-то інших блоків>>).

Здавати треба одну програму, а~не~різні програми для рiзних випадкiв; головна
мета цього опису блокiв\nolinebreak[3] --- не~даючи всіх деталей використаних тестів, дати приблизне уявлення про них. 
%%% Ви~можете хоч розбиратися з цим, хоч просто намагатися розв'язати задачу якнайкраще.

\ifStatementOnly
\vfill\par
Всі задачі цього змагання, включно з цією задачею, заборонено повторно здавати в ejudge після того, як вже здано повнобальний розв'язок.
\fi


% Блок~21 (тести 21--30) має обмеження 
% % $20\dibbb{{<}}N\dibbb{{\<}}333$, 
% $20 < N \< 333$, 
% приносить 14\% загальної кількості балів, 
% перевіряється й оцінюється лише за умови успішного проходження всіх попередніх блоків. 
% 
% Блок~22 (тести 31--40) має обмеження 
% % $333\dibbb{{<}}N\dibbb{{\<}}4444$, 
% $333 < N \< 4444$, 
% приносить 15\% загальної кількості балів, 
% перевіряється й оцінюється лише за умови успішного проходження всіх попередніх блоків. 
% 
% Блок~23 (тести 41--50) має обмеження 
% % $4444\dibbb{{<}}N\dibbb{{\<}}55555$, 
% $4444 < N \< 55555$, 
% приносить 15\% загальної кількості балів, 
% перевіряється й оцінюється лише за умови успішного проходження всіх попередніх блоків. 
% 
% Блок~24 (тести 51--60) має обмеження 
% % $55555\dibbb{{<}}N\dibbb{{\<}}7777777$, 
% $55555 < N \< 7777777$, 
% приносить 20\% загальної кількості балів, 
% перевіряється й оцінюється лише за умови успішного проходження всіх попередніх блоків. 
% 


\end{problemAllDefault}
