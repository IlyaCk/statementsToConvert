\Tutorial	Перш за все, слід зрозуміти, що основна проблема алгоритма \texttt{for \mbox{i:=1 to n} do\linebreak[2] if \mbox{n mod i = 0} then \mbox{write(i, ' ')}}\nolinebreak[3] --- він не\nolinebreak[3] має ніяких шансів устигнути за потрібний час (див.\nolinebreak[3] також стор.~\pageref{text:first-example-how-to-see-algo-will-not-fit-in-time-limit}).

\phantomsection\label{text:about-sqrt-n-in-divisors-list}
Є\nolinebreak[3] сенс перебирати дільники лише до\nolinebreak[2] \verb"sqrt(N)". Дільники, більші за $\sqrt{N}$, можна обчислити діленням $N$ на якийсь із дільників, менших~$\sqrt{N}$. Адже: (1)~якщо $a$\nolinebreak[3] --- дільник\nolinebreak[3] $N$, то $N/a$\nolinebreak[3] --- ціле, й також дільник~$N$; (2)~ч\'{и}сла $a$ та $N/a$ не~можуть одночасно бути більшими~$\sqrt{N}$. 

Можна один раз прокрутити цикл від 1 до\nolinebreak[3] $\sqrt{N}$, і повиводити всі знайдені дільники, а потім наступний (не\nolinebreak[3] вкладений, а\nolinebreak[3] наступний) цикл від $\sqrt{N}$ \texttt{downto}~1 і повиводити \verb"N div i". Такий алгоритм матиме складність $O(\sqrt{N})$, що очевидно поміститься у обмеження часу ($\sqrt{1234567891011}\dib{{\approx}}{1{,}1{\cdot}10^6}$). Можна трохи оптимізувати програму, якщо у першому циклі не\nolinebreak[3] лише виводити дільники, а\nolinebreak[3] ще й запам'ятовувати їх у масив, щоб др\'{у}гий цикл перебирав лише дільники, а\nolinebreak[3] не\nolinebreak[3] всі ч\'{и}сла проміжку. Але така оптимізація \emph{не}\nolinebreak[3] принципова.

Незалежно від % способу 
організації др\'{у}гого циклу, 
% варто 
слід
перевірити, чи\nolinebreak[3] правильно програма розбирається з такими випадками: (а)~$\sqrt{N}$\nolinebreak[1] цілий і є\nolinebreak[2] одним (а\nolinebreak[3] не\nolinebreak[3] двома) з\nolinebreak[2] дільників, як 6 для~36;
(б)~$\sqrt{N}$\nolinebreak[1] не\nolinebreak[3] цілий, але є два дільники близько до $\sqrt{N}$, як 6 і~7 для~42;
(в)~дільників, близьких до\nolinebreak[3] $\sqrt{N}$, нема.

Приклад розв'язку\nolinebreak[3] --- \IdeOne{wY3IY0}\hspace{0.5emplus 1em}
Варто відзначити такі його моменти: \texttt{N} мусить бути 64-\nolinebreak[3]бітовим, але решту величин зручніше лишити 32-\nolinebreak[3]бітовими;
для масиву\nolinebreak[3] \texttt{dividers}, розмір мільйон узятий з\nolinebreak[3] величезним запасом, насправді досить\nolinebreak[3] 7000; але правильно оцінити цю кількість дуже складно, тож якщо не~знати, то краще взяти з запасом (але враховуючи обмеження пам'яті).