{

\PrintEjudgeConstraintsfalse

\begin{problemAllDefault}{ККД}

Різні електролампи можуть мати різні коефіцієнти корисної дії\nolinebreak[3] (ККД).
За~означеннням, ККД електролампи є відношенням потужності, яку вона випромінює у~вигляді світла, до потужності, яку вона споживає з електромережі. 
Діапазон реальних ККД становить десь від 3$\,$\% у поганих ламп розжарювання до 50$\,$\% у якісних світлодіодних.
У~цій задачі вважаємо, що все, що споживається з електромережі й не~випромінюється як світло, переходить лише у тепло. 

Нехай є дві різні лампи, які мають однакову світлову потужність \texttt{p}$\,$Вт, але різні ККД \texttt{k1}$\,$\% та~\texttt{k2}$\,$\%. Наприклад, нехай 
${\texttt{p}\,{=}\,4\,\textnormal{(Вт)}}$, 
${\texttt{k1}\,{=}\,4\,\textnormal{(\%)}}$
та
${\texttt{k2}\,{=}\,50\,\textnormal{(\%)}}$.
Це~можливо, якщо перша лампа споживає з електромережі\nolinebreak[2] 100$\,$Вт (перетворюючи\nolinebreak[2] 96$\,$Вт у~тепло), а~др\'{у}га 8$\,$Вт і 4$\,$Вт відповідно.

{\hyphenpenalty=2000

Напишіть \emph{\underline{вираз}, що знаходитиме, у скільки разів більше тепл\'{а}\linebreak[2] виділяє перша лампа у\nolinebreak[2] порівнянні з другою}. (Наприклад, для щойно згаданих 
\texttt{p=4},
\texttt{k1=4},
\texttt{k2=50}
правильна відповідь$\,$=$\,$24, як $\frac{96\,\textnormal{\scriptsize{Вт}}}{4\,\textnormal{\scriptsize{Вт}}}\dib{{=}}24$;\linebreak[3] 
а для 
\texttt{p=17},
\texttt{k1=42},
\texttt{k2=47},
правильна відповідь$\,\approx\,$1.22461815.)

}

У~цій задачі треба здати не~програму, а~вираз: 
вписати його (сам вираз, не~створюючи файл) у~відповідну сторінку ejudge і відправити. 
У~виразі
можна використовувати цілі десяткові ч\'{и}сла, арифметичні дії ``\verb"+"''~(плюс), 
``\verb"-"''~(мінус), ``\verb"*"''~(множення), ``\verb"/"''~(ділення дробове), круглі дужки ``\verb"("''\nolinebreak[2] та~``\verb")"'' 
для групування та зміни порядку дій.
Дозволяються пропуски (пробіли), але\nolinebreak[3] не~всер\'{е}дині чисел та не~всер\'{е}дині назв змінних.
Множення треба писати явною зірочкою (наприклад, не~можна писати добуток $k_1k_2$ як ``\texttt{k1k2}'', лише~як~\mbox{``\texttt{k1*k2}''}).

Змінні, від яких залежить вираз, повинні називатися с\'{а}ме \texttt{k1},~\texttt{k2},~\texttt{p} (``\texttt{k}''\nolinebreak[3] та~``\texttt{p}''~--- маленькі латинські букви, ``\texttt{1}''\nolinebreak[3] та~``\texttt{2}''~--- цифри). 
Вираз перевірятимуть при цілих значеннях усіх цих змінних, у межах 
$3\dibbb{{\<}}\texttt{k1}\dibbb{{<}}\texttt{k2}\dibbb{{\<}}50$\nolinebreak[3] (відсотків), 
$1\dibbb{{\<}}\texttt{p}\dibbb{{\<}}20$~(Вт).

% \ifStatementOnly
% \vfill\par
% Всі задачі цього змагання, включно з цією задачею, заборонено повторно здавати в ejudge після того, як вже здано повнобальний розв'язок.
% \fi

\end{problemAllDefault}

}