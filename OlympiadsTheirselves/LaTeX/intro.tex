\section{Передмова}

\hspace*{\parindent}Цей збірник містить задачі олімпіад з~інформатики (програмування), що відбувалися у Черкаській області з\nolinebreak[2] \mbox{2012/13} по \mbox{2021/22} навчальні роки. Точніше, він охоплює тури обласних інтернет-олімпіад (крім 2012/13~н.~р., котра була, але не~потрапила в цей збірник, та 2021/22~н.~р., котрої не~було), др\'{у}гі (районні/\linebreak[1]міські) етапи Всеукраїнської олімпіади з~інформатики (крім 2020/21~н.~р., коли його не~було, а\nolinebreak[2] у\nolinebreak[2] \mbox{2020/21}~н.~р. ІІ~етап м.~Черкаси відрізнявся від ІІ~етапу решти області, й\nolinebreak[3] тут наведено той, що\nolinebreak[2] був у\nolinebreak[3] м.~Черкаси), та\nolinebreak[2] ті з третіх (обласних) етапів, у\nolinebreak[3] яких завдання формувалися черкаськими авторами, але\nolinebreak[2] не~охоплює ті з третіх (обласних) етапів, у\nolinebreak[3] яких завдання розроблялися іншими колективами авторів і надсилалися централізовано. Для розглянутих турів, до\nolinebreak[3] кожної задачі наведені умова та вказівки щодо її розв'язування.

{\hyphenpenalty=2000

Умови задач, що входять до збірника, формувалися авторським колективом у~складі: Порубльов~І.$\,$М.\nolinebreak[3] (ЧНУ), Богатирьов~О.$\,$О.\nolinebreak[3] (ЧНУ), Шемшур~В.$\,$М.\nolinebreak[3] (ЧОІПОПП), Фурник~І.$\,$В.\nolinebreak[3] (ЧОІПОПП), Черненко~Р.$\,$В. (випускник ЧНУ 2011~р.), Поліщук~Д.$\,$І. (випускник \mbox{ФіМЛі} 2004~р.), Безпоясний~Б.$\,$С. (ЧОІПОПП), Безпальчук~В.$\,$М.\nolinebreak[3] (ЧНУ), Борзяк~А.$\,$В. (випускник \mbox{ФіМЛі} 2016~р.; лише починаючи з осені 2016~р.). Умови задач наведені у~вигляді, максимально близькому до того, який вони мали на самих етапах олімпіади (лише змінено форматування та виправлено окремі технічні помилки). 

Більшість тексту розборів (пояснень) до задач написана Порубльовим~І.$\,$М., з~урахуванням порад вищезгаданих авторів (з~яких варто особливо відзначити поради Поліщука~Д.$\,$І.). 
Низку слушних пропозицій висловив Полосухін~В.$\,$А. (випускник\nolinebreak[2] \mbox{ФіМЛі} 2014~р., який брав участь лише у\nolinebreak[3] підготовці розборів після відповідних турів). 
Є~й~кілька випадків, коли на пояснення помітно вплинули р\'{о}зв'язки, здані на перевірку в \EjudgeCkipoName{}, чи задані засобами цього ejudge питання; тут окремої подяки заслуговує Граб~Н.$\,$В. (вчителька Золотоніської\nolinebreak[3] СШІТ~\textnumero$\,$2).
Тож, хто (чи то з учителів, чи то з учнів) бажає висловити зауваження та/або пропозиції\nolinebreak[3] --- це, в~розумній мірі, вітається, і, можливо, буде враховано при підготовці наступних версій цього збірника.
% % % після пізніше під учнів\nolinebreak[3] --- учасників відповідних турів олімпіади, або 
% % % У~деяких окремих випадках використані розв'язки учнів\nolinebreak[3] --- учасників відповідних турів олімпіади, а \mbox{3-й} спосіб розв'язання задачі <<Остання ненульова цифра>> (стор.~\mbox{\pageref{text:201819-2-C-start}--\pageref{text:201819-2-C-finish}}) з'явився завдяки тому, що Граб~Н.$\,$В. (Золотоніська\nolinebreak[3] СШІТ~\textnumero$\,$2) донесла до авторів збірника знайдену в~Інтернеті статтю, яку самі автори якось не~помітили.

}


Збірник не~призначений замінити підручник з програмування. Використовуючи конкретні задачі, нереально побудувати збалансований курс, що\nolinebreak[2] розглядає продуманий перелік\nolinebreak[2] тем; та\nolinebreak[3] й\nolinebreak[1] співвідношення обсягу задач %(23\nolinebreak[3] задачі з\nolinebreak[3] 6\nolinebreak[3] турів)
та обсягу збірника %\ifallIdeOneLinksCopiedHere\else(\pageref{LastPage}~сторінок) \fi{}
унеможливлює детальний розгляд усіх потрібних у цих задачах алгоритмів. Тому основна увага у~збірнику приділена поясненням нестандартних рішень у цих задачах. 
Коли задача зводиться до реалізації відомого алгоритму, зазвичай наводиться посилання на джерела в Інтернеті або в літературі. 
Рекомендується використовувати цей збірник у поєднанні із підручниками з\nolinebreak[3] програмування, монографіями та сайтами, де розглянуті ефективні алгоритми.

Тип паче збірник не~є посібником з конкретної мови програмування. Значна частина пояснень сформульована без прив'язки до мови (словесно, математичними формулами, рисунками, тощо). Де~цитати коду необхідні\nolinebreak[3] --- найчастіше використані Pascal та C++, рідше Python та Java.

% % % Розділ~\ref{sec:FAQ}\nolinebreak[3] (стор.~\pageref{sec:FAQ}--\nolinebreak[4]\pageref{text:FAQ-end}) містить огляд специфічних питань, які часто використовуються під час обговорення с\'{а}ме олімпіадних задач.

На олімпіаді з~інформатики (програмування) р\'{о}зв'язком задачі учасником, як\nolinebreak[3] правило, є програма. 
Цей збірник містить трохи таких програм безпосередньо у своєму тексті, але значно більше таких про\-грам-р\'{о}зв'\-яз\-ків доступні як\nolinebreak[3] посилання на\nolinebreak[3] 
сайт \IdeOneName{} (детальніше про його переваги та особливості див.\nolinebreak[3] стор.~\pageref{text:FAQ-section-about-ideone-com}). 

% % % \ifallIdeOneLinksCopiedHere
% % % Тексти цих програм включені безпосередньо у\nolinebreak[3] текст цього збірника. Точніше кажучи, лише у цю верстку, яка поширюється лише у\nolinebreak[3] електронному вигляді. С\'{а}ме через ці включення усіх програм ця верстка у деяких місцях некрасива, та й виковирювати тексти програм з\nolinebreak[3] цього \texttt{.pdf} не\nolinebreak[3] дуже зручно. Тому, хоч вони й є прямо тут, все\nolinebreak[3] ж спробуйте
% % % \else
% % % Цей збірник містить трохи таких програм безпосередньо у своєму тексті, але значно більше таких про\-грам-р\'{о}зв'\-яз\-ків доступні як\nolinebreak[3] посилання на\nolinebreak[3] 
% % % % Збірник майже\nolinebreak[1] не\nolinebreak[3] містить текстів цих програм, але вони доступні як\nolinebreak[3] посилання на\nolinebreak[3] 
% % % \fi
% % % сайт \IdeOneName{} (детальніше про його переваги та особливості див.\nolinebreak[3] стор.~\pageref{text:FAQ-section-about-ideone-com}). 
% % % Крім\nolinebreak[3] того, у\nolinebreak[3] \href{https://cit.ckipo.edu.ua/index.php/forum/olimpiady}{розділі <<Олімпіади>>} форума сайту \href{https://cit.ckipo.edu.ua}{\texttt{\mbox{cit.}\nolinebreak[3]\mbox{ckipo.}\nolinebreak[2]\mbox{edu.ua}}} розміщено і\nolinebreak[3] цю версію збірника, і\nolinebreak[3] версію%
% % % \ifallIdeOneLinksCopiedHere%
% % % \ (яка власне й була надрукована офіційно), що \emph{не}\nolinebreak[3] містить текстів програм (лише посилання на \verb"ideone.com"), і\nolinebreak[3] має краще вивірену верстку.
% % % \else%
% % % , де всі тексти програм розміщено у самому збірнику.
% % % %%% (але через це збільшується об'єм і\nolinebreak[3] погіршується якість верстки).
% % % \fi

Усі наведені у~збірнику задачі (й~не~лише вони) доступні для автоматичної перевірки на сайті \EjudgeCkipoName{} після простої безкоштовної реєстрації%, для якої потрібна лише електронна пошта
. Ця\nolinebreak[3] % автоматична 
перевірка в~цілому відображає і~те, що було на відповідних турах, і~%те, що 
написане у цьому збірнику.
%
Разом з тим, 
% за сім років 
за\nolinebreak[3] 9~років 
(з~2012~р., 
% в~якому відбулося 
коли було
перше змагання на сервері \EjudgeCkipoName, по 2021~р., коли внесені наразі останні правки цей збірник)
відбулося дві повні заміни апаратного забезпечення цього сервера.
\ifnum\number\year > 2021
\ERROR %re-check!
\fi
Через це
не~всі налаштування, які були актуальними на~момент проведення відповідних олімпіад, актуальні зараз. З'явилися і~ситуації, коли ті\nolinebreak[3] с\'{а}мі розв'язки, які раніше проходили менше тестів, почали проходити більше (бо~на\nolinebreak[3] новішому <<залізі>> при використанні новішого компілятора той самий код почав працювати швидше, або тому, що внаслідок переходу з \mbox{32-}\nolinebreak[3]біто\-вої на \mbox{64-}\nolinebreak[3]біто\-ву архітектуру зменшилися чи зникли переповнення типів), і~(рідше) ситуації, коли ті с\'{а}мі розв'язки почали проходити менше тестів (наприклад, тому, що~внаслідок того~ж переходу розв'язок почав перевищувати ліміт пам'яті). Зрідка трапляються навіть випадки, коли старий розв'язок, що нормально працював зі старою версією компілятора, тепер не~компілюється, або зворотні. Том\'{у}, фактична робота \EjudgeCkipoName{} може мати окремі незначні відхилення як від того, що було на відповідних турах, так і від написаного у збірнику, й на те нема ради.

\vspace{1pt plus 1in}