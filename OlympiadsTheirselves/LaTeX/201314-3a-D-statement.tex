\begin{problem}{Графічний пароль}{Клавіатура (stdin)}{Екран (stdout)}{4 сек}{64 мегабайти}

\hyphenpenalty=400

%%%\myflfigaw

{
\def\graphPwdFirstParagarph{Програміст Василь недавно придбав собі новий смартфон і встановив на ньому графічний пароль. Графічний пароль представляє собою ламану лінію, яка проходить через вершини сітки розміром $3{\*}3$ (не\nolinebreak[3] обов'язково через всі).}
\def\graphPwdSecondParagarph{Але згодом Василь зрозумів, що пароль на сітці $3{\*}3$ не\nolinebreak[3] є досить безпечним, і\nolinebreak[3] вирішив виправити цю проблему. Для цього він збільшив розмір сітки до $5000{\*}5000$, але при цьому наклав обмеження на відрізки ламаної\nolinebreak[3] --- тепер вони можуть бути лише горизонтальними та вертикальними. Василь ще хотів добавити функцію, яка\nolinebreak[3] б виводила кількість самоперетинів у графічному паролі, але так склалося, що він не\nolinebreak[3] досить знайомий з ефективними алгоритмами, тому він просить вашої допомоги.}
\def\graphPwdMfPic{\begin{mfpic}[28]{-0.5}{2.5}{-0.5}{2.5}
\def\c##1##2{\circle{(##1,##2),0.06}\gclear\circle{(##1,##2),0.03}}
\c{0}{0}
\c{0}{1}
\c{0}{2}
\c{1}{0}
\c{1}{1}
\c{1}{2}
\c{2}{0}
\c{2}{1}
\c{2}{2}
\pen{2pt}
\circle{(0,0),0.325}
\circle{(0,1),0.325}
\circle{(0,2),0.325}
\circle{(1,0),0.325}
\circle{(1,1),0.325}
\circle{(1,2),0.325}
\circle{(2,0),0.325}
\circle{(2,1),0.325}
\circle{(2,2),0.325}
\drawcolor{gray(0.375)}
\lines{(1,1),(0,1),(0,0),(2,0),(2,2),(0,2)}
\end{mfpic}}

\ifAfour
\myflfigaw{\graphPwdMfPic}
\graphPwdFirstParagarph
\par
\graphPwdSecondParagarph
\else
\mytextandpicture{\graphPwdFirstParagarph}{{\raisebox{-72pt}[0pt][72pt]{\graphPwdMfPic}}}
\par
\graphPwdSecondParagarph
\fi

}


\InputFile
Перший рядок містить ціле число\nolinebreak[3] $N$\nolinebreak[3] --- кількість вершин в ламаній лінії ($2\dib{{\<}}N\dib{{\<}}{1\,000\,000}$). Кожен з наступних $N$ рядків містить два цілих числ\'{а}\nolinebreak[1] $x$ та\nolinebreak[3] $y$\nolinebreak[3] --- координати відповідної вершини ламаної ($0\dib{{\<}}{x, y}\dib{{<}}5000$). Гарантується, що ламана у вхідних даних містить лише горизонтальні та вертикальні відрізки, які можуть перетинатися, але\nolinebreak[1] не\nolinebreak[3] можуть накладатись один на одного. Ніякі дві вершини ламаної (в~т.~ч. старт та\nolinebreak[2] фініш) не\nolinebreak[3] знаходяться в одній і тій самій точці.

\OutputFile
Необхідно вивести єдине ціле число\nolinebreak[3] --- кількість самоперетинів у\nolinebreak[3] ламаній.

\ifAfour\else
\vspace{-0.875\baselineskip plus 0.25 ex}
\fi

{\Examples\makeTableLongfalse
%
%%%\vspace{-\baselineskip} % TODO: доцільність істотно залежить від особливостей верстки
%
\begin{exampleSimple}{5em}{3em}%
\exmp{6
1 1
0 1
0 0
2 0
2 2
0 2}{0}%
\end{exampleSimple}
\begin{exampleSimple}{5em}{3em}%
\exmp{5
0 1
3 1
3 2
2 2
2 0}{1}%
\end{exampleSimple}

}

\end{problem}

