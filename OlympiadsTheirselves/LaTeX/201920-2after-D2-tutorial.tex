~

\smallskip

\vspace{-\baselineskip}

\Tutorial
Розбирати майже нема чого, бо треба лише врахувати і~розбір алгоритмічної складової, який можна знайти через повідомлення у згаданому змаганні~70 сайту \EjudgeCkipoName{} або за \href{https://ejudge.ckipo.edu.ua/coll_new_A4_11pt.pdf#page=110}{\emph{цим прямим посиланням}}, і~розбір цього способу з header-ами та footer-ами з попередньої задачі.
Варто тільки наголосити, що збільшення розмірів масиву та кількості запитів призводить до того, що, як\nolinebreak[3] правило, асимптотично ефективніші способи сильніше проявляють свої переваги (особливо, якщо рахувати у\nolinebreak[2] \emph{відсотках} набраних балів, а~не~кількості набраних балів чи пройдених блоків).

Втім, асимптотично добрий за часом (\texttt{calcRectMax} за $\Theta(1)$) спосіб з чи~то\nolinebreak[2] чотиривимірним масивом розмірів\nolinebreak[2] ${\lceil\log_2{N}\rceil\,{\times}\,\lceil\log_2{M}\rceil\,{\times}\,N\,{\times}\,M}$, чи~то\nolinebreak[2] двовимірним масивом двовимірних масивів тепер не~проходить тести блоків\nolinebreak[2] 5~та~7, бо\nolinebreak[3] не~вкладається у ліміт \emph{пам'яті}. Чи~можна покращити (зменшити) його витрати пам'яті, не~надто погіршивши (збільшивши) його витрати\nolinebreak[2] часу? До~певної (достатньої, щоб пройшли всі тести) міри так, якщо змінити розміри прямокутників з\nolinebreak[2] ${2^a\,{\*}\,2^b}$ на, наприклад, ${6^a\,{\*}\,6^b}$ чи ${8^a\,{\*}\,8^b}$. 
Об'єм пам'яті зменшиться у~${\approx}\,$\mbox{7--7,5}~разів 
(${\lceil\log_8{1234}\rceil}\dib{{=}}{\lceil\log_6{1234}\rceil}\dib{{=}}4$ проти ${\lceil\log_2{1234}\rceil}\dib{{=}}11$, внаслідок чого замість ${11{\times}11}\dib{{=}}121$ масивів по ${1234\,{\times}\,1234}$ лишається тільки ${4{\times}4}\dib{{=}}16$ таких масивів; $\frac{121}{16}$ навіть більше~7,5, але 
є\nolinebreak[3] ще\nolinebreak[3] інші витрати пам'яті).
%%% цей найбільший масив\nolinebreak[3] --- ще\nolinebreak[3] не~всі витрати пам'яті).
Щоправда, взнавати відповідь на запит вибором максимуму з\nolinebreak[3] \mbox{4-х}\nolinebreak[2] (${2\,{\*}\,2}$) елементів масиву вже не~вийде, але можна вибором максимуму з\nolinebreak[3] не~більш, як \mbox{36-ти}\nolinebreak[2] (${6\,{\*}\,6}$) чи не~більш, як \mbox{64-х}\nolinebreak[2] (${8\,{\*}\,8}$) елементів відповідно. 

Або, можна все-таки знайти в~Інтернеті чи літературі та реалізувати двовимірне узагальнення д\'{е}рева відрізків\nolinebreak[3] --- не~таке воно вже й складн\'{е}, а~з~точки зору потенційної корисності в інших задачах набагато краще тим, що дозволяє не~лише виконувати багато запитів щодо незмін\-ного масиву, а~й~змінювати його елементи.
% % % \end{enumerate}

