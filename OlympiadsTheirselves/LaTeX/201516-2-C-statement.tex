\begin{problemAllDefault}{Сучасне мистецтво}

Галерея готує показ творів сучасного мистецтва кількох митців і розглядає різні шляхи оформлення виставки. Так як це сучасне мистецтво, всі твори одного художника мало чим відрізняються один від одного.

Наприклад, припустимо, що є один твір митця~А, два твори митця~В і один твір митця~С. Тоді є 12 різних шляхів оформити виставку в галереї:

\vspace{-0.5\baselineskip}

\begin{multicols}{\ifAfour6\else4\fi}
\begin{enumerate}
\item
ABBC
\item
ABCB
\item
ACBB
\item
BABC
\item
BACB
\item
BBAC
\item
BBCA
\item
BCAB
\item
BCBA
\item
CABB
\item
CBAB
\item
CBBA
\end{enumerate}
\end{multicols}

\vspace{-0.5\baselineskip}

Список наведений в алфавітному порядку.

Напишіть програму, що визначає $n$-ий спосіб оформлення виставки, враховуючи що всі способи слідують в алфавітному порядку.

Ваша програма приймає на вхід 5 цілих чисел: $a$, $b$, $c$ і $d$ (кожне від 0 до 5 включно), що визначають кількість робіт митців A, B, C і~D відповідно, і останнє число $n$ ($1{\<}n{\<}2^{34}$)\nolinebreak[3] --- $n$-ий спосіб оформлення виставки. Гарантується, що хоча~б один митець виставляє свою роботу в галереї. Також гарантується, що $n$ не~перевищує загальної кількості способів.

Як результат ваша програма має вивести рядок (із літер A, B, C,~D), що визначає $n$-ий спосіб оформлення виставки.

\Example
\begin{exampleSimple}{5em}{3em}%
\exmp{1 2 1 0 8}{BCAB}%
\end{exampleSimple}

\end{problemAllDefault}
