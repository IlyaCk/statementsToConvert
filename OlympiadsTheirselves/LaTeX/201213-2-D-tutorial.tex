\Tutorial	Головним є пошук найкоротшого шляху, тож природним є
алгоритм пошуку вшир (вiн\nolinebreak[3] же пошук у ширину, вiн\nolinebreak[3] же\nolinebreak[3] BFS), причому у варіанті з відновленням шляху. Його описи знайдіть у лiтературi чи Iнтернетi.
%, бо цей збірник не~може включити в себе всі потрібні алгоритми. 
А\nolinebreak[3] особливості графу, до якого його слід застосувати, розглянемо.

В умові нема понять <<клітинка>> та її <<сторона>>, але їх варто ввести, бо без них важко формалізувати <<крок>> і <<по\nolinebreak[3] слідам>>. Будемо вважати, що Тезей рухається клітинками деякого прямокутника, і кожен з його <<кроків>> являє собою перехід на клітинку вище\nolinebreak[3] (\texttt{N}), нижче\nolinebreak[3] (\texttt{S}), лівіше\nolinebreak[3] (\texttt{W}) чи правіше~(\texttt{E}). Розміри прямокутника, на\nolinebreak[2] жаль, невідомі; але, на\nolinebreak[2] щастя, при співвідношенні <<до~200\nolinebreak[3] кроків, ліміт пам'яті до\nolinebreak[3] 64~Мб>>, це не~є\nolinebreak[2] серйозною проблемою. Як\nolinebreak[3] один з варіантів (громіздкий, зате ідейно простий), можна вважати, що і\nolinebreak[3] рядки, і\nolinebreak[3] стовпчики занумеровані від 0 до~400 (обидві меж\'{і} включно), і старт знаходиться у клітинці \texttt{[200][200]}. Можна придумати й інші, економніші за обсягом пам'яті, способи, але детальнішу дискусію про це залишимо за межами цього збірника.

Очевидно, що вершинами графа є клітинки; ребрами\nolinebreak[3] --- не~просто сусідні клітинки, а\nolinebreak[2] такі сусідні, що між ними Тезей проходив; граф неорієнтований, тобто те, що Тезей проходив хоча\nolinebreak[3] б в\nolinebreak[2] одному з напрямків, дозволяє перехід між цими клітинками у\nolinebreak[2] будь-якому напрямку. 

{\def\holeInVert#1#2{\gfill\polygon{(#1-0.0625,#2+0.25),(#1-0.0625,#2+0.75),(#1+0.0625,#2+0.75),(#1+0.0625,#2+0.25)}}
%requires fillcolor to be white or light-gray
\def\holeInHor#1#2{\gfill\polygon{(#1+0.25,#2-0.0625),(#1+0.75,#2-0.0625),(#1+0.75,#2+0.0625),(#1+0.25,#2+0.0625)}}

\begin{figure*}

\hyphenpenalty=-1

\ifBigStretch
\renewcommand{\baselinestretch}{1.25}
\else
\renewcommand{\baselinestretch}{1}
\fi

\noindent
\begin{minipage}{\textwidth}

\begin{\mainFontFamily}

\noindent
\begin{minipage}[t]{24em}

\noindent
\begin{minipage}[t]{5em}

\renewcommand{\baselinestretch}{0.75}

\vspace{-0.5\baselineskip}

\noindent
\begin{verbatim}
*********
*.*.*.*.*
*********
*.*.*.*.*
*********
*.*.*.*.*
*********
*.*.*.*.*
*********
\end{verbatim}

\vspace{-0.5\baselineskip}

\myhrulefill

\noindent
\begin{verbatim}
*********
*.*.*...*
*****.*.*
*.*.*...*
*****.***
*.....*.*
*****.***
*.*...*.*
*********
\end{verbatim}

\end{minipage}
\begin{footnotesize}\ifBigStretch\renewcommand{\baselinestretch}{0.75}\else\renewcommand{\baselinestretch}{0.5}\fi\begin{tabular}[t]{@{}c|l@{}}
0	& 			\\
1	& 			\\
2	& 3, 6		\\
3	& 2, 7		\\
4	& 			\\
5	& 			\\
6	& 2, 10, 7	\\
7	& 3, 6		\\
8	& 9			\\
9	& 8, 10		\\
10	& 6, 14, 9	\\
11	& 			\\
12	& 			\\
13	& 13		\\
14	& 10, 13	\\
15	& 			\\
\end{tabular}
\end{footnotesize}
\raisebox{-160pt}[12pt][144pt]{\begin{mfpic}[40]{0}{4}{0}{4}
\pen{2pt}
\lines{(0,0),(4,0)}
\lines{(0,1),(4,1)}
\lines{(0,2),(4,2)}
\lines{(0,3),(4,3)}
\lines{(0,4),(4,4)}
\lines{(0,0),(0,4)}
\lines{(1,0),(1,4)}
\lines{(2,0),(2,4)}
\lines{(3,0),(3,4)}
\lines{(4,0),(4,4)}
\fillcolor{gray(0.75)}
\holeInVert{1}{1}
\holeInVert{2}{1}
\holeInHor{2}{2}
\holeInHor{2}{3}
\holeInVert{3}{3}
\holeInHor{3}{3}
\holeInVert{3}{2}
\holeInHor{2}{1}
\holeInVert{2}{0}
\pen{5pt}
\lines{
(0.5,1.5),
(1.5,1.5),
(2.5,1.5),
(2.5,2.5),
(2.5,3.5),
(3.5,3.5),
(3.5,2.5),
(2.5,2.5),
(2.5,1.5),
(2.5,0.5),
(1.5,0.5),
(2.5,0.5)}
\tlabel[bl](0.55,1.55){0}
\tlabel[bl](1.55,1.55){1}
\tlabel[bl](2.55,1.55){\font\smaller=cmss9 {\smaller 2,8}}
\tlabel[bl](2.55,2.55){\font\smaller=cmss9 {\smaller 3,7}}
\tlabel[bl](2.55,3.55){4}
\tlabel[bl](3.55,3.55){5}
\tlabel[bl](3.55,2.55){6}
% \tlabel[bl](2.55,2.55){7}
% \tlabel[bl](2.55,1.55){8}
\begin{footnotesize}
\tlabel[bl](2.55,0.55){\font\smaller=cmss8 {\smaller 9,11}}
\end{footnotesize}
\tlabel[bl](1.55,0.55){10}
% \tlabel[bl](2.55,0.55){11}
%
%
\tlabel[bl](0.03125,0.0625){$_{0000}$}
\tlabel[bl](1.03125,0.0625){$_{0001}$}
\tlabel[bl](2.03125,0.0625){$_{0010}$}
\tlabel[bl](3.03125,0.0625){$_{0000}$}
%
\tlabel[bl](0.03125,1.0625){$_{0001}$}
\tlabel[bl](1.03125,1.0625){$_{0011}$}
\tlabel[bl](2.03125,1.0625){$_{1110}$}
\tlabel[bl](3.03125,1.0625){$_{0000}$}
%
\tlabel[bl](0.03125,2.0625){$_{0000}$}
\tlabel[bl](1.03125,2.0625){$_{0000}$}
\tlabel[bl](2.03125,2.0625){$_{1101}$}
\tlabel[bl](3.03125,2.0625){$_{0000}$}
%
\tlabel[bl](0.03125,3.0625){$_{0000}$}
\tlabel[bl](1.03125,3.0625){$_{0000}$}
\tlabel[bl](2.03125,3.0625){$_{0101}$}
\tlabel[bl](3.03125,3.0625){$_{0110}$}
%
\tlabel[tl](0.0625,3.9375){$0$}
\tlabel[tl](1.0625,3.9375){$1$}
\tlabel[tl](2.0625,3.9375){$2$}
\tlabel[tl](3.0625,3.9375){$3$}
%
\tlabel[tl](0.0625,2.9375){$4$}
\tlabel[tl](1.0625,2.9375){$5$}
\tlabel[tl](2.0625,2.9375){$6$}
\tlabel[tl](3.0625,2.9375){$7$}
%
\tlabel[tl](0.0625,1.9375){$8$}
\tlabel[tl](1.0625,1.9375){$9$}
\tlabel[tl](2.0625,1.9375){$10$}
\tlabel[tl](3.0625,1.9375){$11$}
%
\tlabel[tl](0.0625,0.9375){$12$}
\tlabel[tl](1.0625,0.9375){$13$}
\tlabel[tl](2.0625,0.9375){$14$}
\tlabel[tl](3.0625,0.9375){$15$}
%
\end{mfpic}}
\end{minipage}

\ifBigStretch
\vspace*{-11.5\baselineskip}
\else
\vspace*{-14\baselineskip}
\fi

\ifBigStretch
\parshape=13
25em 10em
23em 12em
23em 12em
23em 12em
23em 12em
23em 12em
23em 12em
23em 12em
23em 12em
10em 25em
10em 25em
10em 25em
0pt \textwidth
\else
\parshape=16
28.5em 16em
26.5em 18em
26.5em 18em
26.5em 18em
26.5em 18em
26.5em 18em
26.5em 18em
26.5em 18em
26.5em 18em
26.5em 18em
26.5em 18em
26.5em 18em
10em 34.5em
10em 34.5em
10em 34.5em
0pt \textwidth
\fi
На\nolinebreak[3] рисунку зображено приклад з умови; р\'{е}бра виділено сірими розривами у сторонах клітинок. Ч\'{и}сла трохи правіше й вище центрів клітинок позначають порядок, в якому їх відвідував Тезей. Ч\'{и}сла у лівих верхніх кутах клітинок позначають глобальні номери, а між рисунками зображено відповідні цим номерам і способу~(А) списки суміжності. Послідовності з чотирьох нуликів та/або одиничок у лівих нижніх кутах клітинок позначають, як мають описуватися переходи з кожної клітики способом~(Б). Нарешті, окреме поле з самих лише зірочок (<<стін>>) і крапочок (<<проходів>>), зображене ліворуч унизу, позначає поле, зображене праворуч, способом~(В); для порівняння, ліворуч угорі наведено поле з таких само клітинок, яке не\nolinebreak[3] містить ніяких переходів Тезея.

\end{\mainFontFamily}

\end{minipage}

\vspace{\baselineskip}

\myhrulefill

\phantomsection\label{fig:201213-2-Maze}

\end{figure*}

Є щонайменше три природні способи подавати такий граф:\linebreak[2]
(А)~занумерувати клітинки (наприклад, але\nolinebreak[2] не\nolinebreak[3] обов'язково, рядок за рядком), після чого використати звичайні списки суміжності;\linebreak[2]
(Б)~зберігати для кожної клітинки чотири булеві значення: 
чи~є проходи нагору\nolinebreak[3] (\texttt{N}),
униз\nolinebreak[3] (\texttt{S}),
ліворуч\nolinebreak[3] (\texttt{W}),
праворуч\nolinebreak[3] (\texttt{E}) (можна в іншому порядку, с\'{а}ме цей взятий лише тому, що так в умові);\linebreak[2]
(В)~перетворити поле з ${n\,{\times}\,m}$ клітинок, де стінки між клітинками можуть містити чи не~містити проходи, у\nolinebreak[2] поле з ${(2n\,{+}\,1)}\dib{{\times}}{(2m\,{+}\,1)}$ клітинок, де\nolinebreak[2] проходи (р\'{е}бра) є між усіма сусідініми крапочками, а стіни (відсутності проходів) позначені зірочками. Приклади всіх цих підходів зображені на рис. нагорі стор.~\pageref{fig:201213-2-Maze}. Само собою, потреби робити зразу все нема; слід вибрати один найзнайоміший з цих способів, або придумати який-небудь свій.

\myhrulefill

Насамкінець, пояснимо неправильність деяких <<ідей>>, які часто виникають при недостатньо ретельному аналізі цієї задачі. 
По-перше, не~можна <<просто рахувати зсуви по вертикалі й по горизонталі>>, бо сказано <<можна iти лише по слiдам Тезея>>, і, наприклад, для вхідних даних \texttt{EENNESWSSW} (як\nolinebreak[2] в\nolinebreak[2] умові, лише без останнього переходу) не~можна відповідати\nolinebreak[3] ``\texttt{NW}'', бо це проходило~б крізь межу між клітинками, яку Тезей не~перетинав; правильною була~б відповідь\nolinebreak[3] \texttt{ENWW}.

\myflfigaw{\begin{tabular}{@{}c|c@{}}
\begin{mfpic}[18]{0}{4}{0}{4}
\pen{0.75pt}
\lines{(0,0),(4,0)}
\lines{(0,1),(4,1)}
\lines{(0,2),(4,2)}
\lines{(0,3),(4,3)}
\lines{(0,4),(4,4)}
\lines{(0,0),(0,4)}
\lines{(1,0),(1,4)}
\lines{(2,0),(2,4)}
\lines{(3,0),(3,4)}
\lines{(4,0),(4,4)}
\fillcolor{gray(0.75)}
\holeInVert{2}{0}
\holeInHor{2}{1}
\holeInHor{2}{2}
\holeInHor{2}{3}
\holeInVert{2}{3}
\holeInHor{1}{3}
\holeInHor{1}{2}
\holeInVert{2}{1}
\holeInVert{3}{1}
\holeInHor{3}{2}
\holeInVert{3}{2}
\holeInVert{2}{2}
\holeInVert{1}{2}
% \holeInHor{1}{1}
% \holeInHor{1}{2}
% \holeInVert{2}{2}
% \holeInVert{3}{2}
% \holeInHor{3}{2}
% \holeInVert{3}{1}
% \holeInVert{2}{1}
% \holeInVert{1}{1}
% \holeInHor{0}{2}
% \holeInHor{0}{3}
% \holeInVert{1}{3}
% \holeInVert{2}{3}
% \holeInVert{3}{3}
% \holeInVert{3}{2}
% \holeInVert{3}{1}
\pen{1.5pt}
\lines{
(1.5,0.5), 
(2.5,0.5),
(2.5,1.5),
(2.5,2.5),
(2.5,3.5),
(1.5,3.5),
(1.5,2.5),
(1.5,1.5),
(2.5,1.5),
(3.5,1.5),
(3.5,2.5),
(2.5,2.5),
(1.5,2.5),
(0.5,2.5)}
\pen{3pt}
\lines{
(1.5,0.5), 
(2.5,0.5),
(2.5,1.5),
(2.5,2.5),
(1.5,2.5),
(0.5,2.5)}
\tlabel[bl](1.55,0.55){\font\smaller=cmss9 {\smaller 0}}
\tlabel[bl](2.55,0.55){\font\smaller=cmss9 {\smaller 1}}
\tlabel[bl](2.55,1.55){\font\smaller=cmss9 scaled 600{\smaller 2,8}}
\tlabel[bl](2.55,2.55){\font\smaller=cmss9 scaled 400{\smaller 3,11}}
\tlabel[bl](2.55,3.55){\font\smaller=cmss9 {\smaller 4}}
\tlabel[bl](1.55,3.55){\font\smaller=cmss9 {\smaller 5}}
\tlabel[bl](1.55,2.55){\font\smaller=cmss9 scaled 400{\smaller 6,12}}
\tlabel[bl](1.55,1.55){\font\smaller=cmss9 {\smaller 7}}
% \tlabel[bl](2.55,1.55){\font\smaller=cmss9 {\smaller 8}}
\tlabel[bl](3.55,1.55){\font\smaller=cmss9 {\smaller 9}}
\tlabel[bl](3.55,2.55){\font\smaller=cmss9 scaled 800{\smaller 10}}
% \tlabel[bl](2.55,2.55){\font\smaller=cmss9 scaled 800{\smaller 11}}
% \tlabel[bl](1.55,2.55){\font\smaller=cmss9 scaled 800{\smaller 12}}
\tlabel[bl](0.55,2.55){\font\smaller=cmss9 scaled 800{\smaller 13}}
\end{mfpic}
&
\begin{mfpic}[18]{0}{4}{0}{4}
\pen{0.75pt}
\lines{(0,0),(4,0)}
\lines{(0,1),(4,1)}
\lines{(0,2),(4,2)}
\lines{(0,3),(4,3)}
\lines{(0,4),(4,4)}
\lines{(0,0),(0,4)}
\lines{(1,0),(1,4)}
\lines{(2,0),(2,4)}
\lines{(3,0),(3,4)}
\lines{(4,0),(4,4)}
\fillcolor{gray(0.75)}
\holeInHor{1}{1}
\holeInHor{1}{2}
\holeInVert{2}{2}
\holeInVert{3}{2}
\holeInHor{3}{2}
\holeInVert{3}{1}
\holeInVert{2}{1}
\holeInVert{1}{1}
\holeInHor{0}{2}
\holeInHor{0}{3}
\holeInVert{1}{3}
\holeInVert{2}{3}
\holeInHor{2}{3}
\holeInHor{2}{2}
\holeInHor{2}{1}
\pen{1.5pt}
\lines{
(1.5,0.5),
(1.5,1.5),
(1.5,2.5),
(2.5,2.5),
(3.5,2.5),
(3.5,1.5),
(2.5,1.5),
(1.5,1.5),
(0.5,1.5),
(0.5,2.5),
(0.5,3.5),
(1.5,3.5),
(2.5,3.5),
(2.5,2.5),
(2.5,1.5),
(2.5,0.5)}
\pen{3pt}
\lines{
(1.5,0.5),
(1.5,1.5),
(2.5,1.5),
(2.5,0.5)}
\tlabel[bl](1.55,0.55){\font\smaller=cmss9 {\smaller 0}}
\tlabel[bl](1.55,1.55){\font\smaller=cmss9 scaled 600{\smaller 1,7}}
\tlabel[bl](1.55,2.55){\font\smaller=cmss9 {\smaller 2}}
\tlabel[bl](2.55,2.55){\font\smaller=cmss9 scaled 400{\smaller 3,13}}
\tlabel[bl](3.55,2.55){\font\smaller=cmss9 {\smaller 4}}
\tlabel[bl](3.55,1.55){\font\smaller=cmss9 {\smaller 5}}
\tlabel[bl](2.55,1.55){\font\smaller=cmss9 scaled 400{\smaller 6,14}}
% \tlabel[bl](1.55,1.55){\font\smaller=cmss9 {\smaller 7}}
\tlabel[bl](0.55,1.55){\font\smaller=cmss9 {\smaller 8}}
\tlabel[bl](0.55,2.55){\font\smaller=cmss9 {\smaller 9}}
\tlabel[bl](0.55,3.55){\font\smaller=cmss9 scaled 800{\smaller 10}}
\tlabel[bl](1.55,3.55){\font\smaller=cmss9 scaled 800{\smaller 11}}
\tlabel[bl](2.55,3.55){\font\smaller=cmss9 scaled 800{\smaller 12}}
% \tlabel[bl](2.55,2.55){\font\smaller=cmss9 {\smaller 13}}
% \tlabel[bl](2.55,1.55){\font\smaller=cmss9 {\smaller 14}}
\tlabel[bl](2.55,0.55){\font\smaller=cmss9 scaled 800{\smaller 15}}
\end{mfpic}
\\
\footnotesize{\texttt{ENNNWSSEENWWW}}
&
\footnotesize{\texttt{NNEESWWWNNEESSS}}
\end{tabular}}

По-друге, ідея <<шукати та вирізати цикли>> правильна часто, але\nolinebreak[2] теж не\nolinebreak[3] завжди.
Зокрема, як видно з рисунка нагорі\label{text:201213-2-D-Maze-counter-example-2}
\ifnum\getpagerefnumber{text:201213-2-D-Maze-counter-example-2}=\getpagerefnumber{fig:201213-2-Maze}%
цієї сторінки,
\else%
стор.~\pageref{fig:201213-2-Maze},
\fi%
для прикладу з умови, \mbox{2-а} та \mbox{8-а} клітинки збігаються; \mbox{9-а} та \mbox{11-а} теж; значить, можна викинути все, що між \mbox{2-ю} та\nolinebreak[2] \mbox{8-ю}, об'єднавши їх; аналогічно між \mbox{9-ю} та\nolinebreak[3] \mbox{11-ю}.

Але\nolinebreak[2] це погано працює, наприклад, для вхідних даних \texttt{ENNNWSSEENWWW} (де\nolinebreak[2] важливо ще правильно здогадатися, \emph{який} із кількох циклів вирізати, щоб отримати правильну відповідь \texttt{EESSW}, а\nolinebreak[3] не,\nolinebreak[3] наприклад, \texttt{EEESWSW}),\linebreak[2] і геть не~працює, наприклад, для вхідних даних \texttt{NNEESWWWNNEESSS} (де\nolinebreak[2] правильна відповідь \texttt{NWS} взагалі не~отримується таким способом).

}
