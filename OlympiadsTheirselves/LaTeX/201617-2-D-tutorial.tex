\Tutorial
<<Лобовий>> розв'язок (перебирати послідовно \texttt{1*1*1}, \texttt{2*2*2}, \texttt{3*3*3}, тощо, аж доки не~поділиться на~введене~$k$) благополучно і гарантовано проходить перші два блоки тестів, але у\nolinebreak[3] наступних блоках\nolinebreak[3] --- лише деякі окремі тести, бо для більших~$k$ цей підхід має відразу дві проблеми: 
\begin{enumerate}
\item
Значення куба може не~поміщатися у~стандартні типи даних, наприклад при використанні \mbox{64-}\nolinebreak[3]бітового беззна\-ко\-вого цілого (\texttt{QWord}\nolinebreak[2] (Pascal), \texttt{unsigned\nolinebreak[2] long\nolinebreak[2] long}\nolinebreak[2] (C/C++)%, \texttt{ulong}\nolinebreak[2] (C\#)
)\nolinebreak[3] --- починаючи з $2642246^3\dibbb{{=}}18446745128696702936\dibbb{{\>}}18446744073709551616\dibbb{{=}}2^{64}$.
(Якщо взяти менший тип, як-то \mbox{32-}\nolinebreak[3]бітовий, межа буде переходитися % значно 
раніше, й~проблеми виникнуть вже у\nolinebreak[3] \mbox{2-му} блоці тестів.)
\item
Навіть якби перша проблема була вирішена (чи~то використанням мови Java або Python, що мають вбудовану довгу арифметику, чи~то написанням довгої арифметики вручну), такий перебір не~помістився~б у~обмеження часу, причому для деяких з тестів\nolinebreak[3] --- у\nolinebreak[3] мільярди чи трильйони разів.
\end{enumerate}

% % % \begin{color}{red}Буду вдячний, якщо хтось розпише детальніше (але не~надто довго) прості способи отримати помітну частину балів.\end{color}

\MyParagraph{Повний розв'язок.}
Якби провести \emph{факторизацію} (розкладення числ\'{а} на прості множники), можна було~б по~кожному степеню простого множника $p_j^{k_j}$ сказати, що відповідь повинна містити цей простий множник 
$\left\lceil\frac{k_j}{3}\right\rceil$ разів (де\nolinebreak[3] $\lceil{x}\rceil$\nolinebreak[3] --- \emph{заокруглення вгору}, воно~ж \emph{стеля}; наприклад, 
$\left\lceil\frac{1}{3}\right\rceil\dib{{=}}
 \left\lceil\frac{2}{3}\right\rceil\dib{{=}}
 \left\lceil\frac{3}{3}\right\rceil\dib{{=}}1$,
$\left\lceil\frac{4}{3}\right\rceil\dib{{=}}
 \left\lceil\frac{5}{3}\right\rceil\dib{{=}}
 \left\lceil\frac{6}{3}\right\rceil\dib{{=}}2$, тощо).
Наприклад, $2016\dib{{=}}{2^5{\times}3^2{\times}7^1}$, значить 
2\nolinebreak[3] треба брати у\nolinebreak[1] степені $\left\lceil\frac{5}{3}\right\rceil\dib{{=}}2$,
3\nolinebreak[3] у\nolinebreak[2] степені $\left\lceil\frac{2}{3}\right\rceil\dib{{=}}1$ і
7\nolinebreak[3] у\nolinebreak[2] $\left\lceil\frac{1}{3}\right\rceil\dib{{=}}1$,
тож відповіддю для 2016 є $2^2{\cdot}3^1{\cdot}7^1\dib{{=}}84$.
Технічно для заокруглення вгору у\nolinebreak[3] багатьох мовах програмування є бібліотечна функція \texttt{ceil} (чи\nolinebreak[3] \texttt{Ceiling}); інший спосіб\nolinebreak[3] --- виражати $\left\lceil\frac{k_j}{3}\right\rceil$ як ${{\texttt{(k\_j+2)}}\,{\texttt{div}}\,3}$ (де\nolinebreak[3] \texttt{div}\nolinebreak[3] --- цілочисельне ділення, як-то \texttt{17~div~3~=~5}%; цей спосіб кращий тим, що гарантовано не~буде проблем із похибками
).

Як робити факторизацію, написано у багатьох джерелах, зокрема\nolinebreak[3] --- теорії до змагання \textnumero$\,$49 <<День Іллі Порубльова ``Школи Бобра'' (2015, деякі теми теорії чисел)>> цього ж сайту \EjudgeCkipoName{} (зайти у змагання\nolinebreak[3] \textnumero$\,$49 і у\nolinebreak[3] повідомленнях %від 
журі знайти посилання на\nolinebreak[3] розбір). Але проблема в~тім, що <<чесна>> факторизація числ\'{а}~$N$ працює за~час\nolinebreak[3] $O(\sqrt{N})$, що при ${N\,{\sim}\,10^{18}}$ забагато. 

Тут треба помітити, що увесь проміжок від $\sqrt[3]{N}$ до $\sqrt{N}$ перебирається з єдиною метою: розрізнити ситуацію <<число~$N$ є простим>> від ситуації <<число~$N$ є добутком двох простих, не~дуже далеких від~$\sqrt{N}$>>. А~з~точки зору с\'{а}ме цієї задачі їх можна і не~розрізняти: хоч\nolinebreak[3] ${z\,{=}\,p^1}$, хоч\nolinebreak[3] ${z\,{=}\,p_1^1{\times}p_2^1}$ однаково означають, що у\nolinebreak[3] відповідь входитиме сам\'{е} число~$z$, яке досліджуємо. Так\nolinebreak[3] що верхню межу перебору можна знизити з $\sqrt{N}$ до\nolinebreak[3] $\sqrt[3]{N}$, а\nolinebreak[3] $\sqrt[3]{10^{18}}\dib{{=}}10^6$ для\nolinebreak[2] комп'ютера небагато. Звісно, %при\nolinebreak[3] цьому 
треба окремо перевірити, чи\nolinebreak[2] не\nolinebreak[3] є поточне число точним квадратом, бо\nolinebreak[3] спостереження щодо <<хоч\nolinebreak[3] ${z\,{=}\,p^1}$, хоч\nolinebreak[3] ${z\,{=}\,p_1^1{\times}p_2^1}$>> справедливе \emph{лише}\nolinebreak[3] при\nolinebreak[3] ${p_1\,{\neq}\,p_2}$.
