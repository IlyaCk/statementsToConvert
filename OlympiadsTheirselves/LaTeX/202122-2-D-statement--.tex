\begin{problemAllDefault}{Паралелепіпеди}%

\myflfigaw{\hspace{-0.5em}\begin{minipage}{11.5em}
\Examples\\
\begin{exampleSimple}{5em}{5em}
\exmp{7 7}{0 0 0}%
\exmp{6 22}{1 2 3\\
1 3 2\\
2 1 3\\
2 3 1\\
3 1 2\\
3 2 1}%
\exmp{1 6}{1 1 1}%
\exmp{1600 1120}{4 20 20\\
5 8 40\\
5 40 8\\
8 5 40\\
8 40 5\\
20 4 20\\
20 20 4\\
40 5 8\\
40 8 5}%
\end{exampleSimple}
\end{minipage}}

Напишіть програму, яка знаходитиме перелік прямокутних паралелепіпедів, що мають об'єм~$V$ та площу поверхні~$S$. Враховувати лише паралелепіпеди, в\nolinebreak[2] яких всі три розміри виражаються натуральними числами.

\InputFile
Єдиний рядок містить роз\-ді\-лені одинарним пробілом два натуральні числа $V$~$S$.
% Єдиний рядок містить два натуральні числа, відділені одинарним пробілом: $V$,\nolinebreak[3] потім~$S$.
% % спочатку~$V$, потім~$S$, 
% % % Обмеження на 
% % % % значення $V$, $S$ 
% % % їх значення
% % % див.~далі.

\OutputFile
Кожен рядок повинен містити розділені пробілами цілочисельні розміри $a$~$b$~$c$ чергового паралелепіпеда. Ці~трійки обов'язково виводити у\nolinebreak[3] порядку зростання першого розміру~$a$, при рівних~$a$\nolinebreak[3] --- у\nolinebreak[3] порядку зростання др\'{у}гого розміру~$b$\linebreak[3] (а~різних відповідей, в~яких рівні і~$a$, і~$b$, не~буває). 

У~випадку, якщо жодного паралелепіпеда з потрібними $V$,\nolinebreak[3] $S$ не~існує, виводьте єдиний рядок \texttt{0~0~0}.

\Notes
Якщо прямокутний паралелепіпед має розміри $a$~$b$~$c$, то його об'єм дорівнює\nolinebreak[3] ${a\,{\cdot}\,b\,{\cdot}\,c}$, а\nolinebreak[3] площа поверхні ${2\,{\cdot}\bigl(a\,{\cdot}\,b}\dibbb{{+}}{b\,{\cdot}\,c}\dibbb{{+}}{c\,{\cdot}\,a}\bigr)$, 
бо є 
дві \mbox{грані} (наприклад, передня й задня) площі\nolinebreak[2] ${a\,{\cdot}\,b}$, 
дві (наприклад, ліва та права) площі\nolinebreak[2] ${b\,{\cdot}\,c}$ 
та
дві (наприклад, верхня й нижня) площі\nolinebreak[2] ${c\,{\cdot}\,a}$.

% % % У~першому прикладі, для $V\dibbb{{=}}S\dibbb{{=}}7$ не~існує жодного паралелепіпеда.
% % % У~др\'{у}гому, об'єм~6 та площу поверхні~22 має паралелепіпед ${1{\times}2{\times}3}$, і він рахується за~6~штук, бо при поворотах має різний вигляд. У~третьому прикладі, повороти кубика ${1{\times}1{\times}1}$ не~утворюють нового вигляду. У~четвертому, повороти ${5{\times}8{\times}40}$ утворюють 6 різних виглядів, а повороти ${4{\times}20{\times}20}$ лише 3 різні вигляди.



\Scoring

\def\scoringByTestsPart{\begin{minipage}{0.55\textwidth}
\begin{small}
%\begin{footnotesize}
Потестове (можна отримувати бали за окремі тести, навіть якщо інші не~пройшли)
\par
%\end{footnotesize}
\end{small}
\end{minipage}}

\def\scoringByBlocksPart{\begin{minipage}{0.55\textwidth}
\begin{small}
%\begin{footnotesize}
Поблокове (якщо не~пройшов хоч~один тест блоку, весь блок оцінюється на~0); блоки з тестів \mbox{25--29}, \mbox{30--34}, \mbox{35--39}, \mbox{40--44} (усі, крім двох останніх) перевіряються й оцінюються незалежно від того, чи~зараховані інші блоки та тести; два останні (тести\nolinebreak[3] \mbox{45--55} та\nolinebreak[3] \mbox{56--65}) перевіряються й оцінюються, лише якщо успішно пройдено всі попередні тести (\mbox{1--44} чи \mbox{1--55}). Суть відмінності двох останніх блоків є секретною (до\nolinebreak[3] кінця туру).
\par
%\end{footnotesize}
\end{small}
\end{minipage}}

\begin{longtable}{@{}c|c|c|c@{}}
тести
&
обмеження
&
\begin{footnotesize}
\hspace*{-1em}\begin{tabular}{@{}c@{}}
кіль-ть\\
балів
\end{tabular}\hspace*{-1em}
\end{footnotesize}
&
спосіб нарахування балів
\\\hline\endhead
1--4
&
& 
% 0$\,$\%
0
&
\begin{small}
%\begin{footnotesize}
\begin{minipage}{0.55\textwidth}
безпосередньо не~оцінюються (з умови)
\end{minipage}
%\end{footnotesize}
\end{small}
\\\hline
5--14
&
$2\dib{{\<}}{V,\,S}\dib{{\<}}100$
& 
% 20$\,$\%
60
&
\multirow{2}{*}{\scoringByTestsPart}
\\
15--24
&
$500\dib{{\<}}{V,\,S}\dib{{\<}}10^4$
& 
% 20$\,$\%
60
\\\hline
25--29
&
$100\dib{{\<}}{V,\,S}\dib{{\<}}500$
& 
% 10$\,$\%
30
&
\multirow{5}{*}{\scoringByBlocksPart}
\\
30--34
&
$10^4\dib{{\<}}{V,\,S}\dib{{\<}}10^6$
& 
% 10$\,$\%
30
\\
35--39
&
$10^6\dib{{\<}}{V,\,S}\dib{{\<}}10^8$
& 
% 10$\,$\%
30
\\
40--44
&
\begin{small}$10^7\dib{{\<}}S\dib{{\<}}10^9\dib{{\<}}V\dib{{\<}}10^{11}$\end{small}
& 
% 10$\,$\%
30
\\
45--53
&
% $10^{10}\dib{{\<}}S\dib{{\<}}10^{15}\dib{{\<}}V\dib{{\<}}10^{18}$
\begin{small}
$\left\{\begin{array}{c}
10^{10}\dib{{\<}}S\dib{{\<}}10^{17},\\
10^{12}\dib{{\<}}V\dib{{\<}}10^{18}
\end{array}\right.$
\end{small}
& 
% 20$\,$\%
60
\\
54--65
&
% $10^{10}\dib{{\<}}S\dib{{\<}}10^{15}\dib{{\<}}V\dib{{\<}}10^{18}$
\hspace*{-1em}\begin{small}
такі ж, як у попередньому
\end{small}\hspace*{-1em}
& 
% 20$\,$\%
50
\\
\end{longtable}

Гарантовано, що у кожному з тестів, що використовуються для оцінювання цієї задачі, кількість трійок-відповідей строго менша~100.

\end{problemAllDefault}
