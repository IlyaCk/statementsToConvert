{

\PrintEjudgeConstraintsfalse

\begin{problemAllDefault}{Кількість секунд}

Є два моменти часу: 
$a$~годин $b$~хвилин $c$~секунд та
$d$~годин $e$~хвилин $f$~секунд.
Гарантовано, що вони належать одній добі; 
гарантовано, що перший з них ($a:b:c$) відбувся раніше, ніж др\'{у}гий ($d:e:f$);
гарантовано, що 
${0\,{\<}\,a,d\,{\<}\,23}$, 
${0\,{\<}\,b,c,e,f\,{\<}\,59}$.

{

\looseness=-1
Напишіть \emph{вираз}, який знаходитиме, на скільки секунд др\'{у}гий момент часу пізніше, ніж перший. 

\looseness=-1
Змінні, від яких залежить вираз, обов'язково повинні мати с\'{а}ме той смисл, який описаний раніше, і називатися вони повинні с\'{а}ме $a$,~$b$, $c$, $d$, $e$,~$f$ (маленькими латинськими літерами).

\looseness=-1
У~цій задачі треба здати не~програму, а~вираз: 
вписати його (сам вираз, не~назву файлу) у~відповідне поле перевіряючої системи
і відправити на~перевірку. Правила запису виразу:
можна використовувати цілі десяткові ч\'{и}сла, арифметичні дії ``\verb"+"''~(плюс), 
``\verb"-"''~(мінус), ``\verb"*"''~(множення), ``\verb"/"''~(ділення дробове, наприклад, \verb"17/5"${=}3{,}4$), ``\verb"//"''~(ділення цілочисельне, наприклад, \verb"17//5"${=}3$), круглі дужки ``\verb"("''\nolinebreak[2] та~``\verb")"'' 
для групування та зміни порядку дій.
Дозволяються пропуски (пробіли), але\nolinebreak[3] не~всер\'{е}дині чисел.

\looseness=-1
Наприклад: можна здати вираз 
\verb"(d+e+f)-(a+b+c)" 
і отримати 6~балів з~200, бо він неправильний,
але все~ж іноді відповідь випадково збігається з~правильною. 
А~<<цілком аналогічний>> вираз 
\verb"(h+m+s)-(x+y+z)"
буде оцінений на~0~балів, 
бо~змінні повинні називатися так, як вказано.

}

\end{problemAllDefault}

}