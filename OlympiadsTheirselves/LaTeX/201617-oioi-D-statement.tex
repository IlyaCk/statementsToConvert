\begin{problemAllDefault}{Кількість прямокутників}

На перерві Андрійку нема чого робити, друзі пішли за шаурмою, а~дівчинка, яка йому подобається, пішла гуляти з іншими хлопчиками. Однак щоб не~сумувати він вирішив сумувати, тобто рахувати. Він хоче знайти кількість різних прямокутників на своєму листочку в клітинку.

\InputFile
В єдиному рядку задано 2 числ\'{а}\nolinebreak[3] --- 
кількість клітинок на листочку по ширині та висоті.
Кожен з~розмірів завжди не~менший~1, 
обмеження на максимальний розмір наведені далі.

\OutputFile
Виведіть кількість різних прямокутників.

{

\def\RectMyForThisProb#1#2#3#4{
\begin{mfpic}[9]{-0.5}{2.5}{-0.5}{2.5}
%\dotted\bclosed\curve{(0,0),(1,-0.25),(2,0),(2.25,1),(2,2),(1,2.25),(0,2),(-0.25,1)}
\fillcolor{gray(0.5)}
\gfill\polygon{(#1,#3),(#1,#4),(#2,#4),(#2,#3)}
\lines{(0,0),(2,0)}
\lines{(0,1),(2,1)}
\lines{(0,2),(2,2)}
\lines{(0,0),(0,2)}
\lines{(1,0),(1,2)}
\lines{(2,0),(2,2)}
\end{mfpic}}

\savebox{\mypictbox}{\mbox{%
\RectMyForThisProb{0}{1}{0}{1}~
\RectMyForThisProb{0}{2}{0}{1}~
\RectMyForThisProb{1}{2}{0}{1}~
\RectMyForThisProb{0}{1}{0}{2}~
\RectMyForThisProb{0}{2}{0}{2}~
\RectMyForThisProb{1}{2}{0}{2}~
\RectMyForThisProb{0}{1}{1}{2}~
\RectMyForThisProb{0}{2}{1}{2}~
\RectMyForThisProb{1}{2}{1}{2}}}

\Examples

\noindent\begin{tabular}{@{}cc}
\mbox{\begin{exampleSimple}{5em}{5em}
\exmp{2 2}{9}%
\exmp{3 2}{18}%
\end{exampleSimple}}
&
\begin{tabular}{@{}c@{}}
Повний перелік усіх прямокутників тесту \textnumero$\,$1\\
\usebox{\mypictbox}
\end{tabular}
\end{tabular}

}


% \begin{tabular}{p{0.45\textwidth}p{0.45\textwidth}}
% Праворуч наведено повний перелік усіх дев'яти різних прямокутників 
% з\nolinebreak[2] тесту\nolinebreak[2] $\No\,$1.
% &
% \begin{tabular}{ccc}
% \RectMyForThisProb{0}{1}{0}{1}
% &
% \RectMyForThisProb{0}{2}{0}{1}
% &
% \RectMyForThisProb{1}{2}{0}{1}
% \\
% \RectMyForThisProb{0}{1}{0}{2}
% &
% \RectMyForThisProb{0}{2}{0}{2}
% &
% \RectMyForThisProb{1}{2}{0}{2}
% \\
% \RectMyForThisProb{0}{1}{1}{2}
% &
% \RectMyForThisProb{0}{2}{1}{2}
% &
% \RectMyForThisProb{1}{2}{1}{2}
% \\
% \end{tabular}
% \end{tabular}

\Scoring
20\%~балів припадатиме на тести, в~яких обидва розміри не~перевищують~4.
Ще~30\%~балів припадатиме на тести, в~яких більший з~розмірів у~межах від~17 до~42.
Ще~20\%~балів припадатиме на тести, в~яких більший з~розмірів у~межах від~12345 до~54321.
Решта 30\%~балів припадатиме на тести, в~яких більший з~розмірів у~межах від~$10^7$ (десяти мільйонів) до~$10^9$ (мільярда).

Здавати потрібно одну програму, а~не~чотири; різні обмеження вказані, щоб пояснити, 
скільки балів можна отримати, розв’язавши задачу не~повністю.


\end{problemAllDefault}
