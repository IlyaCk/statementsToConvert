\Tutorial
Очевидно, що сумарна вартість буде мінімальною, якщо використати найдорожчу фарбу (ціни~\texttt{c}) для пари граней мінімальної площі, середньої ціни~\texttt{b} для граней середньої площі, найдешевшу\nolinebreak[3] (\texttt{a}) \nolinebreak[3] --- для граней максимальної площі. При бажанні, це твердження можна \emph{довести}, але ми доведення пропустимо; приклади аналогічних доведень (звісно, для інших задач та розв'язків) можна бачити, зокрема, на~стор.~\pageref{text:proof-example-parket-1},~\pageref{text:proof-example-train-to-ship},~\pageref{text:proof-max-num-by-strike-out-one-digit}.

Так\nolinebreak[3] що головна складність\nolinebreak[3] --- знайти, яка з площ максимальна, яка середня і яка мінімальна. Для максимума це, очевидно, 
\texttt{\mbox{max(x*y,$\,$x*z,$\,$y*z)}};
аналогічно, для мінімума 
\texttt{\mbox{min(x*y,$\,$x*z,$\,$y*z)}}. 
Середню ж можна виразити, наприклад, як \texttt{\mbox{max(x,y,z)*min(x,y,z)}} 
(чому? бо, якщо довж\'{и}ни всі різні, то є рівно три дов\-жин\'{и}: мінімальна, середня й максимальна, і з них можна утворити три \mbox{пари}\linebreak[2] 
(мінімальна,\nolinebreak[2] середня),\linebreak[2]
(мінімальна,\nolinebreak[2] максимальна),\linebreak[2]
(середня,\nolinebreak[2] максимальна),\linebreak[2]
і мінімальну площу утоврює перша пара, максимальну\nolinebreak[3] --- остання, 
так\nolinebreak[3] що для середньої площі лишається др\'{у}га пара).
%
Ще один спосіб знайти середню площу
\texttt{\mbox{(x*y$\,$+$\,$x*z$\,$+$\,$y*z) -}}\nolinebreak[1]  
\texttt{\mbox{max(x*y,$\,$x*z$\,$y*z) -}}\nolinebreak[1]  
\texttt{\mbox{min(x*y,$\,$x*z$\,$y*z)}}.
(Якщо додати всі три й відняти з суми мінімум та максимум, залишиться середня.)

{
\def\paralAnsFirst{\begin{ttfamily}{%
\mbox{a*2*max(x*y,x*z,y*z)}\nolinebreak[1]\hspace{0.25em plus 0.5em}+\linebreak[1]\hspace{0.25em plus 0.5em}%
\mbox{c*2*min(x*y,x*z,y*z)}\nolinebreak[2]\hspace{0.25em plus 0.5em}+\linebreak[1]\hspace{0.25em plus 0.5em}%
\mbox{b*2*max(x,y,z)*min(x,y,z)}}\end{ttfamily}}
\def\paralAnsSecond{\begin{ttfamily}{%
\mbox{a*2*max(x*y,x*z,y*z)}\nolinebreak[1]\hspace{0.125em plus 0.5em}+\linebreak[1]\hspace{0.125em plus 0.5em}%
\mbox{c*2*min(x*y,x*z,y*z)}\nolinebreak[1]\hspace{0.125em plus 0.5em}+\linebreak[1]\hspace{0.125em plus 0.5em}%
\mbox{b*2*((x*y+x*z+y*z)}\nolinebreak[3]\hspace{0.0625em plus 0.125em}-\linebreak[1]\hspace{0.0625em plus 0.125em}%
\mbox{max(x*y,x*z,y*z)}\nolinebreak[3]\hspace{0.0625em plus 0.125em}-\linebreak[1]\hspace{0.0625em plus 0.125em}%
\mbox{min(x*y,x*z,y*z)})}\end{ttfamily}}

Насамкінець, досі скрізь шукали площу однієї грані, а сумарна площа пари протилежних граней удвічі більша.
\ifAfour
Це\nolinebreak[3] для\nolinebreak[2] першого способу дає вираз ``\paralAnsFirst'',
а\nolinebreak[3]  для\nolinebreak[2] др\'{у}гого ``\paralAnsSecond''.
\else
Це\nolinebreak[3] дає вираз \paralAnsFirst{}
для першого способу та
\paralAnsSecond{} для другого.
\fi

}

Ще один спосіб розв'язання задачі\nolinebreak[3] --- ігноруючи міркування першого абзацу, не~шукати, яка з площ мінімальна, середня та максимальна, а просто перебрати всі ${3!\,=6}$ варіантів, які грані якою фарбою фарбувати, й вибрати мінімум:
\hspace{0.125em plus 0.5em}\ifAfour``\fi\begin{ttfamily}%
2\nolinebreak[4]\hspace{0.125em plus 0.5em}*\nolinebreak[4]\hspace{0.125em plus 0.5em}min(%
a*x*y\nolinebreak[3]\hspace{0.125em plus 0.125em}+\nolinebreak[3]\hspace{0.125em plus 0.125em}b*x*z\nolinebreak[3]\hspace{0.125em plus 0.125em}+\nolinebreak[3]\hspace{0.125em plus 0.125em}c*y*z,\linebreak[1]\hspace{0.25em plus 0.5em}
a*x*y\nolinebreak[3]\hspace{0.125em plus 0.125em}+\nolinebreak[3]\hspace{0.125em plus 0.125em}c*x*z\nolinebreak[3]\hspace{0.125em plus 0.125em}+\nolinebreak[3]\hspace{0.125em plus 0.125em}b*y*z,\linebreak[1]\hspace{0.25em plus 0.5em}
b*x*y\nolinebreak[3]\hspace{0.125em plus 0.125em}+\nolinebreak[3]\hspace{0.125em plus 0.125em}a*x*z\nolinebreak[3]\hspace{0.125em plus 0.125em}+\nolinebreak[3]\hspace{0.125em plus 0.125em}c*y*z,\linebreak[1]\hspace{0.25em plus 0.5em}
b*x*y\nolinebreak[3]\hspace{0.125em plus 0.125em}+\nolinebreak[3]\hspace{0.125em plus 0.125em}c*x*z\nolinebreak[3]\hspace{0.125em plus 0.125em}+\nolinebreak[3]\hspace{0.125em plus 0.125em}a*y*z,\linebreak[1]\hspace{0.25em plus 0.5em}
c*x*y\nolinebreak[3]\hspace{0.125em plus 0.125em}+\nolinebreak[3]\hspace{0.125em plus 0.125em}a*x*z\nolinebreak[3]\hspace{0.125em plus 0.125em}+\nolinebreak[3]\hspace{0.125em plus 0.125em}b*y*z,\linebreak[1]\hspace{0.25em plus 0.5em}
c*x*y\nolinebreak[3]\hspace{0.125em plus 0.125em}+\nolinebreak[3]\hspace{0.125em plus 0.125em}b*x*z\nolinebreak[3]\hspace{0.125em plus 0.125em}+\nolinebreak[3]\hspace{0.125em plus 0.125em}a*y*z)\end{ttfamily}\ifAfour''\fi.


% % % Його суть\nolinebreak[3] --- ми не~знаємо, яка з площ мінімальна, середня та максимальна, але можемо перебрати всі ${3!\,=\,6}$ варіантів, які грані якою фарбою фарбувати, й вибрати з них мінімум. (Тобто, цей підхід ігнорує міркування першого абзацу цього розбору.)

% % % Ще один спосіб розв'язання задачі такий: 
% % % \begin{ttfamily}
% % % 2*min(%
% % % a*x*y\nolinebreak[3]\hspace{0.125em plus 0.125em}+\nolinebreak[3]\hspace{0.125em plus 0.125em}b*x*z\nolinebreak[3]\hspace{0.125em plus 0.125em}+\nolinebreak[3]\hspace{0.125em plus 0.125em}c*y*z,\linebreak[1]\hspace{0.25em plus 0.5em}
% % % a*x*y\nolinebreak[3]\hspace{0.125em plus 0.125em}+\nolinebreak[3]\hspace{0.125em plus 0.125em}c*x*z\nolinebreak[3]\hspace{0.125em plus 0.125em}+\nolinebreak[3]\hspace{0.125em plus 0.125em}b*y*z,\linebreak[1]\hspace{0.25em plus 0.5em}
% % % b*x*y\nolinebreak[3]\hspace{0.125em plus 0.125em}+\nolinebreak[3]\hspace{0.125em plus 0.125em}a*x*z\nolinebreak[3]\hspace{0.125em plus 0.125em}+\nolinebreak[3]\hspace{0.125em plus 0.125em}c*y*z,\linebreak[1]\hspace{0.25em plus 0.5em}
% % % b*x*y\nolinebreak[3]\hspace{0.125em plus 0.125em}+\nolinebreak[3]\hspace{0.125em plus 0.125em}c*x*z\nolinebreak[3]\hspace{0.125em plus 0.125em}+\nolinebreak[3]\hspace{0.125em plus 0.125em}a*y*z,\linebreak[1]\hspace{0.25em plus 0.5em}
% % % c*x*y\nolinebreak[3]\hspace{0.125em plus 0.125em}+\nolinebreak[3]\hspace{0.125em plus 0.125em}a*x*z\nolinebreak[3]\hspace{0.125em plus 0.125em}+\nolinebreak[3]\hspace{0.125em plus 0.125em}b*y*z,\linebreak[1]\hspace{0.25em plus 0.5em}
% % % c*x*y\nolinebreak[3]\hspace{0.125em plus 0.125em}+\nolinebreak[3]\hspace{0.125em plus 0.125em}b*x*z\nolinebreak[3]\hspace{0.125em plus 0.125em}+\nolinebreak[3]\hspace{0.125em plus 0.125em}a*y*z)\end{ttfamily}.
% % % Він, ігноруючи міркування першого абзацу цього розбору, не~шукає, яка з площ мінімальна, середня та максимальна, а просто перебирає всі ${3!\,=\,6}$ варіантів, які грані якою фарбою фарбувати, й вибирає з них мінімум.

% % % % % % Його суть\nolinebreak[3] --- ми не~знаємо, яка з площ мінімальна, середня та максимальна, але можемо перебрати всі ${3!\,=\,6}$ варіантів, які грані якою фарбою фарбувати, й вибрати з них мінімум. (Тобто, цей підхід ігнорує міркування першого абзацу цього розбору.)


% \noindent
(Само собою, зараховуються не~лише ці вирази, а будь-які правильні.)