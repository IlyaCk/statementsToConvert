\Tutorial	
З юліанським (старим) календарем усе дуже просто: \texttt{if\nolinebreak[3] \verb"year mod 4 = 0" then writeln('YES') else writeln('NO')}.
\ifAfour\else\par\fi
З григоріанським (новим) трохи складніше, але потребує лише акуратності, без\nolinebreak[3] придумок. Буквально слідуючи умові задачі, можна отримати таке: \begin{ttfamily}{if (year~mod~4~=~0)\hspace{0pt plus 1mm} and\hspace{0pt plus 1mm} ((year~mod~100~<>~0)\nolinebreak[2] or~(year~mod~400~=~0)) then writeln('YES') else writeln('NO')}\end{ttfamily}. Інший більш-менш зручний спосіб\nolinebreak[3] --- спочатку, якщо рік взагалі не~кратний~4, то не~високосний; інакше, якщо кратний~400, то високосний; інакше, якщо кратний~100, то не~високосний (випадок кратності~400 вже розглянутий в~іншій гілці); інакше, високосний (раніше вже розглянуто і~взагалі не~кратні~4, і~особливі випадки, лишаються високосні).

Можна і не\nolinebreak[3] писати 
перевірку високосності за григоріанським 
календарем самому, а\nolinebreak[3] використати бібліотечну. Наприклад, мовою \mbox{Free}\discretionary{}{}{}\mbox{Pascal} є функція\nolinebreak[3] \verb"IsLeapYear"; мовою Java --- метод \verb"isLeapYear" класу \verb"GregorianCalendar".

Чи завжди можна використовувати на олімпіаді бібліотечні засоби замість писати сам(ому/ій)? Якщо бібліотека \underline{\emph{не}}~стандартна\nolinebreak[3] --- мабуть, ні (додавати бібліотеки на сервер на\nolinebreak[3] прохання учасника якось не\nolinebreak[3] прийнято). Крім того, на~локальному комп'ютері, де пише учасник, і на\nolinebreak[3] сервері, де\nolinebreak[3] відбувається перевірка, можливі відмінності у переліку бібліотек. Тут лишається тільки або знати конкретні факти (що~в~яких версіях~є), або питати в журі, або пробувати (якщо дозволена багатократна відправка розв'язків, можна витратити 1--2 спроби, щоб узнати, чи\nolinebreak[3] працює на сервері бажана функція бажаної бібіліотеки: ніщо не~заважає, навіть не~пишучи розв'язок повністю, швиденько написати і здати щось, що при наявності бажаної бібліотеки/функції пройде хоча~б пару тестів, а~при відсутності дасть вердикт <<Помилка компіляції>>, і\nolinebreak[3] глянути, що виходить).
