\Tutorial
Щодо позначення кількості офіцерів, див. \mbox{1-й} абзац розбору задачі~\ref{problem:201718-oioi-A-patrol-group-1}.
% % % На\nolinebreak[3] жаль, в умові була допущена помилка: кількість командирів (офіцерів) у одному місці називається~\verb"c", в\nolinebreak[3] інших~\verb"k".
% % % На\nolinebreak[3] щастя, цю помилку було досить швидко помічено, причому її вдалося швидко нівелювати шляхом того, щоб вважати правильними і\nolinebreak[3] розв'язки, де\nolinebreak[3] ця\nolinebreak[3] змінна називається~\verb"k", і\nolinebreak[3] розв'язки, де\nolinebreak[3] вона називається~\verb"c". Це~саме стосується й попередньої задачі.

Знаючи комбінаторику, легко побачити, що відповіддю задачі є $k\cdot{}C_r^2$, що можна перетворити до, наприклад, вигляду \verb"k*r*(r-1)//2". 

Пояснимо, звідки це береться.
З~прикладу видно, що будь-хто з офіцерів
\mbox{К1},\nolinebreak[3] \mbox{К2},\nolinebreak[3] \mbox{К3}
може командувати будь-якою групою рядових 
(\mbox{(Р1, Р2)},\nolinebreak[3] \mbox{(Р1, Р3)},\nolinebreak[3] \mbox{(Р2, Р3)}).
Навіть не~знаючи, що це і є комбінаторне правило добутку, очевидно, що раз будь-якого офіцера можна призначити на будь-яку групу рядових, то треба помножити кількість офіцерів~\texttt{k} на кількість можливих груп рядових. З~цією кількістю трохи складніше, але\nolinebreak[3] не~набагато. Кожного з\nolinebreak[3] \texttt{r}\nolinebreak[3] рядових можна поєднувати в групу з будь-яким з \emph{решти} \mbox{\texttt{r-1}} рядових (ну,\nolinebreak[3] не~поєднувати~ж з\nolinebreak[3] самим\nolinebreak[3] собою\dots). Звідси \texttt{r*(r-1)}. Але цей добуток рахує пари, де\nolinebreak[2] рядові переставлені місцями (\mbox{як-то}\nolinebreak[3] \mbox{(Р1, Р2)}\nolinebreak[2] та\nolinebreak[3] \mbox{(Р2, Р1)}) як різні, й це суперечить пр\'{и}кладу. Значить, треба компенсувати те, що кожна пара врахована двічі, поділивши\nolinebreak[3] на~2.