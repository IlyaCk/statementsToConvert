\begin{problemAllDefault}{Три круги}

Є три круги радіусами $R_1$, $R_2$ та $R_3$.

Чи можна перекласти їх так, щоб відразу два менші круги лежали один поруч з іншим на найбільшому, не~накладаючись один на одного і не~звисаючи за його межі? Торкатися один одного менші круги можуть.

\InputFile  Програма повинна прочитати три цілі числа $R_1$, $R_2$ та $R_3$, в один рядок, через пропуски (пробіли). Усі три значення $R_1$, $R_2$ та $R_3$ є цілими числами в межах від 1 до~1000.

\OutputFile Якщо перекласти круги вказаним чином неможливо, програма повинна вивести єдине слово ``\texttt{NO}'' (без лапок).

Якщо можливо, то в єдиному рядку повинно бути записано ``\texttt{YES, the }\dots{}\texttt{ disk is the maximal}'' (без\nolinebreak[3] лапок), де\nolinebreak[3] замість ``\dots'' повинно бути одне з трьох значень:

\begin{itemize}[leftmargin=*,itemsep=0pt,partopsep=0pt,topsep=0pt,parsep=0pt]
\item
``\texttt{1st}'' (без\nolinebreak[3] лапок), якщо 2-й і 3-й круги можна покласти поверх 1-го;
\item
``\texttt{2nd}'' (без\nolinebreak[3] лапок), якщо 1-й і 3-й круги можна покласти поверх 2-го;
\item
``\texttt{3rd}'' (без\nolinebreak[3] лапок), якщо 1-й і 2-й круги можна покласти поверх 3-го.
\end{itemize}

Зверніть увагу: фраза повинна бути однаковою з правильною байт-у-байт, тобто всі великі чи маленькі літери, всі пропуски (пробіли) та інші подібні дрібниці важливі.

\ifAfour\else
\vspace{-0.875\baselineskip plus 0.25 ex}
\fi

\Examples
\begin{exampleSimple}{5em}{16em}%
\exmp{1 2 3}{YES, the 3rd disk is the maximal}%
\exmp{2 3 4}{NO}%
\exmp{9 3 1}{YES, the 1st disk is the maximal}%
\end{exampleSimple}


\end{problemAllDefault}
