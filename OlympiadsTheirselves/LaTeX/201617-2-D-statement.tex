\begin{problemAllDefault}{Кратний куб}

Напишіть програму, яка для заданого натурального числа~$k$ знаходитиме, куб якого найменшого натурального числа кратний цьому~$k$. 

\InputFile
Єдине число~$k$.

\OutputFile
Виведіть єдине число\nolinebreak[3] --- мінімальне натуральне (ціле строго додатне), куб (третя степінь) якого ділиться націло (без\nolinebreak[2] остачі)\nolinebreak[2] на~$k$.

\Example
\noindent\begin{exampleSimple}{5em}{5em}
\exmp{12}{6}\end{exampleSimple}

\Note
Ні~$1^3{=}1$, 
ні~$2^3{=}8$, 
ні~$3^3{=}27$, 
ні~$4^3{=}64$, 
ні~$5^3{=}125$ не~діляться на~12\nolinebreak[2] націло, а\nolinebreak[3] $6^3{=}216$ ділиться. Тобто, $6^3$ є найменшим натуральним кубом, кратним~12.


\Scoring
20\%~балів припадає на\nolinebreak[3] тести, в\nolinebreak[3] яких $2\dib{{\<}}k\dib{{\<}}100$; 
ще\nolinebreak[3] 20\%~балів\nolinebreak[3] --- на $10^5\dib{{<}}k\dib{{<}}10^6$; 
ще\nolinebreak[3] 20\%~балів\nolinebreak[3] --- на $10^7\dib{{<}}k\dib{{<}}10^8$; 
ще\nolinebreak[3] 20\%~балів\nolinebreak[3] --- на $10^{10}\dib{{<}}k\dib{{<}}10^{12}$;
решта\nolinebreak[3] 20\%~балів\nolinebreak[3] --- на $10^{15}\dib{{<}}k\dib{{<}}10^{18}$.

Писати треба одну програму, а не різні програми для різних випадків; єдина мета цього переліку різних блоків обмежень\nolinebreak[3] --- дати уявлення про те, скільки балів можна отримати, якщо розв’язати задачу правильно, але\nolinebreak[2] не\nolinebreak[3] ефективно.

\end{problemAllDefault}
