\begin{problemAllDefault}{Ракета}

В далекому космосі, де можна нехтувати гравітаційними впливами та тертям, запускають ракету. 
Розмірами ракети теж можна знехтувати, вважаючи її матеріальною точкою.
До~запуску, ракета нерухома; ракета весь час летить прямолінійно, перші $d$ секунд рівноприскорено з прискоренням~$a$~\raisebox{2pt}{м}$\!/\!$\raisebox{-2pt}{с$^2$}, далі рівномірно з тією швидкістю, яку встигла набрати за\nolinebreak[3] ці\nolinebreak[3] $d$~с.

Слід спиратися на такі відомі з фізики факти: поки ракета летить рівноприскорено (час~$t$, що вимірюється від її старту, в~межах ${0\,{\<}\,t\,{\<}\,d}$), її відстань від старту може бути виражена як $\frac{at^2}{2}$, а~швидкість\nolinebreak[3] --- як~$at$;
коли закінчується рівноприскорений та починається рівномірний рух ракети, її швидкість незмінна й дорівнює~$ad$, тож за кожну секунду до пройденої відстані додається\nolinebreak[3] ще~$ad$.
Застосовувати інші фізичні моделі (як-то теорію відносності) не~треба й не~можна.

Напишіть програму, яка рахуватиме, скільки часу знадобилося цій ракеті (яку вважаємо матеріальною точкою), щоб пройти фрагмент свого маршрута, починаючи з відстані $b$~м від старту до відстані $c$~м від старту.

\InputFile
Чотири цілі числ\'{а} $a$~$b$~$c$~$d$ в один рядок через одинарні пропуски (пробіли);
$a$~вимірюється у~\raisebox{2pt}{м}$\!/\!$\raisebox{-2pt}{с$^2$} і перебуває в межах $1\dib{{\<}}a\dib{{\<}}100$;
$b$,~$c$~вимірюються у~метрах і перебувають у межах $0\dib{{\<}}b\dib{{<}}c\dib{{\<}}10^9$;
$d$~вимірюється у~cекундах і перебуває в межах $1\dib{{\<}}d\dib{{\<}}1000$.

\OutputFile
Виведіть єдине число\nolinebreak[3] --- проміжок часу між моментами, коли ракета пролетіла через вказані точки. 

\Scoring
Кожен тест оцінюється окремо.
Якщо відповідь не~є числом (зокрема, містить у собі ще\nolinebreak[3] щось, крім\nolinebreak[2] числ\'{а}), цей тест оцінюється\nolinebreak[2] на~0. \begin{itshape}\underline{Не~виводьте} \underline{ніяку} <<допоміжну>> інформацію, як-то ``\texttt{Raketa projde promivok za}'' чи якось аналогічно. Така заборона стосується не~лише цієї задачі, а й усіх задач змагання, і взагалі переважної більшості задач, що перевіряються автоматично.\end{itshape}
Крім того, заборонено (конкретно у цій задачі), щоб увесь виведений результат містив більше 30~символів.

Запис числ\'{а} може бути хоч суто десятковим, хоч експоненційним; якщо відповідь виявляється цілою, її можна хоч виводити як цілу, хоч дописувати ``\texttt{.000000}'' (з~довільною, але з урахуванням вищезгаданого обмеження, кількістю нулів).
Якщо відповідь на деякий тест правильна з~відносною похибкою до~$10^{-15}$, за тест ставиться повний бал \nolinebreak[3] (10~з~10).
Інакше, якщо відповідь правильна з~відносною похибкою до~$10^{-9}$, за тест ставиться 60$\,$\% балів \nolinebreak[3] (6~з~10).
Інакше, якщо відповідь правильна з~відносною похибкою до~$10^{-3}$, за тест ставиться 30$\,$\% балів \nolinebreak[3] (3~з~10).
Якщо похибка ще~більша, тест оцінюється на~0.

% \myflfigaw{\hspace*{-1em}\begin{tabular}{@{}c@{}}
% \Examples\\
\Examples
\begin{exampleSimple}{14.5em}{8.5em}
\exmp{2 4 9 12}{1}%
\exmp{100 999999999 1000000000 1000}{1.0e-005}%
\exmp{9 2000 3000 20}{5.555555555555556}%
\exmp{1 55 777 12}{60.26191151829848}%
\end{exampleSimple}
% \end{tabular}\hspace*{-1em}}

\Notes
(1)$\,\,$У\nolinebreak[2] першому тесті, через 2~с після старту ракета перебуває якраз за $\frac{2\cdot2^2}{2}\,{=}\,4\,\raisebox{0pt}{(м)}$ від старту, а через 3~с\nolinebreak[3] --- якраз за $\frac{2\cdot3^2}{2}\,{=}\,9\,\raisebox{0pt}{(м)}$. Тому проміжок між цими моментами становить~1~с.\hspace{0.5em plus 0.5em} 
У~др\'{у}гому тесті маємо проміжок, де ракета рухається рівномірно зі швидкістю ${100\,\raisebox{2pt}{м}\!/\!\raisebox{-2pt}{с$^2$}\,}\dibbb{{\times}}{1000\,\raisebox{0pt}{с}\,}\dibbb{{=}}{10^5\,\raisebox{2pt}{м}\!/\!\raisebox{-2pt}{с}}$, тож відстань $1000000000\dibbb{{-}}999999999\dibbb{{=}}1\,\raisebox{0pt}{(м)}$ долатиметься за $10^{-5}\,\raisebox{0pt}{(с)}$.

(2)$\,\,$Смисл поняття <<відносна похибка>> можна уточнити у наступній задачі. І~взагалі, перевірка цієї задачі відбувається в~точності за правилами, описаними у наступній задачі, за єдиним виключенням: там гарантовано, що відповідь учасника являє собою один рядок, а~тут, якщо Ваша програма виводитиме не~один рядок, за тест ставитиметься 0~балів.

\end{problemAllDefault}

