\begin{problemAllDefault}{Цифрові ріки}

Цифрова ріка\nolinebreak[3] --- це послідовність чисел, де\nolinebreak[2] число, що\nolinebreak[2] слідує за\nolinebreak[2] числом\nolinebreak[3] $n$, це $n$ плюс сума його цифр. Наприклад, якщо число ${n{=}12345}$, то за ним буде йти $12345\dib{{+}}{(1{+}2{+}3{+}4{+}5)}\dib{{=}}12360$\nolinebreak[1] і~т.~д. Якщо перше число цифрової річки $N$, ми будемо називати її <<\begin{bfseries}{річка~\begin{itshape}{N}\end{itshape}}\end{bfseries}>>.

Для прикладу, \textbf{річка\nolinebreak[3] 480}\nolinebreak[3] --- це послідовність чисел, яка починається з чисел 
480, 492, 507, 519,~\dots, а\nolinebreak[3] \textbf{річка\nolinebreak[3] 483}\nolinebreak[3] --- послідовність, що починається з 483, 498, 519,~\dots

Напишіть програму, яка приймає на вхід два цілих значення $k$ ($1\dib{{\<}}k\dib{{\<}}16384$) та  $N$ ($1\dib{{\<}}N\dib{{\<}}10000$), і\nolinebreak[3] виводить \mbox{$k$-те}\nolinebreak[1] число річки~$N$.\nopagebreak[3]

\Example
\begin{exampleSimple}{3em}{3em}%
\exmp{4 480}{519}%
\end{exampleSimple}

\end{problemAllDefault}
\pagebreak[2]