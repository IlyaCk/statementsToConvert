\begin{problem}{Відстань між мінімумом та максимумом--1}{\stdinOrInputTxt}{\stdoutOrOutputTxt}{2 сек}{64 мегабайти}

Напишіть програму, яка за заданим масивом цілих чисел знайде місце мінімального її елемента, максимального її елемента, та відстань (кількість проміжних елементів) між ними. У~випадку, якщо масив містить однакові мінімальні та\slash{}або однакові максимальні елементи на різних позиціях, слід вибрати перший з мінімумів та останній з максимумів.

\InputFile
Перший рядок містить єдине число~$N$ ($2\dib{{\<}}N\dib{{\<}}100$)\nolinebreak[3] --- кількість елементів масиву. Др\'{у}гий рядок містить (розділені пробілами) ч\'{и}сла--елементи масиву; значення цих чисел--елементів цілі, від~1 до~100.

\OutputFile
Виведіть єдине ціле число\nolinebreak[3] --- відстань (кількість проміжних елементів) між першим з мінімальних і останнім з максимальних елементів.

{

\Examples

\vspace{-1.125\baselineskip}

\makeTableLongtrue
\noindent\begin{exampleThreeWithSpecNameColTwoAndLineStretch}{7.5em}{5em}{\ifAfour 11cm\else 10cm\fi}{Результати}{Примітки}{\ifAfour -2pt\else -5pt\fi}{-2pt}{0.625}
\exmp{8
3 9 7 5 2 8 6 4}{2}{\noindent\ifAfour\begin{minipage}[t]{11cm}\else\begin{minipage}[t]{10cm}\fi
\setstretch{0.875}
Між елементами 2 та 9 два проміжні елементи 5 та~7
\end{minipage}}%
\exmp{8
4 3 2 1 8 7 6 5}{0}{\noindent\ifAfour\begin{minipage}[t]{11cm}\else\begin{minipage}[t]{10cm}\fi
\setstretch{0.875}
Між елементами 1 та 8 нема проміжних елементів
\end{minipage}}%
\exmp{8
3 1 4 1 5 9 2 6}{3}{\noindent\ifAfour\begin{minipage}[t]{11cm}\else\begin{minipage}[t]{10cm}\fi
\setstretch{0.875}
Між першим з елементів~1 та єдиним (і~тому останнім) елементом~9 три проміжні елементи 4,~1,~5
\end{minipage}}%
\exmp{8
1 1 1 1 1 1 1 1}{6}{\noindent\ifAfour\begin{minipage}[t]{11cm}\else\begin{minipage}[t]{10cm}\fi
\setstretch{0.875}
Оскільки всі елементи однакові, кожен з них мінімальний і максимальний; отже, першим мінімальним є перший з усіх елементів масиву, останнім максимальним\nolinebreak[3] --- останній з усіх
\end{minipage}}%
\end{exampleThreeWithSpecNameColTwoAndLineStretch}

\vspace{-0.75\baselineskip}

% % % \noindent\hspace*{-1em}\begin{exampleSimpleThree}{7.5em}{5em}{10cm}{Примітки}
% % % \exmp{8
% % % 3 9 7 5 2 8 6 4}{2}{\noindent\begin{minipage}[t]{10cm}
% % % Між елементами 2 та 9 два проміжні елементи 5 та~7
% % % \end{minipage}}%
% % % \exmp{8
% % % 4 3 2 1 8 7 6 5}{0}{\noindent\begin{minipage}[t]{10cm}
% % % Між елементами 1 та 8 нема проміжних елементів
% % % \end{minipage}}%
% % % \exmp{8
% % % 3 1 4 1 5 9 2 6}{3}{\noindent\begin{minipage}[t]{10cm}
% % % Між першим з елементів~1 та єдиним (і~тому останнім) елементом~9 три проміжні елементи 4,~1,~5
% % % \end{minipage}}%
% % % \exmp{8
% % % 1 1 1 1 1 1 1 1}{6}{\noindent\begin{minipage}[t]{10cm}
% % % Оскільки всі елементи однакові, кожен з них мінімальний і максимальний; отже, першим мінімальним є перший з усіх елементів масиву, останнім максимальним\nolinebreak[3] --- останній з усіх
% % % \end{minipage}}%
% % % \end{exampleSimpleThree}

}

\end{problem}