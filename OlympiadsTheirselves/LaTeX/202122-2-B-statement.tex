\begin{problemAllDefault}{Гра ``WALLCRSH''}

Гра ``WALLCRSH'' 
вимагає
% полягає в тому, що потрiбно 
розбити усi броньованi плити,
% котрими 
якими
обкладена оборонна башта супротивника. Башта має вигляд 
правильного $N$-кутника, i для кожної з $N$ сторiн вiдома кiлькiсть 
броньованих плит, що її прикривають. Для руйнування плит можна використати 
лише спецiальну двоствольну гармату, котра за один пострiл розбиває або двi 
плити на однiй сторонi башти, або по однiй плитi на двох сусiднiх стронах.
Стрiляти по сторонах башти, де вже нема броньованих плит, дозволяється,
але це вважається марнотратством.

Напишiть програму, котра буде знаходити, за яку найменшу кiлькiсть
пострiлiв з~такої гармати можна зруйнувати описану сукупнiсть плит.

% \pagebreak[3]

%\myflfigaw{\begin{minipage}{11.75em}
%\Example\\
%\begin{exampleSimple}{5.5em}{5em}
%\exmp{6\\
%3 2 2 1 1 3}{6}%
%\end{exampleSimple}
%\end{minipage}}

\mytextandpicture{\InputFile
В~першому рядку вводиться число~$N$ ($5\dibbb{{\<}}N\dibbb{{\<}}12345$), 
в~другому, через пробiли "--- $N$ цілих 
% невід'ємних 
чисел (від~0 до~100 кожне) "--- 
кiлькостi плит на вiдповiдних сторонах, в~порядку обходу 
% за годинниковою стрiлкою; 
многокутника; 
остання сторона сусiдня з~першою.}{\begin{minipage}{11.75em}
\Example\\
\begin{exampleSimple}{5.5em}{5em}
\exmp{6\\
3 2 2 1 1 3}{6}%
\end{exampleSimple}
\end{minipage}}

\pagebreak[3]

\OutputFile
Слiд вивести єдине число "--- мiнiмальну кiлькiсть пострiлiв.

\Scoring
Потестове (проходження кожного тесту оцінюється окремо, інші тести на це не~впливають). 
Перший тест є тестом з умови й не~оцінюється (але\nolinebreak[3] перевіряється, а\nolinebreak[3] детальний протокол показується). 

\end{problemAllDefault}