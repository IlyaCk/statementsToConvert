\Tutorial
Спочатку розглянемо способи дістатися з~$A$\nolinebreak[3] до~$C$.
З~наведеного прикладу-переліку видно, що кожен із\nolinebreak[2] рейсів з~$A$\nolinebreak[3] у~$B$ можна \emph{поєднувати} з кожним із\nolinebreak[2] рейсів з~$B$\nolinebreak[3] у~$C$, тому треба \emph{множити}\nolinebreak[2] ${k_1\cdot{}k_2}$; до цих способів слід \emph{додати} $k_3$ безпосередніх рейсів з~$A$\nolinebreak[3] у~$C$, бо\nolinebreak[2] з~$A$\nolinebreak[3] до~$C$ можна добиратися \emph{або} через~$B$, \emph{або} прямим рейсом. Так отримуємо\nolinebreak[2] ${k_1{\cdot}k_2}\dib{{+}}k_3$. Далі, кожен з цих способів можна \emph{поєднувати} з кожним з\nolinebreak[2] рейсів із~$C$\nolinebreak[3] у~$D$. Отже, остаточна відповідь: \texttt{(k1*k2+k3)*k4}. 

%%% (Само собою, зараховується не~лише цей вираз, а будь-який правильний.)