\Tutorial % \MyParagraph{Розв'язок на 75\%.}
Хоч ця та наступна задачі схожі з задачею <<Палички>> зі\nolinebreak[2] стор.~\mbox{\pageref{text:sticks-simple-begin}--\pageref{text:sticks-simple-end}}, вони \emph{значно складніші}. Там аналіз виграшності зводився до дуже простої ознаки <<чи~кількість паличок кратна~4?>>. А~тут простої ознаки не~виходить, і\nolinebreak[2] треба думати, спочатку\nolinebreak[3] --- як пристосувати до цієї задачі загальні правила перевірки виграшності, потім\nolinebreak[3] --- як це суттєво оптимізувати.

Нагадаємо, що позиція виграшна, коли той, кому вона дісталася, або самим цим фактом вже виграв, або має хоча~б один хід у~програшну. 
Іншими словами: 
% Інакше кажучи,
якщо гравець знає, який це хід (або які це ходи), і використає с\'{а}ме його (або один з них), то супернику дістанеться програшна позиція. Аналогічно, позиція програшна, коли той, кому вона дісталася, або самим цим фактом вже програв, або абсолютно всі його ходи ведуть у виграшні позиції. 
Іншими словами: 
% Інакше кажучи,
хоч би як добре гравець не~вмів грати, він не~має ніяких інших ходів, крім як походити в таку позицію, що починаючи з неї суперник може виграти. 

% % % Якщо гравець отримав виграшну позицію, але не~вміє грати й не~розуміє, 
% % % які ходи з неї ведуть до програшних позицій і які до виграшних, 
% % % такий гравець може втратити свою потенційну можливість виграти. 
% % % С\'{а}ме таке <<перехоплення ініціативи>> у~гравця, 
% % % що\nolinebreak[2] с\'{а}ме не~вміє грати, 
% % % і пропонується робити у значній частині тестів.

Кожна позиція або виграшна, або програшна; не~може бути ні третього варіанту, ні поєднання обох цих варіантів для однієї позиції. (Охочі можуть знайти в літературі чи Інтернеті доведення цього твердження. Воно правильне не\nolinebreak[3] для\nolinebreak[3] абсолютно всіх ігор, але достатньо (одночасного) виконання таких умов: гравців двоє; гравці мають повну інформацію про правила гри; гравці мають повну інформацію про поточну позицію гри; гравці вибирають ходи з зарані відомих можливостей, і ці можливості не~залежать ні від чого, крім поточної позиції; гра завершується перемогою одного з гравців і поразкою іншого, нема ні нічиїх, ні числового вираження, <<наскільки програв>>; кількість позицій гри скінч\'{е}нна; гра не~може зациклюватися, багатократно проходячи через ті\nolinebreak[3] с\'{а}мі позиції.)

З усього цього випливає, що позицією цієї гри не~можна вважати % сам\'{у} лише 
кількість паличок. Хоча~б том\'{у}, що в перших двох прикладах наступної задачі показано неможливість виграти, ходячи першим при ${N\,{=}\,3}$, а~в~останньому прикладі наступної задачі суперник забирає всі три палички одним ходом і вигра\'{є}. 
Це\nolinebreak[3] суперечить умові <<гравці вибирають ходи з зарані відомих можливостей, і ці можливості не~залежать \emph{ні від чого, крім поточної позиції}>>. 

\emph{Позицією цієї гри можна і варто вважати пару <<кількість паличок, що лишилися у купці; кількість паличок, забраних на наразі останньому ході>>.} Причому, хоча на початку гри <<наразі останнього ходу>> не~було, дозволені ходи (забирати або одну, або дві палички) такі с\'{а}мі, які були~б, якби щойно забрали одну паличку й лишилося~$N$. Тож початковою є позиція\nolinebreak[3] $(N; 1)$; якщо 
% перший
\mbox{1-й}
гравець забирає одну паличку, то 
% др\'{у}гому гравцеві 
\mbox{2-му}
дістається позиція\nolinebreak[2] $(N{-}1; 1)$, а якщо дві, то позиція\nolinebreak[3] $(N{-}2; 2)$. І~так далі.

\ifAfour
\begin{figure}[!b]
\vspace*{-1.25\baselineskip}
\noindent\hrulefill
\vspace{-0.75\baselineskip}
\par\noindent\hrulefill
\vspace{-0.75\baselineskip}
\par\noindent\hrulefill
\vspace{-0.25\baselineskip}
\else
\begin{figure}[p]
\fi

\def\phraseI{Зобразимо всі позиції для початкових кількостей паличок 5 (верхня діаграма) та 12 (нижня).}
\def\phraseII{Позначимо позиції круж\'{е}чками й розмістимо так, щоб 
стовпчики від\-по\-ві\-да\-ли 
кількості паличок, що лишилися, 
рядк\'{и}\nolinebreak[3] --- кількості паличок, забраних 
на останньому ході.
% останнім ходом.
}
\def\phraseIII{Літери у кру\-ж\'{е}ч\-ках позначають, виграшна (\textbf{W}in) позиція чи програшна (\textbf{L}ose).}
\def\phraseIV{\mbox{Жирні} стр\'{і}лки позначають переходи, які ведуть до програшних позицій і тим забезпечують виграшність; 
пунк\-тирні\nolinebreak[3] --- інші переходи.}
\def\phraseV{Верхня діаграма є частиною (лівою нижньою) нижньої; 
це\nolinebreak[2] природньо, бо раз подальші ходи з позиції 
не~залежать 
від того, як потрапили в цю позицію,
то не~залежить і виграшність\nolinebreak\hspace{0pt plus 1pt}/\nolinebreak[2]\hspace{0pt plus 1pt}про\-граш\-ність.}

\ifglobalsf
\begin{sffamily}
\fi

\ifAfour
\ifBigStretch
\errorNotCalculatedYet
\begin{Huge}верстка!!!\end{Huge}
\input sticks-with-spec-moves-mfpicture-5
\input sticks-with-spec-moves-mfpicture-12
\phraseI
\phraseII
\phraseIII
\phraseIV
\phraseV
\else
\parshape=15
0.075\textwidth 0.395\textwidth
0.00\textwidth 0.47\textwidth
0.00\textwidth 0.47\textwidth
0.00\textwidth 0.47\textwidth
0.12\textwidth 0.35\textwidth
0.22\textwidth 0.25\textwidth
0.28\textwidth 0.19\textwidth
0.35\textwidth 0.65\textwidth
0.40\textwidth 0.60\textwidth
0.45\textwidth 0.55\textwidth
0.50\textwidth 0.50\textwidth
0.55\textwidth 0.45\textwidth
0.60\textwidth 0.40\textwidth
0.65\textwidth 0.35\textwidth
0.72\textwidth 0.28\textwidth
\phraseI{}
\phraseII{}
\phraseIII{}
\phraseIV{}
\phraseV{}
\par
\vspace{-17\baselineskip}
\par\noindent\hspace*{0.49\textwidth}\input sticks-with-spec-moves-mfpicture-5
\par
\vspace{-2\baselineskip}
\noindent
\input sticks-with-spec-moves-mfpicture-12
\fi
\else
\ifBigStretch
\parshape=7
0em \textwidth
0.21\textwidth 0.79\textwidth
0.43\textwidth 0.57\textwidth
0.47\textwidth 0.53\textwidth
0.52\textwidth 0.48\textwidth
0.59\textwidth 0.41\textwidth
0.71\textwidth 0.29\textwidth
\phraseI{}
\phraseII{}
\phraseIII{}
\par
\vspace{-7\baselineskip}
\noindent
\input sticks-with-spec-moves-mfpicture-5
\parshape=10
0.23\textwidth 0.77\textwidth
0.33\textwidth 0.67\textwidth
0.43\textwidth 0.57\textwidth
0.51\textwidth 0.49\textwidth
0.55\textwidth 0.45\textwidth
0.59\textwidth 0.41\textwidth
0.69\textwidth 0.31\textwidth
0.75\textwidth 0.25\textwidth
0.80\textwidth 0.20\textwidth
0.86\textwidth 0.14\textwidth
\phraseIV{}
\phraseV{}
\par
\vspace{-10\baselineskip}
\noindent
\input sticks-with-spec-moves-mfpicture-12
\else
\errorNotCalculatedYet
\begin{Huge}верстка!!!\end{Huge}
\input sticks-with-spec-moves-mfpicture-5
\input sticks-with-spec-moves-mfpicture-12
\phraseI{}
\phraseII{}
\phraseIII{}
\phraseIV{}
\phraseV{}
\fi
\fi

\ifAfour\vspace{0.25ex}\par\hspace*{0.075\textwidth}\fi%
Далі зображено самі позначки W/L (без стр\'{і}\-лок) для ще більшої кількості позицій,
та вказано мінімальні~$s$, починаючи з яких позиції стають виграшними.

\phantomsection
\begin{small}
\noindent%
\ifAfour
\input sticks-with-spec-moves-tabular-36
\else
\input sticks-with-spec-moves-tabular-30
\fi
\end{small}

\label{fig:stick-with-spec-moves-positions}


\ifglobalsf
\end{sffamily}
\fi
\end{figure}

На перший погляд, висновок <<позицією слід вважати пару~\dots>> украй погано поєднується з обмеженнями задачі: якщо $N\dib{{\approx}}1234567$, а позицією є пара, то кількість позицій перевищує $10^{12}/{4}$, що\nolinebreak[2] % в\nolinebreak[3] умовах олімпіади 
не\nolinebreak[3] вкладеться ні\nolinebreak[3] в\nolinebreak[3] обмеження пам'яті, ні\nolinebreak[3] в\nolinebreak[3] обмеження часу.
%
Але дослідити, що вийде, якщо запрограмувати стандартний аналіз виграшності/програшності для позицій-пар, % все\nolinebreak[3] одно 
варто. 
Хоча~б тому, що така с\'{а}ма гра розглядається й у наступній задачі, а там обмеження % значно 
менші. Але варто було~б, навіть якби наступної задачі взагалі не~було. 
Накрайняк, навіть неефективну правильну програму можна здати, щоб отримати бали хоча~б за деякі з блоків. 
% % % (\ERROR \begin{color}{red}\begin{Huge}{link to ready solution here}\end{Huge}\end{color}) 
Але є й важливіша причина:
буває, що ручний аналіз результатів, згенерованих менш ефективним розв'язком, дозволяє помітити деякі особливості, на основі яких можна придумати більш ефективний розв'язок. Так що дивимося на сукупності виграшних/програшних позицій, зображені\label{text:link-to-fig-stick-with-spec-moves-positions} 
\ifAfour
\ifnum\getpagerefnumber{text:link-to-fig-stick-with-spec-moves-positions}=\getpagerefnumber{fig:stick-with-spec-moves-positions}
внизу цієї сторінки.
\else
внизу стор.~\pageref{fig:stick-with-spec-moves-positions}.
\fi
\else
на стор.~\pageref{fig:stick-with-spec-moves-positions}.
\fi

Легко помітити, що позиції\nolinebreak[3] $(0;s)$ програшні при\nolinebreak[2] всіх~$s$, бо ${n\,{=}\,0}$ означає, що позиція дістається % гравцеві 
після того, як суперник вже\nolinebreak[3] забрав останню паличку і виграв. Також легко помітити, що у\nolinebreak[3] випадку\nolinebreak[3] ${n\,{>}\,0}$ всі позиції $(n;s)$ при\nolinebreak[3] $s\dib{{\>}}\frac{n}{2}$ виграшні (бо\nolinebreak[3] при\nolinebreak[3] $s\dib{{\>}}\frac{n}{2}$ можна забрати всі палички за один хід% і тим виграти
)

А ще можна помітити таке: \begin{itshape}якщо деяка позиція $(n^{\star};s^{\star})$ виграшна, то виграшні й \underline{усі} позиції $(n^{\star};s)$ при тому самому~$n^{\star}$ та $s\dib{{\>}}s^{\star}$\end{itshape} (на\nolinebreak[2] діаграмах і в таблиці\nolinebreak[3] --- кожен стовпчик (крім\nolinebreak[3] ${n\,{=}\,0}$) або складається з сам\'{и}х лише~``W'', або має внизу скінч\'{е}нну кількість~``L'', а\nolinebreak[3] вище сам\'{і} лише~``W'').

Доведемо, що це\nolinebreak[2] не\nolinebreak[3] збіг обставин для с\'{а}ме цих позицій, а так\nolinebreak[2] і\nolinebreak[2] є для\nolinebreak[3] всіх~${n\,{>}\,0}$. Куди можна піти з\nolinebreak[3] позиції\nolinebreak[3] $(n;1)$? Лише в\nolinebreak[3] ${(n\,{-}\,1;\,1)}$ (при\nolinebreak[2] всіх\nolinebreak[2] ${n\,{>}\,0}$) та\nolinebreak[2] в\nolinebreak[3] ${(n\,{-}\,2;\,2)}$ (при\nolinebreak[2] ${n\,{\>}\,2}$). А~куди з\nolinebreak[3] $(n;2)$? 
При\nolinebreak[2] ${n\,{\<}\,2}$, % в~точності 
в~ті\nolinebreak[3] с\'{а}мі позиції;
при\nolinebreak[2] ${n\,{\>}\,3}$ до них додається\nolinebreak[3] ${(n\,{-}\,3;\,3)}$,
а\nolinebreak[3] при\nolinebreak[2] ${n\,{\>}\,4}$ також\nolinebreak[3] ${(n\,{-}\,4;\,4)}$.
Порівнюючи ходи\nolinebreak[2] з\nolinebreak[3] $(n;2)$ та з\nolinebreak[3] $(n;3)$, знов бачимо, що всі ходи з\nolinebreak[3] $(n;2)$ лишаються допустимими, і до них додаються (якщо $n$ досить велике) нові ${(n\,{-}\,5;\,5)}$ та\nolinebreak[3] ${(n\,{-}\,6;\,6)}$. І\nolinebreak[3] так для\nolinebreak[2] всіх збільшень\nolinebreak[3] $s$ при сталому~$n$: всі старі ходи лишаються, і\nolinebreak[2] до\nolinebreak[2] них, можливо, додаються нові. 
% % % Значить, якщо при меншому~$s$ є хід у\nolinebreak[2] деяку програшну позицію 
% % % (оскільки в\nolinebreak[3] цій\nolinebreak[2] грі нема тупикових виграшних позицій, 
% % % це\nolinebreak[2] рівносильно <<позиція при меншому~$s$ виграшна>>), 
% % % він лишається і при всіх більших~$s$, отже ті позиції теж виграшні.
В~цій\nolinebreak[2] грі, позиція може бути виграшною лише за рахунок того,
що з~неї існує хід у\nolinebreak[2] деяку програшну позицію.
Отже, раз цей самий хід можливий і при всіх більших~$s$, всі позиції з більшим~$s$ при тому самому~$n$ теж виграшні.

\myflfigaw{\phantomsection\label{code:201819-2-D-sticks-with-spec-moves-calc-minWinS}\ifAfour\hspace*{-1mm}\begin{minipage}{18em}\else\begin{minipage}{17.25em}\fi\begin{small}\renewcommand{\baselinestretch}{0.875}\begin{alltt}minWinS[0]:=round(2e9); // $+\infty$\\
minWinS[1]:=1;\\
for i:=2 to N do\\
\tabbb{}for s:=1 to (i+1) div 2 do\\
\tabbb\tabbb{}if (minWinS[i-(2*s-1)]$\,$>$\,$2*s-1) or\\
\tabbb\tabbb\tabbb\tabbb{}(minWinS[i-2*s] > 2*s) then\\
\tabbb\tabbb{}begin // при цьому s вперше \\
\tabbb\tabbb\tabbb{}// знайшли хід у програшну\\
\tabbb\tabbb\tabbb{}minWinS[i] := s; // позицію\\
\tabbb\tabbb\tabbb{}break // обриваємо внутрішній\\
\tabbb\tabbb\tabbb{}// цикл for s\begin{tiny}{...}\end{tiny}\\
\tabbb\tabbb{}end // зовнішній цикл for i\begin{tiny}{...}\end{tiny}\\
\tabbb\tabbb{}// продовжується\end{alltt}\end{small}\end{minipage}}
% \myflfigaw{\begin{minipage}{17.5em}\begin{small}\renewcommand{\baselinestretch}{0.875}\begin{alltt}minWinS[0]:=round(2e9); // $+\infty$\\
% minWinS[1]:=1;\\
% for i:=2 to N do\\
% \tabbb{}for s:=1 to (i+1) div 2 do\\
% \tabbb\tabbb{}if (minWinS[i-(2*s-1)] > 2*s-1) or\\
% \tabbb\tabbb\tabbb\tabbb{}(minWinS[i-2*s] > 2*s) then\\
% \tabbb\tabbb{}begin // саме при цьому s вперше\\
% \tabbb\tabbb\tabbb{}minWinS[i] := s; // знайшли хід\\
% \tabbb\tabbb\tabbb{}break\tabbb{} // у програшну позицію\\
% \tabbb\tabbb\tabbb{}// обриваємо (лише) внутрішній\\
% \tabbb\tabbb\tabbb{}// цикл for j..., \\
% \tabbb\tabbb\tabbb{}// зовнішній цикл for i... \\
% \tabbb\tabbb{}end // продовжується\end{alltt}\end{small}\end{minipage}}
% % % \myflfigaw{\begin{minipage}{17.5em}\begin{small}\renewcommand{\baselinestretch}{0.875}\begin{alltt}minWinS[0]:=round(2e9); // $+\infty$\\
% % % minWinS[1]:=1;\\
% % % for i:=2 to N do begin\\
% % % \tabbb{}for s:=1 to (i+1) div 2 do begin\\
% % % \tabbb\tabbb{}if (minWinS[i-(2*s-1)] > 2*s-1) or\\
% % % \tabbb\tabbb\tabbb\tabbb{}(minWinS[i-2*s] > 2*s) then\\
% % % \tabbb\tabbb{}begin // саме при цьому s вперше\\
% % % \tabbb\tabbb\tabbb{}minWinS[i] := s; // знайшли хід\\
% % % \tabbb\tabbb\tabbb{}break\tabbb{} // у програшну позицію\\
% % % \tabbb\tabbb{}end\\
% % % \tabbb{}end\\
% % % end\end{alltt}\end{small}\end{minipage}}

Це дає ключ до такого розв'язку. 
Реалізуємо стандартний аналіз
<<позиція виграшна, коли є хоча~б один хід у\nolinebreak[3] програшну>>, але\nolinebreak[2] не~зберігаючи кожну пару\nolinebreak[2] $(n;s)$, а\nolinebreak[3] подаючи інформацію в одновимірному масиві\nolinebreak[2] \texttt{minWinS} з діапазоном індексів \texttt{[0..}$N$\texttt{]} (обидві межі включно), де\nolinebreak[2] \texttt{minWinS[i]=k} означає, що всі позиції $(i;1)$, $(i;2)$,~\dots, ${(i;\,k\,{-}\,1)}$ (якщо вони взагалі існують, тобто ${k\,{>}\,1}$) програшні, а~всі позиції $(i;k)$, ${(i;\,k\,{+}\,1)}$,~\dots{} виграшні.

По~суті, останній рядок\nolinebreak[3] стор.~\pageref{fig:stick-with-spec-moves-positions} і є таким масивом.

Остат\'{о}чну відповідь можна визначити так: 
якщо $\texttt{minWinS[N]}\dib{{=}}1$, то позиція $(N;1)$ виграшна 
(виграє \mbox{1-й}\nolinebreak[3] гравець),
інакше програшна~(\mbox{2-й}).
% % % при \texttt{minWinS[N]>1}, позиція $(N;1)$ програшна 
% % % %%%(\mbox{2-й}\nolinebreak[3] гравець може забезпечити собі виграш), 
% % % %%%(при\nolinebreak[2] правильній грі виграє \mbox{2-й}\nolinebreak[3] гравець),
% % % (виграє \mbox{2-й}\nolinebreak[3] гравець),
% % % інакше виграшна~(\mbox{1-й}).

Визначити асимптотичну складність цього розв'язку важко. Дивтися на верхню межу внутрішнього циклу \texttt{for s:=1 to (i+1) div 2 do}\dots{} і робити висновок про <<завелику для ${N\,{\approx}\,10^6}$ складність~$O(N^2)$>> нема толку, бо хоч цю межу й підібрано з осмисленого міркування <<при\nolinebreak[2] $2{\cdot}\texttt{s}\,{\>}\,\texttt{i}$ вже можна забрати всі\nolinebreak[2] \texttt{i}\nolinebreak[2] паличок одним ходом, тож ще більші~$s$ не~потрібні>>, фактично цей цикл завжди обривається \texttt{break}-ом, і\nolinebreak[3] справжня кількість його ітерацій незрозуміла.
Експериментально, складність всього алгоритму схожа на $\Theta(N\log{}N)$ з малим константним множником. І\nolinebreak[3] цього досить, щоб пройти всі тести всіх блоків.

% % % Є~приблизні 
% % % % (неточні) міркування 
% % % аргументи
% % % на користь того, що в~середньому цей цикл завершується швидко. 
% % % (1)~Якщо при деякому $\texttt{i}\dib{{=}}i^{\star}$ цей цикл робить багато ітерацій, то виходить велике\nolinebreak[2] \texttt{minWinS[$i^{\star}$]}, 
% % % і при 
% % % $\texttt{i}\dib{{=}}{i^{\star}{+}\,1}$ 
% % % та 
% % % $\texttt{i}\dib{{=}}{i^{\star}{+}\,2}$ 
% % % буде лише одна ітерація (${\texttt{s}\,{=}\,1}$), яка знайде 
% % % це велике\nolinebreak[2] \texttt{minWinS[$i^{\star}$]}:
% % % при $\texttt{i}\dib{{=}}{i^{\star}{+}\,1}$, це буде $\texttt{minWinS[}\!\underbrace{\texttt{i}}_{i^{\star}{+}1}\!\!\!-\underbrace{(2{\cdot}s{-}1)}_{1}\texttt{]}\dib{{>}}\underbrace{2{\cdot}s{-}1}_{1}$,
% % % а при $\texttt{i}\dib{{=}}{i^{\star}{+}\,2}$, $\texttt{minWinS[}\!\underbrace{\texttt{i}}_{i^{\star}{+}2}\!\!\!-\underbrace{2{\cdot}s}_{2}\texttt{]}\dib{{>}}\underbrace{2{\cdot}s}_{2}$;
% % % аналогічно, 
% % % при 
% % % ${i^{\star}{+}\,3}$ 
% % % та 
% % % ${i^{\star}{+}\,4}$ 
% % % ітерацій буде щонайбільше дві;
% % % при 
% % % ${i^{\star}{+}\,5}$ 
% % % та 
% % % ${i^{\star}{+}\,6}$\nolinebreak[3] --- щонайбільше три; і~т.~д.\linebreak[2] 
% % % (2)~Якщо при деякому $i^{\star}\dib{{+}}i^{\star\star}$ вкладений цикл завершується не~дуже скоро, то \texttt{minWinS[$i^{\star}\dib{{+}}i^{\star\star}$]} \emph{теж} виявляється засобом обмеження кількості ітерацій вкладеного циклу при $i^{\star}\dib{{+}}i^{\star\star}\dib{{+}}1$, $i^{\star}\dib{{+}}i^{\star\star}\dib{{+}}2$,\nolinebreak[3] \dots, і ці два обмеження (\texttt{minWinS[$i^{\star}$]} та \texttt{minWinS[$i^{\star}\dib{{+}}i^{\star\star}$]}) діють собі одночасно, не~заважаючи одн\'{е} \'{о}д\-но\-му і не~потребуючи ніяких спеціальних дій. А\nolinebreak[3] де\nolinebreak[2] два обмеження, там і три, і більше.


При\nolinebreak[2] бажанні можна побудувати й ще значно ефективніший алгоритм. 
%
Продовживши аналіз таблиці з виграшними та програшними позиціями та/або масиву\nolinebreak[2] \texttt{minWinS}, можна помітити, що особливо великі значення у\nolinebreak[3] \texttt{minWinS} досягаються на індексах, рівних числам Фібоначчі. 
Тож\nolinebreak[2] можна спробувати, що буде, якщо записувати значення~$N$ у т.~зв. <<фібоначчієвій системі числення>> (що\nolinebreak[2] це\nolinebreak[2] таке, зайдіть самостійно), й отримати, що \mbox{2-й}\nolinebreak[3] гравець може забезпечити собі виграш при тих і тільки для тих початкових значеннях~$N$, які у фібоначчієвій системі закінчуються на хоча~б два нулі. Як\nolinebreak[3] це\nolinebreak[2] строго довести і чи\nolinebreak[3] можна це використати також і в наступній задачі (де\nolinebreak[3] значно більша потреба працювати\nolinebreak[2] з\nolinebreak[3] ${s\,{>}\,1}$), нехай залишиться за\nolinebreak[2] межами цього збірника.

Насамкінець, якщо визначити, що позиціями гри є пари\nolinebreak[3] $(n;s)$, але\nolinebreak[2] не~помітити властивість <<якщо деяка позиція $(n^{\star};s^{\star})$ виграшна, то виграшні й \emph{усі} позиції $(n^{\star};s)$ при\nolinebreak[2] тому\nolinebreak[3] ж\nolinebreak[3] $n^{\star}$ та $s\dib{{\>}}s^{\star}$>>, може мати смисл реалізувати стандартну перевірку позицій на виграшність\nolinebreak\hspace{0pt plus 1pt}/\nolinebreak[2]\hspace{0pt plus 1pt}про\-граш\-ність рекурсією із за\-пам'\-я\-то\-ву\-ва\-н\-ня\-ми, зберігаючи результати вже переглянутих позицій не~у~двовимірному масиві, а~в~\mbox{\texttt{map}-і} (в~деяких мовах програмування це називається \texttt{dictionary}, в~літературі поширена також назва <<асоціативний масив>>), ключами якого є такі пари\nolinebreak[3] $(n;s)$. При грамотній реалізації (зокрема, оголошувати позицію виграшною, щойно знайшовши перший хід у програшну, а~не~крутити цикл завжди до~кінця) значна частина з $\Theta(N^2)$ позицій, теоретично досяжних з позиції\nolinebreak[3] $(N;1)$, практично не~розглядаються, тож не~потрапляють у~\texttt{map}, тож такий розв'язок \emph{можна} написати так, що він пройде передостанній блок тестів ($12345\dib{{\<}}N\dib{{\<}}43210$). Але пройти так останній блок, начебто, неможливо.
