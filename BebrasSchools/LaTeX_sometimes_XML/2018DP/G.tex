\begin{problemAllDefault}{MaxSum (усі стовпчики)}\label{problem:maxsum-all-columns}

\ifisART
\else
Є прямокутна таблиця розміром $N$ рядків на $M$ стовпчиків. У кожній клітинці записано ціле число. 
По ній потрібно пройти згори донизу, починаючи з будь-якої клітинки верхнього рядка, 
далі переходячи щоразу в одну з <<нижньо-сусідніх>> і закінчити маршрут у якій-небудь клітинці нижнього рядка. 
<<Нижньо-сусідня>> означає, що з клітинки $(i,j)$ можна перейти у ${(i\,{+}\,1,j\,{-}\,1)}$, 
або у ${(i\,{+}\,1,j)}$, або у ${(i\,{+}\,1,j\,{+}\,1)}$, але не~виходячи за межі таблиці 
(при ${j\,{=}\,1}$ перший з наведених варіантів стає неможливим, а при ${j\,{=}\,M}$\nolinebreak[3] --- останній).

Напишіть програму, яка знаходитиме максимально можливу суму значень пройдених клітинок 
серед усіх допустимих шляхів ходами, що
\emph{проходять хоча~б по одному разу через кожен зі стовпчиків}. 

\InputFile
У першому рядку записані $N$ і $M$\nolinebreak[3] --- кількість рядків і кількість стовпчиків 
(${1\,{\<}\,N\,{\<}\,1024}$, ${1\,{\<}\,M\,{\<}\,N}$); далі у кожному з наступних $N$ рядків 
записано рівно по $M$ розділених пробілами цілих чисел (кожне не~перевищує за\nolinebreak[2] модулем~$10^6$)\nolinebreak[3] --- 
значення клітинок таблиці. 

\OutputFile
Вивести єдине ціле число\nolinebreak[3] --- знайдену максимальну серед сум за маршрутами зазначеного вигляду.
Оскільки гарантовано, що ${M\,{\<}\,N}$, відповідь існує завжди.

\fi


\ifisART
%
\myflfigaw{\begin{exampleSimple}{5em}{5em}%
\input G-ex
\end{exampleSimple}}
Опис таблиці та дозволених шляхів відповідають розд.~\ref{sec:max-sum-basic} та задачі~\ref{problem:maxsum-basic}, але тепер треба знайти максимальну суму лише серед тих шляхів, які \emph{і}~задовольняють вимоги з~тієї задачі, \emph{і~проходять хоча~б по одному разу через кожен зі стовпчиків}.
Формат вхідних даних в~цілому відповідає задачі~\ref{problem:maxsum-basic}, але тепер можливі більші розміри: ${1\,{\<}\,N\,{\<}\,1024}$, ${1\,{\<}\,M\,{\<}\,N}$. Формат виведення результатів цілком відповідає задачі~\ref{problem:maxsum-basic}; оскільки гарантовано, що ${M\,{\<}\,N}$, відповідь існує завжди.
%
\else
%
\Example
\begin{example}%
\input G-ex
\end{example}
%
\fi


\Notes
Відповідь дорівнює $28\dib{{=}}15\dib{{+}}5\dib{{+}}2\dib{{+}}6$, бо всі шляхи з більшою сумою проходять не через усі стовпчики.

\ARTskip{Також зверніть увагу, що в задачі дуже великий розмір вхідних даних. Може бути важливим (зокрема, для часу виконання програми) вибраний спосіб їх читання.}

\end{problemAllDefault}