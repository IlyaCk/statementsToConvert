\begin{problemAllDefault}{MaxSum (з кількістю шляхів)}\label{problem:maxsum-with-quantity}

Є прямокутна таблиця розміром $N$ рядків на $M$ стовпчиків. У кожній клітинці записано ціле число. 
По ній потрібно пройти згори донизу, починаючи з будь-якої клітинки верхнього рядка, 
далі переходячи щоразу в одну з <<нижньо-сусідніх>> і закінчити маршрут у якій-небудь клітинці нижнього рядка. 
<<Нижньо-сусідня>> означає, що з клітинки $(i,j)$ можна перейти у ${(i\,{+}\,1,j\,{-}\,1)}$, 
або у ${(i\,{+}\,1,j)}$, або у ${(i\,{+}\,1,j\,{+}\,1)}$, але не~виходячи за межі таблиці 
(при ${j\,{=}\,1}$ перший з наведених варіантів стає неможливим, а при ${j\,{=}\,M}$\nolinebreak[3] --- останній).

Напишіть програму, яка знаходитиме максимально можливу суму значень пройдених клітинок 
серед усіх допустимих шляхів, \emph{а~також кількість різних шляхів, на яких ця сума досягається}.

\InputFile  
У першому рядку записані $N$ і $M$\nolinebreak[3] --- кількість рядків і кількість стовпчиків 
(${1\,{\<}\,N, M\,{\<}\,200}$); далі у кожному з наступних $N$ рядків 
записано рівно по $M$ розділених пробілами цілих чисел 
(кожне не~перевищує за\nolinebreak[2] модулем~$10^6$)\nolinebreak[3] --- значення клітинок таблиці.

Гарантовано, що при перевірці будуть використані тільки такі вхідні дані, 
для яких шукана кількість шляхів з максимальною сумою 
не~перевищує\nolinebreak[2] $10^9$\nolinebreak[3] (мільярд).

\OutputFile
Вивести в одному рядку два цілі числа, розділені пробілом: 
максимально можливу суму за маршрутами зазначеного вигляду
та кількість різних маршрутів, уздовж яких 
досягається ця максимальна сума.

\Example
\begin{example}%
\input E-ex
\end{example}

\Note
У першому тесті, максимальне значення 42 можна набрати 
уздовж лише одного шляху ($15\dib{{+}}9\dib{{+}}9\dib{{+}}9$). 
А у другому, максимальне значення 111 можна набрати трьома способами: 
або \verb"a[1][3]=100", \verb"a[2][2]=1", \verb"a[3][1]=10", 
або \verb"a[1][3]=100", \verb"a[2][3]=10", \verb"a[3][2]=1", 
або \verb"a[1][3]=100", \verb"a[2][3]=10", \verb"a[3][3]=1".


\end{problemAllDefault}
