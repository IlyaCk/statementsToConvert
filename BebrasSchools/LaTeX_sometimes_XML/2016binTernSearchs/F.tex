\begin{problemAllDefault}{Дошки}

\begin{footnotesize}

(На~жаль, на~самій школі умова цієї задачі містила деякі неоднозначності, том\'{у} тепер вона трохи переписана.)

\end{footnotesize}

Степан вирішив оновити старий паркан біля дому. Для цього йому потрібно $M$ дощок. Зараз у його сараї є $N$ дощок однакової товщин\'{и} та ширин\'{и}, але різної довжин\'{и}. Так як Степан любить засмагати влітку, він хоче, щоб паркан був якомога більшої однакової цілочисельної висот\'{и}. Степан просить допомогти йому і знайти таку найбільшу цілочисельну висоту. Дошки можна різати (як з метою приведення до однакової довжин\'{и}, так і з метою отримання з однієї дуже довгої кількох коротших), але не~можна доточувати одну до \'{о}дної.

\InputFile
В першому рядку знаходяться два числа $N$, $M$ ($1{\<}N,M{\<}10000$). В~кожному з наступних $N$ рядків знаходиться по одному числу\nolinebreak[3] $A_i$\nolinebreak[3] --- довжина \mbox{$i$-ої} дошки в сараї ($1{\<}A_i{\<}10^7$).


\OutputFile
Виведіть єдине число --- відповідь на задачу. Якщо Степан не~може побудувати такий паркан, виведіть~\verb"0".

\Example

\begin{example}
\exmp{4 3
4
7
3
2}{3}%
\end{example}

\Note
Три дошки висот\'{и}~3 можна отримати аж кількома різними способами: можна взяти дошки довжинами 3, 4, 7 і довші вкоротити до найкоротшої; можна взяти лише дошки довжинами 3 та~7, коротшу взяти цілою, а\nolinebreak[3] довшу розрізати на дві корисні по~3 і 1 у~відходи. Якби в\nolinebreak[3] умові не\nolinebreak[3] було вимоги про цілочисельність відповіді, то відповіддю було~б 3,5 (довжину~7 навпіл плюс ще одна 3,5 із дошки довжин\'{и}~4); але вимога цілочисельності~є, тому~3.

\end{problemAllDefault}

