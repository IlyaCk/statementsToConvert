\begin{problemAllDefault}{Всюдихід--1}

Всюдихід повинен виїхати з бази, перетнути спочатку пустелю, а потім болото і прибути на пост. Перешкод на шляху немає, всюдихід може рухатись у будь-якому напрямку. Максимальна швидкість всюдихода по пустелі і по болоту можуть відрізнятися одна від одної. Відомо, що пряма, яка з'єднує базу і пост, проходить через обидві території. Визначте шлях, по якому всюдихід якнайшвидше прибуде на пост.

\InputFile
\mbox{1-й}\nolinebreak[2] рядок містить максимальну швидкість всюдихода в пустелі $v_1$\nolinebreak[3] (м/с) та максимальну швидкість всюдихода по болоту $v_2$\nolinebreak[3] (м/с). 
\mbox{2-й}\nolinebreak[2] рядок містить координати $x_1$~$y_1$ бази. 
\mbox{3-й}\nolinebreak[2] рядок містить координати $x_2$~$y_2$ посту.
Всередині кожного рядка числа розділені одним пропуском (пробілом).
Відомо, что вісь $Ox$ розділяє пустелю і болото (пустеля вгорі), $y_1{>}0$, $y_2{<}0$, $x_1{\geqslant}0$, $x_2{\geqslant}0$. Всі числа дійсні і не~перевищують за~модулем~$10^7$.

\OutputFile
Програма повинна вивести два дійсних числа\nolinebreak[3] --- абсцису ($x$-координату) точки перетину межі територій і мінімальний час (у~секундах), необхідний всюдиходу на поїздку від бази до посту. Відповідь зараховується, коли для кожного з цих двох чисел хоча~б одна з похибок (абсолютна та/або відносна) не~перевищує $10^{-9}$.

\Examples

\begin{example}
\exmp{1 1
5 3
5 -2
}{5.0 5.0}%
\exmp{3 5
20 10
8 -9
}{15.74651029 5.997280122}%
\end{example}

\end{problemAllDefault}

