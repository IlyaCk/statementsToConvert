\begin{problemAllDefault}{Кубічне рівняння}

Кубічне рівняння $ax^3+bx^2+cx+d=0$ задане чотирма своїми коефіцієнтами $a$,~$b$,~$c$,~$d$ ($a\neq 0$).
Як відомо, кубічне рівняння завжди має хоча~б один дійсний корінь, а максимальна кількість різних дійсних коренів дорівнює~3.
Напишіть програму, яка знайде усі дійсні корені кубічного рівняння.

\InputFile
В~одному рядку через пропуски (пробіли) задано чотири числа $a$,~$b$,~$c$,~$d$ --- коефіцієнти рівняння. Усі чотири коефіцієнти є цілими числами, що не~перевищують за~модулем~1000, при цьому $a\neq 0$ (решта коефіцієнтів можуть бути в тому числі й нулями).

\OutputFile
Виведіть в один рядок через пробіли (пропуски) усі дійсні корені рівняння. Якщо рівняння має менше, ніж три, корені, дозволяється деякі з них повторити, але так, щоб:
\begin{enumerate}
\item
сумарна кількість виведених чисел була не~більша трьох;
\item
кожне з виведених чисел було коренем рівняння (дозволяється абсолютна та/або відносна похибка до~$10^{-6}$);
\item
кожен з коренів був виведений хоча~б один раз.
\end{enumerate}

\Examples

\begin{example}
\exmp{1 -3 3 -1}{1}%
\exmp{1 -6 11 -6}{1 2 3}%
\end{example}

\Note
Правила зарахування відповідей слід сприймати буквально. Наприклад, у першому прикладі зараховуються також відповіді, де 1 виведена два чи три рази. Більш того, зараховується також відповідь з~трьома <<різними>> коренями \verb"1", \verb"1.000000001" та \verb"0.999999999". А~три різні корені \verb"1", \verb"1.001" та \verb"0.999" вже не~зараховуються, бо похибка перевищує дозволену. А~чотири <<різні>> корені \verb"1", \verb"1.0000000001", \verb"1.0000000002" та \verb"0.999999999" не~зараховуються не~через похибки, а~через те, що їх чотири, а~дозволено максимум три.


\end{problemAllDefault}

