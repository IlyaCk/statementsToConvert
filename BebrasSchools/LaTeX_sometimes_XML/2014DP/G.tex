\begin{problemAllDefault}{MaxSum (відвідати усі стовпчики ходами коня)}

\label{problem:max-sum-knight-all-cols}

 
Є прямокутна таблиця розміром $N$ рядків на $M$ стовпчиків. У кожній клітинці записано ціле число. 
По ній потрібно пройти згори донизу, починаючи з будь-якої клітинки верхнього рядка, 
далі рухаючись вниз ходами коня і закінчити маршрут у якій-небудь клітинці нижнього рядка. 
Тобто, з клітинки $(i,j)$ можна перейти у ${(i{+}1,j{-}2)}$, 
або у ${(i{+}2,j{-}1)}$, або у ${(i{+}2,j{+}1)}$, або у ${(i{+}1,j{+}2)}$, 
виключаючи варіанти, що виходять за межі таблиці.

Напишіть програму, яка знаходитиме максимально можливу суму значень пройдених клітинок 
\emph{серед усіх допустимих шляхів ходами коня, 
що проходять хоча б по одному разу через кожен зі стовпчиків}. 


\InputFile
  
У першому рядку записані $N$ і $M$ --- кількість рядків і кількість стовпчиків 
(${1{\<}N{\<}42}$, $1{\<}M{\<}17$); далі у кожному з наступних $N$ рядків 
записано рівно по $M$ розділених пробілами цілих чисел (кожне не перевищує по модулю 10$^6$) --- 
значення клітинок таблиці. 



\OutputFile

Вивести або єдине ціле число (знайдену максимальну серед сум за маршрутами зазначеного вигляду),
або рядок ``\texttt{impossible}'' (без лапок, маленькими латинськими буквами). 
Рядок ``\texttt{impossible}'' має виводитися тільки у~разі, 
коли не існує жодного маршруту ходами коня, що проходить через всі стовпчики хоча б по одному разу. 


\Examples

\begin{example}
\exmp{4 3
1 15 2
9 7 5
9 2 4
6 9 -1}{25}%
\exmp{3 3
1 15 2
9 7 5
9 2 4}{impossible}%
\end{example}
 
 
\Note 

Для поля $4\times3$ є рівно чотири способи спуститися ходами коня, відвідавши кожен стовпчик:



\begin{tabular}{rl}
перший
& 
\verb"a[1][1]"$\rightarrow$\verb"a[2][3]"$\rightarrow$\verb"a[4][2]";
 \\ 

другий
& 
\verb"a[1][2]"$\rightarrow$\verb"a[3][1]"$\rightarrow$\verb"a[4][3]";
 \\ 

третій
& 
\verb"a[1][2]"$\rightarrow$\verb"a[3][3]"$\rightarrow$\verb"a[4][1]";
 \\ 

четвертий
& 
\verb"a[1][3]"$\rightarrow$\verb"a[2][1]"$\rightarrow$\verb"a[4][2]"

\end{tabular}
 
Максимальна можлива сума  $25\dib{{=}}15\dib{{+}}4\dib{{+}}6$ досягається на 3-му з них. 
 
Для поля $3\times3$ таких способів взагалі немає. 

\end{problemAllDefault}