\begin{problemAllDefault}{Дільники на проміжку--1}\label{task:divisors-in-range-1}

Напишіть програму, яка знайде кількості дільників усіх підряд чисел проміжку від~$A$\nolinebreak[2] до~$B$ (обидві межі включно), і~виведе:
\begin{itemize}
\item
суму цих кількостей;
\item
суму квадратів цих кількостей;
\item
суму чисел, утворених з окремо взятих цифр цих кількостей.
\end{itemize}

\InputFile
У єдиному рядку через пробіл задані два натуральні числа $A$ та~$B$: м\'{е}жі проміжку. Виконуються обмеження:

\vspace{-1\baselineskip}

\begin{multicols}{3}
\begin{itemize}
\item
$1\<A$;
\item
$B\<10^{12}$;
\item
$A\<B\<A+100$.
\end{itemize}
\end{multicols}

\vspace{-0.5\baselineskip}

\myflfigaw{\hspace*{-0.5em}\begin{exampleSimple}{3.5em}{4.5em}%
\input F-ex
\end{exampleSimple}\hspace*{-0.25em}}
\OutputFile
Виведіть у одному рядку через пробіли три числ\'{а}: суму кількостей дільників чисел проміжку; суму квадратів кількостей дільників; суму чисел, утворених з окремо взятих цифр цих кількостей.

% \Example

% \begin{example}
% \exmp{119 122}{27 297 18}%
% \end{example}

\Note
Число~119 має 4 дільники (1,~7, 17,~119); число~120 має 16 дільників (1,~2, 3, 4, 5, 6, 8, 10, 12, 15, 20, 24, 30, 40, 60,~120); число~121 має 3\nolinebreak[3] дільника (1,\nolinebreak[3] 11,\nolinebreak[3] 121); число~122 має 4 дільники (1,~2, 61,~122).
%
Звідки, ${4\,{+}\,16\,{+}\,3\,{+}\,4}\dibbb{{=}}27$ дає першу відповідь,\hspace{0.25em plus 0.5em}
${4^2\,{+}\,16^2\,{+}\,3^2\,{+}\,4^2}\dibbb{{=}}{16\,{+}\,256\,{+}\,9\,{+}\,16}\dibbb{{=}}297$ дає другу відповідь,\hspace{0.25em plus 0.5em}
${4\,{+}\,(1\,{+}\,6)\,{+}\,3\,{+}\,4}\dibbb{{=}}18$ дає третю відповідь.

\end{problemAllDefault}