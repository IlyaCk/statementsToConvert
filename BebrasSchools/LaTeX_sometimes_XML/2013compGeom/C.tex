\begin{problemAllDefault}{Чи перетинаються відрізки---2}

Дано чотири точки $A$, $B$, $C$, $D$. Чи мають відрізки $AB$ і $CD$ хоча~б одну спільну точку? Програма повинна працювати в усіх випадках, включаючи в~тому числі й ситуації, коли відрізки накладаються, а також $A=B$ або $C=D$, тобто один чи обидва відрізки вироджені в точку.

\InputFile 
слід прочитати зі стандартного входу (клавіатури). У~першому рядку задано $N$ ($1\<N\<5$) --- кількість пар відрізків у даному тесті, далі йдуть $N$ груп по два рядки у кожній. Перший з рядків кожної групи містить відрізок $AB$ (у вигляді чотирьох чисел $Ax$~$Ay$~$Bx$~$By$), другий і останній рядок кожної групи --- $CD$ як $Cx$~$Cy$~$Dx$~$Dy$. Усі координати є цілими числами, не~перевищують по модулю мільйон.

\OutputFile Для кожної з груп вивести у окремому рядку \texttt{YES} (якщо відрізки перетинаються) або \texttt{NO} (якщо~ні).

\Examples

\begin{example}
\exmp{
3
0 0 1 0
100 -100 100 100
-5 0 5 0
0 5 0 -5
592741 76372 273343 724795
408678 74450 197154 3779
}{
NO
YES
NO
}
\exmp{
1
0 0 3 6
1 2 1 2
}{
YES
}\end{example}

\end{problemAllDefault}
