\begin{problemAllDefault}{Хвіртка у паркані}

Пан Дивак вирішив оновити паркан, що відділяє його подвір'я від вулиці. Він вже закопав $N$ стовпчиків, а потім згадав, що у паркані бажано~б залишити хвіртку, шириною щонайменше~$W$. Тепер йому, мабуть, доведеться викопувати деякі із вкопаних стовпчиків.

Щоб робота не~була даремною, слід викопати якнайменше стовпчиків. Допоможіть панові Диваку визначити, скільки стовпчиків доведеться викопати. \mbox{Після} викопування стовпчиків мусить існувати проміжок (між двома залишеними стовпчиками, або між залишеним стовпчиком і одним із кінців ділянки, або між кінцями ділянки) ширини$\,{\>}\,W$.

\InputFile
Перший рядок містить два цілих числа $N$\nolinebreak[2] та\nolinebreak[2] $W$\nolinebreak[3] --- кількість вкопаних стовпчиків та мінімально необхідну ширину проміжку для хвіртки відповідно. Гарантується, що $0{\<}N{\<}100000$, і що $0{\<}W{\<} 10^9$.
Будемо вважати, що уздовж межі подвір'я введено вісь координат. У~др\'{у}гому рядку вхідного файлу вказано два числа $L$ та\nolinebreak[2] $R$ ($L{<}R$)\nolinebreak[3] --- координати лівого і правого кінців межі подвір'я. Далі йде третій рядок, що містить $N$ чисел\nolinebreak[3] --- координати вкопаних стовпчиків. Усі координати (включаючи $L$ та~$R$)\nolinebreak[3] --- різні цілі числа, які не~перевищують за модулем (абсолютною величиною) $10^9$. Гарантується, що всі стовпчики вкопані між лівим і правим кінцями.

\OutputFile
Слід вивести єдине число у єдиному рядку\nolinebreak[3] --- мінімальну кількість стовпчиків, які треба викопати. Якщо розв'язку не~існує, то виведіть замість кількості число ``$-1$'' (без~лапок).

\Examples
\noindent\begin{exampleSimple}{10em}{7em}
\exmp{3 2
2 6
3 4 5}{1}%
\exmp{3 2
1 6
4 3 5}{0}%
\exmp{3 5
1 7
5 3 4}{3}%
\end{exampleSimple}

\Note
Гарантовано, що хоча~б у половині тестів координати стовпчиків відсортовані за зростанням.

\end{problemAllDefault}
