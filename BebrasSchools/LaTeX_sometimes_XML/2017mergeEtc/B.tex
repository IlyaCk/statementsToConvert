\begin{problemAllDefault}{Всюдисущі ч\'{и}сла}

Дано прямокутну таблицю $N{\*}M$\nolinebreak[3] чисел. Гарантовано, що\nolinebreak[2] у\nolinebreak[3] кожному окремо взятому рядку всі ч\'{и}сла різні й монотонно зростають.

Напишіть програму, яка шукатиме перелік (також у\nolinebreak[3] порядку зростання) всіх тих чисел, які зустрічаються\linebreak[1] в\nolinebreak[3] усіх $N$ рядках.

\InputFile У~першому рядку задано два числ\'{а} $N$ та~$M$. Далі йдуть $N$ рядків, кожен з яких містить рівно~$M$ розділених пропусками чисел (гарантовано у~порядку зростання). 

\OutputFile	Програма має вивести в один рядок через пробіли у порядку зростання всі ті ч\'{и}сла, які зустрілися абсолютно в усіх рядках. Кількість чисел виводити не~треба. Після виведення всіх чисел потрібно зробити одне переведення рядка. Якщо нема жодного числ\'{а}, що зустрілося в усіх рядках, виведення повинно не~містити жодного видимого символу, але містити переведення рядка.


\Example
\begin{exampleSimple}{6em}{3em}%
\exmp{4 5
6 8 10 13 19
8 9 13 16 19
6 8 12 13 15
3 8 13 17 19}{8 13}%
\end{exampleSimple}

\Scoring
У 20\% тестів $3{\<}{N,M}{\<}20$, значення чисел від~0 до~100.
%
У\nolinebreak[3] ще\nolinebreak[3] 20\%\nolinebreak[2], $3{\<}{N,M}{\<}20$, значення чисел від $-10^9$ до~$+10^9$.
%
У\nolinebreak[3] ще\nolinebreak[3] 20\%\nolinebreak[2], $1000{\<}{N,M}{\<}1234$, значення від 0 до 12345.
%
У\nolinebreak[3] решті\nolinebreak[3] 20\%\nolinebreak[2], $1000{\<}{N,M}{\<}1234$, значення від $-10^9$ до~$+10^9$.

Здавати потрібно одну програму, а~не~чотири; різні обмеження вказані, щоб пояснити, скільки балів можна отримати, розв’язавши задачу не~повністю.
(Враховуючи особливості саме цього туру, де досить великий штраф за хоча~б один не~пройдений тест, відсотки балів приблизно дорівнюють трьом чвертям 
згаданих відсотків тестів.)


\end{problemAllDefault}
