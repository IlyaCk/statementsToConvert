% \begin{small}

% \begin{problemAllDefault}{Розподіл станцій по зонам}

% \label{problem:stations}
 
Керівництво Дуже Великої Залізниці (ДВЗ) вирішило встановити нову систему оплати за проїзд. 
ДВЗ являє собою відрізок прямої, на якій послідовно розміщені $N$ станцій. 
Планується розбити їх на $M$ неперервних зон, що слідують підряд, таким чином, 
щоб кожна зона містила хоча~б одну станцію. 
Оплату проїзду від станції $j$ до станції $k$ необхідно встановити рівною 
${1{+}|z_j{-}z_k|}$, де~$z_j$ і~$z_k$ ---
номери зон, яким належать станції $j$ та~$k$ відповідно.
Відома кількість пасажирів, які відправляються за день з кожної станції на кожну іншу.


Напишіть програму, що визначатиме, яку максимальну денну виручку можна отримати 
за новою системою при оптимальному розбитті на зони.

\InputFile

Перший рядок містить два цілих числ\'{а} $N$ та $M$ (${1{\<}M{\<}N{\<}1000}$).
Другий\nolinebreak[3] --- одне число, що означає кількість пасажирів, 
які їдуть між станціями 1 та~2.
Третій\nolinebreak[3] --- два числ\'{а}, що означають кількість пасажирів, 
які їдуть між станціями 1 та 3 та між станціями 2 та 3, 
відповідно.
І~так далі. В\nolinebreak[3] \mbox{$N$-ому} рядку міститься $N{-}1$ число, 
\mbox{$i$-е} з них визначає кількість пасажирів від станції $i$ до станції~$N$. 

Кількість пасажирів для кожної пари станцій дано 
з урахуванням руху в обидві сторони. 
Всі ч\'{и}сла цілі, невід’ємні та не~перевищують 10000.

\OutputFile
Програма повинна вивести єдине ціле число\nolinebreak[3] --- шукану максимальну денну виручку.

\Example

\begin{example}
\exmp{3 2
200
10 20}{440}%
\end{example}
 
\Notes
Іншими словами, рядки вхідних даних з 2-го по $N$-й 
являють собою нижню-ліву половину 
симетричної матриці пасажиропотоку з нулями по\nolinebreak[2] головній діагоналі
(де\nolinebreak[3] симетрично розміщені елементи не~треба додавати, бо\nolinebreak[3] кожен з них\nolinebreak[3] --- вже сума потоків туди й назад).
Зокрема, у прикладі по суті задано матрицю 
\begin{scriptsize}
\begin{tabular}{rrr}
        0
       & 
        200
       & 
        10
       \\ 
        200
       & 
        0
       & 
        20
       \\ 
        10
       & 
        20
       & 
        0
\end{tabular}
\end{scriptsize}

Денну виручку 440 можна отримати, якщо розбити станції на зони 
як <<\mbox{1-а}\nolinebreak[3] станція у \mbox{1-ій} зоні, 
\mbox{2-а} та \mbox{3-я} станції у \mbox{2-ій} зоні>>.
Тоді ціну ${1{+}1}\dib{{=}}2$ платитимуть ${200{+}10}$ пасажирів 
(які їздять між \mbox{1-ю} та \mbox{2-ю} та 
між \mbox{1-ю} та \mbox{3-ю} станціями відповідно),
а ціну ${1{+}0}\dib{{=}}1$ платитимуть 20 пасажирів 
(які їздять між \mbox{2-ю} та \mbox{3-ю} станціями); 
${(200{+}10)\times2}\dib{{+}}{20\times1}\dib{{=}}440$.

Денної виручки, більшої за 440, досягти неможливо.

% \end{problemAllDefault}

% \end{small}