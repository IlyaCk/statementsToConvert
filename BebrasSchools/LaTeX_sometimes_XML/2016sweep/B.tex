\begin{problemAllDefault}{Дужки (чотири типи)}

Напишіть програму, яка з'ясовуватиме, чи~є дужковий вираз правильним. Тобто, чи~правда, що кожній дужці є парна, і у кожній парі відкривна дужка йде раніше, ніж закривна. Дужок є чотири типи: круглі `\texttt{(}' та~`\texttt{)}', квадратні `\texttt{[}' та~`\texttt{]}', фігурні `\texttt{\{}' та~`\texttt{\}}', кутові `\texttt{<}' та~`\texttt{>}'. Кожна пара дужок повинна містити відкривну та закривну дужки однакового типу. Дозволено лише щоб пара або починалася тоді, коли інша пара закінчилася, або повністю включалася в~іншу пару. Часткове накладання пар вважається неправильним (див.~останній рядок прикладу).

\InputFile
1-й рядок містить число $N$ ($2{\<}N{\<}12$) --- кількість різних рядків, які треба перевірити на правильність. Кожен з подальших $N$ рядків містить послідовність символів, яку треба перевірити на правильність. Послідовності гарантовано не містять ніяких інших символів, крім дужок `\texttt{(}' та/або `\texttt{)}' та/або `\texttt{[}' та/або `\texttt{]}' та/або `\texttt{\{}' та/або `\texttt{\}}' та/або `\texttt{<}' та/або~`\texttt{>}'. Довжина кожного рядка не~перевищує~1~мільйона символів; сумарна довжина усіх рядків не~перевищує 2~мільйонів символів.

\OutputFile
Виведіть рівно $N$ рядків, у~кожному єдиний символ 1~або~0 (1,~якщо відповідний рядок утворює правильну дужкову послідовність; 0,~якщо неправильну).


\Example

\begin{example}
\exmp{4
(<>)()
(<>)<\}\{>
[)(]
(<)>
}{1
0
0
0
}%
\end{example}

\end{problemAllDefault}

