\begin{problemAllDefault}{Лінія горизонту}

У~місцевості з~абсолютно плоскою поверхнею землі уздовж прямої вулиці стоять будинки--прямокутні паралелепіпеди. З~іншого боку цієї вулиці --- чисте поле. Спостерігач дивиться на будинки з деякої віддаленої точки цього чистого поля і бачить їх як прямокутники. Будинки завжди розміщені паралельно вулиці, але можуть бути як безпосередньо понад дорогою, так і у~глибині кварталу; за~рахунок цього, спостерігач може бачити накладання різних будинків-прямокутників. 
Потрібно знайти лінію горизонту, тобто ламану, що (з~точки зору описаного спостерігача) <<відокремлює будинки від неба>>.

\InputFile
1-й рядок містить єдине число~$N$ ($2{\<}N{\<}123456$)\nolinebreak[3] --- кількість будинків. Кожен з подальших $N$ рядків містить рівно по три числа кожен\nolinebreak[3] --- \mbox{$x$-коор}\-ди\-нату початку будинку, його довжину і висоту. Всі розміри та координати\nolinebreak[3] --- цілі числа від~1 до~$10^9$.

\OutputFile
Результат (опис лінії горизонту) повинен складатися з упорядкованих зліва направо за $x$-координатою описів окремих її горизонтальних фрагментів. Кожен окремий фрагмент слід описати трійкою (початок, висота, кінець). Початок кожного наступного фрагменту повинен бути або строго більшим кінця попереднього (якщо між ними лінія горизонту проходить по~землі), або дорівнювати (у~випадку, якщо висота міняється з~однієї ненульової на \emph{іншу} ненульову). Фрагменти, що проходять по поверхні землі, не~виводити. Сусідні фрагменти, що мають спільну $x$-координату та однакову висоту, слід об'єднувати (навіть якщо вони утворені різними будинками).

\Scoring
Не~менш ніж 30\% балів припадає на~тести, де $2{\<}N{\<}2500$. 

Ще~не~менш ніж 30\% балів --- на тести, де $N{\<}123456$, зате будинки розміщені не~більш як у~10 паралельних вулиці рядів (і~тому для будь-якої $x$-координати справедливо, що у~ній накладаються не~більш ніж 10~прямокутників). 

У~решті тестів $N{\<}123456$ і нема обмежень на~кількість накладань по одній і тій самій $x$-координаті.

\Example

\begin{example}
\exmp{4
20 10 30
1 49 10
60 30 12
80 20 10
}{1 10 20
20 30 30
30 10 50
60 12 90
90 10 100
}%
\end{example}


\end{problemAllDefault}

