\begin{problemAllDefault}{Площа простого многокутника}

Многокутник на площині задано цілочисельними координатами $N$ вершин. Потрібно знайти його площу. 

Многокутник простий, тобто його сторони не~перетинаються і не~дотикаються (за~винятком сусідніх, у вершинах), але він не~обов'язково опуклий.

\InputFile слід прочитати зі стандартного входу (клавіатури). У~першому рядку задано кількість вершин $N$ ($1\<N\<50000$). У~наступних $N$ рядках записані пари чисел --- координати вершин. Сторони многокутника --- відрізки між 1-ою і 2-ою, 2-ою і 3-ьою, \ldots, ($N-1$)-ою і $N$-ою, $N$-ою і 1-ою вершинами. Значення координат --- цілі числа, не~перевищують по модулю мільйон.

\OutputFile Вивести єдине число --- знайдену площу многокутника. Виводити можна хоч у експоненційній формі, хоч стандартним десятковим дробом. Результат зараховується, коли похибка (абсолютна або відносна, тобто хоча~б одна з них) не~перевищує $10^{-6}$.

\Example

\begin{example}
\exmp{
4
0 4
0 0
3 0
1 1
}{
3.5
}\end{example}

\end{problemAllDefault}
