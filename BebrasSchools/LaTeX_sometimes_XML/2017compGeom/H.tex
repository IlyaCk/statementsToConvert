\begin{problemAllDefault}{Чи є маршрут приємним?}

Туристу набридло подорожувати уздовж координатної вісі, тому він вирішив помандрувати ще по координатній площині. Він розпочинає зі своєї бази в точці $A_1$ з координатами $(x_1,y_1)$, рухається найкоротшим маршрутом до визначної пам'ятки $A_2$ з координатами $(x_2,y_2)$, далі, не~зупиняючись, рухається найкоротшим маршрутом до визначної пам’ятки $A_3$ з координатами $(x_3,y_3)$, і~так далі. Дійшовши до останньої визначної пам’ятки $A_n$ з координатами $(x_n,y_n)$, він, не~зупиняючись, рухається до своєї бази. Турист вважає свій маршрут \emph{непри\-єм\-ним}, якщо існує така пряма, що він уздовж неї не~рухався, і разом з тим перетинав її строго більше двох разів. Якщо маршрут не~є~неприємним, турист вважає його \emph{приємним}. 

Турист вважає, що перетинав пряму, якщо в деякий момент часу перебував у одній півплощині відносно неї, а через деякий проміжок часу\nolinebreak[3] --- в іншій півплощині (сама пряма не~належить жодній з півплощин).

Напишіть програму, яка, прочитавши описи кількох маршрутів, визначить, чи приємний кожен з них.

\InputFile
Програма повинна прочитати спочатку кількість маршрутів $K$ ($2{\<}K{\<}12$), потім $K$ однотипних блоків, кожен з яких описує маршрут. Кожен блок опису маршруту починається числом $n$ ($2{\<}n{\<}98765$), далі йдуть $n$ пар цілих чисел, що не~перевищують $10^8$ за абсолютною величиною — координати $x_1$ $y_1$ $x_2$ $y_2$\dots $x_n$ $y_n$. Всі числа всіх маршрутів записані в одному рядку й розділені одинарними пробілами. Сумарна кількість всіх вершин усіх маршрутів, які програма має обробити за один запуск, не~перевищуватиме~123456.

\OutputFile
Програма повинна вивести у один рядок $K$ розділених пробілами нулів та/або одиниць, які позначають, приємними~(1) чи неприємними~(0) були відповідні маршрути.

\Example

\noindent
\begin{exampleSimple}{23em}{7em}
\exmp{2 3 0 0 4 0 4 3 7 0 3 0 0 2 0 3 1 4 0 5 0 3 4}{1 0}%
\end{exampleSimple}


\vspace{-2\baselineskip}

\noindent
\begin{tabular}{@{}p{0.83\textwidth}p{0.15\textwidth}@{}}
\Note
Многокутники з прикладу вхідних даних зображені на рисунку.
У першому випадку неможливо провести таку пряму, щоб турист уздовж неї не~рухався, але перетинав її строго більше двох разів. У~др\'{у}гому випадку наведено один з багатьох можливих прикладів с\'{а}ме такої прямої.
&
\begin{mfpic}[8]{-1}{5}{-1}{4}
\axes
\dotted\lines{(-1,-1),(-1,4)}
\dotted\lines{( 1,-1),( 1,4)}
\dotted\lines{( 2,-1),( 2,4)}
\dotted\lines{( 3,-1),( 3,4)}
\dotted\lines{( 4,-1),( 4,4)}
\dotted\lines{(5, -1),( 5,4)}
%
\dotted\lines{(-1,-1),(5,-1)}
\dotted\lines{(-1, 1),(5, 1)}
\dotted\lines{(-1, 2),(5, 2)}
\dotted\lines{(-1, 3),(5, 3)}
\dotted\lines{(-1, 4),(5, 4)}
\pen{2pt}
\polygon{(0,0),(4,0),(4,3)}
\end{mfpic}
\begin{mfpic}[8]{-1}{6}{-1}{4}
\axes
\dotted\lines{(-1,-1),(-1,4)}
\dotted\lines{( 1,-1),( 1,4)}
\dotted\lines{( 2,-1),( 2,4)}
\dotted\lines{( 3,-1),( 3,4)}
\dotted\lines{( 4,-1),( 4,4)}
\dotted\lines{(5, -1),( 5,4)}
\dotted\lines{(6, -1),( 6,4)}
%
\dotted\lines{(-1,-1),(6,-1)}
\dotted\lines{(-1, 1),(6, 1)}
\dotted\lines{(-1, 2),(6, 2)}
\dotted\lines{(-1, 3),(6, 3)}
\dotted\lines{(-1, 4),(6, 4)}
\lines{(-1,0),(6,1)}
\pen{2pt}
\polygon{(0,3),(0,0),(2,0),(3,1),(4,0),(5,0),(3,4)}
% \polygon{(0,3),(0,0),(0,2),(1,3),(0,4),(0,5),(3,4)}
\end{mfpic}

\end{tabular}


\end{problemAllDefault}