\begin{problemAllDefault}{Спільні дотичні--2}

\begin{small}

Як відомо, дотичною до кола є пряма, яка має рівно одну спільну точку з цим колом. Можлива ситуація, коли одна й та сама пряма є дотичною  відразу до двох кіл. Тоді вона називається спільною дотичною. Напишіть програму, яка знаходитиме спільні дотичні для заданих двох кіл. При виведенні \mbox{врахуйте} стародавню традицію приписувати числу~7 значення <<багато>>. Тобто, коли кількість спільних дотичних строго більша~6, незалежно від справжньої кількості виводьте будь-які сім з усіх можливих спільних дотичних.

\InputFile
Шість цілих чисел, розділених пропусками (пробілами) $X1$, $Y1$, $R1$, $X2$, $Y2$, $R2$\nolinebreak[3] --- відповідно координати центра і радіуси \mbox{1-го} і \mbox{2-го} кола. Для всіх координат, абсолютна величина (модуль) не~перевищує мільйон. Для обох радіусів, значення у межах від~1 до мільйона (обидві межі включно).

\OutputFile
Програма виводить у~першому рядку єдине число\nolinebreak[3] $K$\nolinebreak[3] --- кількість шуканих спільних дотичних (з~урахуванням згаданої стародавньої традиції);
далі повинно йти рівно\nolinebreak[3] $K$\nolinebreak[3] рядків, кожен з яких повинен містити чотири дійсні числ\'{а}\nolinebreak[3] --- координати двох різних точок з відповідної дотичної (спочатку $x$- та $y$-координати однієї точки, потім $x$- та $y$-координати іншої).

% \begin{footnotesize}

Відповідь зараховуватиметься, коли виконуватимуться всі вимоги:
\begin{enumerate}
\item
Кількість спільних дотичних знайдено правильно (з~урахуванням вказаної стародавньої традиції).
\item
Кожна дотична описується двома помітно різними точками (відстань між двома точками, що задають одну дотичну, не~менша~1).
\item
Пряма, що проходить через кожну з пари точок, або справді є дотичною (має рівно одну спільну точку з колом), або є добрим наближенням до дотичної (або проходить поза колом на відстані не~більш як $10^{-6}$ від нього, або заходить всер\'{е}дину кола так, що довжина частини цієї прямої всер\'{е}дині цього кола не~перевищує однієї мільйонної від радіуса цього кола).
\item
Кожен з\nolinebreak[3] $K$\nolinebreak[3] рядків, що описують дотичні, описує свою власну дотичну, відмінну від інших; формально кажучи, якщо взяти \mbox{$j$-й} та \mbox{$k$-й} рядки ($j{\neq}k$), де дотичні задані точками $A_j$, $B_j$ та $A_k$, $B_k$, то повинно виконуватися

\begin{center}

\vspace{-\baselineskip}

$({(dist(A_j,A_kB_k)\>0{,}1)}\texttt{or}{(dist(B_j,A_kB_k)\>0{,}1)})$

\texttt{and}

$({(dist(A_k,A_jB_j)\>0{,}1)}\texttt{or}{(dist(B_k,A_jB_j)\>0{,}1)})$

\vspace{-\baselineskip}

\end{center}

\noindent%
де\nolinebreak[3] $dist$\nolinebreak[3] --- відстань від точки до прямої, що рахується уздовж перпендикуляру; смисл усього виразу разом узятого\nolinebreak[3] --- хоча~б одна з двох точок, які задають пряму, знаходиться на відстані хоча~б $0{,}1$ від іншої прямої.

\end{enumerate}


\Example

\noindent
\begin{exampleSimple}{7em}{28em}
\exmp{20 0 4 50 0 10}{4
48 -9.79795897113 19.2 -3.91918358845
48 9.79795897113 19.2 3.91918358845
45.3333333333 -8.84433277428 21.8666666667 3.53773310971
45.3333333333 8.84433277428 21.8666666667 -3.53773310971}%
\end{exampleSimple}

% \end{footnotesize}

\end{small}

\end{problemAllDefault}