\begin{problemAllDefault}{З\'{а}мок}


Стародавній з\'{а}мок має прямокутну форму. Замок містить щонайменше дві кімнати. Підлогу замка можна умовно поділити на $M{\*}N$ клітин. Кожна така клітинка містить <<0>> або <<1>>, які задають порожні ділянки та стіни замку відповідно.

Напишіть програму, яка б знаходила кількість кімнат у зам\-ку та площу найбільшої кімнати (яка вимірюється кількістю клітинок).

\InputFile 
План замку задається у вигляді послідовності чисел, по одному числу, яке характеризує кожну клітинку. Перший рядок містить два цілих числа $M$ та\nolinebreak[2] $N$\nolinebreak[3] --- кількість рядків та кількість стовпчиків (${3\,{\<}\,M\,{\<}\,1000}$, ${3\,{\<}\,N\,{\<}\,1000}$). $M$~наступних рядків містить по\nolinebreak[3] $N$\nolinebreak[3] нулів або одиниць, що\nolinebreak[3] йдуть поспіль (без\nolinebreak[2] пробілів). Перший та останній рядок, а\nolinebreak[3] також перший та останній стовпчик формують зовнішні стіни замку і складаються лише з одиниць.

\OutputFile Виведіть кількість кімнат та площу найбільшої кімнати замку (по\nolinebreak[3] одному числу в\nolinebreak[3] рядку).


\Examples
\begin{exampleSimple}{5em}{5em}
\exmp{6 8
11111111
10011001
10011001
11111001
10101001
11111111}{4
8}%
\end{exampleSimple}
\begin{exampleSimple}{6em}{5em}
\exmp{9 12
111111111111
101001000001
111001011111
100101000001
100011111101
100001000101
111111010101
100000010001
111111111111}{4
28}%
\end{exampleSimple}

\Scoring Значна частина тестів містить план замку з кімнатами лише прямокутної форми, але повні (чи~навіть близькі до повних) бали може отримати лише програма, яка правильно враховує, що <<кімнати>> можуть бути як завгодно <<закручені>>. Не~менше половини тестів такі, що ${3\,{\<}\,M\,{\<}\,20}$, ${3\,{\<}\,N\,{\<}\,50}$, але повні (чи~навіть близькі до повних) бали може отримати лише програма, яка виходить з повних обмежень ${M,\,N\,{\<}\,1000}$.

\end{problemAllDefault}

%%%\pagebreak % TODO: check!!!