\begin{problemAllDefault}{MaxSum (щаслива)}

Є\nolinebreak[3] прямокутна таблиця розміром $N$ рядків на $M$ стовпчиків. У\nolinebreak[3] кожній клітинці записане невід'ємне ціле число. 
По\nolinebreak[3] ній потрібно пройти згори донизу, починаючи з будь-якої клітинки верхнього рядка, 
далі переходячи щоразу в одну з <<нижньо-сусідніх>> і закінчити маршрут у якій-небудь клітинці нижнього рядка. 
<<Нижньо-сусідня>> означає, що з клітинки $(i,j)$ можна
перейти в\nolinebreak[2] ${(i+1, j-1)}$,\linebreak[2] 
або в\nolinebreak[2] ${(i+1,j)}$,\linebreak[2]
або в\nolinebreak[2] ${(i+1,j+1)}$, 
але не~виходячи за межі таблиці (при\nolinebreak[2] ${j=1}$ перший з наведених варіантів стає неможливим, а при ${j=M}$\nolinebreak[3] --- останній).

Напишіть програму, яка знаходитиме максимально можливу \emph{щасливу} суму значень пройдених клітинок 
серед усіх допустимих шляхів.
Як широко відомо у вузьких колах, щасливими є ті\nolinebreak[2] й\nolinebreak[2] тільки\nolinebreak[2] ті ч\'{и}сла, десятковий запис яких містить лише цифри 4\nolinebreak[3] \mbox{та/або}~7 
(можна обидві, можна лише якусь одну; але ніяких інших цифр використовувати не~можна).
Зверніть увагу, що щасливою повинна бути с\'{а}ме сума, а\nolinebreak[3] обмежень щодо окремих доданків нема.

\InputFile
У першому рядку записані $N$ та\nolinebreak[3] $M$\nolinebreak[3] --- кількість рядків і кількість стовпчиків 
(${1\<N,\,M\<12}$); далі у кожному з наступних $N$ рядків 
записано рівно по $M$ розділених пробілами невід'ємних цілих чисел, 
кожне не~більш ніж з\nolinebreak[3] 12\nolinebreak[3] десяткових цифр\nolinebreak[3] --- значення клітинок таблиці.

\OutputFile
Вивести або єдине ціле число (знайдену максимальну серед щасливих сум за маршрутами зазначеного вигляду), 
або рядок <<\texttt{impossible}>> (без\nolinebreak[2] лапок, маленькими латинськими буквами). 
Рядок <<\texttt{impossible}>> має виводитися тільки у\nolinebreak[2] разі, коли жоден з допуcтимих маршрутів не~має щасливої суми.

\myflfigaw{\begin{tabular}{@{}c@{}}
\Example\\
\begin{exampleSimple}{5em}{5em}
\exmp{3 4\\
3 0 10 10\\
5 0 7 4\\
4 10 5 4}{7}%
\end{exampleSimple}
\end{tabular}}


\end{problemAllDefault}

% \begin{small}

\Notes
Взагалі-то максимально можливою сумою є 27$\dib{{=}}$10$\dib{{+}}$7$\dib{{+}}$10,\linebreak[2] але число 27 не~є щасливим.
Тому відповіддю буде максимальна серед щасливих сума 7$\dib{{=}}$3$\dib{{+}}$0$\dib{{+}}$4, 
яка досягається уздовж маршруту \texttt{a[1][1]}$\dib{{\to}}$\texttt{a[2][2]}$\dib{{\to}}$\texttt{a[3][1]}.

Наскільки відомо автору задачі, автором <<широко відомого у вузьких колах>> с\'{а}ме такого трактування поняття <<щасливого числа>>
є Василь Білецький, випускник Львівського національного університету імені Івана Франка,
котрий був капітаном першої з українських команд, що вибороли золоту медаль на фіналі першості світу ACM~ICPC,
і\nolinebreak[3] тривалий час входив у десятку найсильніших спортивних програмістів світу за рейтингом TopCoder.

% \end{small}