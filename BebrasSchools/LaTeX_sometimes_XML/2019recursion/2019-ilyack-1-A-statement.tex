\begin{problemAllDefault}{Швидке пiднесення до степеню (за~модулем, до~$2^{63}$)}

Для натуральних чисел $a$, $N$,\nolinebreak[3] $p$
(${1\,{\<}\,a\,{<}\,2^{63}}$, 
${1\,{\<}\,N\,{<}\,2^{63}}$, 
${1\,{\<}\,p\,{<}\,2^{63}}$)
виведiть $(a^N) \bmod p$, тобто залишок вiд цiлочисельного дiлення $a^N$ на~$p$.

\myflfigaw{\begin{tabular}{@{}c@{}}
\Example\\
\begin{exampleSimple}{8em}{5em}%
\exmp{2 10 100 10 2 32}{24\\
4}%
\end{exampleSimple}
\end{tabular}}

\InputFile 
В~один рядок через пропуски (пробіли) будуть вводитися багато трійок натуральних чисел $a$, $N$,\nolinebreak[3] $p$, 
всер\'{е}дині кожної з трійок $a$, $N$,\nolinebreak[3] $p$ задані в~указаному порядку, 
та взятi з дiапазонiв, указаних у попередньому абзацi. 
Кількість трійок ні\'{я}к не вказується, слід читати, доки вони не~вичерпаються.
Гарантовано, що між числами вс\'{е}редині рядка рівно по одному пропуску (пробілу);
після останнього числ\'{а} останньої трійки нема пропуску і є переведення рядка;
зразу після переведення рядка вхідні дані закінчуються.

\OutputFile	
Вивести стільки знайдених значень $(a^N) \bmod p$, скільки у вхідних даних було трійок;
кожне число виводити в окремому рядку.

% \Note
% При здачі у ejudge, перший тест є тестом з умови, і (\emph{лише} для нього) протокол тестування показує,
% яку відповідь вивела Ваша програма.


\end{problemAllDefault}