\begin{problemAllDefault}{Драконова ламана--2}

Означення самоподібної <<драконової ламаної>> див. у попередній задачі.
Якщо говорити не~про\nolinebreak[2] текст програми для <<черепашки>>, а\nolinebreak[3] про\nolinebreak[2] хід її застосування, то 
<<черепашка>> завжди малює відрізки однаковісінької довжин\'{и}, а красивий малюнок формується суто послідовністю поворотів. Більш того, тут є лише два види поворотів:  
лівий (позначимо його буквою~``\texttt{L}'') та
правий (позначимо~``\texttt{R}''), обидва на~$90^\circ$.

{% \sloppy
Таким чином, драконову ламану \mbox{1-го} порядку можна описати єдиним поворотом~``\texttt{\small{R}}'' (враховуємо лише повороти \emph{всер\'{е}дині} ламаної); аналогічно, ламану \mbox{2-го} порядку можна описати послідовністю  
``\texttt{\small{RRL}}'', \mbox{3-го}\nolinebreak[3] --- послідовністю 
``\texttt{\small{RRLRRLL}}'', \mbox{4-го}\nolinebreak[3] --- 
``\texttt{\small{RRLRRLLRRRLLRLL}}'', і так далі.

}

Це, в~принципі, можна використати, щоб побудувати ламану, не~використовуючи рекурсію (побудувати послідовність таких рядків суто ітеративними, тобто циклічними, засобами, потім виконати послідовність поворотів). Але з\'{а}раз розглянемо зворотню задачу: слід, не~будуючи цих рядків у~пам'яті, вивести підря\-док (неперервний фрагмент) такого рядка. Наприклад, вище наведено, який вигляд мають рядки, відповідні \mbox{2-му}, \mbox{3-му} та \mbox{4-му} порядкам.
І~можна поставити задачу: вивести увесь підря\-док (неперервний фрагмент), починаючи з символу $\No\,i$ та кількістю символів~$k$. Наприклад, при ${i\,{=}\,3}$, ${k\,{=}\,10}$ це буде ``\texttt{\small{LRRLLRRRLL}}'' (вважаючи, що нумерація символів починається\nolinebreak[3] з~1); який при цьому бр\'{а}ти порядок ламаної, насправді несуттєво (аби\nolinebreak[2] досить великий), бо\nolinebreak[2] рядок кожного подальшого поряд\-ку починається в~точності з рядка попереднього \mbox{порядку}.

\myflfigaw{\begin{tabular}{c}\Examples\\
\begin{exampleSimple}{5em}{5em}
\exmp{3 10}{LRRLLRRRLL}%
\exmp{8 8}{RRRLLRLL}%
\end{exampleSimple}
\end{tabular}}
\InputFile
Єдиний рядок вхідних даних містить два числ\'{а} $i$ ($1\dib{{\<}}i\dib{{\<}}10^{18}$) та $k$ ($1\dib{{\<}}k\dib{{\<}}10^{5}$), розділені одинарним пробілом.

\OutputFile
Підрядок (неперервний фрагмент), починаючи з символу $\No\,i$ та кількістю символів~$k$
(вважаючи, що символи рядка занумеровані\nolinebreak[3] з~1, і що розглядається досить великий порядок ламаної, щоб символи з такими номерами існували).

\end{problemAllDefault}

