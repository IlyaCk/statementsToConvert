\begin{problemAllDefault}{Об'єм об'єднання паралелепіпедів}

{\tolerance=8000
\hyphenpenalty=1000
Дано $N$ ($2\dib{{\<}}N\dib{{\<}}100$) прямокутних паралелепіпедів, р\'{е}бра  яких пара\-ле\-льні вісям координат. Про ці паралелепіпеди достеменно відомо, що вони частенько перетинаються по~\mbox{2--3}, але лише ${K\,{\<}\,\min(15,N)}$ з усіх $N$ паралелепіпедів можуть брати участь у тих перетинах, де одна й та сама частина простору нену\-льо\-в\'{о}го об'єму належить більш, ніж трьом паралелепіпедам одночасно. Чому дорівнює об'єм об'єднання всіх цих паралелепіпедів? (Зі~стандартного означення об'єднання випливає, що об'єми частин простору, належних відразу кільком паралелепіпедам, рахуються рівно один раз, незалежно від того, у~скільки паралелепіпедів вони входять.)



\myflfigaw{\begin{tabular}{@{}c@{}}
\Example\\
\begin{exampleSimple}{9em}{5em}
\exmp{3\\
0 0 0 10 10 10\\
19 19 19 9 9 9\\
20 30 20 30 20 30}{2999}%
\end{exampleSimple}
\end{tabular}}
\InputFile 
Перший рядок містить натуральне число $N$ ($2\dib{{\<}}N\dib{{\<}}100$). Далі йдуть $N$\nolinebreak[3] рядків по\nolinebreak[3] 6\nolinebreak[3] цілих чисел\nolinebreak[3] --- спочатку координати якої-небудь вершини чергового паралелепіпеда, потім координати протилежної його вершини. Всі значення координат не~перевищують за модулем (абсолютною величиною) мільйон. Паралелепіпеди задовольняють всім вищеописаним обмеженням.

\OutputFile 
Виведіть єдине ціле число\nolinebreak[3] --- шуканий об'єм об'єднання.

\Note
Властивість <<лише ${K\,{\<}\,\min(15,N)}$ з усіх $N$ паралелепіпедів можуть брати участь у тих перетинах, де~\dots>> \emph{сильно} впливає на вибір найдоцільнішого способу розв'язання; цілком аналогічну задачу, де така вимога відсутня, слід було~б розв'язувати зовсім інакше.

}

\end{problemAllDefault}

